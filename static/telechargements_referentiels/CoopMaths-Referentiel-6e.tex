\documentclass[a4paper,12pt,fleqn]{article}
\input{preambule_coop}
\usepackage{pdfpages}
\begin{document}

\theme{nombres}{Référentiel}{6\up{e}}{Numération et fractions}

\titleformat*{\subsection}{\color{couleur_theme}\bfseries}
\renewcommand{\labelitemi}{}

\subsection*{Numération et fractions - Niveau 1}

\begin{itemize}[itemsep=1em]
	\item \reference{6N10}Connaitre le système décimal.
	\item \reference{6N11}Comparer, ranger, encadrer, repérer des grands nombres entiers.
	\item \reference{6N12}Multiplier un entier par 10, 100, 1~000\ldots
	\item \reference{6N13}Utiliser les préfixes multiplicateurs (déca à kilo).
	\item \reference{6N14}Comprendre et utiliser la notion de fraction dans des cas simples.
	%\item \reference{6N15}Résoudre un problème avec des grands nombres.

\end{itemize}

\subsection*{Numération et fractions - Niveau 2}

\begin{itemize}[itemsep=1em]
	\item \reference{6N20}Faire le lien entre les fractions et les nombres entiers.
	\item \reference{6N21}Repérer et placer des fractions sur une demi-droite graduée (origine visible).
	\item \reference{6N22}Faire des calculs simples avec des fractions à l’aide d’un dessin.
	\item \reference{6N23}Comprendre et utiliser différentes écritures d’un nombre.
	\item \reference{6N24}Utiliser les préfixes multiplicateurs et diviseurs (milli à kilo).
\end{itemize}

\subsection*{Numération et fractions - Niveau 3}

\begin{itemize}[itemsep=1em]
	\item \reference{6N30}Repérer et placer des nombres décimaux sur une demi-droite graduée adaptée.
	\item \reference{6N31}Comparer, ranger, encadrer, intercaler des nombres décimaux.
	\item \reference{6N32}Repérer et placer des fractions sur une demi-droite graduée (origine non visible).
	\item  \reference{6N33}Calculer la fraction d’une quantité.
	\item \reference{6N34}Utiliser les préfixes multiplicateurs et diviseurs (milli à téra).
\end{itemize}

\subsection*{Numération et fractions - Niveau 4}

\begin{itemize}[itemsep=1em]
	\item \reference{6N40}Repérer et placer des nombres décimaux sur une demi-droite graduée - niveau 2.
	\item \reference{6N41}Établir des égalités entre des fractions simples.
	\item \reference{6N42}Multiples et diviseurs des nombres d’usage courant.
	\item \reference{6N43}Critères de divisibilité.

\end{itemize}
\newpage

%%%%%%%%%%%%%%%%%%%
%%%%% CALCULS %%%%%
%%%%%%%%%%%%%%%%%%%

\theme{nombres}{Référentiel}{6\up{e}}{Calculs}


\titleformat*{\subsection}{\color{couleur_theme}\bfseries}

\subsection*{Calculs - Niveau 1}

\begin{itemize}[itemsep=1em]
	\item \reference{C10}Additionner, soustraire et multiplier des nombres entiers.
	\item \reference{C11}Calculer des divisions euclidiennes simples.
	\item \reference{C12}Résoudre des problèmes de niveau 1.

\end{itemize}

\subsection*{Calculs - Niveau 2}

\begin{itemize}[itemsep=1em]
	\item \reference{C20}Additionner et soustraire des nombres décimaux.
	\item \reference{C21}Calculer une division euclidienne de niveau 2.
	\item \reference{C22}Résoudre des problèmes de niveau 2.

\end{itemize}

\subsection*{Calculs - Niveau 3}

\begin{itemize}[itemsep=1em]
	\item \reference{C30}Multiplier des nombres décimaux.
	\item \reference{C31}Effectuer une division décimale.
	\item \reference{C32}Résoudre des problèmes de niveau 3
\end{itemize}

\newpage
%%%%%%%%%%%%%%%%%%%%%%%%%%%%
%%%%% Proportionnalité %%%%%
%%%%%%%%%%%%%%%%%%%%%%%%%%%%

\theme{gestion}{Référentiel}{6\up{e}}{Gestion de données}

\subsection*{Proportionnalité}

\begin{itemize}[itemsep=1em]
	\item \reference{6P10}Reconnaitre des problèmes relevant de la proportionnalité.
	\item \reference{6P11}Résoudre un problème relevant de la proportionnalité avec les propriétés de linéarité.	
	\item \reference{6P12}Calculer et utiliser un coefficient de proportionnalité.
	\item \reference{6P13}Appliquer un pourcentage.
	\item \reference{6P14}Reproduire une figure en respectant une échelle donnée.
	\item \reference{6P15}Résoudre un problème impliquant des échelles ou des vitesses.
\end{itemize}

\subsection*{Statistiques}

\begin{itemize}[itemsep=1em]
	\item \reference{6S10}Lire une représentation de données (tableaux ; diagrammes en bâtons, circulaires ou semi-circulaires ; graphiques cartésiens).
	\item \reference{6S11}Organiser des données en vue de les traiter.
\end{itemize}

\newpage
%%%%%%%%%%%%%%%%%%
%%%%% Mesure %%%%%
%%%%%%%%%%%%%%%%%%

\theme{grandeurs}{Référentiel}{6\up{e}}{Grandeurs et mesures}

\subsection*{Grandeurs et mesures - Niveau 1}

\begin{itemize}[itemsep=1em]
	\item \reference{6M10}Déterminer le périmètre d’un polygone.
	\item \reference{6M11}Déterminer l’aire d’un carré ou d’un rectangle (ou de figures composées de rectangles et carrés).
	\item \reference{6M12}Convertir des longueurs.
\end{itemize}

\subsection*{Grandeurs et mesures - Niveau 2}

\begin{itemize}[itemsep=1em]
	\item \reference{6M20}Déterminer l’aire d’un triangle.
	\item \reference{6M21}Déterminer l’aire d’un polygone par assemblage ou par découpage.
	\item \reference{6M22}Déterminer le périmètre ou l’aire d’un disque.
	\item \reference{6M23}Convertir des aires.
	\item \reference{6M24}Résoudre un problème en utilisant les périmètres et les aires.


\end{itemize}

\subsection*{Grandeurs et mesures - Niveau 3}

\begin{itemize}[itemsep=1em]
	\item \reference{6M30}Déterminer le volume d’un pavé droit.
	\item \reference{6M31}Convertir des volumes (et faire le lien avec les contenances).
\end{itemize}

\subsection*{Les durées}

\begin{itemize}[itemsep=1em]
	\item \reference{6D10}Réaliser des conversions de durées (avec une ou deux étapes de traitement) et utiliser les heures décimales.
	\item \reference{6D11}Calculer des durées ou déterminer un horaire.
	\item \reference{6D12}Effectuer des calculs avec des durées.
\end{itemize}

\newpage
%%%%%%%%%%%%%%%%%%%%%
%%%%% Géométrie %%%%%
%%%%%%%%%%%%%%%%%%%%%

\theme{geo}{Référentiel}{6\up{e}}{Géométrie}

\subsection*{Constructions géométriques - Niveau 1}

\begin{itemize}[itemsep=1em]
	\item \reference{G10}Connaitre le vocabulaire et les notations des points, des droites, des segments, des demi-droites et des cercles. 
	\item \reference{G11}Tracer des perpendiculaires.
	\item \reference{G12}Tracer des parallèles.
	\item \reference{G13}Tracer des rectangles et des carrés de longueurs données.
	\item \reference{G14}Exécuter un programme de construction de niveau 1.
\end{itemize}

\subsection*{Constructions géométriques - Niveau 2}

\begin{itemize}[itemsep=1em]
	\item \reference{G20}Connaître le vocabulaire des polygones.
	\item \reference{G21}Tracer un polygone avec le compas et l’équerre.
	\item \reference{G22}Connaître le vocabulaire et les notations des angles.
	\item \reference{G23}Utiliser le rapporteur pour tracer ou mesurer un angle.
	\item \reference{G24}Tracer le symétrique d’une figure.
	\item \reference{G25}Tracer la médiatrice d’un segment.
\end{itemize}

\subsection*{Constructions géométriques - Niveau 3}

\begin{itemize}[itemsep=1em]
	\item \reference{G30}Exécuter un programme de construction complexe.
	\item \reference{G31}Agrandissement ou réduction de figures.
	\item \reference{G32}Connaître et utiliser les propriétés de conservation de la symétrie axiale.
	\item \reference{G33}Connaître et utiliser les propriétés des polygones particuliers.
\end{itemize}

\subsection*{Espace}

\begin{itemize}[itemsep=1em]
	\item \reference{6G40}Reconnaître des solides (pavé droit, cube, cylindre, prisme droit, pyramide, cône, boule) à partir d’un objet réel, d’une image, d’une représentation en perspective cavalière.
	\item \reference{6G41}Construire et mettre en relation une représentation en perspective cavalière et un patron d’un pavé droit.
    \item \reference{6G42}Connaître et utiliser les propriétés géométriques des cubes et pavés droits.
    \item \reference{6G43}Connaître et utiliser le vocabulaire des pavés droits afin de les décrire.

\end{itemize}



\end{document}