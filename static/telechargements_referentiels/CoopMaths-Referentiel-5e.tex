\documentclass[a4paper,12pt,fleqn]{article}
\input{preambule_coop}
\usepackage{pdfpages}
\begin{document}

\titleformat*{\subsection}{\color{couleur_theme}\bfseries}
\renewcommand{\labelitemi}{}


\theme{nombres}{Référentiel}{5\up{e}}{Nombres et calculs}


	
\subsection*{Calculs}

\begin{itemize}[itemsep=1em]
	\item \reference{5C10}	Calculer le quotient et le reste dans une division euclidienne.																								
	\item \reference{5C11}	Traduire un enchaînement d’opérations à l’aide d’une expression avec des parenthèses.																								
	\item \reference{5C12}	Effectuer un enchaînement d’opérations en respectant les priorités opératoires.	
\end{itemize}

\subsection*{Arithmétique}

\begin{itemize}[itemsep=1em]																									
	\item \reference{5A10}	Déterminer si un nombre entier est ou n’est pas multiple ou diviseur d’un autre nombre entier.																								
	\item \reference{5A11}	Utiliser les critères de divisibilité (par 2, 3, 5, 9, 10).																								
	\item \reference{5A12}	Déterminer les nombres premiers inférieurs ou égaux à 30.																								
	\item \reference{5A13}	Décomposer un nombre entier strictement positif en produit de facteurs premiers inférieurs à 30.																								
	\item \reference{5A14}	Modéliser et résoudre des problèmes faisant intervenir les notions de multiple, de diviseur, de quotient et de reste.	
\end{itemize}

\subsection*{Numération et fractions}

\begin{itemize}[itemsep=1em]																									
	\item \reference{5N10}	Utiliser les écritures décimales et fractionnaires et passer de l’une à l’autre.																								
	\item \reference{5N11}	Relier fractions, proportions et pourcentages.																								
	\item \reference{5N12}	Décomposer une fraction sous la forme d’une somme (ou d’une différence) d’un entier et d'une fraction																								
	\item \reference{5N13}	Reconnaître et produire des fractions égales.																								
	\item \reference{5N14}	Comparer, ranger, encadrer des fractions dont les dénominateurs sont égaux ou multiples l’un de l'autre.
\end{itemize}

\subsection*{Calculs avec des fractions}

\begin{itemize}[itemsep=1em]							
	\item \reference{5N20}	Additionner ou soustraitre des fractions dont les dénominateurs sont égaux ou multiples l’un de l’autre.																								
	\item \reference{5N21}	Utiliser la décomposition en facteurs premiers inférieurs pour produire des fractions égales.	
\end{itemize}

\newpage
\subsection*{Relatifs - Niveau 1}

\begin{itemize}[itemsep=1em]																								
	\item \reference{5R10}	Utiliser la notion d'opposé.																								
	\item \reference{5R11}	Repérer un point sur une droite graduée les nombres décimaux relatifs.																								
	\item \reference{5R12}	Repérer un point dans le plan muni d’un repère orthogonal.		
\end{itemize}

\subsection*{Relatifs - Niveau 2}

\begin{itemize}[itemsep=1em]																																								
	\item \reference{5R20}	Additionner des nombres décimaux relatifs.																								
	\item \reference{5R21}	Soustraire des nombres décimaux relatifs.
	\item \reference{5R22} Effectuer une somme algébrique.
\end{itemize}

\subsection*{Calcul littéral}

\begin{itemize}[itemsep=1em]																								
	\item \reference{5L10}	Produire une expression littérale pour élaborer une formule ou traduire un programme de calcul.																								
	\item \reference{5L12}	Utiliser le calcul littéral pour démontrer une propriété générale.																								
	\item \reference{5L13}	Utiliser la distributivité simple pour réduire une expression littérale de la forme $ax + bx$ où $a$ et $b$ sont des nombres décimaux.																								
	\item \reference{5L14}	Calculer la valeur d’une expression littérale.																								
	\item \reference{5L15}	Tester si une égalité où figurent une ou deux indéterminées est vraie quand on leur attribue des valeurs numériques.																							
\end{itemize}

\newpage
%%%%%%%%%%%%%%%
%%% Données %%%
%%%%%%%%%%%%%%%

\theme{gestion}{Référentiel}{5\up{e}}{Grandeurs et mesure}



\subsection*{Proportionnalité}

\begin{itemize}[itemsep=1em]
	\item \reference{5P10}Reconnaître une situation de proportionnalité ou de non proportionnalité́ entre deux grandeurs.
	\item \reference{5P11}Résoudre des problèmes de proportionnalité avec des procédures variées (additivité, homogénéité, passage à l’unité, coefficient de proportionnalité).
	\item \reference{5P12}Partager une quantité en deux ou trois parts selon un ratio donné.
	\item \reference{5P13}Utiliser l’échelle d’une carte.
\end{itemize}

\subsection*{Statistiques}

\begin{itemize}[itemsep=1em]
	\item \reference{5S10}Recueillir et organiser des données.
	\item \reference{5S11}Lire et interpréter des données brutes ou présentées sous forme de tableaux, de diagrammes et de graphiques.
	\item \reference{5S12}Représenter, sur papier ou à l’aide d’un tableur-grapheur, des données sous la forme d’un tableau, d’un diagramme ou d’un graphique.
	\item \reference{5S13}Calculer des effectifs et des fréquences.
	\item \reference{5S14}Calculer et interpréter la moyenne d’une série de données.
\end{itemize}

\subsection*{Probabilités}

\begin{itemize}[itemsep=1em]
	\item \reference{5S20}Placer un événement sur une échelle de probabilités.
	\item \reference{5S21}Calculer des probabilités dans des situations simples d’équiprobabilité.
\end{itemize}

\newpage
%%%%%%%%%%%%%%
%%% Mesure %%%
%%%%%%%%%%%%%%

\theme{grandeurs}{Référentiel}{5\up{e}}{Grandeurs et mesure}


\subsection*{Périmètre et aire}

\begin{itemize}[itemsep=1em]
	\item \reference{5M10}Calculer le périmètre et l’aire des figures usuelles (rectangle, parallélogramme, triangle,disque).
	\item \reference{5M11}Calculer le périmètre et l’aire d’un assemblage de figures.
	\item \reference{5M12}Effectuer des conversions d’unités de longueurs.
	\item \reference{5M13}Effectuer des conversions d’unités d’aires.
\end{itemize}

\subsection*{Volume}

\begin{itemize}[itemsep=1em]
	\item \reference{5M20}Calculer le volume d’un pavé droit, d’un prisme droit, d’un cylindre.
	\item \reference{5M21}Calculer le volume d’un assemblage de pavés, prismes et/ou cylindres.
	\item \reference{5M22}Effectuer des conversions d’unités de volumes.
	\item \reference{5M23}Utiliser la correspondance entre les unités de volume et de contenance pour effectuer des conversions.
\end{itemize}

\subsection*{Durée}

\begin{itemize}[itemsep=1em]
	\item \reference{5M30}Effectuer des conversions d’unités de durées.
	\item \reference{5M31}Effectuer des calculs de durées et d’horaires.
\end{itemize}

\newpage
%%%%%%%%%%%%%%%%%
%%% Géométrie %%%
%%%%%%%%%%%%%%%%%

\theme{geo}{Référentiel}{5\up{e}}{Géométrie}


\subsection*{Symétries}

\begin{itemize}[itemsep=1em]
	\item \reference{5G10}Transformer une figure par symétrie axiale.
	\item \reference{5G11}Transformer une figure par symétrie centrale.
	\item \reference{5G12}Identifier des symétries dans des frises, des pavages, des rosaces.
	\item \reference{5G13}Utiliser les propriétés de conservation du parallélisme, des longueurs et des angles.
\end{itemize}

\subsection*{Triangles}

\begin{itemize}[itemsep=1em]
	\item \reference{5G20}Construire des triangles connaissant des longueurs et/ou des angles.
	\item \reference{5G21}Connaître et utiliser l'inégalité triangulaire.
	\item \reference{5G22}Connaître et utiliser la définition de la médiatrice.
	\item \reference{5G23}Connaître et utiliser la définition des hauteurs d’un triangle.
\end{itemize}

\subsection*{Angles}

\begin{itemize}[itemsep=1em]
	\item \reference{5G30}Connaître et utiliser les caractérisations angulaires du parallélisme (angles alternes internes, angles correspondants).
	\item \reference{5G31}Connaître et utiliser la somme des angles d’un triangle.
\end{itemize}

\subsection*{Parallélogrammes}

\begin{itemize}[itemsep=1em]
	\item \reference{5G40}Connaître et construire un parallélogramme.
	\item \reference{5G41}Connaître et construire un parallélogramme particulier.
	\item \reference{5G42}Connapitre et utiliser les propriétés des parallélogrammes.
\end{itemize}

\subsection*{Espace}

\begin{itemize}[itemsep=1em]
	\item \reference{5G50}Reconnaître des solides (pavé droit, cube, cylindre, prisme droit, pyramide, cône, boule) à partir d’un objet réel, d’une image, d’une représentation en perspective cavalière.
	\item \reference{5G51}Construire et mettre en relation une représentation en perspective cavalière et un patron d’un pavé droit, d’un cylindre et d'un prisme droit.
    \item\reference{5G52}Connaître et utiliser les propriétés géométriques des cubes, prismes droits et cylindres.
\end{itemize}




\end{document}


\end{document}