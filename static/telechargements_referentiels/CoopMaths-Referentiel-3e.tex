\documentclass[a4paper,12pt,fleqn]{article}
\input{preambule_coop}
\usepackage{pdfpages}
\begin{document}

\titleformat*{\subsection}{\color{couleur_theme}\bfseries}
\renewcommand{\labelitemi}{}

\theme{nombres}{Référentiel}{3\up{e}}{Nombres et Calculs}


\subsection*{Calcul littéral}

\begin{itemize}
	\item \reference{3L10}Déterminer l’opposé d’une expression littérale.
	\item \reference{3L11}Développer (par simple et double distributivités), factoriser, réduire des expressions algébriques simples.
	\item \reference{3L12}Factoriser une expression du type $a^2-b^2$ et développer des expression du type $(a+b)(a-b)$.
	\item \reference{3L13}Résoudre algébriquement une équation du premier degré.
	\item \reference{3L14}Résoudre algébriquement une équation produit.
	\item \reference{3L15}Résoudre algébriquement une équations de la forme $x^2=a$ sur des exemples simples.
	\item \reference{3L16}Résoudre des problèmes se ramenant à une équation, qui peuvent être internes aux mathématiques ou en lien avec d’autres disciplines.

\end{itemize}

\subsection*{Arithmétique}

\begin{itemize}
	\item \reference{3A10}Décomposer un nombre entier en produit de facteurs premiers (à la main, à l’aide d’un tableur ou d’un logiciel de programmation).
	\item \reference{3A11}Simplifier une fraction pour la rendre irréductible.
	\item \reference{3A12}Modéliser et résoudre des problèmes mettant en jeu la divisibilité (engrenages, conjonction de phénomènes...).
\end{itemize}

\subsection*{Nombres et calculs}

\begin{itemize}
	\item \reference{3N10}Utiliser les puissances d’exposants positifs ou négatifs pour simplifier l’écriture des produits.
	\item \reference{3N11}Calculer avec les nombres rationnels, notamment dans le cadre de résolution de problèmes.
	\item \reference{3N12}Résoudre des problèmes mettant en jeu des racines carrées.
	\item \reference{3N13}Résoudre des problèmes avec des puissances, notamment en utilisant la notation scientifique.
\end{itemize}


\newpage
%%%%%%%%%%%%%%
%%% Mesure %%%
%%%%%%%%%%%%%%

\theme{gestion}{Référentiel}{3\up{e}}{Organisation et gestion de données, fonctions}

\subsection*{Généralités sur les fonctions}

\begin{itemize}
	\item \reference{3F10}Utiliser les notations et le vocabulaire fonctionnels.
	\item \reference{3F11}Passer d’un mode de représentation d’une fonction à un autre.
	\item \reference{3F12}Déterminer, à partir de tous les modes de représentation, l’image d’un nombre.
	\item \reference{3F13}Déterminer un antécédent à partir d‘une représentation graphique ou d’un tableau de valeurs d’une fonction.
	\item \reference{3F14}Modéliser un phénomène continu par une fonction.
	\item \reference{3F15}Résoudre des problèmes modélisés par des fonctions en utilisant un ou plusieurs modes de représentation.
\end{itemize}

\subsection*{Fonctions affines et linéaires}

\begin{itemize}
	\item \reference{3F20}Représenter graphiquement une fonction linéaire, une fonction affine.
	\item \reference{3F21}Interpréter les paramètres d’une fonction affine suivant l’allure de sa courbe représentative.
	\item \reference{3F22}Modéliser une situation de proportionnalité à l’aide d’une fonction linéaire.
	\item \reference{3F23}Déterminer de manière algébrique l’antécédent par une fonction, dans des cas se ramenant à la résolution d’une équation du premier degré.
\end{itemize}

\subsection*{Proportionnalité}

\begin{itemize}
	\item \reference{3P10}Utiliser le lien entre pourcentage d’évolution et coefficient multiplicateur.
	\item \reference{3P11}Mener des calculs sur des grandeurs mesurables, notamment des grandeurs composées, et exprimer les résultats dans les unités adaptées.
	\item \reference{3P12}Résoudre des problèmes utilisant les conversions d’unités sur des grandeurs composées.
	\item \reference{3P13}Vérifier la cohérence des résultats du point de vue des unités pour les calculs de grandeurs simples ou composées.
	\item \reference{3P14}Résoudre des problèmes en utilisant la proportionnalité dans le cadre de la géométrie.
\end{itemize}

\subsection*{Statistiques}

\begin{itemize}
	\item \reference{3S10}Lire, interpréter et représenter des données sous forme d’histogrammes pour des classes de même amplitude.
	\item \reference{3S11}Calculer et interpréter l’étendue d’une série présentée sous forme de données brutes, d’un tableau, d’un diagramme en bâtons, d’un diagramme circulaire ou d’un histogramme.
	\item \reference{3S12}Calculer des effectifs et des fréquences.
\end{itemize}

\subsection*{Probabilités}

\begin{itemize}
	\item \reference{3S20}À partir de dénombrements, calculer des probabilités pour des expériences aléatoires simples à une ou deux épreuves.
	\item \reference{3S21}Faire le lien entre stabilisation des fréquences et probabilités.
\end{itemize}

%%%%%%%%%%%%%%%%%
%%% Géométrie %%%
%%%%%%%%%%%%%%%%%

\theme{geo}{Référentiel}{3\up{e}}{Géométrie}

\subsection*{Homothétie et rotation}

\begin{itemize}
	\item \reference{3G10}Transformer une figure par rotation et comprendre l’effet d’une rotation.
	\item \reference{3G11}Transformer une figure par homothétie et comprendre l’effet d’une homothétie.
	\item \reference{3G12}Identifier des rotations et des homothéties dans des frises, des pavages et des rosaces.
	\item \reference{3G13}Mobiliser les connaissances des figures, des configurations, de la rotation et de l’homothétie pour déterminer des grandeurs géométriques.
	\item \reference{3G14}Calculer des grandeurs géométriques (longueurs, aires et volumes) en utilisant les transformations (symétries, rotations, translations, homothétie).
	

\end{itemize}

\subsection*{Théorème de Thalès}

\begin{itemize}
	\item \reference{3G20}Calculer une longueur avec le théorème de Thalès.
	\item \reference{3G21}Démontrer que des droites sont parallèles avec le théorème de Thalès.
	\item \reference{3G22}Connaître et utiliser une définition et une propriété caractéristique des triangles semblables.
\end{itemize}


\subsection*{Trigonométrie}

\begin{itemize}
	\item \reference{3G30}Calculer une longueur dans un triangle rectangle.
	\item \reference{3G31}Calculer la mesure d'un angle dans un triangle rectangle.
	\item \reference{3G32}Résoudre un problème géométrique.
\end{itemize}

\subsection*{Espace}

\begin{itemize}
	\item \reference{3G40}Se repérer sur une sphère (latitude, longitude).
	\item \reference{3G41}Construire et mettre en relation différentes représentations des solides étudiés au cours du cycle (représentations en perspective cavalière, vues de face, de dessus, en coupe, patrons) et leurs sections planes.
	\item \reference{3G42}Calculer le volume d’une boule.
	\item \reference{3G43}Calculer les volumes d’assemblages de solides étudiés au cours du cycle.
\end{itemize}

\end{document}