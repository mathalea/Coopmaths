\documentclass[a4paper,12pt,fleqn]{article}
\input{preambule_coop}
\usepackage{pdfpages}
\begin{document}

\titleformat*{\subsection}{\color{couleur_theme}\bfseries}
\renewcommand{\labelitemi}{}

\theme{nombres}{Référentiel}{4\up{e}}{Calculs}


\subsection*{Relatifs}

\begin{itemize}[itemsep=1em]
	\item \reference{4C10}Effectuer des produits ou des quotients avec des nombres relatifs.
	\item \reference{4C11}Calculer avec des nombres relatifs.
\end{itemize}

\subsection*{Fractions}

\begin{itemize}[itemsep=1em]
	\item \reference{4C20}Comparer, ranger et encadrer des nombres rationnels (positifs ou négatifs).
	\item \reference{4C21}Additionner ou soustraire des nombres relatifs en écriture fractionnaire.
	\item \reference{4C22}Multiplier ou diviser des nombres relatifs en écriture fractionnaire.
	\item \reference{4C23}Effectuer un calcul avec des nombres relatifs et fractionnaires.
	\item \reference{4C24}Utiliser les nombres premiers pour reconnaître et produire des fractions égales ; pour simplifier des fractions.
	\item \reference{4C25}Résoudre des problèmes avec des nombres rationnels.
\end{itemize}

\subsection*{Puissances}

\begin{itemize}[itemsep=1em]
	\item \reference{4C30}Utiliser les puissances de 10 d’exposants positifs ou négatifs.
	\item \reference{4C31}Utiliser les préfixes de nano à giga.
	\item \reference{4C32}Associer, dans le cas des nombres décimaux, écriture décimale, écriture fractionnaire et notation scientifique.
	\item \reference{4C33}Utiliser les puissances d’exposants strictement positifs d’un nombre pour simplifier l’écriture des produits.
\end{itemize}

\subsection*{Calcul littéral}

\begin{itemize}[itemsep=1em]
	\item \reference{4L10}Utiliser la propriété de distributivité simple pour développer un produit ou réduire une expression littérale.
	\item \reference{4L11}Utiliser la propriété de distributivité simple pour factoriser une somme.
	\item \reference{4L12}Démontrer l’équivalence de deux programmes de calcul.
	\item \reference{4L13}Introduire une lettre pour désigner une valeur inconnue et mettre un problème en équation.
	\item \reference{4L14}Tester si un nombre est solution d’une équation.
	\item \reference{4L15}Résoudre algébriquement une équation du premier degré.
\end{itemize}

\newpage
%%%%%%%%%%%%%%
%%% Mesure %%%
%%%%%%%%%%%%%%

\theme{gestion}{Référentiel}{4\up{e}}{Gestion de données}

\subsection*{Statistiques}

\begin{itemize}[itemsep=1em]
	\item \reference{4S10}Lire, interpréter et représenter des données sous forme de diagrammes circulaires.
	\item \reference{4S11}Calculer et interpréter la médiane d’une série de données de petit effectif total.
\end{itemize}

\subsection*{Probabilités}

\begin{itemize}[itemsep=1em]
	\item \reference{4S20}Utiliser le vocabulaire des probabilités : expérience aléatoire, issues, événement, probabilité, événement certain, événement impossible, événement contraire.
	\item \reference{4S21}Reconnaître des événements contraires et s’en servir pour calculer des probabilités.
	\item \reference{4S22}Calculer des probabilités.
	\item \reference{4S23}Exprimer des probabilités sous diverses formes (nombre compris entre 0 et 1, pourcentage, fraction).
\end{itemize}

\subsection*{Proportionnalité}

\begin{itemize}[itemsep=1em]
	\item \reference{4P10}Reconnaître sur un graphique une situation de proportionnalité ou de non proportionnalité.
	\item \reference{4P11}Calculer une quatrième proportionnelle par la procédure de son choix.
	\item \reference{4P12}Utiliser une formule liant deux grandeurs dans une situation de proportionnalité.
	\item \reference{4P13}Résoudre des problèmes en utilisant la proportionnalité dans le cadre de la géométrie.
	\item \reference{4P14}Construire un agrandissement ou une réduction d’une figure donnée.
	\item \reference{4P15}Utiliser un rapport d’agrandissement ou de réduction pour calculer, des longueurs, des aires, des volumes.
\end{itemize}

\subsection*{Notion de fonction}

\begin{itemize}[itemsep=1em]
	\item \reference{4F10}Produire une formule littérale représentant la dépendance de deux grandeurs.
	\item \reference{4F11}Représenter la dépendance de deux grandeurs par un graphique.
	\item \reference{4F12}Utiliser un graphique représentant la dépendance de deux grandeurs pour lire et interpréter différentes valeurs sur l’axe des abscisses ou l’axe des ordonnées.
\end{itemize}


\newpage
%%%%%%%%%%%%%%%%%
%%% Géométrie %%%
%%%%%%%%%%%%%%%%%

\theme{geo}{Référentiel}{4\up{e}}{Géométrie}

\subsection*{Translation et rotation}

\begin{itemize}[itemsep=1em]
	\item \reference{4G10}Transformer une figure par translation.
	\item \reference{4G11}Identifier des translations dans des frises et des pavages.
	\item \reference{4G12}Comprendre et utiliser l’effet d’une translation : conservation du parallélisme, des longueurs, des aires et des angles.
	\item \reference{4G13}Mener des raisonnements en utilisant des propriétés des figures, des configurations et de la translation.
\end{itemize}

\subsection*{Théorème de Pythagore}

\begin{itemize}[itemsep=1em]
	\item \reference{4G20}Calculer une longueur avec le théorème de Pythagore.
	\item \reference{4G21}Démontrer qu'un triangle est rectangle ou non.
	\item \reference{4G22}Résoudre un problème géométrique en ayant recours au théorème de Pythagore.
\end{itemize}

\subsection*{Théorème de Thalès}

\begin{itemize}[itemsep=1em]
	\item \reference{4G30}Calculer une longueur avec le théorème de Thalès.
	\item \reference{4G31}Démontrer que des droites sont parallèles avec le théorème de Thalès.
	\item \reference{4G32}Résoudre un problème géométrique en ayant recours aux théorèmes de Thalès et de Pythagore.
\end{itemize}

\subsection*{Cosinus d'un angle}

\begin{itemize}[itemsep=1em]
	\item \reference{4G40}Calculer une longueur avec le cosinus d'un angle.
	\item \reference{4G41}Calculer la mesure d'un angle à partir de son cosinus.
	\item \reference{4G42}Résoudre un problème géométrique.
\end{itemize}

\subsection*{Espace}

\begin{itemize}[itemsep=1em]
	\item \reference{4G50}Construire et mettre en relation une représentation en perspective cavalière et un patron d’une pyramide.
	\item \reference{4G51}Construire et mettre en relation une représentation en perspective cavalière et un patron d’un d’un cône de révolution.
	\item \reference{4G52}Se repérer dans un pavé droit et utiliser le vocabulaire du repérage : abscisse, ordonnée, altitude.
	\item \reference{4G53}Calculer le volume d’une pyramide, d’un cône.
\end{itemize}



\end{document}

