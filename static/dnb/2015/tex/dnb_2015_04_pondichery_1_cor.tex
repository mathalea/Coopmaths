
\medskip

%Cet exercice est un QCM (questionnaire à choix multiples).
%
%Pour chaque ligne du tableau, une seule affirmation est juste.
%
%Sur votre copie, indiquer le numéro de la question et recopier l'affirmation juste.
%
%On ne demande pas de justifier.
%
%\begin{center}
%\begin{tabularx}{\linewidth}{|c|m{3.5cm}|*{3}{>{\centering  \arraybackslash}X|}}\hline
%	&Questions 			&A			& B			& C\\ \hline
%1	& La forme développée de $(x - 1)^2$ est :&$(x - 1)(x + 1)$&  $x^2 - 2x + 1$& $x^2 + 2x + 1$.\\ \hline
%2	& Une solution de l'équation : $2x^2 + 3x - 2 = 0$ est&0&2&$- 2$\\ \hline
%3	& On considère la fonction $f : x \longmapsto  3x+2$.  Un antécédent de $- 7$ par la fonction $f$ est :&$- 19$& $- 3$& $- 7$\\ \hline
%4	& Lorsqu'on regarde un angle de 18\,\degres{} à la loupe de grossissement 2, on voit 
%un angle de :&9\degres& 36\degres& 18\degres\\ \hline
%5	& On considère la fonction $g :  x \longmapsto x^2 + 7$.
%Quelle est la formule à entrer dans la cellule B2 pour calculer $g(- 2)$ ?
%%\psset{unit=0.5cm}
%\begin{tabular}{|c|c|c|}\hline
%	&A		&B \\ \hline
%1	&$x$	&$g(x)$\\ \hline
%2	&$- 2$	&\\ \hline
%3	&		&\\ \hline
%\end{tabular} &= A2 $\hat{}$ 2 + 7& $= - 2^2 + 7$& = A$2*2$ + 7\\ \hline
%\end{tabularx}
%\end{center}
\begin{enumerate}
\item $(x - 1)^2 = x^2 + 1 - 2x$. Réponse B
\item $2\times (- 2)^2 + 3 \times (- 2) - 2 = 2 \times 4  - 6 - 2 = 8 - 8 = 0$. Réponse C
\item Il faut résoudre l'équation $3x + 2 = - 7$ soit $3x = - 9$ et enfin $x = - 3$. Réponse B.
\item L'angle de 18\,\degres{} reste un angle de 18\,\degres{}. Réponse C
\item Réponse A.
\end{enumerate}

\vspace{0.5cm}

