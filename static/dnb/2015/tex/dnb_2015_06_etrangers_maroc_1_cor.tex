
\medskip

%L'objectif du passage à l'heure d'été est de faire correspondre au mieux les heures
%d'activité avec les heures d'ensoleillement pour limiter l'utilisation de l'éclairage
%artificiel.
%
%Le graphique ci-dessous représente la puissance consommée en mégawatts (MW),
%en fonction des heures (h) de deux journées J1 et J2, J1 avant le passage à l'heure
%d'été et J2 après le passage à l'heure d'été.
%
%\begin{center}
%\psset{xunit=0.45cm,yunit=0.0005cm}
%\begin{pspicture}(-3,-200)(24,20800)
%\multido{\n=0+1}{25}{\psline[linewidth=0.15pt](\n,0)(\n,20400)}
%\multido{\n=0+1700}{13}{\psline[linewidth=0.15pt](0,\n)(24,\n)}
%\psaxes[linewidth=1.25pt,Dx=3,Oy=51100,Dy=1700,labelFontSize=\scriptstyle](0,0)(0,0)(24,20400)
%\pscurve(0,12750)(0.25,11000)(0.5,10200)(1,7800)(1.4,9360)(2,8700)(2.5,7630)(3,6800)(3.5,5100)(4,3400)(4.5,3390)(5,3550)(5.5,6800)(6,10200)(7,17000)(7.5,17900)(8,18600)(8.5,18700)(9,18800)(9.5,18400)(10,17880)(11,16250)(12,16050)(12.25,15100)(13,15150)(14,12850)(15,10500)(16,8580)(17,7150)(18,8800)(19,15300)(19.5,17750)(20,18800)(20.5,17750)(21,13100)(21.5,11000)(22,9650)(22.5,10200)(22.75,11100)(23,12700)(23.5,12000)(24,11800)
%\pscurve[linestyle=dashed](0,8700)(0.5,6800)(1,5950)(1.5,5250)(2,5180)(2.5,4850)(3,3400)(3.5,2560)(4,1950)(4.5,1800)(5,1700)(5.5,3200)(6,5100)(6.5,7100)(7,10500)
%(7.5,12750)(8,14650)(8.5,15300)(9,15320)(9.5,15300)(10,15320)(11,15000)(12,13620)
%(13,12000)(14,10300)(15,8500)(16,6800)(17,5900)(18,5950)(19,6800)(19.5,8500)(20,8800)(21,10200)
%(22,8400)(23,9350)(24,8500)
%\uput[u](22.5,0){Heure (h)}
%\uput[r](0,20700){Puissance consommée (MW)}
%\end{pspicture}
%
%\end{center}
%
%\begin{pspicture}(1,0.1)\psline(0,0)(1,0)\end{pspicture}  J1 : avant le passage à l'heure d'été
%
%\begin{pspicture}(1,0.1)\psline[linestyle=dashed](0,0)(1,0)\end{pspicture}  J2 : après le passage à l'heure d'été
%
%
%\emph{Par lecture graphique, répondre aux questions suivantes.\\
%On arrondira, si nécessaire, les résultats à la demi-heure.}
%
%\medskip

\begin{enumerate}
\item %Pour la journée J1, quelle est la puissance consommée à 7~h ?
On lit à 7 h une consommation de \np{68100}~MW.
\item %Pour la journée J2, à quelle(s) heure(s) de la journée a-t-on une puissance
%consommée de \np{54500}~MW ?
La consommation est de \np{54500}~MW à 3 h et à 5 h 30 min.
\item %À quel moment de la journée le passage à l'heure d'été permet-il le plus d'économies ?
L'écart le plus grand entre les deux courbes se situe vers 19 h 30 min.
\item %Quelle puissance consommée a-t-on économisée à 19~h30 ?
La différence précédente se monte à \np{10200}~MW.
\end{enumerate}

\vspace{0,5cm}

