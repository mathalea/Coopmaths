
\medskip

%Marc veut fabriquer un bonhomme de neige en bois.
%
%Pour cela, il achète deux boules : une boule pour la tête de rayon 3~cm et
%une autre boule pour le corps dont le rayon est 2 fois plus grand.
%
%\medskip

\begin{enumerate}
\item 
	\begin{enumerate}
		\item %Vérifier que le volume de la boule pour la tête est bien $36\pi$~cm$^3$.
		Volume d'une boule de rayon 3~cm : $\dfrac{4}{3}\pi \times 3^3 = 4\pi \times 3^2 = 36\pi$~cm$^3$.
		\item %En déduire le volume exact en cm$^3$ de la boule pour le corps.
		Le rayon étant 2 fois plus grand le volume est $2^3 = 8$ fois plus grand donc égal à $8 \times 36\pi = 288\pi$~cm$^3$.
	\end{enumerate}
\item ~
%Marc coupe les deux boules afin de les assembler pour obtenir le bonhomme de neige.

%Il coupe la boule représentant la tête par un plan situé à 2 cm de son centre.

%Quelle est l'aire de la surface d'assemblage de la tête et du corps ? Arrondir le résultat au cm$^2$.
\begin{center}
\psset{unit=0.5cm}
\begin{pspicture}(-3,-3)(3,3)
%\psgrid
\pscircle(0,0){3}
\pspolygon(0,0)(-2.236,-2)(2.236,-2)
\uput[l](0,-1){2}\uput[ur](1.1,-1){3}\uput[d](1.1,-2){$r$}
\psline(0,0)(0,-2)
\end{pspicture}
\end{center}

La coupe est un disque dont le rayon $r$ est la longueur d'un triangle rectangle de côté 2 et d(hypoténuse 3 ; d'après le théorème de Pythagore, on a :

$r^2 + 2^2 = 3^2$, soit $r^2 = 9 - 4 = 5$, donc $r = \sqrt{5}$~cm.

L'aire du disque est donc égale à : $\pi \times r^2 = \pi \times 5 = 5\pi$~cm$^2 \approx 15,708$ soit environ 16~cm$^2$.
\end{enumerate}

\vspace{0.25cm}

