
\medskip

%Lors d'une étape cycliste, les distances parcourues par un cycliste ont été relevées
%chaque heure après le départ.
%
%Ces données sont précisées dans le graphique ci-dessous :
%
%\begin{center}
%\psset{xunit=2.1cm,yunit=0.0325cm}
%\begin{pspicture}(-0.5,-10)(6,220)
%\multido{\n=0.0+0.5}{13}{\psline[linestyle=dotted,linecolor=cyan](\n,0)(\n,220)}
%\multido{\n=0+10}{23}{\psline[linestyle=dotted,linecolor=cyan](0,\n)(6,\n)}
%\psaxes[linewidth=1.25pt,Dy=20]{->}(0,0)(0,0)(6,220)
%\psline[linewidth=1.25pt](0,0)(1,40)(2,70)(3,130)(4,170)(4.5,190)
%\uput[u](4.75,0){Durée de parcours (en heure)}
%\uput[r](0,215){Distance parcourue (en kilomètre)}
%\psdots[dotstyle=+,dotangle=45,dotscale=1.4](1,40)(2,70)(3,130)(4,170)(4.5,190)
%\end{pspicture}
%\end{center}
%
%Par lecture graphique, répondre aux questions suivantes.
%
%\emph{Aucune justification n'est demandée.}
%
%\medskip

\begin{enumerate}
\item 
	\begin{enumerate}
		\item %Quelle est la distance totale de cette étape ?
La distance totale de cette étape est de 190km.
		\item %En combien de temps le cycliste a-t-il parcouru les cent premiers kilomètres ?
Le cycliste a parcouru les cent premiers kilomètres en 2 heures et 30 minutes.
		\item %Quelle est la distance parcourue lors de la dernière demi-heure de course ?
La distance parcourue lors de la dernière demi-heure de course est 20~km ($= 190 – 170$).
	\end{enumerate}
\item  %Y-a-t-il proportionnalité entre la distance parcourue et la durée de parcours de cette étape ?
	
%Justifier votre réponse et proposer une explication.
Non car les points correspondants  ne sont pas alignés.
\end{enumerate}

\vspace{0,5cm}

