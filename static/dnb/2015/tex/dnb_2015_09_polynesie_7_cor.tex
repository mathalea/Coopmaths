
\medskip

%La distance d'arrêt est la distance que parcourt un véhicule entre le moment où son conducteur voit un obstacle et le moment où le véhicule s'arrête.
%
%Une formule permettant de calculer la distance d'arrêt est :
%
%\[D = \dfrac{5}{18} \times  V + 0,006 \times  V^2 \qquad \begin{tabular}{l}
%$\bullet$~~ D : \text{est la distance d'arrêt en m}\\
%$\bullet$~~ V : \text{la vitesse en km/h}
%\end{tabular}\]

\begin{enumerate}
\item %Un conducteur roule à 130~km/h sur l'autoroute. Surgit un obstacle à 100~m de lui. Pourra-t-il s'arrêter à temps ?
On a $D = \dfrac{5}{18} \times  130 + 0,006 \times  130^2 \approx 137,5$~(m) : le conducteur ne pourra pas s’arrêter à temps.
\item %On a utilisé un tableur pour calculer la distance d'arrêt pour quelques vitesses. Une copie de l'écran obtenu est donnée ci-dessous. La colonne B est configurée pour afficher les résultats arrondis à l'unité.

%\begin{center}
%\begin{tabularx}{0.6\linewidth}{|c|*{2}{>{\centering \arraybackslash}X|}}\hline
%&A &B\\ \hline
%1&Vitesse en km/h& Distance d'arrêt en m\\ \hline
%2&30& 14\\ \hline
%3&40& 21\\ \hline
%4&50& 29\\ \hline
%5&60& 38\\ \hline
%6&70& 49\\ \hline
%7&80& 61\\ \hline
%8&90& 74\\ \hline
%9&100& 88\\ \hline
%\end{tabularx}
%\end{center}

%Quelle formule a-t-on saisie dans la cellule B2 avant de la recopier vers le bas ?
En formatant la colonne B à l’unité près on tape en B2 : 

=A2 *5/18 + A2\verb|^| 2  *0,006.
\item %On entend fréquemment l'affirmation suivante: \og Lorsqu'on va deux fois plus vite, il faut une
%distance deux fois plus grande pour s'arrêter \fg. Est-elle exacte ?
Non : 38~(m) à la vitesse de 60~(km/h) est plus du double de 14~(m) pour s’arrêter à 30~(km/h).
\item %Au code de la route, on donne la règle suivante pour calculer de tête sa distance d'arrêt:
%\og Pour une vitesse comprise entre 50 km/h et 90 km/h, multiplier par lui-même le chiffre des dizaines de la vitesse \fg.
 
%Le résultat calculé avec cette règle pour un automobiliste qui roule à 80~km/h est-il cohérent avec celui calculé par la formule ?
On a $5^2 = 25$ pour une distance de 29 ;

$6^2 = 36$ pour une distance de 38 ;

$7^2 = 49$ pour une distance de 49 ;

$8^2 = 64$ pour une distance de 61 ; 

$9^2 = 81$ pour une distance de 74

Cette règle est à peu près cohérente avec la formule exacte.
\end{enumerate}
