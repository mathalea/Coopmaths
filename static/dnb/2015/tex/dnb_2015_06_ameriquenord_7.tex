
\medskip

La Pyramide du Louvre est une oeuvre de l'architecte Leoh Ming Pei.

Il s'agit d'une pyramide régulière dont la base est un carré de côté 35,50~mètres et dont
les quatre arêtes qui partent du sommet mesurent toutes 33,14~mètres.

\medskip
%\begin{center}
%\includegraphics*[angle=-90,width=8cm]{PyramidePei}
%\end{center}


\parbox{0.55\linewidth}{\begin{enumerate}
\item La Pyramide du Louvre est schématisée comme ci-contre.

Calculer la hauteur réelle de la Pyramide du Louvre.

On arrondira le résultat au centimètre.
\item On veut tracer le patron de cette pyramide à l'échelle $1/800$.
	\begin{enumerate}
		\item Calculer les dimensions nécessaires de ce patron en les arrondissant au millimètre.
		\item Construire le patron en faisant apparaître les traits de construction.
		
On attend une précision de tracé au mm.
	\end{enumerate}
\end{enumerate}}\hfill
\parbox{0.42\linewidth}{\psset{unit=1cm}
\begin{pspicture}(7,7)
%\psgrid
\pspolygon(0.5,0.5)(5.5,0.5)(6.7,1.8)(3.6,6)%ABCSA
\psline(5.5,0.5)(3.6,6)%BS
\pspolygon[linestyle=dashed](0.5,0.5)(1.7,1.8)(6.7,1.8)%ADCA
\pspolygon[linestyle=dashed](5.5,0.5)(1.7,1.8)(3.6,6)(3.6,1.15)%BDSH
%\psline[linestyle=dashed](3.7,6)(3.6,1.15)%SH
\uput[dl](0.5,0.5){A} \uput[dr](5.5,0.5){B} \uput[ur](6.7,1.8){C} 
\uput[ul](1.7,1.8){D} \uput[u](3.6,6){S} \uput[d](3.6,1.15){H}
\psline(3.8,1.1)(3.8,1.3)(3.6,1.4)(3.4,1.3)(3.4,1.1) 
\end{pspicture}}
