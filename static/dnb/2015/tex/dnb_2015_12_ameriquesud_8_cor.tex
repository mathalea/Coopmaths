
\medskip

%Sophie habite Toulouse et sa meilleure amie vient de déménager à Bordeaux. Elles
%décident de continuer à se voir. Sophie consulte les tarifs de train entre les deux
%villes :
%
%\setlength\parindent{8mm}
%\begin{itemize}
%\item un aller-retour coûte 40~\euro
%\item si elle achète un abonnement pour une année à 442~\euro, un aller-retour coûte
%alors moitié prix.
%\end{itemize}
%\setlength\parindent{0mm}
%
%Aider Sophie à choisir la formule la plus avantageuse en fonction du nombre de
%voyages.
%
%\emph{Dans cet exercice, toute trace de recherche, même non aboutie, sera prise en
%compte dans l'évaluation.}
Soit $x$ le nombre d'aller(s)-retour(s)

Sans abonnement Sophie paiera : $40x$ dans l'année.

Avec l'abonnement Sophie paiera : $442 + 20x$.

$\bullet~~$$40x < 442 + 20x$ ou $20x < 442$ ou $10x < 221$ et enfin $x < 22,1$ : jusqu'à 22 allers-retours il vaut mieux ne pas prendre l'abonnement.

$\bullet~~$$40x > 442 + 20x$ ou $20x > 442$ ou $10x > 221$ et enfin $x > 22,1$ : à partir de 23 allers-retours il est plus intéressant pour Sophie de prendre l'abonnement.

\emph{Remarque} : on peut aussi faire la représentation graphique de la fonction linéaire et de la fonction affine et lire pour quelles valeurs de $x$ l'une est en dessous de l'autre.
