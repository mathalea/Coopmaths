
\medskip

%Dans cet exercice, pour chaque affirmation numérotée 1., 2. et 3. des réponses sont
%proposées. Une seule est exacte.
%
%Écrire sur la copie pour chaque numéro la réponse correspondante.
%
%Aucune justification n'est attendue.
%
%\begin{center}
%\begin{tabularx}{\linewidth}{|c|m{3cm}|*{3}{>{\centering \arraybackslash}X|}}\hline
%\textbf{1.}& Les solutions de l'équation $(4x + 5)(x - 3) = 0$ sont :&$- \dfrac{5}{4}$ et 3&$ \dfrac{5}{4}$ et $- 3$&$- \dfrac{5}{4}$ et $- 3$\\ \hline
%\textbf{2.}&$\dfrac{8 \times 10^3 \times 28 \times 10^{-2}}{14 \times 10^{- 3}}$ est égal à :& \np{16000}& 0,16& $1,6 \times 10^5$\\ \hline
%\textbf{3.}&\rule[-3mm]{0mm}{9mm}$\dfrac{\sqrt{32}}{2}$ est égal :&$\sqrt{16}$&$\sqrt{8}$& 2,8\\ \hline
%\end{tabularx}
%\end{center}
\begin{enumerate}
\item $(4x + 5)(x - 3) = 0$ est équivalente à $4x + 5 = 0$ ou $x - 3 = 0$, c'est-à-dire à $4x = - - 5$ ou $x = 3$, soit finalement à $x = - \dfrac{5}{4}$ ou $x = 3$.
\item $\dfrac{8 \times 10^3 \times 28 \times 10^{-2}}{14 \times 10^{- 3}} = \dfrac{8 \times 2 \times 14}{14} \times \dfrac{10^1}{10^{-3}} = 16 \times 10^{1 + 3} = 1,6 \times 10^4$.
\item $\dfrac{\sqrt{32}}{2} = \dfrac{\sqrt{2 \times 16}}{2} = \dfrac{\sqrt{2}\times \sqrt{16}}{2} = \sqrt{16}\times \dfrac{\sqrt{2}}{2} = 4 \times \dfrac{\sqrt{2}}{2} = 2\sqrt{2}$.
\end{enumerate}

\vspace{0,25cm}

