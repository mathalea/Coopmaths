
\medskip

%Chacune des affirmations suivantes est-elle vraie ou fausse ? Justifier votre réponse.
%
%\bigskip

\textbf{Affirmation 1 :}

%Un billet d'avion Paris - New York coûte 400~\euro. La compagnie aérienne Air
%International propose une réduction de 20\,\%. Le billet ne coûte plus que 380~\euro.
Sur 100~\euro{}, la réduction serait de 20~\euro, donc sur 400~\euro{} la réduction est de $4 \times 20 = 80$~\euro. Le billet ne coûte plus que $400 - 80 = 320$~\euro. L'affirmation est fausse.
\bigskip

\textbf{Affirmation 2 :}

%$f$ est la fonction affine définie par $f(x) = 4x - 2$.
%
%L'image de 2 par la fonction $f$ est aussi le double de l'antécédent de $10$.
%76 cm
On a $f(2) = 4 \times 2 - 2 = 8 - 2 = 6$.

La moitié de 6 est 3 et :

$f(3) = 4 \times 3 - 2 = 12 - 2 = 10$.

L'affirmation 2 est vraie.
\bigskip

%\parbox{0.48\linewidth}{\textbf{Affirmation 3 :}

%Les plateaux représentés par (AB) et (CD)
%pour la réalisation de cette desserte en bois
%sont parallèles.} \hfill 
%\parbox{0.52\linewidth}{\psset{unit=0.8cm}
%\begin{pspicture}(-1,0)(5.5,5.5)
%%\psgrid
%\psframe[fillstyle=solid,fillcolor=lightgray](0.3,1.1)(4.6,1.4)
%\psframe[fillstyle=solid,fillcolor=lightgray](0.8,4.3)(4.5,4.6)
%\pspolygon[fillstyle=solid,fillcolor=lightgray](4.7,0)(5.1,0)(1.2,4.3)(0.8,4.3)
%\pspolygon[fillstyle=solid,fillcolor=lightgray](0.65,0.9)(0.9,0.75)(4.3,4.45)(3.9,4.35)
%\pscurve[fillstyle=solid,fillcolor=lightgray](4.3,4.45)(4.8,4.9)(5.4,5.1)(5.3,5.2)(4.6,5.)(4.2,4.8)(4,4.6)(4.3,4.45)
%\pscircle[fillstyle=solid,fillcolor=white](0.5,0.5){0.5}
%\pscircle[fillstyle=solid,fillcolor=lightgray](0.5,0.5){0.2}
%\psdots[dotstyle=+,dotangle=45](1.18,1.3)(3.75,1.3)(1,4.45)(4.2,4.45)(2.5,2.65)%CDAB
%\uput[u](1,4.45){A} \uput[u](4.2,4.45){B} \uput[ul](1.18,1.3){C} \uput[ur](3.75,1.3){D}
%\psline{<->}(1.4,1.15)(2.65,2.5)
%\psline{<->}(2.65,2.5)(4.35,4.35)
%\uput[u](2.5,2.65){O} \rput(2.3,1.6){\scriptsize 50 cm}
%\rput(3.7,3.15){\scriptsize 45 cm}
%\psline{<->}(1,5)(4.2,5) \rput(2.8,5.15){\scriptsize 75 cm}
%\psline{<->}(1.18,0.8)(3.75,0.8)\rput(2.42,0.6){\scriptsize 100 cm}
%\end{pspicture}}
\textbf{Affirmation 3 :}

Si (AB) et (CD) sont parallèles, le théorème de Thalès permet d'écrire :

$\dfrac{\text{OB}}{\text{OC}} = \dfrac{\text{AB}}{\text{CD}}$, soit 

$\dfrac{45}{50} = \dfrac{75}{100}$ ou encore $\dfrac{9}{10} = \dfrac{3}{4}$, c'est-à-dire 0,9 = 0,75 qui est une égalité fausse.

L'affirmation 3 est faussee.
\vspace{0,5cm}

