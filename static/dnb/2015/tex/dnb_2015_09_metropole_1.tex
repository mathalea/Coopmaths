
\medskip

On appelle $f$ la fonction définie par $f(x) = (x - 1)(2x - 5)$. 

On a utilisé un tableur pour calculer les images de différentes valeurs par cette fonction $f$ :

\medskip
\begin{tabularx}{\linewidth}{|c|*{10}{>{\centering \arraybackslash}X|}}\hline 
\multicolumn{4}{|c|}{A2}&\multicolumn{7}{|l|}{$f(x)$}\\ \hline
&A&B&C&D&E&F&G&H&I&J\\ \hline
1&$x$&0&1&2&3&4&5&6&7&8\\ \hline
2&$f(x)$&5&0&$-1$&2&9&20&35&54&77\\ \hline
3&&&&&&&&&&\\ \hline
\end{tabularx}
\medskip

\begin{enumerate}
\item Pour chacune des affirmations suivantes, indiquer si elle est vraie ou fausse. On rappelle que les réponses doivent être justifiées. 

Affirmation 1 : $f(2) = 3$. 

Affirmation 2 : L'image de $11$ par la fonction $f$ est 170. 

Affirmation 3 : La fonction $f$ est linéaire. 
\item  Une formule a été saisie dans la cellule B2 puis recopiée ensuite vers la droite. Quelle formule a-t-on saisie dans cette cellule B2 ? 
\item  Quels sont les deux nombres $x$ pour lesquels $(x - 1)(2x - 5) = 0$ ? 
\end{enumerate}

\vspace{0.5cm}

