
\medskip

\begin{enumerate}
\item %Voici un programme de calcul:

%\begin{center}
%\begin{tabular}{|l|}\hline
%\multicolumn{1}{|c|}{\textbf{Programme A}}\\
%$\bullet~~$Choisir un nombre.\\
%$\bullet~~$Ajouter 3.\\
%$\bullet~~$Calculer le carré du résultat obtenu.\\
%$\bullet~~$Soustraire le carré du nombre de départ.\\ \hline
%\end{tabular}
%\end{center}

	\begin{enumerate}
		\item %Eugénie choisit 4 comme nombre de départ. Vérifier qu'elle obtient 33 comme résultat du programme.
		On obtient successivement :
		
		$4~;~4 + 3 = 7~;~7^2 = 49~;~49 - 4^2 = 49 - 16 = 33$.
		\item %Elle choisit ensuite $-5$ comme nombre de départ. Quel résultat obtient-elle ?
		$- 5~;~- 5 + 3 = - 2~;~(- 2)^2 = 4~;~4 - (- 5)^2 = 4 - 25 = - 21$.
	\end{enumerate}
\item %Voici un deuxième programme de calcul:

%\begin{center}
%\begin{tabular}{|l|}\hline
%\multicolumn{1}{|c|}{\textbf{Programme B}}\\
%$\bullet~~$Choisir un nombre.\\
%$\bullet~~$Multiplier par 6.\\
%$\bullet~~$Ajouter 9 au résultat obtenu.\\ \hline
%\end{tabular}
%\end{center}
%
%Clément affirme: \og Si on choisit n'importe quel nombre et qu'on lui applique les deux programmes, on
%obtient le même résultat. \fg
%Prouver que Clément a raison.
Premier programme : si $x$ est le nombre choisi au départ, on obtient successivement :

$x~;~x + 3~;~(x + 3)^2~;~(x + 3)^2 - x^2 $

Deuxième programme : si $x$ est le nombre choisi au départ, on obtient successivement :

$x~;~6x~;~6x + 9$.

Or $(x + 3)^2 - x^2 = (x + 3 + x)(x + 3 - x) = 3(2x + 3) = 6x + 9$.

Les deux programmes donnent le même résultat.

item %Quel nombre de départ faut-il choisir pour que le résultat des programmes soit $54$ ?
Il faut trouver un nombre $x$ tel que $6x + 9 = 54$ soit $6x = 45$ ou $2x = 15$ et $x = 7,5$.
\end{enumerate}
 
\vspace{0,5cm}

