
\medskip

Des ingénieurs de l'Office National des Forêts font le marquage d'un lot de pins
destinés à la vente.

\medskip

\begin{enumerate}
\item Dans un premier temps, ils estiment la hauteur des arbres de ce lot, en plaçant
leur oeil au point O.

\parbox{0.55\linewidth}{\psset{unit=1cm}
\begin{pspicture}(7.3,5)
%\psgrid
\psline{<->}(7.2,0.3)(7.2,4.5)\uput[l](7.2,2.4){$h$}
\psline(0,0.3)(7.3,0.3)
\psline[linestyle=dotted](5.9,1.1)(0.6,1.1)(5.9,4.5)%AOS
\psline[linestyle=dotted](0.6,1.1)(5.9,0.3)(5.9,4.5)%OPS
\psarc(0.6,1.1){0.8}{0}{35}
%(5.6,1.1)(5.6,1.4)(5.9,1.4)
\uput[ul](5.8,1.2){A}\uput[u](5.9,4.5){S}
\uput[d](5.9,0.3){P}\uput[d](0.6,1.1){O (œil)}
\psline(5.7,1.1)(5.7,1.3)(5.9,1.3)
\pscurve(5.8,0.3)(5.85,1.2)(5.8,3)(5.6,3.3)(5,3.4)(4.7,4)(4.8,4.2)(5,4.3)(5.9,4.5)(6.3,4.4)(6.7,4.2)(6.9,4)(6.5,3.6)(6.2,3.3)(6.12,2.8)(6.11,1.5)(6.14,0.3)
\psarc(0.6,1.1){0.5}{-11}{0}
\rput(1.65,1.3){\footnotesize 45\degres}\rput(2.2,1){\footnotesize 25\degres}
%\psline
\end{pspicture}}\hfill
\parbox{0.32\linewidth}{Ils ont relevé les données suivantes :

OA = 15 m

$\widehat{\text{SOA}} = 45\degres$ et $\widehat{\text{AOP}} = 25\degres$}

\medskip

Calculer la hauteur $h$ de l'arbre arrondie au mètre.
\item  Dans un second temps, ils effectuent une mesure de diamètre sur chaque arbre et
répertorient toutes les données dans la feuille de calculs suivante :

\begin{center}
\begin{tabularx}{\linewidth}{|c|m{2.25cm}|*{12}{>{\centering \arraybackslash}X|}}\hline
	&\centering A &B &C &D &E &F &G &H &I &J &K &L &M\\ \hline
1	& Diamètre (cm)& 30 &35 &40 &45 &50 &55 &60 &65 &70 &75 &80&\\ \hline
2 	&Effectif &2 &4 &8 &9 &10 &12 &14 &15 &11 &4 &3	&\\ \hline
\end{tabularx}
\end{center}

	\begin{enumerate}
		\item Quelle formule doit-on saisir dans la cellule M2 pour obtenir le nombre total
d'arbres ?
		\item Calculer, en centimètres, le diamètre moyen de ce lot. On arrondira le résultat à
l'unité.
	\end{enumerate}
\item Pour calculer le volume commercial d'un pin en mètres cubes, on utilise la formule
suivante :

\[V = \dfrac{10}{24} \times D^2 \times h\]

où $D$ est le diamètre moyen d'un pin en mètres et $h$ la hauteur en mètres.

Le lot est composé de 92 arbres de même hauteur 22 m dont le diamètre moyen
est 57 cm.

Sachant~ qu'un mètre cube de pin rapporte 70~\euro, combien la vente de ce lot
rapporte+elle ? On arrondira à l'euro.
\end{enumerate}

\vspace{0,5cm}

