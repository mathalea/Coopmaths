
\medskip

\begin{enumerate}
\item %Le bar et le P.S.I. (Pound per Square Inch ou livre par pouce carré) sont deux unités utilisées pour mesurer la pression.

%Le graphique ci-dessous donne la correspondance entre ces 2 unités.
%
%\begin{center}
%\psset{xunit=0.11cm,comma=true}
%\begin{pspicture}(-5,-0.5)(90,6)
%\multido{\n=0.5+2.5}{35}{\psline[linestyle=dotted,linewidth=0.2pt](\n,0)(\n,6)}
%\multido{\n=0.00+0.25}{24}{\psline[linestyle=dotted,linewidth=0.2pt](0,\n)(90,\n)}
%\psaxes[linewidth=1.25pt,Dx=5,Dy=0.5]{->}(0,0)(-4,-0.4)(90,6)
%\psaxes[linewidth=1.25pt,Dx=5,Dy=0.5](0,0)(90,6)
%\psline[linewidth=1.5pt,linecolor=blue](80,5.5)
%\uput[u](80,0){Pression en P. S. I.}
%\uput[r](0,5.8){Pression en bar}
%\end{pspicture}
%\end{center}
%
%Avant de prendre la route, Léa vérifie la pression des pneus de sa voiture. La pression conseillée sur le manuel du véhicule est de 36 P.S.I.
%
%Déterminer à l'aide du graphique la pression conseillée en bar. Aucune justification n'est attendue.
La verticale contenant le point d’abscisse 36 coupe la droite en un point d’ordonnée à peu près égale à $2,5$~(bar).
\item %Léa se rend à Brest en prenant la route N12 qui passe par Morlaix. Alors qu'elle se trouve à 123 km de Brest, elle voit le panneau-ci-dessous 

%\begin{center}
%\psset{unit=1cm}
%\begin{pspicture}(5,3.25)
%\psframe[linewidth=2pt](5,1.85)
%\uput[r](0,1.5){\textbf{BREST}}
%\uput[r](3.5,1.5){123}
%\uput[r](0,0.5){\textbf{MORLAIX}}\uput[r](3.5,0.5){\textbf{64}}
%\psframe[fillstyle=solid,fillcolor=red](2,2)(3,2.75)
%\rput(2.5,2.375){\white N 12}
%\end{pspicture}
%\end{center}
%
%Dans combien de kilomètres la distance qui la sépare de Morlaix sera la même que celle de Morlaix à Brest?
D’après le panneau la distance de Morlaix à Brest est égale à :

$123 - 64 = 59$, donc Léa sera à 59 km de Morlaix dans $64 - 59 = 5$~(km).
\end{enumerate}
 
\vspace{0,5cm}

