
\medskip


Une coopérative collecte le lait dans différentes exploitations agricoles.

Le détail, de la collecte du jour ont été saisis dans une feuille de calcul d'un tableur.

\begin{center}
\begin{tabularx}{0.7\linewidth}{|c|*{2}{>{\centering \arraybackslash}X|}}\hline
&A&B \\ \hline
1&Exploitation agricole& Quantité de lait collecté (en L)\\ \hline
2& Beausejour& \np{1250}\\ \hline
3&Le Verger& \np{2130}\\ \hline 
4&La  Fourragère& \np{1070}\\ \hline
5& Petit pas& \np{2260}\\ \hline
6&La  Chausse Pierre& \np{1600}\\ \hline
7& Le Palet& \np{1740}\\ \hline
8&Quantité totale de lait collecté&\\ \hline
\end{tabularx}
\end{center}

\begin{enumerate}
\item Une formule doit être saisie dans la cellule B8 pour obtenir la quantité totale de lait collecté. Parmi les quatre propositions ci-dessous, recopier celle qui convient.

\begin{center}
\begin{tabularx}{\linewidth}{|*{4}{>{\centering \arraybackslash}X|}}\hline
SOMME(B2~:~B7)&SOMME(B2~:~B8)&=SOMME(B2~:~B7)&=SOMME(B2~:~B8)\\ \hline
\end{tabularx}
\end{center}
\medskip

\item  Calculer la moyenne des quantités de lait collecté dans ces exploitations.
\item  Quel pourcentage de la collecte provient de l'exploitation \og Petit Pas \fg{} ? On arrondira le résultat à l'unité.
\end{enumerate}
 
\vspace{0,5cm}

