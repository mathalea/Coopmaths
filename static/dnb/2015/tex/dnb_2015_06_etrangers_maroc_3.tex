
\medskip

\parbox{0.6\linewidth}{À l'entrée du garage à vélos du collège, un digicode
commande l'ouverture de la porte.

Le code d'ouverture est composé d'une lettre A ; B ou C
suivie d'un chiffre 1 ; 2 ou 3.} \hfill
\parbox{0.4\linewidth}{\psset{unit=1cm}
\begin{pspicture}(-0.7,0)(4.1,3.5)
%\psgrid
\psframe(4.1,3.5)
\psframe[fillstyle=solid,fillcolor=lightgray](0.4,2.3)(1.3,3.1)\rput(0.85,2.7){A}
\psframe[fillstyle=solid,fillcolor=lightgray](1.5,2.3)(2.4,3.1)\rput(1.95,2.7){B}
\psframe[fillstyle=solid,fillcolor=lightgray](2.6,2.3)(3.5,3.1)\rput(3.05,2.7){C}
\psframe[fillstyle=solid,fillcolor=lightgray](0.4,1.3)(1.3,2.1)\rput(0.85,1.7){1}
\psframe[fillstyle=solid,fillcolor=lightgray](1.5,1.3)(2.4,2.1)\rput(1.95,1.7){2}
\psframe[fillstyle=solid,fillcolor=lightgray](2.6,1.3)(3.5,2.1)\rput(3.05,1.7){3}
\rput(2.05,0.8){ACCESS CONTROL}
\pscircle(1.5,0.4){0.2} \pscircle[fillstyle=solid,fillcolor=lightgray](2.55,0.4){0.2} 
\end{pspicture}}

\medskip

\begin{enumerate}
\item Quelles sont les différents codes possibles ?
\item Aurélie compose au hasard le code A1.
	\begin{enumerate}
		\item Quelle probabilité a-t-elle d'obtenir le bon code ?
		\item En tapant ce code A1, Aurélie s'est trompée à la fois de lettre et de chiffre. Elle change donc ses choix.
		
Quelle probabilité a-t-elle de trouver le bon code à son deuxième essai ?
		\item Justifier que si lors de ce deuxième essai, Aurélie ne se trompe que de lettre,
elle est sûre de pouvoir ouvrir la porte lors d'un troisième essai.
	\end{enumerate}
\end{enumerate}

\vspace{0,5cm}

