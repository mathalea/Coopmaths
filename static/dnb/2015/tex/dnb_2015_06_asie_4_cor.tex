
\medskip

%À la fin d'une fête de village, tous les enfants présents se partagent équitablement les 397
%ballons de baudruche qui ont servi à la décoration. Il reste alors 37 ballons. 
%
%\smallskip
%
%L'année suivante, les mêmes enfants se partagent les 598 ballons utilisés cette année-là.
%Il en reste alors 13.
%
%\smallskip
%
%Combien d'enfants, au maximum, étaient présents ?

%\emph{Toute trace de recherche, même incomplète, sera prise en compte dans le notation.}
S’il reste 37 ballons la première année, les enfants se sont partagés équitablement 360
ballons car $397 - 37 = 360$.

S’il reste 13 ballons l'année suivante, les enfants se sont partagés équitablement 585
ballons car $598 - 13 = 585$.

Pour connaître le nombre maximum d’enfants présents à la fête, je recherche le PGCD,
plus grand diviseur commun à $360$ et $585$. J’utilise l’algorithme d’Euclide.

$585 = 360 \times 1 + 225$

$360 = 225 \times 1 + 135$

$225 =135 \times 1 + 90$

$135 = 90 \times 1 + 45$

$90 = 45 \times 2 + 0$

Le dernier reste non nul est 45, donc PGCD$(585~;~360) = 45$.

Le nombre maximum d’enfants présents était de $45$.
\vspace{0,5cm}

