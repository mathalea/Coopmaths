
\medskip

Voici un programme de calcul.

\begin{center}
\begin{tabularx}{0.5\linewidth}{|X|}\hline
$\bullet~~$Choisir un nombre\\
$\bullet~~$Ajouter 1\\
$\bullet~~$Calculer le carré de cette somme\\
$\bullet~~$Soustraire 9 au résultat\\ \hline
\end{tabularx}
\end{center}

\begin{enumerate}
\item Vérifier qu'en choisissant 7 comme nombre de départ, le résultat obtenu avec ce programme est 55.
\item Lorsque le nombre choisi est $- 6$, quel résultat obtient-on ?
\item Jim utilise un tableur pour essayer le programme de calcul avec plusieurs nombres. Il a fait apparaître les résultats obtenus à chaque étape. Il obtient la feuille de calcul ci-dessous :

\begin{center}
\begin{tabularx}{\linewidth}{|c|*{4}{>{\centering \arraybackslash}X|}}\hline
	&A		&B		&C		&D\\ \hline
1	&nombre de départ&résultat de la 1\up{e} étape&résultat de la 2\up{e} étape&résultat final\\ \hline
2	&$-0,4$	&0,6	&0,36	&$- 8,64$\\ \hline 
3	&$-0,2$	&0,8	&0,64	&$- 8,36$\\ \hline
4	&0 		&1 		&1		&$- 8$\\ \hline
5	&0,2 	&1,2	&1,44	&$- 7,56$\\ \hline
6	&0,4 	&1,4	&1,96	&$- 7,04$\\ \hline
7	&0,6 	&1,6	&2,56	&$- 6,44$\\ \hline
8	&0,8 	&1,8 	&3,24 	&$- 5,76$\\ \hline
9	&1 		&2 		&4 		&$- 5$\\ \hline
10	&1,2	&2,2	&4,84	&$- 4,16$\\ \hline
11	&1,4	&2,4	&5,76	&$- 3,24$\\ \hline
12	&1,6	&2,6	&6,76	&$- 2,24$\\ \hline
13	&1,8	&2,8	&7,84	&$- 1,16$\\ \hline 
14	&2 		&3 		&9 		&0\\ \hline
15	&2,2	&3,2	&10,24	&1,24\\ \hline
16	&2,4 	&3,4 	&11,56 	&2,56\\ \hline
\end{tabularx}
\end{center}

La colonne B est obtenue à partir d'une formule écrite en B2, puis recopiée vers le bas.

Quelle formule Jim a-t-il saisie dans la cellule B2 ?
\item Le programme donne 0 pour deux nombres. Déterminer ces deux nombres.
\end{enumerate}

\vspace{0,5cm}

