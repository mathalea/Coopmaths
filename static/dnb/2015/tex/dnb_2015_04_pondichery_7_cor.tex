
\medskip

%Trois triangles équilatéraux identiques sont découpés dans les coins d'un triangle
%équilatéral de côté 6~cm. La somme des périmètres des trois petits triangles est égale
%au périmètre de l'hexagone gris restant. Quelle est la mesure du côté des petits
%triangles ?
%
\begin{center}
\psset{unit=1cm}
\begin{pspicture}(5,4.6)
\pspolygon(0,0)(5,0)(2.5,4.33)
\pspolygon[fillstyle=solid,fillcolor=lightgray](1.3,0)(3.7,0)(4.35,1.126)(3.15,3.204)(1.85,3.204)(0.65,1.126)
\rput{60}(0.3,0.8){$c$}\rput{60}(2.05,4.05){$c$}\rput{60}(1.3,2.6){$6 - 2c$}
\end{pspicture}
\end{center}
%
%\bigskip
%
%\textbf{Toute trace de recherche, même non aboutie, figurera sur la copie et sera prise
%en compte dans la notation.}
Soit $c$ la mesure d'un côté de l'un des petites triangles équilatéraux.

Dans l'hexagone gris il y a trois côtés de longueur $c$ et trois côtés de longueur $6 - 2c$.

On a donc :

$3 \times 3c = 3c + 3(6 - 2c)$ soit 

$9c = 3c + 18 - 6c$ soit

$12c = 18$ soit en simplifiant par 6 :

$2c = 3$ et enfin 

$c = \frac{3}{2} = 1,5$~cm.
