
\medskip

%On appelle $f$ la fonction définie par $f(x) = (x - 1)(2x - 5)$. 
%
%On a utilisé un tableur pour calculer les images de différentes valeurs par cette fonction $f$ :
%
%\medskip
%\begin{tabularx}{\linewidth}{|c|*{10}{>{\centering \arraybackslash}X|}}\hline 
%\multicolumn{4}{|c|}{A2}&\multicolumn{7}{|l|}{$f(x)$}\\ \hline
%&A&B&C&D&E&F&G&H&I&J\\ \hline
%1&$x$&0&1&2&3&4&5&6&7&8\\ \hline
%2&$f(x)$&5&0&$-1$&2&9&20&35&54&77\\ \hline
%3&&&&&&&&&&\\ \hline
%\end{tabularx}
%\medskip

\begin{enumerate}
\item %Pour chacune des affirmations suivantes, indiquer si elle est vraie ou fausse. On rappelle que les réponses doivent être justifiées. 

Affirmation 1 : On a $f(2) = (2 - 1)(4 - 5) = 1 \times (- 1) = - 1$. Affirmation fausse. 

Affirmation 2 : %L'image de $11$ par la fonction $f$ est 170. 
On a $f(11) = (11 - 1)(22 - 5) = 10 \times 17 = 170$. Affirmation vraie.

Affirmation 3 : %La fonction $f$ est linéaire.
On a $f(x) = 2x^2 - 5x - 2x + 5 = 2x^2 - 7x + 5$ : ce n'est pas une fonction linéaire. Affirmation fausse. 
\item  %Une formule a été saisie dans la cellule B2 puis recopiée ensuite vers la droite. Quelle formule a-t-on saisie dans cette cellule B2 ?
\fbox{=(B1 - 1)*( 2*B1 - 5)}
\item  %Quels sont les deux nombres $x$ pour lesquels $(x - 1)(2x - 5) = 0$ ?
$(x - 1)(2x - 5) = 0$. Un produit de facteurs est nul si l'un des facteurs est nul soit si 

$x - 1 = 0$ soit $x = 1$ ou

$2x - 5 = 0$ ou $2x  = 5$ et  $x = \dfrac{5}{2}$.

Les deux nombres qui annulent $f(x)$ sont 1 et $\dfrac{5}{2}$. 
\end{enumerate}

\vspace{0.5cm}

