
\medskip



On propose les deux programmes de calcul suivants :

\begin{center}
\psset{unit=0.71cm}
\begin{pspicture}(16.8,10.7)
\psframe[fillstyle=solid,fillcolor=lightgray](16.8,10)
%\psgrid
\rput(3.3,9.5){\textbf{Programme A}} \rput(12.3,9.5){\textbf{Programme B}}
\psframe(16.8,10.7)\psline(0,10)(16.8,10)
\psframe[fillstyle=solid,fillcolor=white](0.4,7.6)(4.5,8.5)\psline{->}(4.5,8.05)(5.2,8.05)
\psframe[fillstyle=solid,fillcolor=white](5.2,7.3)(6.8,8.8)\psline{->}(6,7.3)(6,6.3)
\psframe[fillstyle=solid,fillcolor=white](5.2,4.8)(6.8,6.3)\psline{->}(6,4.8)(6,4)
\psframe[fillstyle=solid,fillcolor=white](5.2,2.5)(6.8,4)\psline{->}(4.4,3.4)(5.2,3.4)
\psframe[fillstyle=solid,fillcolor=white](2.3,2.9)(4.4,3.8)
\psframe[fillstyle=solid,fillcolor=white](9.7,7.6)(13.8,8.5)\psline{->}(13.8,8.05)(14.6,8.05)
\psframe[fillstyle=solid,fillcolor=white](14.6,7.3)(16.2,8.8)\psline{->}(15.4,7.3)(15.4,6.3)
\psframe[fillstyle=solid,fillcolor=white](14.6,6.3)(16.2,4.8)\psline{->}(15.4,4.8)(15.4,4)
\psframe[fillstyle=solid,fillcolor=white](14.6,2.5)(16.2,4)\psline{->}(15.4,2.5)(15.4,1.8)
\psframe[fillstyle=solid,fillcolor=white](14.6,0.3)(16.2,1.8)
\psframe[fillstyle=solid,fillcolor=white](11.1,0.7)(13.7,1.4)
\rput(2.5,8){Nombre de départ}\rput(11.65,8){Nombre de départ}
\rput(4.5,6.7){Ajouter 2}\rput(4.4,4.4){Élever au carré}
\rput(3.4,3.4){Résultat}\rput(12.4,1.1){Résultat}
\rput(14.2,2.2){Ajouter 4}\rput(12.5,4.5){Multiplier par le nombre de}
\rput(12.5,3.9){départ}\rput(14.2,6.8){Ajouter 4}
\end{pspicture}
\end{center}

\medskip

\begin{enumerate}
\item Montrer que si on choisit 3 comme nombre de départ, les deux programmes
donnent 25 comme résultat.
\item Avec le programme A, quel nombre faut-il choisir au départ pour que le résultat
obtenu soit 0 ?
\item Ysah prétend que, pour n'importe quel nombre de départ, ces deux programmes
donnent le même résultat.

A-t-elle raison ? Justifier votre réponse.
\end{enumerate}

\vspace{0,5cm}

