
\medskip

\begin{enumerate}
  \item Le triangle $AKD$ étant rectangle en $K$, on peut appliquer le théorème de Pythagore et on a :
  
$DA^2 = DK^2 + KA^2$.
  
D'où $KA^2 = DA^2 - DK^2$.
  
Donc $KA = \sqrt{DA^2-DK^2} = \sqrt{60^2-11^2} = \sqrt{3479} \approx 59,0$ cm.
  \item Les droites $(DK)$ et $(PH)$ étant toutes les deux perpendiculaires à la droite $(KA)$, elles sont parallèles.
  
  On peut donc appliquer le théorème de Thalès et on a : $\dfrac{AP}{AD}=\dfrac{AH}{AK}=\dfrac{HP}{KD}$.
  
Or $AP = AD - DP = 60 - 45 = 15$ cm.
  
D'où $\dfrac{15}{60}=\dfrac{HP}{11}$.
  
Et donc $HP = \dfrac{15 \times 11}{60} = 2,75$ cm.
\end{enumerate}

\medskip

