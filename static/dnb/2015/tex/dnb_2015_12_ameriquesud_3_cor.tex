
\medskip

%Un « DJ »\footnote{DJ signifie « disk jokey » c'est à dire animateur musical} possède 96 titres de musique rap et 104 titres de musique électro. Lors
%de ses concerts, il choisit les titres qu'il mixe au hasard.
%
%\medskip

\begin{enumerate}
\item %Calculer la probabilité que le premier titre soit un titre de musique rap.
Il y a en tout $96 + 104 = 200$ titres. La probabilité que le premier titre soit un titre de musique rap est donc égale à $\dfrac{96}{200} = \dfrac{48}{100} = 48\,\%= 0,48$.
\item %Pour varier ses concerts, le DJ souhaite répartir tous ses titres en réalisant des « mix »\footnote{mix est une abréviation de mixage} identiques, c'est-à-dire comportant le même nombre de titres et la même répartition de titres de musique « rap » et de musique « électro ».
	\begin{enumerate}
		\item %Quel est le nombre maximum de concerts différents pourra-t-il réaliser ?
Il faut répartir tous les titres donc il faut trouver un nombre qui divise 104 et 96 le plus grand : c'est donc le PGCD de 104 et 96.

On utilise l'algorithme d'Euclide :

$104 = 96 \times 1 + 8$ ;

$96 = 8 \times 12 + 0$.

On a donc PGCD(104~;~96) = 8, soit 8 concerts différents.
		\item %Combien y aura-t-il dans ce cas de titres de musique rap et de musique électro par concert ?
On a $104 = 8 \times 13$ et $96 = 8 \times 12$.

Il y aura dans chaque concert 13 titres d'électro et 12 titres de rap.
	\end{enumerate}
\end{enumerate}
%%%%%%%%%%%%%%
\vspace{0.25cm}

