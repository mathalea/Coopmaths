
\medskip

\parbox{0.35\linewidth}{\psset{unit=1cm}
\begin{pspicture}(4,3)
%\psgrid
\psline[linewidth=7pt,linecolor=red]{cc-cc}(0.5,0.5)(3.5,0.5)
\psarc(3.5,0.5){0.18}{-90}{30}\psline(0.5,0.3)(3.5,0.3)
\psarc(2,2.6){0.2}{25}{155}\psline(3.65,0.6)(2.2,2.65)
\psarc(0.5,0.5){0.18}{120}{270}\psline(1.82,2.7)(0.4,0.65)
\psline[linewidth=7pt,linecolor=red]{cc-cc}(0.5,0.5)(2,2.6)
\psline[linewidth=7pt,linecolor=red]{cc-cc}(3.5,0.5)(2,2.6)
\pspolygon*(1,0.8)(3,0.8)(1.4,1.3)\rput{-15}(2,1.4){10\,\%}
\end{pspicture}}\hfill \parbox{0.6\linewidth}{Ce panneau routier indique une descente dont la pente est de 10\,\%.}

\medskip

Cela signifie que pour un déplacement horizontal de 100 mètres, le dénivelé est de 10 mètres.

Le schéma ci-dessous n'est pas à l'échelle.

\begin{center}
\psset{unit=0.5cm}
\begin{pspicture}(17,6)
\pspolygon(2.5,1.5)(15.8,1.5)(2.5,5.2)%BCA
\psframe(2.5,1.5)(2.8,1.8)
\uput[ul](2.5,5.2){A} \uput[dl](2.5,1.5){B} \uput[r](15.8,1.5){C}
\rput{-15}(9.15,3.8){Route} \uput[l](2.5,3.45){Dénivelé :}
\rput(9.15,1){Déplacement horizontal : 100~m}
\uput[l](1.9,2.7){10 m}
\end{pspicture}
\end{center}
\medskip

\begin{enumerate}
\item Déterminer la mesure de l'angle $\widehat{\text{BCA}}$ que fait la route avec l'horizontale.

Arrondir la réponse au degré.
\item Dans certains pays, il arrive parfois que 1a pente d'une route ne soit pas donnée par un pourcentage, mais par une indication telle que \og 1 : 5 \fg, ce qui veut alors dire que pour un déplacement horizontal de 5~mètres, le dénivelé est de 1 mètre.

Lequel des deux panneaux ci-dessous indique la pente la plus forte ?

\begin{center}
\begin{pspicture}(4,3)
%\psgrid
\psline[linewidth=7pt,linecolor=red]{cc-cc}(0.5,0.5)(3.5,0.5)
\psarc(3.5,0.5){0.18}{-90}{30}\psline(0.5,0.3)(3.5,0.3)
\psarc(2,2.6){0.2}{25}{155}\psline(3.65,0.6)(2.2,2.65)
\psarc(0.5,0.5){0.18}{120}{270}\psline(1.82,2.7)(0.4,0.65)
\psline[linewidth=7pt,linecolor=red]{cc-cc}(0.5,0.5)(2,2.6)
\psline[linewidth=7pt,linecolor=red]{cc-cc}(3.5,0.5)(2,2.6)
\pspolygon*(1,0.8)(3,0.8)(1.4,1.3)\rput{-15}(2,1.4){15\,\%}
\end{pspicture}\qquad \quad \begin{pspicture}(4,3)
%\psgrid
\psline[linewidth=7pt,linecolor=red]{cc-cc}(0.5,0.5)(3.5,0.5)
\psarc(3.5,0.5){0.18}{-90}{30}\psline(0.5,0.3)(3.5,0.3)
\psarc(2,2.6){0.2}{25}{155}\psline(3.65,0.6)(2.2,2.65)
\psarc(0.5,0.5){0.18}{120}{270}\psline(1.82,2.7)(0.4,0.65)
\psline[linewidth=7pt,linecolor=red]{cc-cc}(0.5,0.5)(2,2.6)
\psline[linewidth=7pt,linecolor=red]{cc-cc}(3.5,0.5)(2,2.6)
\pspolygon*(1,0.8)(3,0.8)(1.4,1.3)\rput{-15}(2,1.4){1 : 5}
\end{pspicture}
\end{center}

\hspace{2.25cm}Panneau A \hspace{3.15cm}   Panneau B
\end{enumerate}


