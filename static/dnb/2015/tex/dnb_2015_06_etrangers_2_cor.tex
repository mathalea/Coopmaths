
\medskip

%Le 14 octobre 2012, Félix Baumgartner, a effectué un saut d'une altitude de \np{38969,3}~mètres.
%
%La première partie de son saut s'est faite en chute libre (parachute fermé).
%
%La seconde partie, s'est faite avec un parachute ouvert.
%
%Son objectif était d'être le premier homme à \textbf{\og dépasser le mur du son \fg}.
%
%\begin{center}\textbf{\og dépasser le mur du son \fg{}} : signifie atteindre une vitesse supérieure ou égale à la vitesse du son, c'est à dire $340$ m.s$^{-1}$.\end{center}
%
%La Fédération Aéronautique Internationale a établi qu'il avait atteint la vitesse maximale de
%\np{1357,6} km.h$^{-1}$ au cours de sa chute libre.
%
%\medskip

\begin{enumerate}
\item ~%A-t-il atteint son objectif ? Justifier votre réponse.
\begin{center}
\begin{tabularx}{\linewidth}{|l|*{3}{>{\centering \arraybackslash}X|}}\hline
distance parcourue par le son (km)&0,340&20,4&\np{1224}\\ \hline
temps (s)&1&60&\np{3600}\\ \hline
\end{tabularx}
\end{center}

Le son a donc une vitesse de \np{1224} km/h inférieure à celle de Félix Baumgartner ; celui-ci a atteint son objectif.
\item %Voici un tableau donnant quelques informations chiffrées sur ce saut :

%\begin{center}
%\begin{tabularx}{0.7\linewidth}{|l|X|}\hline
%Altitude du saut 					&\np{38969,3} m\\ \hline
%Distance parcourue en chute libre	&\np{36529} m\\ \hline
%Durée totale du saut				&9 min 3 s\\ \hline
%Durée de la chute libre			&4 min 19 s\\ \hline
%\end{tabularx}
%\end{center}
%
%Calculer la vitesse moyenne de Félix Baumgartner en chute avec parachute ouvert
%exprimée en m.s$^{-1}$. On arrondira à l'unité.
Avec le parachute, Félix a parcouru : $\np{38969,3} - \np{36529} = \np{2440,3}$~m en 

9 min 3 s $-$ 4 min 19 = 4 min 44 s, ou 284 s soit une vitesse moyenne de $\dfrac{\np{2440,3}}{284} \approx 8,59$ en mètres par seconde, soit 9~m/s à l'unité près.
\end{enumerate}

\vspace{0,5cm}

