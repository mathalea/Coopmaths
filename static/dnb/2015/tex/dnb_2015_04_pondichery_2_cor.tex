
\medskip

%Un chocolatier vient de fabriquer \np{2622}~oeufs de Pâques et \np{2530}~poissons en chocolat.
%
%Il souhaite vendre des assortiments d'oeufs et de poissons de façon que :
%
%\setlength\parindent{6mm}
%\begin{itemize}
%\item[$\bullet~~$] tous les paquets aient la même composition ;
%\item[$\bullet~~$] après mise en paquet, il reste ni oeufs, ni poissons.
%\end{itemize}
%\setlength\parindent{0mm}
%
%\medskip

\begin{enumerate}
\item %Le chocolatier peut-il faire 19 paquets ? Justifier.
On a $\dfrac{\np{2622}}{19} = 138$, mais $\dfrac{\np{2530}}{19} \approx  133,2$.

Ce qui veut dire que l'on ne pas répartir les \np{2530} poissons dans 19 paquets (il en reste 3)
\item %Quel est le plus grand nombre de paquets qu'il peut réaliser ? Dans ce cas, quelle sera la composition de chaque paquet ?
Le plus grand nombre de paquets qu'il peut réaliser est un diviseur commun à \np{2622} et à \np{2530}. Puisque c'est le plus grand c'est donc leur PGCD que l'on calcule grâce à l'algorithme d'Euclide :

$\np{2622}  \np{2530} \times 1 + 92$ ;

$\np{2530} =  92 \times 27  + 46$ ;

$92 = 46 \times 2 + 0$.

Le PGCD est le dernier reste non nul, donc 46.

Effectivement : $\dfrac{\np{2622}}{46} = 57$ et   $\dfrac{\np{2530}}{46} = 55$

Dans chacun des 46 paquets il y aura 57 œufs et 55 poissons.
\end{enumerate}

\vspace{0.5cm}

