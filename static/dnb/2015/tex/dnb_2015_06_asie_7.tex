
\medskip

\parbox{0.5\linewidth}{Un aquarium a la forme d'une sphère de 10~cm de
rayon, coupée en sa partie haute: c'est une \og calotte
sphérique \fg.

La hauteur totale de l'aquarium est 18 cm.}\hfill
\parbox{0.47\linewidth}{\psset{unit=0.9cm}
\begin{pspicture*}(0,-0.1)(5.6,4.1)
%\psgrid
\psarc(2.5,2.5){2.5}{144}{36}
\psline(0.45,4)(4.55,4)
\psline{<->}(2.5,2.5)(5,2.5)\uput[u](3.75,2.5){$r$}
\psline{<->}(5.3,0)(5.3,4)\rput{90}(5.45,2){$h$}
\end{pspicture*}}

\medskip

\begin{enumerate}
\item Le volume d'une calotte sphérique est donné par la formule :

\[V \dfrac{\pi}{3} \times h^2 \times (3r - h)\]

où $r$ est le rayon de la sphère et $h$ est la hauteur de la calotte sphérique.
	\begin{enumerate}
		\item Prouver que la valeur exacte du volume en cm$^3$ de l'aquarium est $\np{1296}\pi$.
		\item Donner la valeur approchée du volume de l'aquarium au litre près.
	\end{enumerate}
\item On remplit cet aquarium à ras bord, puis on verse la totalité de son contenu dans
un autre aquarium parallélépipédique. La base du nouvel aquarium est un rectangle
de $15$~cm par $20$~cm.

Déterminer la hauteur atteinte par l'eau (on arrondira au cm).

* Rappel: 1 $\ell$ = 1 dm$^3 = \np{1000}$ cm$^3$
\end{enumerate}
