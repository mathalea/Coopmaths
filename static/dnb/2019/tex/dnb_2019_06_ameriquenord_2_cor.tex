
\medskip


%Voici quatre affirmations. Pour chacune d'entre elles, dire si elle est vraie ou fausse. On rappelle que la réponse doit être justifiée.
%
%\medskip

\begin{enumerate}[itemsep=5mm]
	\item %\textbf{Affirmation 1 :} $\dfrac{3}{5} + \dfrac{1}{2} = \dfrac{3 + 1}{5 + 2}$.
$\bullet~~$ $\dfrac{3}{5} + \dfrac{1}{2}  = \dfrac{3\times 2}{5\times 2} + \dfrac{1\times 5}{2\times 5} = \dfrac{6 + 5}{10} = \dfrac{11}{10}$ ;

$\bullet~~$$\dfrac{3 + 1}{5 + 2} = \dfrac{4}{7}$.
Le premier nombre est supérieur à 1, le second est inférieur à 1 : ils ne sont donc pas égaux. \hfill \textbf{Affirmation fausse}	
	\item %On considère la fonction $f: x \longmapsto 5 - 3x$.
		
%\textbf{Affirmatíon 2 :} l'image de $-1$ par $f$ est $-2$.	
On a $f(- 1) = 5 - - 3\times (- 1) = 5 + 3 = 8 \ne - 2$. \hfill \textbf{Affirmation fausse}

	\item %On considère deux expériences aléatoires :
%		\begin{itemize}
%			\item \emph{expérience n\up{o}\,$1$ :} choisir au hasard un nombre entier compris entre 1 et 11 (1 et 11 inclus).			
%			\item \emph{expérience n\up{o}\,$2$ :} lancer un dé équilibré à six faces numérotées de 1 à 6 et annoncer le nombre qui apparait sur la face du dessus.
%		\end{itemize}
	
%\textbf{Affirmation 3 :} il est plus probable de choisir un nombre premier dans l'expérience n\up{o}\,1 que d'obtenir un nombre pair dans l'experience n\up{o}\,2.

De 1 à 11, il y a 2 ; 3 ; 5 ; 7 ; 11 soit 5 nombres sur 11 qui sont des naturels premiers. La probabilité de choisir un naturel premier est donc égale à $\dfrac{5}{11}$.

2 ; 4 ; 6 sont pairs  ; il y a donc $\dfrac{3}{6} = \dfrac{1}{2}$.

$\dfrac{5}{11} < \dfrac{5,5}{11} = \dfrac{1}{2}$. Donc \hfill \textbf{Affirmation fausse}.	
	\item %\textbf{Affirmation 4 :} pour tout nombre $x$, \quad $(2x + 1)^2 - 4 = (2x + 3)(2x - 1)$.
	Quel que soit le nombre $x$, \: $(2x + 1)^2 - 4 = (2x + 1)^2 - 2^2\: \left(\text{identité}\: a^2 - b^2\right) = (2x + 1 + 2)(2x + 1- 2) = (2x + 3)(2x - 1)$. \hfill \textbf{Affirmation vraie}.
\end{enumerate}
\bigskip

