
\medskip

Guillaume aimerait savoir combien de cheveux il a sur la tête. Pour cela il représente sa tête par une sphère de rayon $R$.

Il mesure le tour de sa tête comme indiqué sur le schéma ci-dessous et obtient $56$~cm.

\medskip

\parbox{0.65\linewidth}{\begin{tabularx}{\linewidth}{|Xm{3cm}|}\hline
\multicolumn{2}{|c|}{Rappels :}\\
Périmètre d'un cercle de rayon $R$ :& $\mathcal{P} = 2\pi R$\\
Aire d'une sphère de rayon $R$ :	& $\mathcal{A} = 4\pi R^2$.\\ \hline
\end{tabularx}} \hfill
\parbox{0.23\linewidth}{\psset{unit=1cm}
\begin{pspicture}(-1.9,-1.9)(1.9,1.9)
\pscircle(0,0){1.9}
\scalebox{.99}[0.3]{\psarc[linewidth=2pt](0,0){1.9}{180}{0}}%
\scalebox{.99}[0.3]{\psarc[linestyle=dashed,linewidth=2pt](0,0){1.9}{0}{180}}%
\psline(0,0)(1.9,0)\uput[u](0.95,0){$R$}\rput(0,-1){Tour de tête}
\psline{->}(0.3,-0.9)(-0.3,-0.55)
\end{pspicture}}

\medskip

\begin{enumerate}
\item Montrer que le rayon d'un cercle de périmètre $56$~cm est environ égal à $9$~cm.
\item Guillaume considère que ses cheveux recouvrent la moitié de la surface de sa tête. Sur 1 cm$^2$ de son crâne, il a compté 250 cheveux.

Estimer le nombre de cheveux de Guillaume.

\emph{Pour cette question toute trace de recherche sera valorisée lors de la notation.}
\end{enumerate}

\vspace{0,5cm}

