
\medskip

\begin{multicols}{3}
%\begin{minipage}[]{4.7cm}
On veut réaliser un dessin constitué de deux types
d'éléments (tirets et carrés) mis bout à bout.\\
Chaque script ci-contre trace un élément, et déplace le stylo.\\
On rappelle que \og s'orienter à 90 \fg{} signifie qu'on
oriente le stylo vers la droite.
%\end{minipage}

%\columnbreak

%\begin{minipage}[]{3.5cm}
{\small \begin{scratch}
\initmoreblocks{définir \namemoreblocks{carré }}
\blockmove{s’orienter à \ovalnum{90\selectarrownum} degrés}
\blockmove{tourner de \turnleft{} de \ovalnum{90} degrés}
\blockrepeat{répéter \ovalnum{4} fois}{
\blockmove{avancer de \ovalnum{5}}
\blockmove{tourner \turnright{} de \ovalnum{90} degrés}
\blockmove{avancer de \ovalnum{5}}
}
\blockpen{relever le stylo}
\blockmove{s’orienter à \ovalnum{90\selectarrownum} degrés}
\blockmove{avancer de \ovalnum{10}}
\blockpen{stylo en position d’écriture}
\end{scratch}}
%\end{minipage}

%\hspace{2cm}

%\begin{minipage}[]{3.5cm}
{\small\begin{scratch}
\initmoreblocks{définir \namemoreblocks{Tiret}}
\blockmove{s’orienter à \ovalnum{90\selectarrownum} degrés}
\blockmove{avancer de \ovalnum{10}}
\end{scratch}}
%\end{minipage}
\end{multicols}

\begin{enumerate}
\item En prenant 1 cm pour 2 pixels, représenter la figure obtenue si on exécute le script Carré.

Préciser les positions de départ et d'arrivée du stylo sur votre figure.

\medskip

Pour tracer le dessin complet, on a réalisé 2 scripts qui se servent des blocs \og  Carré\fg{} et \og Tiret \fg.
ci-dessus :

\begin{multicols}{2}
%\parbox{0.38\linewidth}{
Script 1\\ \\ \\ \\

{\small \begin{scratch}
\blockinit{quand \selectmenu{flèche haut} est pressé}
\blockmove{aller à x: \ovaloperator{\ovalmove{ -230} } y: \ovalmove{0}}
\blockmove{s’orienter à \ovalnum{90\selectarrownum} degrés}
\blockpen{effacer tout}
\blockpen{stylo en position d’écriture}
\blockrepeat{répéter \ovalnum{23} fois}
{
\blockmoreblocks{carré}
\blockmoreblocks{Tiret}
}
\end{scratch}}
%} 
%\hfill
%\parbox{0.58\linewidth}{

Script 2\\
{\small \begin{scratch}
\blockinit{quand \selectmenu{flèche bas} est pressé}
\blockmove{aller à x: \ovaloperator{\ovalmove{ -230} } y: \ovalmove{0}}
\blockmove{s’orienter à \ovalnum{90\selectarrownum} degrés}
\blockpen{effacer tout}
\blockpen{stylo en position d’écriture}
\blockrepeat{répéter \ovalnum{46} fois}{
\blockifelse{si \boolsensing{\ovaloperator{nombre aléatoire entre \ovalnum{1} et \ovalnum{2}}=\ovalnum{1}} alors}
{\blockmoreblocks{carré}}
{\blockmoreblocks{Tiret}}
}
\end{scratch}}
%}
\end{multicols}

\medskip
 
On exécute les deux scripts et on obtient les deux dessins ci-dessous.

\medskip

\psset{unit=0.3cm}
\begin{pspicture}(46,5.5)
\rput(2,5){Dessin A}\rput(2,2){Dessin B}
\psframe(0,3)(1,4)\psline(1,3.5)(2,3.5)\psframe(2,3)(3,4)\psframe(3,3)(4,4)\psline(4,3.5)(5,3.5)\psline(5,3.5)(6,3.5)\psframe(6,3)(7,4)\psline(7,3.5)(8,3.5)\psframe(8,3)(9,4)
\psframe(9,3)(10,4)\psframe(10,3)(11,4)\psframe(11,3)(12,4)
\psline(12,3.5)(13,3.5)\psline(13,3.5)(14,3.5)\psline(14,3.5)(15,3.5)
\psframe(15,3)(16,4)\psline(16,3.5)(17,3.5)\psline(17,3.5)(18,3.5)
\psframe(18,3)(19,4)\psframe(19,3)(20,4)\psframe(20,3)(21,4)
\psline(21,3.5)(22,3.5)\psline(22,3.5)(23,3.5)\psline(23,3.5)(24,3.5)
\psline(24,3.5)(25,3.5)\psline(25,3.5)(26,3.5)\psframe(26,3)(27,4)
\psframe(27,3)(28,4)\psline(28,3.5)(29,3.5)\psline(29,3.5)(30,3.5)
\psline(30,3.5)(31,3.5)\psframe(31,3)(32,4)\psline(32,3.5)(33,3.5)
\psframe(33,3)(34,4)\psline(34,3.5)(35,3.5)\psframe(35,3)(36,4)
\psline(36,3.5)(37,3.5)\psframe(37,3)(38,4)\psline(38,3.5)(39,3.5)
\psframe(39,3)(40,4)\psframe(40,3)(41,4)\psline(41,3.5)(42,3.5) \psline(42,3.5)(43,3.5)\psline(43,3.5)(44,3.5)
\psline(44,3.5)(45,3.5)\psframe(45,3)(46,4)
\multido{\n=0+2,\na=1+2,\nb=2+1}{23}{\psframe(\n,0)(\na,1)}
\multido{\n=1+2,\na=2+2}{23}{\psline(\n,0.5)(\na,0.5)}
\end{pspicture}

\item  Attribuer à chaque script la figure dessinée. Justifier votre choix.
\item  On exécute le script 2.
	\begin{enumerate}
			\item Quelle est la probabilité que le premier élément tracé soit un carré ?
			\item Quelle est la probabilité que les deux premiers éléments soient des carrés ?
	\end{enumerate}
\item  Dans le script 2, on aimerait que la couleur des différents éléments, tirets ou carrés, soit aléatoire, avec à chaque fois $50$\,\% de chance d'avoir un élément noir et $50$\,\% de chance d'avoir un élément rouge.
	
Écrire la suite d'instructions qu'il faut alors créer et préciser où l'insérer dans le script 2.
	
\textbf{Indication }: on pourra utiliser les instructions \begin{scratch}\blockvariable{mettre la couleur du stylo \`a \ovalvariable{\red rouge}\selectarrownum}\end{scratch} 

et \begin{scratch}\blockvariable{mettre la couleur du stylo \`a \ovalvariable{\black noir}\selectarrownum}\end{scratch} pour choisir la couleur du stylo.

\end{enumerate}

\bigskip

