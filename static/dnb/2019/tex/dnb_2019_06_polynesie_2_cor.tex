
\medskip

\begin{enumerate}
\item %On a utilisé une feuille de calcul pour obtenir les images de différentes valeurs de $x$ par une fonction affine $f$.

%Voici une copie de l'écran obtenu :

%\begin{center}
%\begin{tabularx}{\linewidth}{|c|*{8}{>{\centering \arraybackslash}X|}}\hline
%\multicolumn{2}{|l|}{B2}&\multicolumn{7}{l|}{=3*B1$-4$}\\ \hline
%	&A		&B		&C		&D		&E		&F	&G	&H\\ \hline
%1	&$x$	&$-2$	&$-1$	&0		&1		&2	&3	&4\\ \hline
%2	&$f(x)$	&\multicolumn{1}{>{\columncolor{lightgray}}c|}{$- 10$}	&$- 7$	&$- 4$	&$- 1$	&2	&5	&8\\ \hline
%\end{tabularx}
%\end{center}

	\begin{enumerate}
		\item Les antécédents sont dans la ligne 1, les images dans la ligne 2.
		
%Quelle est l'image de $- 1$ par la fonction $f$ ?
L'image de $- 1$ par la fonction $f$ est $f(-1)  = - 7$.
		\item %Quel est l'antécédent de $5$ par la fonction $f$?
L'antécédent de $5$ par la fonction $f$ est $3$.
		\item %Donner l'expression de $f(x)$.
On a $f(x) = 3x - 4$.
		\item %Calculer $f(10)$.
Donc $f(10) = 3 \times 10 - 4 = 30 - 4 = 26$.
 	\end{enumerate}
\item  %On donne le programme suivant qui traduit un programme de calcul.

%\begin{center}
%\begin{scratch}
%\blockinit{Quand \greenflag est cliqué}
%\blocksensing{demander \txtbox{Choisir un nombre} et attendre}
%\blockmove{mettre \ovalnum{A \selectarrownum} à \ovalvariable{réponse}}
%\blockmove{mettre \ovalnum{A\selectarrownum} à \ovalnum{\ovalnum{A} + \ovalnum{3}}}
%\blockmove{mettre \ovalnum{A\selectarrownum} à \ovalnum{\ovalnum{A} * \ovalnum{2}}}
%\blockmove{mettre \ovalnum{A\selectarrownum} à \ovalnum{\ovalnum{A} $- \ovalnum{3}$}}
%\blockmove{dire\ovalnum{regroupe} {\txtbox{Le programme de calcul donne }}\ovalnum{A}}
%\end{scratch}
%\end{center}
	\begin{enumerate}
		\item Écrire sur votre copie les deux dernières étapes du programme de calcul:
\begin{center}
\begin{tabularx}{0.4\linewidth}{|X|}\hline
$\bullet~~$ Choisir un nombre.\\
$\bullet~~$ Ajouter 3 à ce nombre.\\
$\bullet~~$ {\blue Multiplier ce nombre par 2}\\
$\bullet~~$ {\blue Retrancher 5 de ce nombre}\\ \hline
\end{tabularx}
\end{center}
		\item  %Si on choisit le nombre $8$ au départ, quel sera le résultat ?
		8 donne successivement $8 \to 11 \to 22 \to 17$.
		\item  %Si on choisit $x$ comme nombre de départ, montrer que le résultat obtenu avec ce programme de calcul sera $2x + 1$.
$x$ donne successivement $x \to x + 3 \to 2(x + 3) \to 2(x + 3) - 5$.

Or $2(x + 3) - 5 = 2x + 6 - 5 = 2x + 1$.
		\item %Quel nombre doit-on choisir au départ pour obtenir 6 ?
		
$\bullet~~$Il faut trouver $x$ tel que $2(x + 3) - 5 = 2x + 6 - 5 = 2x + 1 = 6$ soit $2x = 5$ et enfin $x = 2,5$.

$\bullet~~$On peut \og remonter \fg{} les opérations :

$  5,5 - 3 = 2,5 \gets\dfrac{11}{2} = 5,5 \gets 6 + 5 = 11\gets 6$.
 	\end{enumerate}
\item  %Quel nombre faudrait-il choisir pour que la fonction $f$ et le programme de calcul
%donnent le même résultat ?
Il faut trouver $x$ tel que :

$3x - 4 = 2x + 1$ soit en ajoutant $- 2x$ à chaque membre : $x - 4 = 1$ et en ajoutant 4 à chaque membre : $x = 5$.

Par $f$ et par le programme de calcul 5 donne 11.
\end{enumerate}

\bigskip

