
\medskip

On considère le programme de calcul : \:\begin{tabular}{|l|}\hline
$\bullet~~$Choisir un nombre.\\
$\bullet~~$Prendre le carré de ce nombre.\\
$\bullet~~$Ajouter le triple du nombre de départ.\\
$\bullet~~$Ajouter 2.\\ \hline
\end{tabular}

\medskip

\begin{enumerate}
\item Montrer que si on choisit 1 comme nombre de départ, le programme donne $6$ comme résultat.
\item Quel résultat obtient-on si on choisit $-5$ comme nombre de départ ?
\item On appelle $x$ le nombre de départ, exprimer le résultat du programme en fonction de $x$.
\item Montrer que ce résultat peut aussi s'écrire sous la forme $(x + 2)(x + 1)$ pour toutes les valeurs de $x$.
\item  La feuille du tableur suivante regroupe des résultats du programme de calcul précédent.

\medskip
\begin{tabularx}{\linewidth}{|c|l|*{9}{>{\centering \arraybackslash}X|}}\hline
	&A 			&B		&C		&D		&E		&F		&G		&H		&I		&J\\ \hline
1	&$x$		& $-4$	&$-3$	&$-2$	&$-1$	&0		&1		&2		&3		&4\\ \hline
2	&$(x+2)(x+1)$&6		&2 		&0 		&0 		&2 	&6 &12 &20 &30\\ \hline
\end{tabularx}
\medskip

	\begin{enumerate}
		\item Quelle formule a été écrite dans la cellule B2 avant de l'étendre jusqu'à la cellule J2 ?
		\item Trouver les valeurs de $x$ pour lesquelles le programme donne 0 comme résultat.
 	\end{enumerate}
\end{enumerate}

\vspace{0,5cm}

