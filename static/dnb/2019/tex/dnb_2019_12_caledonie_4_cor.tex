
\medskip

%\parbox{0.55\linewidth}{Thomas attache son cerf-volant au sol au point T.
%
%Il fait 20 pas pour parcourir la distance TH.
%
%Un pas mesure 0,6 mètre.
%
%Le schéma ci-contre illustre la situation. Il n'est pas à l'échelle.

\begin{enumerate}
\item %Montrer que la hauteur CH du cerf-volant est égale à 9~m.

On a TH $ = 20 \times 0,6 = 12$~(m).

Dans le triangle CTH rectangle en H le théorème de Pythagore s'écrit :

$\text{CT}^2  = \text{TH}^2 + \text{HC}^2$ ou $15^2 = 12^2 + \text{HC}^2$ soit $\text{HC}^2 = 15^2 - 12^2 = (15 + 12)(15 - 12) = 27 \times 3 = 81 = 9^2$, d'où CH $ = 9$~(m).
\item %Thomas souhaite que son cerf-volant atteigne une hauteur EF de 13,5 m.

%Calculer la longueur TE de la corde nécessaire.
Les droites (CH) et (EF) étant toutes deux perpendiculaires à la droite (TH) sont parallèles ; on a donc une configuration de Thalès ce qui permet d'écrire l'égalité des rapports :

$\dfrac{\text{EF}}{\text{CH}} = \dfrac{\text{TE}}{\text{CT}}$ soit $\dfrac{13,5}{9} = \dfrac{\text{TE}}{\text{15}}$, d'où en multipliant par 15 :

TE $ = 15 \times \dfrac{13,5}{9} = 5 \times \dfrac{13,5}{3} = 5 \times 4,5 = 22,5$~(m)
\end{enumerate}
%}\hfill
%\parbox{0.44\linewidth}{
%\psset{unit=0.9cm}
%\begin{pspicture}(7.5,5.3)
%\def\cerf{\psline[doubleline=true,doublesep=1.25pt](-0.6,0)(0.6,0)
%\psline[doubleline=true,doublesep=1.25pt](0,0.9)(0,-0.9)
%\pspolygon(-0.6,0)(0,0.9)(0.6,0)(0,-0.9)
%\pscurve(0,-0.9)(0.15,-1.1)(0,-1.4)(-0.2,-1.6)
%\pscurve(0,-0.9)(0.05,-1.1)(0,-1.4)(0.15,-1.6)}
%\psline(0,0.4)(7.5,0.4)\psline(1.2,0.4)(5.4,3.3)
%\psline[linestyle=dashed](5.4,0.4)(5.4,3.3)%HC
%\psline[linestyle=dashed](7.1,0.4)(7.1,4.5)%FE
%\psline[linestyle=dashed](5.4,3.3)(7.1,4.5)
%\psframe(5.4,0.4)(5.2,0.6)\psframe(7.1,0.4)(6.9,0.6)
%\uput[d](1.2,0.4){T}\uput[d](5.4,0.4){H} \uput[d](7.1,0.4){F}
%\uput[u](5.4,3.3){C}\uput[ur](7.1,4.5){E}
%\rput{-90}(7.3,2.3){13,5 m}\uput[u](3.4,0.4){20 pas}
%\uput[d](0.4,0.4){sol}
%\rput{55}(5.4,3.3){\cerf}
%\uput[r](0,5){Les points T, C et E sont alignés.}
%\uput[r](0,4.6){Les points T, H et F sont alignés.}
%\uput[r](0,4.2){TC = 15~m}
%\psframe(0,3.9)(5.4,5.3)
%\end{pspicture}}

\vspace{0,5cm}

