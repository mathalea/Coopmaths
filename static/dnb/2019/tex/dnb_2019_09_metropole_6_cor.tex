
\medskip

%Deux amis Armelle et Basile jouent aux dés en utilisant des dés bien équilibrés mais dont les faces ont
%été modifiées. Armelle joue avec le dé A et Basile joue avec le dé B.
%
%\medskip
%
%Lors d'une partie, chaque joueur lance son dé et celui qui obtient le plus grand numéro gagne un point.
%
%\medskip
%
%Voici les patrons des deux dés:
%
%\begin{center}
%\begin{tabularx}{\linewidth}{*{2}{>{\centering \arraybackslash}X}}
%Patron du dé A &Patron du dé B\\
%\psset{unit=1cm}
%\begin{pspicture}(4,3)
%\pspolygon(0,1)(0,2)(1,2)(1,3)(2,3)(2,2)(4,2)(4,1)(2,1)(2,0)(1,0)(1,1)
%\psline(1,1)(1,2)\psline(2,1)(2,2)\psline(3,1)(3,2)\psline(1,2)(2,2)\psline(1,1)(2,1)
%\psdots(0.2,1.2)(0.8,1.8)(1.2,1.2)(1.8,1.8)(2.2,1.2)(2.8,1.8)(3.2,1.2)(3.8,1.8)
%(1.2,2.25)(1.2,2.5)(1.2,2.75)(1.2,0.25)(1.2,0.5)(1.2,0.75)
%(1.8,2.25)(1.8,2.5)(1.8,2.75)(1.8,0.25)(1.8,0.5)(1.8,0.75)
%\end{pspicture}&
%\psset{unit=1cm}
%\begin{pspicture}(4,3)
%\pspolygon(0,1)(0,2)(1,2)(1,3)(2,3)(2,2)(4,2)(4,1)(2,1)(2,0)(1,0)(1,1)
%\psline(1,1)(1,2)\psline(2,1)(2,2)\psline(3,1)(3,2)\psline(1,2)(2,2)\psline(1,1)(2,1)
%\psdots(0.5,1.5)(2.5,1.5)(3.5,1.5)(1.5,1.5)(1.25,1.25)(1.75,1.25)(1.25,1.75)(1.75,1.75)
%(1.25,0.25)(1.75,0.25)(1.25,0.75)(1.75,0.75)(1.5,0.5)(1.25,2.25)(1.25,2.75)(1.5,2.5)(1.75,2.25)(1.75,2.75)
%\end{pspicture}\\
%\end{tabularx}
%\end{center}
%
%\smallskip

\begin{enumerate}
\item %Une partie peut-elle aboutir à un match nul ?
Armelle peut tirer 2 ou 6 et Basile 1 ou 5 : il ne peut y avoir égalité, donc de match nul.
\item 
	\begin{enumerate}
		\item %Si le résultat obtenu avec le dé A est 2, quelle est la probabilité que Basile gagne un point ?
Basile a $\dfrac{3}{6} = \dfrac{1}{2}$ chances de sortir un 5 donc de battre Armelle.
		\item %Si le résultat obtenu avec le dé B est 1, quelle est la probabilité qu'Armelle gagne un point?
Dans tous les cas Armelle  tire un 2  un 6 et tous les deux sont supérieurs à 1 : sa probabilité de battre  Basile est donc égale à 1.
 	\end{enumerate}
\end{enumerate}
 
%Les joueurs souhaitent comparer leur chance de gagner. Ils décident de simuler un match de
%soixante mille duels à l'aide d'un programme informatique.
%	
%Voici une partie du programme qu'ils ont réalisé.
%
%\begin{small}
%
%\begin{tabularx}{\linewidth}{l X}	
%Programme principal &Sous-programmes\\
%\begin{scratch}
%\blockinit{quand \greenflag est cliqué}
%\blockvariable{mettre \ovalvariable{Victoire de A\selectarrownum} \`a \ovalnum{0}}
%\blockvariable{mettre \ovalvariable{Victoire de B\selectarrownum} \`a \ovalnum{0}}
%\blockrepeat{répéter \ovalnum{60000} fois}
%{
%\blockmove{Lancer le dé A}
%\blockmove{Lancer le dé B}
%\blockifelse{si \booloperator{\ldots < \ldots} alors}
%{
%\blockvariable{ajouter \`a {Victoire de A~ \selectarrownum} \ovalnum{1}}
%}
%{\blockvariable{ajouter \`a {Victoire de B~ \selectarrownum} \ovalnum{1}}}
%}
%\end{scratch}&\begin{scratch}
%\initmoreblocks{définir \ovalmoreblocks{Lancer le dé A}}
%{\blockvariable{mettre \ovalvariable{tirage  de dé~ \selectarrownum} \`a \ovalvariable{nombre aléatoire entre \ovalnum{1} et \ovalnum{6}}}
%\blockifelse{si \booloperator{\ovalvariable{tirage de dé } < 5} alors}
%{
%\blockvariable{mettre  {\ovalvariable{FaceA~ \selectarrownum}} \`a \ovalnum{2}}
%}
%{\blockvariable{mettre  {\ovalvariable{FaceA~ \selectarrownum}} \`a \ovalnum{6}}}
%}
%\end{scratch}
%
%\begin{scratch}
%\initmoreblocks{définir \namemoreblocks{Lancer le dé B}}
%\end{scratch}\\ \end{tabularx}
%\end{center}
%\end{small}
%
%\medskip
%
%On précise que l'expression (\textbf{nombre aléatoire entre 1 et 6}) renvoie de manière équiprobable un
%nombre pouvant être 1 ; 2 ; 3 ; 4 ; 5 ou 6.
%	
%Les variables FaceA et FaceB enregistrent les résultats des dés A et B. Par exemple, la variable FaceA
%peut prendre soit la valeur 2 soit la valeur 6, puisque ce sont les seuls nombres présents sur le dé A.
%	
%Les variables Victoire de A et Victoire de B comptent les victoires des joueurs.

\begin{enumerate}[resume]
\item 
	\begin{enumerate}
		\item %Lorsqu'on exécute le sous-programme \og Lancer le dé A \fg, quelle est la probabilité que la variable FaceA prenne la valeur 2 ?
		Il y a $\dfrac{4}{6} = \dfrac{2}{3}$ de chances que le nombre tiré soit inférieur à 5 donc que FaceA prenne la valeur 2.
		\item %Recopier la ligne 7 du programme principal en la complétant.
Si FaceB < FaceA alors
		\item %Rédiger un sous-programme \og Lancer le dé B \fg{} qui simule le lancer du dé B et enregistre le nombre obtenu dans la variable FaceB.
C'est le même sous-programme que Lancer le dé A en remplaçant Lancer le dé A par Lancer le dé B, 5 par 4 (troisième ligne), 2 par 1 (quatrième ligne) et 6 par 5 (cinquième ligne).
	\end{enumerate}
\item  %Après exécution du programme principal, on obtient les résultats suivants:
	
%\emph{Victoire de A}$ = \np{39901}$ \qquad  \emph{Victoire de A}$ = \np{20099}$
	\begin{enumerate}
		\item %Calculer la fréquence de gain du joueur A, exprimée en pourcentage. On donnera une valeur approchée à 1\,\% près.
La fréquence de victoires de A est égale à $\dfrac{\np{39901}}{\np{39901} + \np{20099}} = \dfrac{\np{39901}}{\np{60000}}\approx 0,665$, soit 66,5\,\% à 0,1 près
		\item %Conjecturer la probabilité que A gagne contre B.
La probabilité que A gagne contre B est d'environ 66,666\,\% soit 2 chances sur 3.
	\end{enumerate}
\end{enumerate}
