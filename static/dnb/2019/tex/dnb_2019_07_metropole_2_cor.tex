
\medskip

\begin{enumerate}
\item Dans ADM rectangle en A, on a  $\tan \widehat{\text{ADM}} = \dfrac{\text{côté opposé à D}}{\text{côté adjacent à D}} = \dfrac{\text{AM}}{\text{AD}}$, soit $\tan (60) = \frac{\text{AM}}{2}$, d'où

AM $= 2 \times \tan 60 \approx \np{3,4641}$ soit 3,46~m au centième près.
 
[AM] mesure environ 3,46 m. 
\item  Comme M appartient à  [AB], on a $\text{MB} = \text{AB} - \text{AM}$, 
soit MB $\approx 4 - 3,46 \approx  0,54$.

La proportion de plaque non utilisée est  $\dfrac{\text{MB}}{\text{AB}} \approx  0,14$ au centième près.
\item  Comme dans un triangle la somme des angles est égale à $180$\degres,
on a dans AMD, un angle de $90$\degres{} en A, un angle de $60$\degres{} en D et un angle de $30$\degres{} en M.

Dans le triangle DPN rectangle en P, on a donc un angle de 90\degres{} en P.
De plus, son angle en D mesure $90 - 60$, soit $30$\degres.

Le triangle DPN ayant deux angles de 90\degres{} et 30\degres{} comme le triangle ADM, ces deux triangles sont semblables.

Dans le triangle MPN rectangle en P, on a donc un angle de $90$\degres{} en P.
De plus, son angle en M mesure $90 - 30$, soit $60$\degres.

Le triangle MPN ayant deux angles de 90\degres et 60\degres{} comme le triangle ADM, ces deux triangles sont semblables.

Les trois triangles AMD, PNM et PDN sont semblables.
 
Deux triangles sont semblables si deux angles de l'un des triangles ont les mêmes mesures que deux angles de l'autre triangle.
\item  Les triangles DNP et ADM sont semblables.

Le rapport d'agrandissement pour passer de DNP à ADM est par exemple, le rapport $\dfrac{\text{DM}}{\text{DN}}$ des hypoténuses.

On a DN = AM $\approx 3,46$.

Dans ADM rectangle en A, on a  $\cos \widehat{\text{ADM}} = \dfrac{\text{AD}}{\text{DM}}$, soit $\cos 60 = \dfrac{2}{\text{DM}}$.

On a donc $\cos 60 \times \text{DM} = 2$, soit $\text{DM} = \dfrac{2}{\cos 60}$.

On a DM $= 4$.
Le rapport d'agrandissement est $\dfrac{4}{3,46}$, soit environ $1,16$.

Il est donc  inférieur à $1,5$.
\end{enumerate}

\bigskip

