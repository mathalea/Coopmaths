
\medskip

%Angelo va sur le site \og météo NC \fg{} pour avoir une idée des meilleurs moments pour faire du cerf-volant avec ses enfants.
%
%Il obtient le graphique ci-dessous qui donne la prévision de la vitesse du vent, en nœuds, en fonction de l'heure de la journée.
%
%Répondre aux questions par lecture graphique. Aucune justification n'est demandée.
%
%\begin{center}
%\psset{xunit=0.55cm,yunit=0.3cm}
%\begin{pspicture}(-1,-1.5)(23,30)
%\multido{\n=0+1}{24}{\psline[linestyle=dotted,dotsep=1pt](\n,0)(\n,26)}
%\multido{\n=0+1}{27}{\psline[linestyle=dotted,dotsep=1pt](0,\n)(23,\n)}
%\psaxes[linewidth=1.25pt,labelFontSize=\scriptstyle]{->}(0,0)(0,0)(23,26)
%\psaxes[linewidth=1.25pt,labelFontSize=\scriptstyle](0,0)(0,0)(23,26)
%\pscurve[linewidth=1.25pt](0,17)(1,12)(2,10)(3,10)(4,9)(5,7)(6,10)(7,12)(8,14)(9,21)(10,21)(11,24)(12,20)(13,17)(14,19)(15,17)(16,16)(17,15)(18,18)(19,19)(20,17)(21,15)
%\uput[d](22.2,-1.5){Heure}\uput[r](0,27){Vitesse du vent en nœuds}
%\rput(11,29){\large Vitesse moyenne des vents (en nœuds) par heure}
%\end{pspicture}
%\end{center}

\begin{enumerate}
\item 
	\begin{enumerate}
		\item %Quelle est la vitesse du vent prévue à 14 h ?
À 14 h la vitesse du vent prévue est de 19 nœuds par heure.
		\item %À quelles heures prévoit-on 12 nœuds de vent?
La vitesse du vent sera de 12 nœuds par heure à 1~h et à 7~h.
		\item %À quelle heure la vitesse du vent prévue est-elle la plus élevée?
La vitesse maximale de 23 nœuds par heure est prévue à 11~h.
		\item %À quelle heure la vitesse du vent prévue est-elle la plus faible ?
		
La vitesse la plus faible (7 nœuds par heure) est prévue à 5~h.
	\end{enumerate}
\item %La pratique du cerf-volant est dangereuse au-dessus de 20 nœuds.

%De quelle heure à quelle heure ne faut-il pas faire de cerf-volant ? 

%\emph{On répondra avec la précision permise par le graphique}.
La pratique du cerf-volant sera dangereuse entre 8~h 30 et 12~h.
\end{enumerate}

\vspace{0,5cm}

