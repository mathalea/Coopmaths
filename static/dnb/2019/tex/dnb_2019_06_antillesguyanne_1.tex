
\medskip

Damien a fabriqué trois dés à six faces parfaitement équilibrés mais un peu particuliers.

Sur les faces du premier dé sont écrits les six plus petits nombres pairs strictement positifs : 2 ; 4; 6 ; 8 ; 10; 12.

Sur les faces du deuxième dé sont écrits les six plus petits nombres impairs positifs.

Sur les faces du troisième dé sont écrits les six plus petits nombres premiers.

Après avoir lancé un dé, on note le nombre obtenu sur la face du dessus.

\medskip

\begin{enumerate}
\item Quels sont les six nombres figurant sur le deuxième dé ? 

Quels sont les six nombres figurant sur le troisième dé ?
\item  Zoé choisit le troisième dé et le lance. Elle met au carré le nombre obtenu.
Léo choisit le premier dé et le lance. Il met au carré le nombre obtenu.
	\begin{enumerate}
		\item Zoé a obtenu un carré égal à $25$. Quel était le nombre lu sur le dé qu'elle a lancé ?
		\item Quelle est la probabilité que Léo obtienne un carré supérieur à celui obtenu par Zoé ?
 	\end{enumerate}
\item  Mohamed choisit un des trois dés et le lance quatre fois de suite. Il multiplie les quatre nombres obtenus et obtient 525.
	\begin{enumerate}
		\item Peut-on déterminer les nombres obtenus lors des quatre lancers ? Justifier.
		\item Peut-on déterminer quel est le dé choisi par Mohamed ? Justifier.
 	\end{enumerate}
\end{enumerate}

\bigskip

