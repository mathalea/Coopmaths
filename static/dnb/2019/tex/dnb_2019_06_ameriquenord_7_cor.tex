
\medskip

%Pour ranger les boulets de canon, les soldats du XVI\up{e} siècle utilisaient souvent un type d'empilement pyramidal à base carrée, comme le montrent les dessins suivants :
%
%\hspace{-16mm}
%\begin{tabular}{>{\centering \arraybackslash} p{2.6cm}
%		>{\centering \arraybackslash} p{3.6cm}
%		>{\centering \arraybackslash} p{4.6cm}
%		>{\centering \arraybackslash} p{5.6cm}}
%\begin{tikzpicture}[x={(6:8mm)},y=(145:4mm),z={(90:8.50mm)},baseline={(current bounding box.center)}]
%\foreach \x in {2.5,3.5}{
%	\shade[ball color=gray!30] (\x,3.5,1.697) circle (4mm);}
%\shade[ball color=gray!30] (3,3,2.263) circle (4mm);
%\foreach \x in {2.5,3.5}{
%	\shade[ball color=gray!30] (\x,2.5,1.697) circle (4mm);}
%\end{tikzpicture}
%&
% \begin{tikzpicture}[x={(6:8mm)},y=(145:4mm),z={(90:8.50mm)},baseline={(current bounding box.center)}]
%\foreach \x in {2,3,4}{
%	\shade[ball color=gray!30] (\x,4,1.131) circle (4mm);}
%\foreach \x in {2.5,3.5}{
%	\shade[ball color=gray!30] (\x,3.5,1.697) circle (4mm);}
%\foreach \x in {2,3,4}{
%	\shade[ball color=gray!30] (\x,3,1.131) circle (4mm);}
%\shade[ball color=gray!30] (3,3,2.263) circle (4mm);
%\foreach \x in {2.5,3.5}{
%	\shade[ball color=gray!30] (\x,2.5,1.697) circle (4mm);}
%\foreach \x in {2,3,4}{
%	\shade[ball color=gray!30] (\x,2,1.131) circle (4mm);}
%\end{tikzpicture}
%&
%\begin{tikzpicture}[x={(6:8mm)},y=(145:4mm),z={(90:8.50mm)},baseline={(current bounding box.center)}]
%\foreach \x in {1.5,2.5,...,4.5}{
%	\shade[ball color=gray!30] (\x,4.5,0.5657) circle (4mm);}
%\foreach \x in {2,3,4}{
%	\shade[ball color=gray!30] (\x,4,1.131) circle (4mm);}
%\foreach \x in {1.5,2.5,...,4.5}{
%	\shade[ball color=gray!30] (\x,3.5,0.5657) circle (4mm);}
%\foreach \x in {2.5,3.5}{
%	\shade[ball color=gray!30] (\x,3.5,1.697) circle (4mm);}
%\foreach \x in {2,3,4}{
%	\shade[ball color=gray!30] (\x,3,1.131) circle (4mm);}
%\shade[ball color=gray!30] (3,3,2.263) circle (4mm);
%\foreach \x in {1.5,2.5,...,4.5}{
%	\shade[ball color=gray!30] (\x,2.5,0.5657) circle (4mm);}
%\foreach \x in {2.5,3.5}{
%	\shade[ball color=gray!30] (\x,2.5,1.697) circle (4mm);}
%\foreach \x in {2,3,4}{
%	\shade[ball color=gray!30] (\x,2,1.131) circle (4mm);}
%\foreach \x in {1.5,2.5,...,4.5}{
%	\shade[ball color=gray!30] (\x,1.5,0.5657) circle (4mm);}
%\end{tikzpicture}
%&
% \begin{tikzpicture}[x={(6:8mm)},y=(145:4mm),z={(90:8.50mm)},baseline={(current bounding box.center)}]
% \foreach \x in {1,...,5}{
% 	\shade[ball color=gray!30] (\x,5,0) circle (4mm);}
% \foreach \x in {1.5,2.5,...,4.5}{
%	\shade[ball color=gray!30] (\x,4.5,0.5657) circle (4mm);}
% \foreach \x in {1,...,5}{
% 	\shade[ball color=gray!30] (\x,4,0) circle (4mm);}
% \foreach \x in {2,3,4}{
%	\shade[ball color=gray!30] (\x,4,1.131) circle (4mm);}
% \foreach \x in {1.5,2.5,...,4.5}{
%	\shade[ball color=gray!30] (\x,3.5,0.5657) circle (4mm);}
% \foreach \x in {2.5,3.5}{
%	\shade[ball color=gray!30] (\x,3.5,1.697) circle (4mm);}
% \foreach \x in {1,...,5}{
%	\shade[ball color=gray!30] (\x,3,0) circle (4mm);}
% \foreach \x in {2,3,4}{
%	\shade[ball color=gray!30] (\x,3,1.131) circle (4mm);}
%	\shade[ball color=gray!30] (3,3,2.263) circle (4mm);
% \foreach \x in {1.5,2.5,...,4.5}{
%	\shade[ball color=gray!30] (\x,2.5,0.5657) circle (4mm);}
% \foreach \x in {2.5,3.5}{
%	\shade[ball color=gray!30] (\x,2.5,1.697) circle (4mm);}
% \foreach \x in {1,...,5}{
%	\shade[ball color=gray!30] (\x,2,0) circle (4mm);}
% \foreach \x in {2,3,4}{
%	\shade[ball color=gray!30] (\x,2,1.131) circle (4mm);}
% \foreach \x in {1.5,2.5,...,4.5}{
%	\shade[ball color=gray!30] (\x,1.5,0.5657) circle (4mm);}
%  \foreach \x in {1,...,5}{
% 	\shade[ball color=gray!30] (\x,1,0) circle (4mm);}
% \end{tikzpicture}\\ [1.8cm]
% Empilement \linebreak à 2 niveaux&Empilement à 3 niveaux&
% Empilement à 4 niveaux&Empilement à 5 niveaux
%\end{tabular}

\begin{enumerate}
	\item %Combien de boulets contient l'empilement à 2 niveaux ?
L'empilement à 2 niveaux contient $4 + 1 = 5$~(boulets).	
	\item %Expliquer pourquoi l'empilement à 3 niveaux contient 14 boulets.	
L'empilement à 3 niveaux contient $9 + 4 + 1 = 14$~(boulets).
	\item %On range 55 boulets de canon selon cette méthode. Combien de niveaux comporte alors l'empilement obtenu ?	
Avec 4 niveaux on peut ranger $16 + 9 + 4 + 1 = 30$~(boulets). Il faut donc un niveau de plus de $5 \times 5 = 25$ ~(boulets).

Sur 5 niveaux il y aura $25 + 16 + 9 + 4 + 1 = 55$~(boulets exactement).
	\item %Ces boulets sont en fonte; la masse volumique de cette fonte est de \np[kg/m^3]{7300}.
-- Volume d'un boulet : $\dfrac{4}{3} \times \pi \times 6 \times 6 \times 6 = 288\pi$~cm$^3$.

-- L'empilement à 3 niveaux contient 14~boulets qui ont un volume de $14 \times 288\pi = \np{4032}\pi$~cm$^3$.

1 m$^3$ de fonte a une masse de \np{7300}~kg, donc 1~dm$^3$ de fonte a une masse de 7,3~kg et 1~cm$^3$ de fonte a une masse de \np{0,0073}~kg, donc les 14 boulets ont une masse de :

$\np{4032}\pi \times \np{0,0073}= \np{29,4336}\pi \approx 92,46$~kg, soit 92~kg au kilogramme près.
	
%On modélise un boulet de canon par une boule de rayon \np[cm]{6}.
%	
%Montrer que l'empilement à 3 niveaux de ces boulets pèse \np[kg]{92}, au kg près.
	
%\emph{Rappels:}
%\begin{itemize}
%		\item $\emph{volume d'une boule} = \dfrac{4}{3}\times \pi \times \text{\emph{rayon}} \times \text{\emph{rayon}} \times \text{\emph{rayon}}$.	
%		\item une masse volumique de \np[kg/m^3]{7300} signifie que \np[m^3]{1} pèse \np[kg]{7300}.
%\end{itemize}
\end{enumerate}
 
\vspace{0,5cm}

