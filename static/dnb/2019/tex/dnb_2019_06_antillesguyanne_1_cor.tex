
\medskip

%Damien a fabriqué trois dés à six faces parfaitement équilibrés mais un peu particuliers.
%
%Sur les faces du premier dé sont écrits les six plus petits nombres pairs strictement positifs : 2 ; 4 ; 6 ; 8 ; 10 ; 12.
%
%Sur les faces du deuxième dé sont écrits les six plus petits nombres impairs positifs.
%
%Sur les faces du troisième dé sont écrits les six plus petits nombres premiers.
%
%Après avoir lancé un dé, on note le nombre obtenu sur la face du dessus.
%
%\medskip

\begin{enumerate}
\item %Quels sont les six nombres figurant sur le deuxième dé ? 
Sur les faces du deuxième dé sont écrits : 1 ; 3 ; 5 ; 7 ; 9 ; 11.

%Quels sont les six nombres figurant sur le troisième dé ?
Sur les faces du troisième dé sont écrits : 2 ; 3 ; 5  ; 7 ; 11 ; 13.
\item %Zoé choisit le troisième dé et le lance. Elle met au carré le nombre obtenu.
%Léo choisit le premier dé et le lance. Il met au carré le nombre obtenu.
	\begin{enumerate}
		\item %Zoé a obtenu un carré égal à $25$. Quel était le nombre lu sur le dé qu'elle a lancé ?
On a $25 = 5^2$ : Zoé a obtenu 5.
		\item %Quelle est la probabilité que Léo obtienne un carré supérieur à celui obtenu par Zoé ?
Les nombres du premier dé dont le carré est supérieur à 25 sont : 6 ; 8 ; 10 ; 12 soit 4 nombres sur 6.
		
La probabilité que Léo obtienne un carré supérieur à celui obtenu par Zoé est donc égale à $\dfrac{4}{6} = \dfrac{2}{3}$.
 	\end{enumerate}
\item  %Mohamed choisit un des trois dés et le lance quatre fois de suite. Il multiplie les quatre nombres obtenus et obtient 525.
	\begin{enumerate}
		\item %Peut-on déterminer les nombres obtenus lors des quatre lancers ? Justifier.
$\bullet~~$525 est impair, donc Mohamed n'a pas choisi le premier dé qui ne donne que des nombres pairs ;
		
$\bullet~~$On a $525 = 25 \times 21 = 3 \times 5^2 \times 7$.

On peut donc exclure de la liste des numéros sortis 9 et 11 pour le dé 2 et 11 et 13 pour le dé 3 ainsi que 2, car le produit serait pair.

On voit qu'avec le dé 2 ou le dé 3, il a pu obtenir (dans n'importe quel ordre) : 3 ; 5 ; 5 ; 7 qui sont les quatre nombres obtenus.

Ce sont les nombres obtenus, mais ils peuvent provenir du dé 2 ou du dé 3, donc
		\item %Peut-on déterminer quel est le dé choisi par Mohamed ? Justifier.
Ce sont les nombres obtenus, mais ils peuvent provenir du dé 2 ou du dé 3, donc on ne peut pas savoir quel est le dé choisi par Mohamed.
 	\end{enumerate}
\end{enumerate}

\bigskip

