
\medskip
 
%Voici les copies d'écran d'un programme qui permet d'obtenir une frise.
%
%\begin{center}
%\begin{tabularx}{\linewidth}{|*{2}{>{\centering \arraybackslash}X|}m{4cm}|}\hline
%\textbf{Script de la frise}&\textbf{Bloc motif}&\textbf{Pour information}\\
%\small{\begin{scratch}
%\blockinit{Quand \greenflag est cliqu\'e}
%\blockmove{s'orienter à \ovalnum{90} }
%\blockmove{aller à x: \ovalnum {- 150} y: \ovalnum 0}
%\blockpen{effacer tout}
%\blockrepeat{répéter \ovalnum{4} fois}
%{\blockevent{Motif}
%\blockmove{avancer de \ovalnum{30}}}
%\end{scratch}}&
%\small{\begin{scratch}
%\initmoreblocks{définir \namemoreblocks{Motif}}
%\blockpen{stylo en position d'écriture}
%\blockmove{avancer de \ovalnum{20}}
%\blockmove{tourner \turnright{} de \ovalnum{60} degrés}
%\blockmove{avancer de \ovalnum{40}}
%\blockmove{tourner \turnleft{} de \ovalnum{120} degrés}
%\blockmove{avancer de \ovalnum{40}}
%\blockmove{s'orienter \`a \ovalnum{90}}
%\blockpen{relever le stylo}
%\end{scratch}}&L'instruction 
%
%\small{\begin{scratch}\blockmove{s'orienter à \ovalnum{90}}\end{scratch}}
%
%signifie qu'on s'oriente en vue
%
% de se diriger vers la droite.\\ \hline
%\multicolumn{3}{|c|}{\textbf{Frise obtenue avec le script}}\\
%\multicolumn{3}{|c|}{\psset{unit=0.5cm}\def\zig{\psline(0,0)(0.4,0)(0.8,-0.7)(1.2,0)} 
%\begin{pspicture}(-5,-1)(8,0.5)
%\rput(-2.5,-0.5){Point de départ}
%\rput(1,0){\zig}\rput(3,0){\zig}\rput(5,0){\zig}\rput(7,0){\zig}\psline{->}(0,-0.5)(1,0)\end{pspicture}}\\ \hline
%\end{tabularx}
%\end{center}
%\medskip

\begin{enumerate}
\item %Quelle distance le lutin a-t-il parcourue pour tracer \textbf{un seul motif} de la frise?
Pour un motif le lutin parcourt : 

$20 + 40 + 40 = 100$~pixels.
\item %On modifie le programme, dans cette question seulement :

%\setlength\parindent{6mm}
%\begin{itemize}[label=$\bullet~~$]
%\item on ne modifie pas le script de la frise.
%\item dans le bloc motif, il enlève l'instruction : \small{\begin{scratch} \blockpen{relever le stylo} \end{scratch}}
%\end{itemize}
%\setlength\parindent{0mm}

%Dessiner à main levée la frise obtenue avec ce nouveau programme.
On obtient le dessin continu suivant :

\psset{unit=0.5cm}\def\zig{\psline(0,0)(0.4,0)(0.8,-0.7)(1.2,0)} 
\begin{pspicture}(-5,-1)(10,0.5)
%\psgrid
%\rput(-2.5,-0.5){Point de départ}
\rput(1,0){\zig}\psline[linecolor=red](2.2,0)(3,0)\rput(3,0){\zig}\psline[linecolor=red](4.2,0)(5,0)\rput(5,0){\zig}\psline[linecolor=red](6.2,0)(7,0)\rput(7,0){\zig}\psline[linecolor=red](8.2,0)(9,0)\end{pspicture}

\item %On utilise maintenant le bloc motif ci-dessous. Laquelle des deux frises obtient-il ? Expliquer pourquoi.
On obtient la frise \no 2 : avec le nouveau motif à la fin de son exécution on ne change pas l'orientation contrairement à la frise \no 1.

%\begin{center}
%\begin{tabularx}{\linewidth}{|*{3}{>{\centering \arraybackslash}X|}}\hline
%\textbf{Bloc motif modifié}&\textbf{Frise \no 1}&\textbf{Frise \no 2}\\
%\psset{unit=1cm}
%\begin{pspicture}(4,6.5)
%\rput(2,3.1){
%\small{\begin{scratch}
%\initmoreblocks{définir \namemoreblocks{Motif}}
%\blockpen{stylo en position d'écriture}
%\blockmove{avancer de \ovalnum{20}}
%\blockmove{tourner \turnright{} de \ovalnum{60} degrés}
%\blockmove{avancer de \ovalnum{40}}
%\blockmove{tourner \turnleft{} de \ovalnum{120} degrés}
%\blockmove{avancer de \ovalnum{40}}
%\blockpen{relever le stylo}
%\end{scratch}}}\end{pspicture}&\psset{unit=0.58cm}\def\zig{\psline(0,0)(0.4,0)(0.8,-0.6928)(1.2,0)}
%\psset{unit=0.58cm}\begin{pspicture}(1.3,-4)(5.5,3.4)\rput(1.5,0.5){\zig}\rput(3,1){\zig}\rput(4.5,1.5){\zig}\rput(6,2){\zig}\rput(1.5,2){Point de départ}\psline{->}(1.5,1.8)(1.5,0.45)\end{pspicture}&\psset{unit=0.6cm}\def\zig{\psline(0,0)(0.4,0)(0.8,-0.6928)(1.2,0)}
%\begin{pspicture}(-2,-3)(4,5.2)
%\rput(0,0){\zig}\rput{60}(1.5,0.52){\zig} \rput{120}(1.8,2.1){\zig}\rput{180}(0.6,3.15){\zig} \rput(0,-1.5){Point de départ}\psline{->}(0,-1.2)(0,0)\end{pspicture}\\ \hline
%\end{tabularx}
%\end{center}
\end{enumerate}
