
\medskip

%Le tableau ci-dessous regroupe les résultats de la finale du $200$~m hommes des Jeux
%Olympiques de Rio de Janeiro en 2016, remporté par Usain Bolt en $19,78$ secondes.
%
%\begin{center}
%\begin{tabularx}{\linewidth}{|c|*{3}{>{\centering \arraybackslash}X|}}\hline
%\textbf{Rang}& \textbf{Athlète}& \textbf{Nation}& \textbf{Performance en seconde}\\ \hline
%1 &U. Bolt& Jamaïque& 19,78\\ \hline
%2 &A. De Grasse& Canada& 20,02\\ \hline
%3 &C. Lemaitre& France& 20,12\\ \hline
%4 &A. Gemili& Grande-Bretagne& 20,12\\ \hline
%5 &C. Martina& Hollande& 20,13\\ \hline
%6 &L. Merritt& USA& 20,19\\ \hline
%7 &A. Edward& Panama& 20,23\\ \hline
%8 &R. Guliyev& Turquie& 20,43\\ \hline
%\end{tabularx}
%\end{center}
%
%\medskip

\begin{enumerate}
\item %Calculer la vitesse moyenne en m/s de l'athlète le plus rapide. Arrondir au centième.
Usain Bolt a parcouru 200~m en 19,78~s, soit $\dfrac{200}{19,78}$ mètres par seconde donc $ \approx 10,11$~m/s (au centième près).
\item %Calculer la moyenne des performances des athlètes. Arrondir au centième.
Le temps moyen pour les huit finalistes est :

$\dfrac{19,78 + 20,02 + \ldots + 20,43}{8} = \dfrac{161,02}{8} = \np{20,1275}$, soit 20,13~s au centième près.
\item %En 1964 à Tokyo, la moyenne des performances des athlètes sur le $200$m hommes était de $20,68$~s et l'étendue était de $0,6$ s. En comparant ces résultats à ceux de 2016, qu'observe-t-on ? 
En 2016, l'étendue des performances est de 0,65~s et la moyenne de 20,13~s : donc les étendues sont sensiblement les mêmes mais la moyenne  a baissé de 0,55~s.
\end{enumerate}

\bigskip

