
\medskip

Voici un tableau (document 1) concernant les voitures particulières \og diesel ou essence \fg{} en circulation en France en 2014.

\begin{center} \textbf{Document 1}\end{center}

\begin{tabularx}{\linewidth}{|c|*{2}{>{\centering \arraybackslash}X|}}\hline
		&Nombre de voitures en circulation (en milliers)& Parcours moyen annuel
 (en km/véhicule)\\ \hline
Diesel 	&\np{19741} &\np{15430}\\ \hline
Essence &\np{11984} &\np{8344}\\ \hline
\multicolumn{3}{r}{\small \emph{Source : INSEE}}\\
\end{tabularx}


\begin{enumerate}
\item Vérifier qu'il y avait \np{31725000} voitures \og \emph{diesel ou essence}\fg{} en circulation en France en 2014.
\item Quelle est la proportion de voitures \emph{essence} parmi les voitures \og \emph{diesel ou essence} \fg{} en circulation en France en 2014 ?

Exprimer cette proportion sous forme de pourcentage. 

\emph{On arrondira le résultat à l'unité}.
\item  Fin décembre 2014, au cours d'un jeu télévisé, on a tiré au sort une voiture parmi les voitures \og \emph{diesel ou essence}\fg{} en circulation en France. On a proposé alors au propriétaire de la voiture tirée au sort de l'échanger contre un véhicule électrique neuf.

Le présentateur a téléphoné à Hugo, l'heureux propriétaire de la voiture tirée au sort.

Voici un extrait du dialogue (\textbf{document 2}) entre le présentateur et Hugo:

\begin{center} \textbf{Document 2}\end{center}

\begin{tabularx}{\linewidth}{|X|}\hline
\textbf{Le présentateur} : \og Bonjour Hugo, quel âge a votre voiture? \fg,\\
\textbf{Hugo} : \og Là, elle a 7 ans! \fg.\\
\textbf{Le présentateur} : \og Et combien a-t-elle de kilomètres au compteur ? \fg,\\
\textbf{Hugo} : \og Un peu plus de \np{100000}~km. Attendez, j'ai une facture du garage qui date d'hier \ldots elle a exactement \np{103824}~km \fg,\\
\textbf{Le présentateur}: \og Ah! Vous avez donc un véhicule diesel je pense ! \fg\\\hline
\end{tabularx}

\medskip

À l'aide des données contenues dans le \textbf{document 1} et dans le \textbf{document 2} :

	\begin{enumerate}
		\item Expliquer pourquoi le présentateur pense que Hugo a un véhicule \emph{diesel}.
		\item Expliquer s'il est possible que la voiture de Hugo soit un véhicule \emph{essence}.
	\end{enumerate}
 \end{enumerate}

\bigskip

