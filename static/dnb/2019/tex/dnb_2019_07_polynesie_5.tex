
\medskip

\parbox{0.5\linewidth}{Lorsqu'un voilier est face au vent, il ne peut pas avancer.

Si la destination choisie nécessite de prendre une direction face
au vent, le voilier devra progresser en faisant des zigzags.

Comparer les trajectoires de ces deux voiliers en calculant la
distance, en kilomètres et arrondie au dixième que chacun a
parcourue.}\hfill
\parbox{0.38\linewidth}{

\psset{unit=0.9cm}
\begin{pspicture}(0,0.3)(6,8)
\def\voilier{\psline[unit=0.5cm](0,-0.4)(0,0)(0,0.4)\pscurve[unit=0.5cm](0,0.4)(0.8,0.3)(1.2,0)(0.4,-0.3)(0,-0.4)
\pscurve[linewidth=2pt,unit=0.5cm](1.1,0)(0.6,0.2)(0.1,0.1)}
%\psgrid
\psdots(2.5,1.5)(2.5,7.3)
\uput[r](2.5,7.3){A}\uput[r](2.5,1.5){C}
\uput[l](0.4,6.4){D}\uput[r](5,3){B}
\pspolygon(2.5,7.3)(5,3)(2.5,1.5)(0.4,6.4)
\rput{-64}(0.4,6.4){\psframe*(0.3,0.3)}
\rput{120}(5,3){\psframe*(0.3,0.3)}
\psline[linestyle=dashed](2.5,4.2)(2.5,1.5)
\psline[linestyle=dashed](2.5,4.5)(2.5,7.3)
\rput(4,6){4,8~km}\rput(2.5,4.35){5,6~km}
\psarc(2.5,1.5){0.5}{90}{115}
\rput(2.3,2.5){\scriptsize 24\degres}
\rput(2.5,1){\textbf{Départ}}\rput(2.5,7.8){\textbf{Arrivée}}
\rput(0.8,7.8){Sens du vent}\psline[linewidth=1.8pt]{->}(0.8,7.6)(0.8,6.8)
\rput{41}(3.4,2){\voilier}\rput{110}(1.8,3){\voilier}
\rput(4.3,1.8){voilier 1}\rput(0.8,3.2){voilier 2}
\rput(3,0.3){La figure n'est pas à l'échelle}
\end{pspicture}
}

\vspace{0.5cm}

