
\medskip

Pour fêter son anniversaire, Pascale a acheté à la boutique deux boîtes de macarons.

La boîte \textbf{numéro 1} est composée de : 4 macarons chocolat, 3 macarons café, 2 macarons vanille
et 3 macarons caramel.

La boîte \textbf{numéro 2} est composée de : 2 macarons chocolat, 1 macaron fraise, 1 macaron
framboise et 2 macarons vanille.

On suppose dans la suite que les macarons sont indiscernables au toucher.

\medskip

\begin{enumerate}
\item Si on choisit au hasard un macaron dans la boîte numéro 1, quelle est la probabilité que ce soit
un macaron au café ?
\item Au bout d'une heure il reste 3 macarons chocolat et 2 macarons café dans la boîte numéro 1
et 2 macarons chocolat et 1 macaron fraise dans la boîte numéro 2.

Carole n'aime pas le chocolat mais apprécie tous les autres parfums. Si elle choisit un macaron
au hasard dans la boîte numéro 1, puis un second dans la boîte numéro 2, quelle est la probabilité
qu'elle obtienne deux macarons qui lui plaisent ?
\end{enumerate}

\bigskip

