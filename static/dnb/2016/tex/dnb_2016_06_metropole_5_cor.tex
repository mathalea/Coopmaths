
%\medskip 
%
%\parbox{0.55\linewidth}{La figure PRC ci-contre représente un terrain appartenant à une commune. 
%
%Les points P{}, A et R sont alignés. 
%
%Les points P{}, S et C sont alignés. 
%
%Il est prévu d'aménager sur ce terrain: 
%
%\begin{itemize}
%\item[$\bullet~~$] une \og zone de jeux pour enfants\fg{} sur la partie PAS ; 
%\item[$\bullet~~$] un \og skatepark \fg{} sur la partie RASC. 
%\end{itemize}
%
%On connaît les dimensions suivantes : 
%
%PA = 30~m ; AR = 10~m ; AS = 18~m.}\hfill \parbox{0.45\linewidth}{\psset{unit=0.9cm}
%\begin{pspicture}(6,8)
%%\psgrid
%\pspolygon(1.5,0.5)(5,1.2)(0.5,7)%RCP
%\psline(1.2,2.2)(3.9,2.7)
%\rput{10}(1.5,0.5){\psframe(0.3,0.3)}
%\rput{10}(1.22,2.2){\psframe(0.3,0.3)}
%\uput[u](3.5,6){zone de jeux pour enfants}
%\psline{->}(3.5,6)(2,4)
%\uput[u](5.1,2.3){skatepark}\psline{->}(5,2.4)(3.8,1.8)
%\uput[l](1.5,0.5){R} \uput[dr](5,1.2){C} \uput[u](0.5,7){P} \uput[l](1.2,2.2){A} \uput[ur](3.9,2.7){S}
%\pspolygon[fillstyle=hlines](1.22,2.2)(3.86,2.7)(5,1.2)(1.5,0.5) 
%\end{pspicture}
%}

\begin{enumerate}
\item %La commune souhaite semer du gazon sur la \og zone de jeux pour enfants\fg. Elle décide d'acheter des sacs de $5$~kg de mélange de graines pour gazon à 13,90~\euro{} l'unité. Chaque sac permet de couvrir une surface d'environ 140~m$^2$. 

%Quel budget doit prévoir cette commune pour pouvoir semer du gazon sur la totalité de la \og zone de jeux pour enfants\fg{} ?
L'aire du  triangle PAS rectangle en A est égale à :

$\dfrac{\text{PA} \times  \text{AS}}{2}$, soit $\dfrac{30 \times 18}{2} = 270$~~m$^2$.

Il faut donc acheter deux sacs de gazon (car $2 \times 140 = 280 > 270$) à 13,90~\euro{} l'un soit une dépense de $2\times 13,90 = 27,80$~\euro.
\item %Calculer l'aire du \og skatepark \fg.
Les droites (AS) et (RC) sont perpendiculaires à (PA) : elles sont donc parallèles. On peut donc appliquer la propriété de Thalès et par exemple :

$\dfrac{\text{PA}}{\text{PR}} =  \dfrac{\text{AS}}{\text{RC}}$ soit $\dfrac{30}{30+10} =  \dfrac{18}{\text{RC}}$ ou $\dfrac{3}{4} = \dfrac{18}{\text{RC}}$ soit $3\text{RC} = 4 \times 18$ ou $\text{RC} = 4 \times 6 = 24$~(m).

L'aire du triangle PRC est donc égale à :

$\dfrac{\text{PR}\times \text{RC}}{2} = \dfrac{40 \times 24}{2} = 40 \times 12 = 480$~m$^2$.

L'aire du \og skatepark \fg{} est donc égale à : $480 - 270 = 210$~m$^2$.
\end{enumerate}

\bigskip

