
\medskip

Une entreprise de fabrication de bonbons souhaite vérifier la qualité de sa nouvelle
machine de conditionnement. Cette machine est configurée pour emballer environ $60$
bonbons par paquet. Pour vérifier sa bonne configuration, on a étudié $500$~paquets à
la sortie de cette machine.

\medskip

\textbf{Document 1 : Résultats de l'étude}

\begin{center}
\begin{tabularx}{\linewidth}{|m{2cm}|*{9}{>{\centering \arraybackslash}X|}}\hline
Nombre de bonbons	&56 &57 &58 &59 &60 	&61 &62 &63 &64\\ \hline
Effectifs 			&4 	&36 &53 &79 &145 	&82 &56 &38 &7\\ \hline
\end{tabularx}
\end{center}

\textbf{Document 2 : Critères de qualité}

\medskip

Pour être validée par l'entreprise, la machine doit respecter trois critères de qualité:

\setlength\parindent{8mm}
\begin{itemize}
\item[$\bullet~~$] Le nombre moyen de bonbons dans un paquet doit être compris entre 59,9 et
60,1.
\item[$\bullet~~$] L'étendue de la série doit être inférieure ou égale à $10$.
\item[$\bullet~~$] L'écart interquartile (c'est-à-dire la différence entre le troisième quartile et le premier quartile) doit être inférieur ou égal à 3.
\end{itemize}
\setlength\parindent{0mm} 
 
La nouvelle machine respecte-t-elle les critères de qualité ?
 
\emph{Il est rappelé que, pour l'ensemble du sujet, les réponses doivent être justifiées.}

\bigskip

