
\medskip

%Romane souhaite préparer un cocktail pour son anniversaire.
%
%\begin{center}
%\begin{tabularx}{\linewidth}{|*{2}{>{\centering \arraybackslash}X|}}\hline
%Document 1 : Recette du cocktail
%
%Ingrédients pour 6 personnes :&Document 2 : Récipient de Romane\\ 
%
%\begin{pspicture}(5,4)
%\uput[r](0,3.75){$\bullet~~$60 cl de jus de mangue}
%\uput[r](0,3.25){$\bullet~~$30 cl de jus de poire}
%\uput[r](0,2.75){$\bullet~~$12 cl de jus de citron vert}
%\uput[r](0,2.25){$\bullet~~$12 cl de sirop de cassis}
%\end{pspicture}&\psset{unit=1cm}
%\begin{pspicture}(5,4)
%\psellipse(2.5,3)(2,0.4)
%\psarc(2.5,3){2cm}{-180}{0}
%\end{pspicture}\\
%Préparation : &\\
%Verser les différents ingrédients dans un récipient et remuer.
%
%Garder au frais pendant au moins 4~h.&On considère qu'il a la forme d'une
%demi-sphère de diamètre 26 cm.\\ \hline
%\end{tabularx}
%\end{center}
%\emph{Rappels :}
%
%$\bullet~~$Volume d'une sphère : $V = \dfrac{4}{3}\pi r^3$
%
%$\bullet~~$1~L = 1~dm$^3$ = \np{1000}~cm$^3$
%
%\medskip
%
%Le récipient choisi par Romane est-il assez grand pour préparer le cocktail pour 20
%personnes ?
Pour un cocktail le volume des ingrédients est égal à :

$\dfrac{60 + 30 + 12 + 12}{6} = \dfrac{114}{6} = 19$~cl.

Donc pour 20 cocktails le volume est égal à :

$20 \times 19 = \np{380}$~cl soit 3,8~l.

Le volume du récipient de Romane est égal à :

$\dfrac{1}{2} \times \dfrac{4}{3}\times \pi \times 13^3 \approx \np{4601}~$cm$^3$ soit environ 4,601~dm$^3$ ou 4,6~l : le récipient est assez grand pour préparer tous les cocktails.
\medskip

\textbf{Il est rappelé que, pour l'ensemble du sujet, les réponses doivent être justifiées.\\
Il est rappelé que toute trace de recherche sera prise en compte dans la
correction.}
