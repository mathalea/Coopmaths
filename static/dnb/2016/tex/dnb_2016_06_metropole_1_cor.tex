
\medskip

%Une société commercialise des composants électroniques qu'elle fabrique dans deux usines. Lors d'un contrôle de qualité, 500 composants sont prélevés dans chaque usine et sont examinés pour déterminer s'ils sont \og bons\fg{} ou \og défectueux \fg. 
%
%Résultats obtenus pour l'ensemble des \np{1000} composants prélevés : 
%
%\begin{center}
%\begin{tabularx}{0.65\linewidth}{|*{3}{>{\centering \arraybackslash}X|}}\cline{2-3}
%\multicolumn{1}{c|}{~}	&Usine A	&Usine B\\ \hline     
%Bons   					&473  		&462 \\ \hline   
%Défectueux   			&27   		&38 \\ \hline
%\end{tabularx}
%\end{center}
   
\begin{enumerate}
\item %Si on prélève un composant au hasard parmi ceux provenant de l'usine A, quelle est la probabilité qu'il soit défectueux ?
Il y a 27 composants défectueux sur 500 ; la probabilité est donc égale à $\dfrac{27}{500} = \dfrac{54}{\np{1000}} = 0,054 = \dfrac{5,4}{100} = 5,4\,\%$. 
\item %Si on prélève un composant au hasard parmi ceux qui sont défectueux, quelle est la probabilité qu'il provienne de l'usine A ? 
Sur les $27 + 38 = 65$ composants défectueux, 27 proviennent de l'usine A.

La probabilité qu'il provienne de l'usine A est donc égale à $\dfrac{27}{65} \approx 0,415$ ou 41,5\,\%.
\item %Le contrôle est jugé satisfaisant si le pourcentage de composants défectueux est inférieur à 7\,\% dans chaque usine. Ce contrôle est-il satisfaisant ?
Dans l'usine A la proportion de composants défectueux est de $5,4 < 7\,\%$.

Dans l'usine B  la proportion de composants défectueux est de $\dfrac{38}{500} = \dfrac{76}{\np{1000}} = \dfrac{7,6}{100} = 7,6\,\%$ donc supérieur à 7\,\%.

Conclusion : le contrôle n'est pas satisfaisant.
\end{enumerate}

\bigskip

