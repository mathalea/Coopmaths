
\medskip

\begin{enumerate}
\item En I2, il faut saisir la formule : = SOMME(B2:H2)
\item  $m =  \dfrac{324+240+310+204+318+386+468}{7} = \dfrac{\np{2250}}{7} \approx 321$.

Le nombre moyen de macarons vendus par jour est d'environ 321.
\item  Je range les valeurs correspondantes au nombre de macarons vendus dans l'ordre
croissant : 204 \quad 240\quad  310\quad  318\quad  324\quad  386\quad  468

L'effectif est 7 (impair) et $\dfrac{7 + 1}{2} = 4$, la médiane est la 4\up{e} valeur de la série ordonnée, c'est-à-dire 318.  Le nombre médian de macarons est donc de 318.
\item  $468 - 204 = 264$.

La différence entre le nombre de macarons vendus le dimanche et ceux vendus le jeudi
est 264, cette valeur correspond à l'étendue de la série.
\end{enumerate}

\vspace{0,5cm}

