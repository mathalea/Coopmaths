
\medskip

\begin{enumerate}
\item Si on choisit au hasard un macaron dans la boîte \no 1, la probabilité que ce soit un
macaron au café est $\dfrac{3}{12} = \dfrac{1}{4}$.
\item~

\begin{center}
\pstree[treemode=R,nodesep=8pt,levelsep=3.5cm]{\TR{}}
{\pstree{\TR{\footnotesize Chocolat}\taput{$\frac{3}{5}$}}
	{\TR{\footnotesize Chocolat}\taput{$\frac{2}{3}$}
	\TR{\footnotesize Fraise}\tbput{$\frac{1}{3}$}
	}
\pstree{\TR{\footnotesize Café}\tbput{$\frac{2}{5}$}}
	{\TR{\footnotesize Chocolat}\taput{$\frac{2}{3}$}
	\TR{\footnotesize Fraise}\tbput{$\frac{1}{3}$}
	}
}
\end{center} 

Pour obtenir deux macarons qui lui plaisent, Carole doit choisir un macaron au café dans
la boîte \no 1 et un macaron à la fraise dans la boîte \no 2.

Je calcule : $\dfrac{2}{5} \times \dfrac{1}{3} = \dfrac{2}{15}$.

La probabilité que Carole obtienne deux macarons qui lui plaisent est donc de $\dfrac{2}{15}$.
\end{enumerate}
 
\vspace{0,5cm}

