
\medskip

%Un confiseur lance la fabrication de bonbons au chocolat et de bonbons au
%caramel pour remplir 50 boîtes. Chaque boîte contient 10 bonbons au chocolat et 8 bonbons au caramel.
%
%\medskip

\begin{enumerate}
\item %Combien doit-il fabriquer de bonbons de chaque sorte?
Le confiseur doit fabriquer $50\times 10 = 500$~bonbons au chocolat et $50 \times 8 = 400$~bonbons au caramel.
\item %Jules prend au hasard un bonbon dans une boite. Quelle est la probabilité qu'il
%obtienne un bonbon au chocolat ?
Dans une boîte il y a 10 bonbons au chocolat sur 18 bonbons. La probabilité est donc égale à $\dfrac{10}{18} \approx 0,56$.
\item %Jim ouvre une autre boîte et mange un bonbon. Gourmand, il en prend sans regarder un
%deuxième. Est-il plus probable qu'il prenne alors un bonbon au chocolat ou un bonbon au caramel ?
$\bullet~~$S'il a pris un bonbon au chocolat, il reste 9 bonbons au chocolat et 8 au caramel.

$\bullet~~$S'il a pris un bonbon au caramel, il reste 10 bonbons au chocolat et 7 au caramel.

Dans chaque cas il reste plus de bonbons au chocolat que de bonbons au caramel : le second bonbon a plus de chances d'être un bonbon au chocolat qu'un bonbon au caramel.
\item %Lors de la fabrication, certaines étapes se passent mal et, au final, le confiseur a 473 bonbons
%au chocolat et 387 bonbons au caramel.
	\begin{enumerate}
		\item %Peut-il encore constituer des boîtes contenant 10 bonbons au chocolat et 8 bonbons au
%caramel en utilisant tous les bonbons? Justifier votre réponse.
Avec 473 bonbons au chocolat il peut faire 47 boîtes de 10 bonbons et avec 387 bonbons au caramel 48 boîtes.

Il peut donc faire 47 boîtes de 10 bonbons au chocolat et 8 bonbons au caramel. Il lui restera 3 bonbons au chocolat et 11 bonbons au caramel.
		\item %Le confiseur décide de changer la composition de ses boîtes. Son objectif est de faire le plus de boîtes identiques possibles en utilisant tous ses bonbons. Combien peut-il faire de boîtes ?
%Quelle est la composition de chaque boîte ?
On a $473 = 430 + 43 = 43\times 10 + 43\times 1 = 43 \times (10 + 1) = 43 \times 11$.

387 est un multiple de 9 car $3 + 8 + 7 = 18$ l'est aussi. $387 = 9 \times 43 = 43 \times 9$.

On peut donc faire en utilisant tous les bonbons 43 boîtes contenant chacune 11 bonbons au chocolat et 9 bonbons au caramel.
	\end{enumerate}
\end{enumerate}

\vspace{0,5cm}

