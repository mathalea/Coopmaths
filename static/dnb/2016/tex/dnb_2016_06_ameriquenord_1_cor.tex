
\medskip
		
\textbf{Affirmation 1 :} La solution de l'équation $5x + 4 = 2x + 17$ est un nombre entier.

$5x + 4 = 2x + 17$
 
$5x - 2x + 4 = 2x -2x + 17$
 
$3x + 4 = 17$

$3x + 4 - 4 = 17 - 4$

$3x = 13$

$\dfrac{3x}{3}  = \dfrac{13}{3}$

$x = \dfrac{13}{3}$.

13 n'étant pas un multiple de 3, $\dfrac{13}{3}$ n'est pas un nombre entier.

L'affirmation est fausse.

\textbf{Affirmation 2 :} Le triangle CDE est rectangle en C.

Dans le triangle CDE, [DE] est le  côté de plus grande longueur.

Je calcule séparément : 

D'une part, DE$^2 = \left(13\sqrt{7}\right)^2 = 13^2 \times \left(\sqrt{7}\right)^2 = 169 \times  7 = \np{1183}$.

D'autre part, $\text{DC}^2 + \text{CE}^2 = \left(\sqrt{175}\right)^2 + \left(12\sqrt{7}\right)^2 = 175 + 12^2  \times 7 = 175  + 144 \times 7 = 175 + \np{1008} = \np{1183}$.

Je constate que : $\text{DE}^2 = \text{DC}^2 + \text{CE}^2$.

D'après la réciproque du théorème de Pythagore, le triangle CDE est rectangle en C.

L'affirmation est vraie.		

\textbf{Affirmation 3 :} Manu affirme que, sur ces étiquettes, le pourcentage de réduction sur la
montre est supérieur à celui pratique sur les lunettes.

\emph{Méthode }1 :

$45 - 31,50 = 13,50$. Le montant de la réduction sur les lunettes est de $13,50$~\euro.

$\dfrac{13,50}{45} \times 100 = 30$. Le pourcentage de réduction sur les lunettes est de $30$\,\%.

$56 - 42 =14$. Le montant de la réduction sur la montre est de $14$~\euro.

$\dfrac{14}{56} \times 100 = 25$. Le pourcentage de réduction sur la montre est de $25\,\%$.

Le pourcentage de réduction sur les lunettes est supérieur à celui sur la montre.

L'affirmation est fausse.

\emph{Méthode } 2 :

$\dfrac{31,50}{45} = 0,7 = 1 - 0,3 = 1 - \dfrac{30}{100}$ ; \quad $\dfrac{42}{56} = 0,75 = 1 - 0,25 = 1 - \dfrac{25}{100}$.

Le pourcentage de réduction sur les lunettes est de 30\,\% et est supérieur à celui sur les
lunettes qui est de 25\,\%.

L'affirmation est fausse.

\vspace{0,5cm}

