
\medskip

%Avec des ficelles de 20~cm, on construit des polygones comme ci-dessous :
%\medskip
%
%\begin{center} 
%
%\textbf{Méthode de construction des polygones}
%
%\begin{tabularx}{\linewidth}{|l|m{6cm}|X|}\hline 
%Étape 1&\psset{unit=1cm}\begin{pspicture}(6,1)%\psgrid
%\psline(0,0.5)(6,0.5)\psline(1.5,0.8)(1.7,0.4)\rput{-65}(1.78,0.25){\psellipse(0,0)(0.16,0.1)}
%\psline(1.8,0.8)(1.55,0.4)\rput{-125}(1.45,0.25){\psellipse(0,0)(0.16,0.1)} \end{pspicture}&On coupe la ficelle de 20~cm en deux   morceaux.\\ \hline   
%Étape 2&\psset{unit=1cm}\begin{pspicture}(6,1.5)%\psgrid
%\psline(0,0.5)(1.6,0.5)\psline(1.8,0.5)(6,0.5)  \rput(0.8,1){morceau \no 1}
%   \rput(4,1){morceau \no 2}\end{pspicture}   &On sépare les deux morceaux.\\ \hline   
%Étape 3&\psset{unit=0.65cm}\begin{pspicture}(6,4.5)%\psgrid
%\psframe(0.5,1)(1.5,2) \pspolygon(4,0.5)(7,0.5)(5.5,3.398)
%\end{pspicture}
%&$\bullet~~$Avec le \og morceau \no 1 \fg,  on construit un carré.
%   
%$\bullet~~$Avec le \og morceau \no 2 \fg,  on construit un triangle équilatéral.\\ \hline   
%\end{tabularx}
%\end{center}
%
%\textbf{Partie 1 :}
%
%\smallskip 
%
%Dans cette partie, on découpe à l'étape 1 une ficelle pour que le \og morceau \no 1 \fg{} mesure 8~cm.
%
%\medskip 

\begin{enumerate}
\item %Dessiner en grandeur réelle les deux polygones obtenus.
Avec le morceau \no 1, on construit un carré de côté $c$, donc $8 = 4c$ soit $c = 2$~(cm).

Avec le morceau \no 2 de longueur $20 - 8 = 2$, on construit un triangle équilatéral de côté $d$ tel que $3d = 12$, soit $d = 4$~(cm). D'où la construction :

\begin{center}
\psset{unit=1cm}
\begin{pspicture}(9,5)
\psframe(1,1)(3,3) \pspolygon(4,0.5)(8,0.5)(6,3.9641)
\end{pspicture}
\end{center}

\item %Calculer l'aire du carré obtenu.
L'aire du carré est égale à $c^2 = 2^2 = 4$~cm$^2$. 
\item %Estimer l'aire du triangle équilatéral obtenu en mesurant sur le dessin.
Les hauteurs du triangle équilatéral mesurent environ 3,4~cm (au mm près).

L'aire de ce triangle est donc à peu près $\dfrac{4 \times 3,4}{2} = 4 \times 1,7 = 6,8$~cm$^2$. 
\end{enumerate}

\medskip

\textbf{Partie 2 :}

\medskip 

%Dans cette partie, on cherche maintenant à étudier l'aire des deux polygones obtenus à l'étape 3 en fonction de la longueur du \og morceau \no 1 \fg. 
%
%\medskip

\begin{enumerate}
\item %Proposer une formule qui permet de calculer l'aire du carré en fonction de la longueur du  \og morceau \no 1 \fg.
Si $\ell$ est la longueur  \og morceau \no 1 \fg{}, le côté du carré a pour longueur $\dfrac{\ell}{4}$ et par conséquent l'aire du carré est $\mathcal{A}_{\text{carré}} = \left(\dfrac{\ell}{4} \right)^2 = \dfrac{\ell^2}{16}$. 
\item %Sur le graphique ci-dessous: 

%\setlength\parindent{8mm}
%\begin{itemize}
%\item[$\bullet~~$] la courbe A représente la fonction qui donne l'aire du carré en fonction de la longueur du \og morceau \no 1 \fg{} ; 
%\item[$\bullet~~$]la courbe B représente la fonction qui donne l'aire du triangle équilatéral en fonction de la longueur du \og morceau \no 1 \fg. 
%\end{itemize}
%\setlength\parindent{0mm}
%
\textbf{Graphique représentant les aires des polygones en fonction de la longueur du \og morceau \no 1 \fg }

\begin{center}
\psset{unit=0.4cm}
\begin{pspicture}(-1.5,-2)(21,26)
\multido{\n=0+2}{11}{\psline[linewidth=0.3pt,linecolor=cyan](\n,0)(\n,26)}
\multido{\n=0+2}{14}{\psline[linewidth=0.3pt,linecolor=cyan](0,\n)(21,\n)}
\psaxes[linewidth=1.25pt,Dx=2,Dy=2](0,0)(0,0)(21,26)
\uput[d](15.6,-1){Longueur du \og morceau \no 1 \fg{} (en cm)}
\uput[r](0,25.5){Aire $\left(\text{en cm}^2\right)$}
\psplot[plotpoints=4000,linewidth=1.25pt,linecolor=blue]{0}{20}{x 4 div dup mul}
\psplot[plotpoints=4000,linewidth=1.25pt]{0}{20}{20 x sub 3 div dup mul 0.866025 mul 2 div}
\rput(18.2,16.5){\blue Courbe A}
\rput(3.7,16.5){Courbe B}
\psset{arrowsize=3pt 4}
\psline{->}(0,14)(2.95,14)
\psline{->}(2.95,14)(2.95,0)
\psline{->}(9.36,0)(9.36,5.65) 
\end{pspicture}
\end{center}
% 
%En utilisant ce graphique, répondre aux questions suivantes. Aucune justification n'est attendue. 
	\begin{enumerate}
		\item %Quelle est la longueur du \og morceau \no 1 \fg{} qui permet d'obtenir un triangle équilatéral d'aire 14~cm$^2$ ?
		On trace l'horizontale partant du point de coordonnées (0~;~14) qui coupe la courbe B en un point dont l'abscisse est obtenue en projetant ce point sur l'axe des abscisses (voir la figure) ; on lit environ 2,95~cm. 
		\item %Quelle est la longueur du \og morceau \no 1 \fg{} qui permet d'obtenir deux polygones d'aires égales ?
Le point commun aux deux courbes a pour ordonnée l'aire commune aux deux polygones (environ 5,5) ; l'abscisse de ce point est environ 9,4~(cm). 
	\end{enumerate}
\end{enumerate}

\bigskip

