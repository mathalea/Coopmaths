
\medskip

On considère les deux programmes de calcul ci-dessous. 

%\begin{center}
%\begin{tabularx}{0.75\linewidth}{|X|m{0.4cm}|X|}\cline{1-1}\cline{3-3}
%\multicolumn{1}{|c|}{\textbf{Programme A}}&&\multicolumn{1}{|c|}{\textbf{Programme B}}\\\cline{1-1}\cline{3-3}  
%1. Choisir un nombre.			&&1. Choisir un nombre. \\ 
%2. Multiplier par $-2$.			&&2. Soustraire 7.  \\
%3. Ajouter 13.					&&3.  Multiplier par 3. \\\cline{1-1}\cline{3-3}
%\end{tabularx}
%\end{center} 

\begin{enumerate}
\item %Vérifier qu'en choisissant 2 au départ avec le programme A, on obtient 9.
Avec le programme A, on obtient :

$2 \to 2 \times (- 2) = - 4 \to - 4 + 13 = 9$. 
\item %Quel nombre faut-il choisir au départ avec le programme B pour obtenir 9 ? 
Avec le programme B :

$\bullet~~$Méthode 1 : en partant du nombre $x$ :

$x \to x - 7 \to (x - 7) \times 3 = 9$.

Il faut résoudre l'équation :

$3(x - 7) = 9$ ou $3(x - 7) = 3\times 3$, soit $x - 7 = 3$ et enfin $x = 10$.

$\bullet~~$Méthode 2 : on peut \og reculer \fg{} :

$9 \to \dfrac{9}{3} = 3 \to 3 + 7 = 10$.

Pour trouver le même résultat 9 avec le programme B il faut partir de 10. 
\item %Peut-on trouver un nombre pour lequel les deux programmes de calcul donnent le même résultat ?
Si on part de $a$ avec le programme A,on obtient la suite :

$a \to a \times (- 2) = - 2a \to - 2a + 13 = 13 - 2a$.

Si on part de $a$ avec le programme B, on obtient la suite :

$a \to a - 7  \to 3(a - 7)$.

Il faut donc résoudre l'équation :

$13 - 2a = 3(a - 7)$ soit $13 - 2a = 3a - 21$ ou $13 + 21 = 2a + 3a$ ou $34 = 5a$ ou  $\dfrac{1}{5}\times 34 = \dfrac{1}{5}\times 5a$ et enfin $\dfrac{34}{5} = a = 6,8$.

Dans les deux cas le résultat final est $- 0,6$.

Le nombre 6,8 donne avec les deux programmes le même résultat.
\end{enumerate} 

\bigskip

