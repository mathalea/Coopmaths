
\medskip

%Le tableau ci-dessous fournit le nombre d'exploitations agricoles en France, en fonction de leur
%surface pour les années 2000 et 2010.
%
%\begin{center}
%\begin{tabularx}{\linewidth}{|c|>{\footnotesize}c|*{3}{>{\centering \arraybackslash}X|}}\hline
%&	A	&B	&C	& D\\ \hline
%1&Surface de l'exploitation&\multicolumn{2}{|m{4cm}|}{Nombre d'exploitations agricoles (en milliers)}&\\ \hline
%2&								&En 2000 	&En 2010&\\ \hline
%3&Inférieure à 20 ha			&359		&235	&\\ \hline
%4&Comprise entre 20 et 50 ha 	&138 		&88		&\\ \hline
%5&Comprise entre 50 et 100 ha	&122		&98		&\\ \hline
%6&Comprise entre 100 et 200 ha	&64			&73		&\\ \hline
%7&Supérieure à 200 ha			&15			&21		&\\ \hline
%8& Total&&&\\ \hline
%9&&&&\\ \hline
%\end{tabularx}
%\end{center}

\begin{enumerate}
\item %Quelles sont les catégories d'exploitations qui ont vu leur nombre augmenter entre 2000 et 2010 ?
Seules les exploitations de plus de 100~ha ont vu leur nombre augmenter.
\item %Quelle formule doit-on saisir dans la cellule B8 pour obtenir le nombre total d'exploitations agricoles en 2000 ?
=SOMME(B3:B7)
\item %Si on étire cette formule, quel résultat s'affiche dans la cellule C8 ?
En C8 on aura $235 + 88 + 98 + 73 + 21 = 515$ (exploitations)
\item %Peut-on dire qu'entre 2000 et 2010 le nombre d'exploitations de plus de 200 ha a augmenté de
%40\,\% ? Justifier.
Le nombre est passé de 15 à 21 soit une augmentation de $\dfrac{21 - 15}{15} \times 100 = \dfrac{6}{15}\times 100 = \dfrac{2}{5} \times 100 = 40$~\%. L'affirmation est vraie.
\end{enumerate} 

\vspace{0,5cm}

