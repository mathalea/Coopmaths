
\medskip

%Mélanie est une étudiante toulousaine qui vit en colocation dans un appartement. Ses parents habitent à Albi et elle retourne chez eux les week-ends.
%
%Elle rentre à Toulouse le dimanche soir.
%
%Sur sa route, elle passe prendre ses 2 colocataires à la sortie \no 3, dernière sortie avant le péage.
%
%Elle suit la route indiquée par l'application GPS de son téléphone portable, dont l'affichage est
%reproduit ci-après.
%
%\begin{center}
%\begin{pspicture}(10,6)
%\def\aeroport{\psset{unit=0.4cm}\psframe(1,1)
%\psline[linewidth=1.8pt](0.35,0)(0.5,0.15)(0.65,0)(0.5,0.15)(0.5,1)\pspolygon*[fillstyle=solid](0.5,0.8)(0,0.3)(0.5,0.4)(1,0.3)(0.5,0.8)}
%%\psgrid
%\pspolygon[linewidth=1.8pt,fillstyle=solid,fillcolor=lightgray](0.3,0.9)(1.2,0.6)(1.5,1.1)(1.2,1.8)(1,2)(0.9,1.6)(0.7,1.45)(0.6,1.3)(0.3,1.1)
%\rput(1,0.2){\textbf{TOULOUSE}}
%\uput[r](4.4,2){Légende :}
%\uput[r](5,1.4){Sortie d'autoroute}
%\uput[r](5,0.85){kilomètres}
%\uput[r](5,0.2){Aéroport}
%\psline[linewidth=1.8pt](1.65,1)(1.7,1.4)(1.55,1.6)(1.4,1.8)(1.1,2.1)
%\psline[linewidth=1.8pt](1.4,1.8)(1.8,2)(1.9,2.2)(2,2.5)(2.4,3.)(3,3)(3.3,3.3)(3.8,3.7)(4.3,4)(4.7,4)(5.5,4.2)(6.5,4.5)(7.5,4.4)(8.6,5.2)(8.8,5)(9.2,5.05)(9.4,5.3)(9.3,5.7)
%\psline[linewidth=1.8pt](1.6,2.2)(2,2)
%\psdots[dotscale=1.4](1.7,1.2)(1.55,1.55)(1.4,1.8)(8.8,5)(9.1,5.05)(9.4,5.25)(9.3,5.6)
%\pscurve[linewidth=1.8pt,fillstyle=solid,fillcolor=lightgray](8.8,5.2)(9.1,5.25)(9.2,5.4)(9,5.6)(8.7,5.4)(8.8,5.2)
%\rput(8.4,5.7){\textbf{ALBI}}
%\rput(1.2,2.4){péage}
%%\rput(2.1,2.4){\small\ding{204}}
%\scalebox{0.5}{\pscircle*(4.2,4.8){0.3}\rput(4.2,4.8){\white \bf 3}}
%%\rput(3.35,3.2){\small\ding{207}}
%\scalebox{0.5}{\pscircle*(6.7,6.4){0.3}\rput(6.7,6.4){\white \bf 6}}
%%\rput(3.85,3.6){\small\ding{208}}
%\scalebox{0.5}{\pscircle*(7.7,7.2){0.3}\rput(7.7,7.2){\white \bf 7}}
%%\rput(5.8,4.4){\small\ding{210}}
%\scalebox{0.5}{\pscircle*(11.6,9){0.3}\rput(11.6,9){\white \bf 9}}
%\rput(1.8,1.8){7}\rput(2.7,2.8){13}\rput(3.3,3.7){6}
%\rput(4.6,3.8){16}\rput(7,4.6){16}
%\psframe(4.2,0)(8,2.4)
%\rput(0,1.5){\aeroport}
%\rput(4.6,0.1){\aeroport}
%\rput(4.7,1.4){\small\ding{203}}
%\rput(4.7,0.85){16}
%
%\scalebox{0.5}{\pscircle*(16.1,10.2){0.3}\rput(16.1,10.2){\white \bf 11}}
%
%\end{pspicture}
%\end{center}
%
%Elle est partie à 16 h 20 et entre sur l'autoroute au niveau de la sortie \no 11 à 16~h~33.
%
%Le rendez-vous est à 17 h.
%
%Sachant qu'il lui faut 3 minutes pour aller de la sortie \no 3 au lieu de rendez-vous, à quelle vitesse moyenne doit-elle rouler sur l'autoroute pour arriver à l'heure exacte ? Vous donnerez votre réponse
%en km/h.
%
%\textbf{Toute recherche même incomplète, sera valorisée dans la notation.}
Sur l'autoroute de la sortie 11 à la sortie 3 il y a $16 + 16 +  6 + 13 = 51$~km

Elle est entrée à la sortie 11 à 16 h 33 et doit être à la sortie 3 à 16 h 57.

Il lui faut donc parcourir 51~km en 24 minutes ou 17 kilomètres en 8 minutes ou 8,5 kilomètres en 4 minutes et enfin $15 \times 8,5$~km en $15 \times 4 = 60$~minutes soit 127,5~km/h.

\emph{Remarque:} la vitesse maximale étant de 130~km/h cette moyenne  de 127,5~km/h est pratiquement impossible à réaliser.
\vspace{0,5cm}

