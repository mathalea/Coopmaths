
\medskip

\begin{enumerate}
\item On sait que le triangle USO est rectangle en O. 

On a OS $ = 396 - 220 = 176$.

Pour calculer la valeur de l'angle $\widehat{\text{GUS}}$, on recourt à
la formule du sinus.

$\sin \widehat{\text{OUS}} = \dfrac{\text{côté opposé}}{\text{hypoténuse}} = \dfrac{\text{OS}}{\text{US}} = \dfrac{176}{762} \approx  0,231$.

Il ne reste plus qu'à calculer avec la calculatrice et l'inverse du sinus : on obtient $\widehat{\text{OUS}} \approx  13$\degres.

\emph{Remarque} : on ne peut pas utiliser la formule du cosinus car nous ne disposons pas de la longueur du côté adjacent à l'angle, donné par [UO]
\item On utilise la formule de la vitesse :

$\text{vitesse} = \dfrac{\text{distance}}{\text{temps}}$.

Le temps est de 6 min 30 s soit $360 + 30 = 390$~s.

$\text{vitesse} = \dfrac{762}{390} = \dfrac{254}{130} \approx 1,954$ soit 2~m/s à l'unité près. 
\item 
	\begin{enumerate}
		\item Le nombre calculé avec la formule est égal à la somme des affluences de de 8 h à 20 h par tranches de 2 h, 615 visiteurs sur la journée.
		\item La somme des nombres visibles est : $122 + 140 + 63 + 75 + 118 = 518$.
		
Le nombre de visiteurs entre 12~h et 14~h est donc égal à $615 - 518 = 97$
	\end{enumerate}
\item Pour qu'un tableur puisse appliquer un calcul, il faut toujours commencer par le symbole \og = \fg. Il suffit alors de taper \og  =MOYENNE(B2:G2)/2\fg.

\emph{Remarque} : on peut également pour avoir cette moyenne entrer la formule : \og =H2/12 \fg.
\end{enumerate}
