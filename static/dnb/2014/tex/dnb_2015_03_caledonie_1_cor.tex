
%\medskip

%\emph{Cet exercice est un questionnaire à choix multiple (QCM). Pour chaque ligne du tableau, une seule des trois réponses proposées est exacte. Sur la copie, indiquer le numéro de la question et recopier, sans justifier, la réponse choisie. Aucun point ne sera enlevé en cas de mauvaises réponses.}

%\begin{center}
%\begin{tabularx}{\linewidth}{|c|m{5cm}|*{3}{>{\centering \arraybackslash}X|}}\hline
%&Question &A &B &C\\ \hline
%1&Quelles sont les solutions de l'équation $(x - 3)(3x + 2) = 0$ ?&$0$ et $- \dfrac{5}{2}$& 3 et $- \dfrac{2}{3}$& $-3$ et $\dfrac{3}{2}$\\ \hline
%2&Une plante de 56~cm grandit de 15\,\% par trimestre sous serre. Quelle sera sa taille dans 3 mois?&64,4 cm& 71 cm& 8,4 cm\\ \hline
%3&\psset{unit=0.6cm}
%\begin{pspicture*}(-2.2,-3.5)(4.2,4.5)
%\psgrid[gridlabels=0pt,subgriddiv=1,gridwidth=0.2pt]
%\psaxes[linewidth=1.25pt,labelFontSize=\scriptstyle]{->}(0,0)(-2,-3.5)(4,4.5)
%\psaxes[linewidth=1.25pt,labelFontSize=\scriptstyle](0,0)(4,4.5)
%\psplot[plotpoints=2000,linewidth=1.25pt,linecolor=blue]{-2}{4}{x 1 sub dup mul 3 sub}
%\end{pspicture*}
%
%Quelle est l'image du nombre 1 par la fonction représentée ci-dessus ?&3&0&$- 3$\\ \hline
%4& Quel est le PGCD de 108 et 189 ?& 81& 9& 27\\ \hline
%5& Quelle est l'écriture égale à $\sqrt{45}$ ?\rule[-3mm]{0mm}{8mm}& 22,5& $3\sqrt{5}$& $5\sqrt{3}$\\ \hline
%\end{tabularx}
%\end{center}

\begin{enumerate}
\item \textbf{Réponse B}

Un produit de facteurs est nul si et seulement si un des facteurs est nul.

$x-3=0 \iff x=3$ et $3x+2=0 \iff 3x=-2 \iff x=-\dfrac{2}{3}$

\item \textbf{Réponse A}

Ajouter 15\,\% c'est multiplier par 1,15; $56\times 1,15 = 64,4$

\item \textbf{Réponse C}

La courbe dessinée passe par le point de coordonnées $(1,\,-3)$ donc l'image de $1$ est $-3$.

\item \textbf{Réponse C}

$189=108\times 1 + 81$; $108 = 81\times 1 + 27$; $81=27\times 3+0$;
le dernier reste non nul, donc le PGCD, est 27.

\item \textbf{Réponse B}

$\sqrt{45}=\sqrt{9\times 5}=\sqrt{9}\times \sqrt{5}=3\sqrt{5}$

\end{enumerate}

\vspace{0,5cm}

