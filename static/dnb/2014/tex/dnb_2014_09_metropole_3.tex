
\medskip

 Dans une classe de collège, après la visite médicale, on a dressé le tableau suivant : 

\begin{center}
\begin{tabularx}{0.8\linewidth}{|*{3}{>{\centering \arraybackslash}X|}}\hline
&Porte des lunettes& Ne porte pas de lunettes\\ \hline 
Fille &3 &15\\ \hline 
Garçon &7 &5\\ \hline 
\end{tabularx}
\end{center} 

Les fiches individuelles de renseignements tombent par terre et s'éparpillent.

\medskip
 
\begin{enumerate}
\item Si l'infirmière en ramasse une au hasard, quelle est la probabilité que cette fiche soit :
	\begin{enumerate}
		\item celle d'une fille qui porte des lunettes ? 
		\item celle d'un garçon ? 
	\end{enumerate} 
\item Les élèves qui portent des lunettes dans cette classe représentent 12,5\,\% de ceux qui en portent dans tout le collège. Combien y a-t-il d'élèves qui portent des lunettes dans le collège ? 
\end{enumerate} 

\vspace{0,5cm}

