
\medskip 

Julien veut mesurer un jeune chêne avec une croix de bûcheron comme le montre le schéma ci-dessous. 

\begin{center}
\psset{unit=0.8cm}
\begin{pspicture}(14,8)
%\psgrid
\psframe*(0,0)(14,0.2)
\psframe*(2.4,0.2)(2.7,7)
\pscurve*(2.4,2.5)(2,2)(1.5,2.7)(1,2.9)(0.5,4.4)(1,4.9)(1.5,6)(2,6.6)(2.4,7.4)(2.7,7.3)(2.9,7.2)(3,7)(3.5,4.5)(4,4)(4.2,3)(2.4,2.5)
\psframe*(10.7,5.3)(10.9,7.6)
\psframe*(10.9,6.1)(13.9,6.3)
\psline(2.6,0.2)(10.8,2.5)(2.6,7.4)
\psline(10.8,2.5)(2.4,2.5)
\psframe*(9.2,2.1)(9.35,3.42)
\psframe*(9.35,2.5)(10.8,2.65)
\psframe(9.35,2.65)(9.55,2.85)
\uput[u](9.35,3.42){D} \uput[d](9.35,2.1){E} \uput[ul](9.2,2.5){F}
\uput[ur](10.6,2.6){O}
\psarc(11.4,1.5){1.}{90}{120}
\psarc(11.4,3.6){1.}{-120}{-90}
\psellipse*(11.05,2.55)(0.05,0.1)
\rput(12,3.2){\footnotesize {\oe}il de l'observateur}
\rput(12.6,6.6){\footnotesize croix du bûcheron}
\psline[linestyle=dashed](10.8,2.5)(10.8,0.2) 
\psframe(10.8,0.2)(10.6,0.4)
\uput[ur](10.8,0.2){C}
\uput[dr](2.7,0){B}\uput[u](2.6,7.4){A}
\psframe(2.7,0.2)(3,0.5)
\end{pspicture}
\end{center}
 
Il place la croix de sorte que O, D et A d'une part et O, E et B d'autre part soient alignés. 

Il sait que DE = 20 cm et OF = 35 cm. Il place [DE] verticalement et [OF] horizontalement. 

Il mesure au sol BC = 7,7 m. 


\medskip

\begin{enumerate}
\item Le triangle ABO est un agrandissement du triangle ODE. Justifier que le coefficient d'agrandissement est 22. 
\item Calculer la hauteur de l'arbre en mètres. 
\item Certaines croix du bûcheron sont telles que DE = OF. 
Quel avantage apporte ce type de croix? 
\item Julien enroule une corde autour du tronc de l'arbre à 1,5 m du sol. Il mesure ainsi une circonférence de 138~cm. 

Quel est le diamètre de cet arbre à cette hauteur? Donner un arrondi au centimètre près. 
\end{enumerate} 

\vspace{0,5cm}

