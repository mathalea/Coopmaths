
\bigskip
 
Un couple a acheté une maison avec piscine en vue de la louer. Pour cet achat, le couple a effectué un prêt auprès de sa banque. Ils louent la maison de juin à septembre et la maison reste inoccupée le reste de l'année.

\medskip
 
\textbf{Information 1 : Dépenses liées à cette maison pour l'année 2013}

\medskip
 
Le diagramme ci-dessous présente, pour chaque mois, le total des dépenses dues aux différentes taxes, aux abonnements (électricité, chauffage, eau, internet), au remplissage et au chauffage de la piscine.
 
\begin{center}
\psset{xunit=0.75cm,yunit=0.008cm}
\begin{pspicture}(-1,-110)(13,700)
\psaxes[linewidth=1.25pt,Dx=20,Dy=100](0,0)(13,601)
\multido{\n=0+50}{13}{\psline(0,\n)(13,\n)}
\rput(2,650){Dépenses (en \euro)}
\psframe[fillstyle=solid,fillcolor=lightgray](0.75,0)(1.25,250)
\psframe[fillstyle=solid,fillcolor=lightgray](1.75,0)(2.25,250)
\psframe[fillstyle=solid,fillcolor=lightgray](2.75,0)(3.25,250)
\psframe[fillstyle=solid,fillcolor=lightgray](3.75,0)(4.25,250)
\psframe[fillstyle=solid,fillcolor=lightgray](4.75,0)(5.25,450)
\psframe[fillstyle=solid,fillcolor=lightgray](5.75,0)(6.25,550)
\psframe[fillstyle=solid,fillcolor=lightgray](6.75,0)(7.25,550)
\psframe[fillstyle=solid,fillcolor=lightgray](7.75,0)(8.25,550)
\psframe[fillstyle=solid,fillcolor=lightgray](8.75,0)(9.25,550)
\psframe[fillstyle=solid,fillcolor=lightgray](9.75,0)(10.25,300)
\psframe[fillstyle=solid,fillcolor=lightgray](10.75,0)(11.25,150)
\psframe[fillstyle=solid,fillcolor=lightgray](11.75,0)(12.25,150)
\rput(1,-80){\rotatebox{45}{janvier}}
\rput(2,-80){\rotatebox{45}{février}}
\rput(3,-80){\rotatebox{45}{mars}}
\rput(4,-80){\rotatebox{45}{avril}}
\rput(5,-80){\rotatebox{45}{mai}}
\rput(6,-80){\rotatebox{45}{juin}}
\rput(7,-80){\rotatebox{45}{juillet}}
\rput(8,-80){\rotatebox{45}{août}}
\rput(9,-80){\rotatebox{45}{septembre}}
\rput(10,-80){\rotatebox{45}{octobre}}
\rput(11,-80){\rotatebox{45}{novembre}}
\rput(12,-80){\rotatebox{45}{décembre}}
\end{pspicture}
\end{center} 

\textbf{Information 2 : Remboursement mensuel du prêt}

\medskip
 
Chaque mois, le couple doit verser 700~euros à sa banque pour rembourser le prêt.

\medskip
 
\textbf{Information 3 : Tarif de location de la maison}

$\bullet~~$ Les locations se font du samedi au samedi. 

$\bullet~~$Le couple loue sa maison du samedi 7 juin au samedi 27 septembre 2014. 

$\bullet~~$Les tarifs pour la location de cette maison sont les suivants :

\begin{center}
\begin{tabularx}{0.85\linewidth}{|*{3}{>{\centering \arraybackslash}X|}c|}\hline 
\textbf{Début}& \textbf{Fin}& \textbf{Nombre de semaines}&\textbf{Prix de la location}\\ \hline 
07/06/2014& 05/07/2014& 4 semaines &750 euros par semaine\\ \hline  
05/07/2014 &23/08/2014 &7 semaines &... euros par semaine\\ \hline  
23/08/2014 &27/09/2014 &5 semaines &750 euros par semaine\\ \hline 
\end{tabularx}
\end{center}
 
Pour l'année 2014, avec l'augmentation des différents tarifs et taxes, le couple prévoit que le montant des dépenses liées à la maison sera 6\,\% plus élevé que celui pour 2013.
 
Expliquer pourquoi le total des dépenses liées à la maison s'élèvera à \np{4505}~\euro{} en 2014.
 
On suppose que le couple arrive à louer sa maison durant toutes les semaines de la période de location. À quel tarif minimal (arrondi à la dizaine d'euros) doit-il louer sa maison entre le 5/07 et 23/08 pour couvrir les frais engendrés par la maison sur toute l'année 2014 ? 




