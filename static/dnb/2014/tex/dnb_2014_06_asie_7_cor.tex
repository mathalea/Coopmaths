
\medskip
 
%\textbf{Dans cet exercice, toute trace de recherche même non aboutie sera prise en compte dans l'évaluation.}

%\medskip
% 
%Les gérants d'un centre commercial ont construit un parking 
%souterrain et souhaitent installer un trottoir roulant pour accéder 
%de ce parking au centre commercial.
% 
%Les personnes empruntant ce trottoir roulant ne doivent 
%pas mettre plus de 1 minute pour accéder au centre commercial.
% 
%La situation est présentée par le schéma ci-dessous.
%
%\begin{center}
%\psset{unit=0.5cm}
%\begin{pspicture}(22,5.5)
%\psline(0,0.5)(22,0.5)
%\psline[linewidth=1.5pt](3.5,0.5)(16.3,2)(22,2)
%\psline(16.3,0.5)(16.3,2)
%\psframe(16.3,0.5)(16,0.8) 
%\rput(11.5,5){\scriptsize Trottoir roulant}
%\psline{->}(11.5,4.3)(11.5,1.5) 
%\uput[ul](16.3,2){C} \uput[u](19,2){\scriptsize Sol du centre commercial} 
%\uput[r](16.3,1){4 m} 
%\uput[d](1.75,0.5){\scriptsize Sol du parking} 
%\uput[d](3.5,0.5){P} 
%\uput[d](10.3,0.5){25 m} 
%\uput[dr](16.3,0.5){H}
%\end{pspicture}
%\end{center}
%
%\begin{center}
%\begin{tabularx}{\linewidth}{|X|m{0.5cm}|X|}\hline 
%\textbf{Caractéristiques du trottoir roulant} : &~&\textbf{Caractéristiques du trottoir roulant} :\\
%Modèle 1 &&Modèle 2 \\
%$\bullet~~$ Angle d'inclinaison maximum avec l'horizontale : 12~\degres&& 
%$\bullet~~$ Angle d'inclinaison maximum avec l'horizontale : 6~\degres.\\ 
%$\bullet~~$  Vitesse : 0,5 m/s&& $\bullet~~$  Vitesse : 0,75 m/s.\\ \hline
%\end{tabularx}
%\end{center}
 
%Est-ce que l'un de ces deux modèles peut convenir pour équiper ce centre commercial ? 

%Justifier. 
Modèle 1 : l’angle $a$ du trottoir roulant avec l’horizontale est tel que :

$\tan a = \dfrac{4}{25} = \dfrac{16}{100} = 0,16$.

La calculatrice donne $a \approx 9,1\degres$ : l’angle est acceptable ;

Dans le triangle rectangle CHP, on a :

$\text{CP}^2 = 4^2 + 25^2 = 16 + 625 = 641$, d’où $\text{CP} \approx 25,318$~m.

Pour gravir cette pente il faudra un temps de :

$\dfrac{25,318}{0,5}  \approx 50,6$~s soit moins d’une minute. 

Le modèle 1 est acceptable.

Par contre le modèle 2 ne peut convenir car la pente est trop forte.
