
\medskip
 
%Léa pense qu'en multipliant deux nombres impairs consécutifs (c'est-à-dire qui se suivent) et en ajoutant 1, le résultat obtenu est toujours un multiple de 4.
%
%\medskip
 
\begin{enumerate}
\item Étude d'un exemple :
 
%5 et 7 sont deux nombres impairs consécutifs.
	\begin{enumerate}
		\item %Calculer $5 \times 7 + 1$. 
$5 \times 7 + 1 = 35 + 1 = 36 = 4 \times 9$.
		\item %Léa a-t-elle raison pour cet exemple ?
		Oui	
	\end{enumerate} 
\item %Le tableau ci-dessous montre le travail qu'elle a réalisé dans une feuille de calcul. 

%\begin{center}
%\begin{tabularx}{\linewidth}{|c|c|*{4}{>{\centering \arraybackslash}X|}}\hline
%&A&B&C&D&E\\ \hline
%1&&Nombre impair&Nombre impair suivant&Produit de ces nombres impairs consécutifs&Résultat obtenu\\ \hline
%2&$x$&$2x + 1$&$2x + 3$&$(2x + 1)(2x + 3)$&$(2x + 1)(2x + 3) + 1$\\ \hline
%3	&0	&1	&3	&3	&4\\ \hline
%4	&1	&3	&5	&15	&16\\ \hline
%5	&2	&5	&7	&35	&36\\ \hline
%6	&3	&7	&9	&63	&64\\ \hline
%7	&4	&9	&11	&99	&100\\ \hline
%8	&5	&11	&13	&143&144\\ \hline
%9	&6	&13	&15	&195&196\\ \hline
%10	&7	&15	&17	&255&256\\ \hline
%11	&8	&17	&19	&323&324\\ \hline
%12	&9	&19	&21	&399&400\\ \hline
%\end{tabularx}
%\end{center}

	\begin{enumerate}
		\item %D'après ce tableau, quel résultat obtient-on en prenant comme premier nombre impair 17 ?
		$17 \times 19 + 1 = 324$. 
		\item %Montrer que cet entier est un multiple de 4. 
$324 = 320 + 4 = 4 \times 80 + 4\times 1 = 4 \times (80 + 1) = 4 \times 81$.
		\item %Parmi les quatre formules de calcul tableur suivantes, deux formules ont pu être saisies dans la cellule D3. Lesquelles ? Aucune justification n'est attendue.

%\setlength\parindent{12mm}
%\begin{description}
%\item[ ] Formule 1 : \fbox{ =(2*A3+1)*(2*A3+3)} 
%\item[ ] Formule 2 :  \fbox{= (2*B3 + 1)*(2*C3 + 3)} 
%\item[ ] Formule 3 :  \fbox{= B3*C3} 
%\item[ ] Formule 4 :  \fbox{= (2*D3+1)*(2*D3+3)}
%\end{description}
%\setlength\parindent{0mm}
Les formules 2 et 3 donnent le bon produit.
	\end{enumerate} 
\item Étude algébrique : 
	\begin{enumerate}
		\item %Développer et réduire l'expression $(2x + 1)(2x + 3) + 1$.
$(2x + 1)(2x + 3) + 1 = 4x^2 + 6x + 2x + 3 + 1 = 4x^2 + 8x + 4$.		 
		\item %Montrer que Léa avait raison: le résultat obtenu est toujours un multiple de 4.
$4x^2 + 8x + 4 = 4 \left(x^2 + 2x + 1\right)$. On a même $(2x + 1)(2x + 3) + 1 = 4(x + 1)^2$.
	\end{enumerate} 
\end{enumerate} 

\vspace{0,5cm}

