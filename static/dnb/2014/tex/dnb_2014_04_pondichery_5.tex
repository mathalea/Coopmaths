
\medskip 

\emph{Pense-bête : toutes les formules données ci-dessous correspondent bien \`{a} des formules d'aires ou de volumes. On ne sait pas \`{a} quoi elles correspondent, mais elles peuvent quand même être utiles pour résoudre l'exercice ci-dessous.}

\begin{center}
\psset{unit=1cm}
\begin{pspicture}(12,4)
\psellipse(6,2)(6,1.6) 
\psellipse[fillstyle=solid,fillcolor=white](2,3)(0.9,0.4)
\psellipse[fillstyle=solid,fillcolor=white](1.5,3.2)(0.6,0.3)
\psellipse[fillstyle=solid,fillcolor=white](1.,3.3)(0.3,0.15)
\rput(5.3,3){$\dfrac{1}{3} \times  \text{aire de la base} \times  \text{hauteur}$} 
\rput(8,2){$\pi r^2$} 
\rput(2,2){$\dfrac{4}{3}\pi r^3$} 
\rput(6,1){aire de la base $\times$ hauteur} 
\end{pspicture} 
\end{center}

\parbox{0.65\linewidth}{Voici une bouteille constituée d'un cylindre et d'un tronc de cône surmonté par un goulot cylindrique. La	bouteille est pleine lorsqu'elle est remplie	jusqu'au goulot.

Les dimensions sont notées sur le schéma.

 
\textbf{1.} Calculer le volume exact de la partie cylindrique de la bouteille puis en donner un arrondi au cm$^3$.}\hfill 
\parbox{0.32\linewidth}{\psset{unit=0.8cm}
\begin{pspicture}(0,-1)(4,8)
%\psgrid 
\scalebox{.99}[0.25]{\psarc(1.5,22.4){1.5}{180}{0}}%
\scalebox{.99}[0.25]{\psarc[linestyle=dashed](1.5,22.4){1.5}{0}{180}}%
\psline(0,0.3)(0,5.6)
\psline(3,0.3)(3,5.6)
\scalebox{.99}[0.25]{\psarc(1.5,1.35){1.5}{180}{0}}%
\scalebox{.99}[0.25]{\psarc[linestyle=dashed](1.5,1.35){1.5}{0}{180}}%
\psline(0,5.6)(1,6.7)
\psline(3,5.6)(2,6.7)
\scalebox{.99}[0.25]{\psarc(1.5,27){0.5}{180}{0}}%
\scalebox{.99}[0.25]{\psarc[linestyle=dashed](1.5,27){0.5}{0}{180}}%
\psellipse(1.5,7.67)(0.5,0.18)
\psline(1,6.7)(1,7.67)
\psline(2,6.7)(2,7.67)
\psline{<->}(3.6,0.3)(3.6,5.6)\uput[r](3.6,2.8){15 cm}
\psline{<->}(0,-0.5)(3,-0.5)\uput[d](1.5,-0.5){10 cm}
\psline[linestyle=dashed](3,0.3)(3.6,0.3)
\psline[linestyle=dashed](3,5.6)(3.6,5.6)
\psline[linestyle=dashed](0,0.3)(0,-0.5)
\psline[linestyle=dashed](3,0.3)(3,-0.5)
\end{pspicture}}

\medskip

\textbf{2.~} Pour obtenir le tronc	de cône, on a coupé un cône par un plan parallèle \`{a} la base passant par O$'$. La hauteur SO du grand cône est de 6 cm et la hauteur SO' du petit est égale \`{a} 2 cm. 	Le rayon de la base du grand cône est de 5 cm.

\parbox{0.4\linewidth}{\psset{unit=1cm}\begin{pspicture}(5,4.2)
%\psgrid
\scalebox{.99}[0.25]{\psarc(2.5,2){2.5}{180}{0}}%
\scalebox{.99}[0.25]{\psarc[linestyle=dashed](2.5,2){2.5}{0}{180}}%
\psellipse(2.5,2.75)(0.8,0.3)
\psline(0,0.58)(1.72,2.82)
\psline(5,0.58)(3.28,2.82)
\psline[linestyle=dashed](1.72,2.82)(2.5,3.75)(3.28,2.82)
\uput[ul](2.5,3.75){S}
\psdots[dotstyle=+,dotangle=45](2.5,2.75)(2.5,0.3)
\uput[ur](2.5,2.75){O$'$}
\uput[r](2.5,0.3){O}
\psline(2.5,0.3)(2.5,3.75)
\end{pspicture}}\hfill 
\parbox{0.55\linewidth}{\textbf{a.~} Calculer le volume $V_{1}$ du grand cône de hauteur SO (donner la valeur exacte).
 
\textbf{b.~} Montrer que le volume $V_{2}$ du tronc de cône est égal 
\`{a} $\dfrac{\np{1300} \pi}{27}$ cm$^3$. En donner une valeur arrondie  au 
cm$^3$.}

\medskip

\textbf{3.~}  Parmi les quatre graphiques ci-dessous, l'un d'entre eux représente le volume $V(h)$ de la bouteille en fonction de la hauteur $h$ de remplissage du bidon.
 
Quel est ce graphique ? Pourquoi les autres ne sont-ils pas convenables ? 

\begin{center}
\begin{tabularx}{\linewidth}{XX}
\psset{xunit=0.22cm,yunit=0.0022cm}
\begin{pspicture}(-3,-400)(28,2700)
\multido{\n=0+3}{8}{\psline[linewidth=0.2pt,linecolor=orange](\n,0)(\n,2400)}
\multido{\n=0+300}{9}{\psline[linewidth=0.2pt,linecolor=orange](0,\n)(21,\n)}
\psaxes[linewidth=1.5pt,Dx=3,Dy=3000]{->}(0,0)(23,2700)
\multido{\n=0+300}{9}{\uput[l](0,\n){\np{\n}}}
\uput[u](22,0){$h$}\uput[l](0,2650){$V(h)$}
\psline(0,0)(15,1200)(19,1325)(21,1350)
\rput(12,-350){Graphique 1}
\end{pspicture}&\psset{xunit=0.22cm,yunit=0.0022cm}
\begin{pspicture}(-3,-400)(26,2700)
\multido{\n=0+3}{8}{\psline[linewidth=0.2pt,linecolor=orange](\n,0)(\n,2400)}
\multido{\n=0+300}{9}{\psline[linewidth=0.2pt,linecolor=orange](0,\n)(21,\n)}
\psaxes[linewidth=1.5pt,Dx=3,Dy=3000]{->}(0,0)(23,2700)
\multido{\n=0+300}{9}{\uput[l](0,\n){\np{\n}}}
\uput[u](22,0){$h$}\uput[l](0,2650){$V(h)$}
\psline(0,0)(15,1200)(19,1000)(21,1050)
\rput(12,-350){Graphique 2}
\end{pspicture}\\
\psset{xunit=0.22cm,yunit=0.0022cm}
\begin{pspicture}(-3,-400)(28,2700)
\multido{\n=0+3}{8}{\psline[linewidth=0.2pt,linecolor=orange](\n,0)(\n,2400)}
\multido{\n=0+300}{9}{\psline[linewidth=0.2pt,linecolor=orange](0,\n)(21,\n)}
\psaxes[linewidth=1.5pt,Dx=3,Dy=3000]{->}(0,0)(23,2700)
\multido{\n=0+300}{9}{\uput[l](0,\n){\np{\n}}}
\psline(0,0)(15,1200)(19,2550)(21,2600)
\rput(12,-350){Graphique 3}
\uput[u](22,0){$h$}\uput[l](0,2650){$V(h)$}
\end{pspicture}&\psset{xunit=0.22cm,yunit=0.0022cm}
\begin{pspicture}(-3,-400)(26,2700)
\multido{\n=0+3}{8}{\psline[linewidth=0.2pt,linecolor=orange](\n,0)(\n,2400)}
\multido{\n=0+300}{9}{\psline[linewidth=0.2pt,linecolor=orange](0,\n)(21,\n)}
\psaxes[linewidth=1.5pt,Dx=3,Dy=3000]{->}(0,0)(23,2700)
\multido{\n=0+300}{9}{\uput[l](0,\n){\np{\n}}}
\uput[u](22,0){$h$}\uput[l](0,2650){$V(h)$}
\rput(12,-350){Graphique 4}
\psline(0,150)(15,1350)(19.5,1450)(21,1500)
\end{pspicture}\\
\end{tabularx}
\end{center}


\vspace{0,25cm}

