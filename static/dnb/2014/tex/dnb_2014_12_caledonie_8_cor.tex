
\begin{enumerate}
\item La plus grande sphère du dépôt a un diamètre de 19,7~m, donc un rayon de 9,85~m. \\$V_{\text{grande sphère}} = \dfrac{4}{3} \times \pi \times 9,85^3$ \\[2mm]
\phantom{$V_{\text{grande sphère}} $} $\approx 4~003$ m$^3$ \\
Le volume de stockage de la plus grande sphère du dépôt est bien d'environ \np{4000}~m$^{3}$.
\item 1 m$^3$ de butane pèse 580 kg soit 0,58 tonne. \\
On a une situation de proportionnalité : \\[2mm]
\renewcommand{\arraystretch}{1.5}
%\hspace{1.5cm} 
\begin{tabular}{|>{\raggedright\hspace{0pt}}m{3cm}|>{\centering\hspace{1pt}}m{1.7cm}|>{\centering\hspace{0pt}}m{1.7cm}|}
			\hline
			\textbf{Volume en m$^3$} & 1 & $V$  \tabularnewline
			\hline
			\textbf{Masse en tonne} & 0,58 &  1~200  \tabularnewline
			\hline
			\end{tabular}

Le volume $V$ correspondant aux \np{1200}~tonnes est :  $V=\dfrac{1\times 1~200}{0,58}\approx 2~069$ m$^3$. 
\item  Le volume total des deux plus petites sphères est de $1~000+600=1~600$ m$^3$. \\
Ce volume est inférieur aux \np{2069} m$^3$ correspondant à \np{1200}~tonnes de \linebreak butane, donc la grande sphère sera nécessaire.
\end{enumerate}
