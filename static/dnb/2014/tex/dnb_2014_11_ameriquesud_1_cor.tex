
\medskip

%\emph{Pour chacune des questions suivantes, plusieurs propositions de réponse sont faites. Une seule des propositions est exacte. Aucune justification n'est attendue.\\ 
%Une bonne réponse rapporte $1$ ou $2$ points. Une mauvaise réponse ou une absence de réponse rapporte $0$ point. \\
%\medskip
%Reporter sur votre copie le numéro de la question et donner la bonne réponse.} 
%
%\medskip

\begin{enumerate}
\item %Une école de musique organise un concert de fin d'année. Lors de cette manifestation la recette s'élève à \np{1300}~\euro. 

%Dans le public il y a $100$ adultes et $50$ enfants. Le tarif enfant coûte 4~\euro{} de moins que le tarif adulte. 

%Le tarif enfant est : 
%
%\medskip
%\begin{tabularx}{\linewidth}{*{3}X}
%\textbf{a.~~}10~\euro&\textbf{b.~~}8~\euro&\textbf{c.~~}6~\euro
%\end{tabularx}
%\medskip
Si $t$ est le tarif enfant, la tarif adulte est $t + 4$.

La recette est donc :

$50t + 100 (t + 4) = \np{1300}$ soit $150t + 400 = \np{1300}$ ou encore $150t = 900$, donc $t = 6$~\euro. Réponse \textbf{c.}
\item %On considère la figure ci-dessous où AEFD est un rectangle avec AB $= \sqrt{15}  - 1$ et BE $= 2$.

%\begin{center} 
%\psset{unit=0.75cm}
%\begin{pspicture}(5,4.5)
%\def\iso{\psline(0,-0.1)(0,0.1) \psline(0.1,-0.1)(0.1,0.1)}
%\psframe(0.5,0.5)(5.64,4.14)
%\psline(3.64,0.5)(3.64,4.14)
%\uput[ul](0.5,4.14){A} \uput[u](3.64,4.14){B}  
%\uput[ur](5.64,4.14){E} \uput[dl](0.5,0.5){D} 
%\uput[d](3.64,0.5){C}  \uput[dr](5.64,0.5){F}
%\rput(2.07,0.5){\iso}\rput(2.07,4.14){\iso}
%\rput{-90}(0.5,2.37){\iso} \rput{-90}(3.64,2.37){\iso} 
%\end{pspicture}
%\end{center}
 
%L'aire du rectangle AEFD est: 
%
%\medskip
%\begin{tabularx}{\linewidth}{*{3}X}
%\textbf{a.~~}$2\sqrt{15} - 2$&\textbf{b.~~}29&\textbf{c.~~}14
%\end{tabularx}
%\medskip
La figure se décompose en un carré de côté $\sqrt{15}  - 1$ et un rectangle de côtés $\sqrt{15}  - 1$ et $2$. L'aire est donc égale à :

$\left(\sqrt{15}  - 1\right)^2 + 2\left(\sqrt{15}  - 1\right) =  \left(\sqrt{15}  - 1\right)\left(\sqrt{15}  - 1 + 2\right) = \left(\sqrt{15}  - 1\right)\left(\sqrt{15}  + 1\right) =$

$ 15 - 1 = 14$. Réponse \textbf{c.}
\item %Le 27 janvier 2012, peu avant 16~h, un séisme de magnitude 5,4 s'est produit dans la province de Parme dans le nord de l'Italie. La secousse a été ressentie fortement à Gênes, Milan, Turin mais également dans une moindre mesure à Cannes dans les Alpes Maritimes. 

%Les ondes sismiques ont mis 59~secondes pour parvenir à Cannes, située à 320~km de l'épicentre. 

%On rappelle que la relation qui relie le temps $t$, la distance $d$ et la vitesse $v$ est : $v = \dfrac{d}{t}$. 
%
%La vitesse de propagation des ondes sismiques, exprimée en kilomètres par seconde, arrondie au dixième, est : 
%
%\medskip
%\begin{tabularx}{\linewidth}{*{3}X}
%\textbf{a.~~}5,4 km/s&\textbf{b.~~}10,8 km/s &\textbf{c.~~}59,3 km/s
%\end{tabularx}
%\medskip
On a $v = \dfrac{320}{59} \approx 5,42$ soit au dixième près 5,4~km/s. Réponse \textbf{a.}
\end{enumerate}

\vspace{0,5cm}

