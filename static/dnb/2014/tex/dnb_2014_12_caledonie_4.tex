
\medskip 

Mathilde et Eva se trouvent à la Baie des Citrons. 

Elles observent un bateau de croisière quitter le port de Nouméa. Mathilde pense qu'il navigue à une vitesse de 20 noeuds. 

Eva estime qu'il navigue plutôt à 10 noeuds. 

Elles décident alors de déterminer cette vitesse mathématiquement. 

Sur son téléphone, Mathilde utilise d'abord la fonction chronomètre.

Elle déclenche le chronomètre quand l'avant du navire passe au niveau d'un cocotier et l'arrête quand l'arrière du navire passe au niveau du même cocotier ; il s'écoule 40 secondes.

Ensuite, Eva recherche sur Internet les caractéristiques du bateau. Voici ce qu'elle a trouvé: 

\begin{center}\begin{tabular}{|l|}\hline
\textbf{Caractéristiques techniques :}\\ 
Longueur : 246 m\\
Largeur : 32 m\\
Calaison: 6 m\\ 
Mise en service : 1990\\ 
Nombre maximum de passagers : \np{1596}\\
Membres d'équipage : 677\\\hline
\end{tabular}
\end{center}

\textbf{Questions :}

\medskip

\begin{enumerate}
\item Quelle distance a parcouru le navire en 40 secondes ? 
\item Qui est la plus proche de la vérité, Mathilde ou Eva ? Justifier la réponse. 
\end{enumerate}

\medskip

Rappel : Le \og nœud \fg{} est une unité de vitesse. 

\emph{Naviguer à $1$ nœud signifie parcourir $0,5$ mètre en $1$ seconde.}

\medskip 

\emph{Dans cet exercice, toute trace de recherche, même incomplète ou non fructueuse, sera prise en compte dans l'évaluation.}  

\vspace{0,5cm}

