
\medskip

Certaines écluses ont des portes dites \og busquées \fg{} qui forment un angle pointé vers l'amont de manière à résister à la pression de l'eau, 

\begin{center}
\psset{unit=1cm}
\begin{pspicture}(10.8,9.6)
%\psgrid 
\psframe[fillstyle=solid,fillcolor=lightgray](0,0.5)(1.2,9.6)
\psframe[fillstyle=solid,fillcolor=lightgray](6.6,0.5)(7.8,9.6)
\psline(1.2,4.7)(6.6,4.7)
\psline[linewidth=3pt](1.2,4.7)(4,6)(6.6,4.7)
\psline[linewidth=3pt]{->}(3.7,9)(3.7,6.7)
\psline(3.9,6)(3.9,4.3)
\psframe(3.9,5)(3.6,4.7)
\rput(3.7,9.5){Amont} 
\rput(5,9){Sens du courant} 
\rput(9.4,7.1){Portes}
\rput(9.4,6.7){\og busquées\fg}
\rput(9.4,6.3){de même}
\rput(9.4,5.9){longueur}
\rput(1,4.7){A}\rput(6.8,4.7){B}
\rput(4,6.3){P} 
\psline{<->}(1.2,4.3)(6.6,4.3)
\uput[d](3.9,4.3){5,8 m} 
\rput(3.6,0.2){Aval}
\pscurve(2.2,9.3)(2.4,8.3)(2,7.3)(2.15,6) 
\pscurve(4.5,9.3)(4.7,8.3)(4.3,7.3)(4.45,6) 
\pscurve(2.2,3.7)(2.4,2.7)(2,1.7)(2.15,0.7) 
\pscurve(4.5,3.7)(4.7,2.7)(4.3,1.7)(4.45,0.7) 
\psline{->}(8.5,6.7)(3.35,5.6) 
\psline{->}(8.5,6.7)(5.6,5.3)
\psarc(1.2,4.7){1cm}{25}{90}\rput(1.8,5.8){55~\degres}
\end{pspicture}
\end{center}
 
En vous appuyant sur le schéma ci-dessus, déterminer la longueur des portes au cm près.
 
\textbf{Si le travail n'est pas terminé, laisser tout de même une trace de la recherche. Elle sera prise en compte dans la notation.} 


