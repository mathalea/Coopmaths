
\medskip
 
%La copie d'écran ci-dessous montre le travail effectué par Léa pour étudier trois fonctions $f$, $g$ et $h$ telles 
%que :
%
%\setlength\parindent{5mm}
%\begin{itemize}
%\item[$\bullet~~$] $f(x) = x^2+ 3x - 7$ 
%\item[$\bullet~~$] $g(x) = 4x + 5$
%\item[$\bullet~~$] $h$ est une fonction affine dont Léa a oublié d'écrire l'expression dans la cellule A4. 
%\end{itemize}
%\setlength\parindent{0mm}
%
%\begin{center}
%\begin{tabularx}{\linewidth}{|c|m{3cm}|*{5}{X|}}\hline
%\multicolumn{2}{|r|}{$\Sigma =$}&\multicolumn{5}{l|}{=B1*B1+3*B1-7}\\ \hline 
%	&A	&B	&C	&D	&E	&F\\ \hline
%1	&$x$&$- 2$&0&2&4&6\\ \hline
%2	&$f(x) = x^2 + 3x - 7$&$- 9$&$- 7$&3&21&47 \\ \hline
%3	&$g(x) = 4x + 5$&$- 3$&5&13&21&29\\ \hline
%4	&$h(x)$	&9	&5	&1	&$- 3$	&$- 7$\\ \hline
%\end{tabularx}
%\end{center}

\begin{enumerate}
\item %Donner un nombre qui a pour image $- 7$ par la fonction $f$.
On voit sur les lignes 1 et 2 que $f(0) = - 7$. Donc $0$ a pour image $- 7$ par $f$. 
\item %Vérifier à l'aide d'un calcul détaillé que $f(6) = 47$.
On a $f(6) = 6^2 + 3 \times 6 - 7 = 36 + 18 - 7 = 54 - 7 = 47$. 
\item %Expliquer pourquoi le tableau permet de donner une solution de l'équation: $x^2 + 3x - 7 = 4x + 5$.
On voit dans la colonne E que 4 a la même image par $f$ et par $g$. Donc 4 est une solution de l'équation $f(x) = g(x)$ ou  $x^2+ 3x - 7 =  4x + 5$.
%Quelle est cette solution ? 
\item %À l'aide du tableau, retrouver l'expression algébrique $h(x)$ de la fonction affine $h$.
On sait que $h$ est de la forme $h(x) = ax + b$.

Comme $h(0) = b = 5$, on a déjà $b = 5$.

D'autre part $h(2) = 2 \times a + 5 = 1$ soit $2a = - 4$ et $a = - 2$.

On a donc $h(x) = -2x + 5$ (fonction affine).
\end{enumerate}

\bigskip

