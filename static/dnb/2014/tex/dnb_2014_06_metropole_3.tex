
\medskip

Voici un programme de calcul: 

\begin{center}
\psset{xunit=1cm,yunit=0.6cm}
\begin{pspicture}(8,10)
\psline{->}(4,8.7)(4,8)
%\psgrid
\rput(4,9.){Choisir un nombre}
\rput(1,6){Soustraire 6} 
\rput(7,6){Soustraire 2} 
\psframe(0,4)(2,5.1)
\psframe(6,4)(8,5.1)
\psframe(3,7)(5,8)
\rput(4,3.5){Multiplier les deux }
\rput(4,3){nombres obtenus }
\psline{->}(3.4,7)(1,5.1)
\psline{->}(4.6,7)(7,5.1)
\rput(4,1){Résultat}
\psframe(3,0.7)(5,1.3)
\psline(1,4)(4,2)
\psline(7,4)(4,2) 
\psline{->}(4,2)(4,1.3)
\end{pspicture}
\end{center}

\begin{enumerate}
\item Montrer que si on choisit 8 comme nombre de départ, le programme donne 12 comme résultat. 
\item Pour chacune des affirmations suivantes, indiquer si elle est vraie ou fausse. On rappelle que les réponses doivent être justifiées. 

\begin{description}
\item[ ] \textbf{Proposition 1 : } Le programme peut donner un résultat négatif.
\item[ ] \textbf{Proposition 2 : } Si on choisit $\frac{1}{ 2}$ comme nombre de départ, le programme donne $\frac{33}{4}$ comme résultat. 
\item[ ] \textbf{Proposition 3 :} Le programme donne 0 comme résultat pour exactement deux nombres. 
\item[ ] \textbf{Proposition 4 :} La fonction qui, au nombre choisi au départ, associe le résultat du programme est une fonction linéaire. 
\end{description}
\end{enumerate}

\vspace{0,5cm} 

