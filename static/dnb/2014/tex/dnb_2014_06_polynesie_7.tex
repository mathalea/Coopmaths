
\medskip

Dans tout cet exercice, on travaille avec des triangles ABC isocèles en A tels que : BC = 5 cm. La mesure de l'angle $\widehat{\text{ABC}}$ peut varier. 

\smallskip

On va alors s'intéresser aux angles extérieurs de ces triangles, c'est-à-dire, comme l'indique la figure ci-après, aux angles qui sont supplémentaires et adjacents avec les angles de ce triangle.

\begin{center}
\psset{unit=1cm}
\begin{pspicture}(12,7.5)
%\psgrid 
\pspolygon(4,2)(6.3,2.8)(4,5.7)
\uput[l](4,2){C} \uput[dr](6.3,2.8){B} \uput[ur](4,5.7){A}
\rput(2.6,6.9){$x$}
\rput(9.,3.6){$y$}
\rput(3.85,0.2){$z$}
\psline(3.9,3.7)(4.1,3.5)\psline(3.9,3.6)(4.1,3.4)
\psline(4.9,4.4)(5.1,4.6)\psline(4.9,4.3)(5.1,4.5)
\psline(6.3,2.8)(8.6,3.6)
\psline(4,5.7)(2.85,7.15)
\psline(4,2)(4,0.1)
\psarc(4,2){5mm}{-90}{26}
\psarc(6.3,2.8){5mm}{26}{122}
\psarc(4,5.7){5mm}{125}{270}
\rput(1.6,6){Angle extérieur}\psline{->}(5,1)(4.5,1.6) 
\rput(7,4.8){Angle extérieur}\psline{->}(6.8,4.6)(6.4,3.4) 
\rput(6.2,1){Angle extérieur}\psline{->}(2.7,5.8)(3.4,5.5)
\end{pspicture}
\end{center}

\begin{enumerate}
\item Dans cette question uniquement, on suppose que $\widehat{\text{ABC}} = 40\degres$. 
	\begin{enumerate}
		\item Construire le triangle ABC en vraie grandeur. Aucune justification n'est attendue pour cette construction. 
		\item Calculer la mesure de chacun de ses 3 angles extérieurs. 
		\item Vérifier que la somme des mesures de ces 3 angles extérieurs est égale à 360\degres.
	\end{enumerate} 
\item Est-il possible de construire un triangle ABC isocèle en A tel que la somme des mesures de ses trois angles extérieurs soit différente de $360$\degres ? 
\end{enumerate}
