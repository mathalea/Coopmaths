
\begin{enumerate}
\item Représentation d'un agrandissement de cet octogone en l'inscrivant dans un cercle de rayon 3~cm. 
\begin{center}
\begin{tikzpicture}
\tkzDefPoints{0/0/O,3/0/J}
\tkzDefPointBy[rotation=center O angle 22.5](J) \tkzGetPoint{H}
\tkzDefPointBy[rotation=center O angle 45](H) \tkzGetPoint{A}
\tkzDefPointBy[rotation=center O angle 45](A) \tkzGetPoint{B}
\tkzDefPointBy[rotation=center O angle 45](B) \tkzGetPoint{C}
\tkzDefPointBy[rotation=center O angle 45](C) \tkzGetPoint{D}
\tkzDefPointBy[rotation=center O angle 45](D) \tkzGetPoint{E}
\tkzDefPointBy[rotation=center O angle 45](E) \tkzGetPoint{F}
\tkzDefPointBy[rotation=center O angle 45](F) \tkzGetPoint{G}
\tkzDrawPolygon(A,B,C,D,E,F,G,H) \tkzDrawPoints(O,A,B,C,D,E,F,G,H)
\tkzDrawSegments(O,A O,B)
\tkzLabelPoint[below](O){$O$} 
\tkzLabelPoint[above](A){$A$} \tkzLabelPoint[above left](B){$B$}
\tkzLabelPoint[left](C){$C$} \tkzLabelPoint[left](D){$D$}
\tkzLabelPoint[below](E){$E$} \tkzLabelPoint[below](F){$F$}
\tkzLabelPoint[right](G){$G$} \tkzLabelPoint[right](H){$H$}
\tkzCompass[color=blue,length=1.5](A,B)
\tkzCompass[color=blue,length=1.5](B,C)
\tkzCompass[color=blue,length=1.5](C,D)
\tkzCompass[color=blue,length=1.5](D,E)
\tkzCompass[color=blue,length=1.5](E,F)
\tkzCompass[color=blue,length=1.5](F,G)
\tkzCompass[color=blue,length=1.5](G,H)
\tkzDrawCircle[color=red](O,A)
%\tkzProtractor[scale=.55,rotate=67.5,with=half](O,A)
\tkzMarkAngle[color=red](A,O,B)
\tkzLabelAngle[pos=1.2](A,O,B){$45^{\circ}$}
%\tkzCompass[color=blue,length=1.5](A,B)
\end{tikzpicture}
\end{center}
On place le point $A$ sur le cercle de centre $O$ et de rayon 3~cm. On place le point $B$ sur le cercle tel que $\widehat{AOB}=\dfrac{360}{8}=45^{\circ}$. \`A l'aide d'un compas, on reporte,  avec un écartement de $AB$, on définit les autres points.
\item Le triangle $DAH$ est rectangle.

On a: $\text{mes}(\widehat{DOH})=4\times\text{mes}(\widehat{HOA})=4\times45^{\circ}=180^{\circ}$; les points $D$, $O$ et $H$ sont donc alignés et $D$ et $H$ sont ainsi diamétralement opposés. $[DH]$ est un diamètre du cercle, $A$ est sur le cercle.

Ainsi, $DAH$ est rectangle.
\item Dans un cercle, si un angle inscrit (ici \textcolor{blue}{$\widehat{BEH}$}) et un angle au centre (ici \textcolor{blue}{$\widehat{BOH}$}) interceptent le même arc, alors la mesure de l'angle au centre (ici \textcolor{blue}{$\text{mes}(\widehat{BOH})=2\times 45= 90^{\circ}$}) est le double de la mesure de l'angle inscrit (ici $\textcolor{red}{\text{mes}(\widehat{BEH})}=\dfrac{2\times 45}{2}\textcolor{red}{= 45^{\circ}}$).
\end{enumerate}

\vspace{0.5cm}

