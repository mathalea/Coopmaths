
\medskip 

Dans le jeu \emph{pierre--feuille--ciseaux} deux joueurs choisissent en même temps l'un des trois \og  coups \fg{} suivants : 

\textbf{pierre} en fermant la main 

\textbf{feuille} en tendant la main 

\textbf{ciseaux} en écartant deux doigts 

\medskip
\begin{itemize}
\item La \textbf{pierre} bat les \textbf{ciseaux} (en les cassant). 
\item Les \textbf{ciseaux} battent la \textbf{feuille} (en la coupant). 
\item La \textbf{feuille} bat la \textbf{pierre} (en l'enveloppant). 
\item Il y a match nul si les deux joueurs choisissent le même coup (par exemple si chaque joueur choisit \og \textbf{feuille} \fg). 
\end{itemize}

\medskip

\begin{enumerate}
\item Je joue une partie face à un adversaire qui joue au hasard et je choisis de jouer \og pierre ». 
	\begin{enumerate}
		\item Quelle est la probabilité que je perde la partie ? 
		\item Quelle est la probabilité que je ne perde pas la partie ? 
	\end{enumerate}
\item  Je joue deux parties de suite et je choisis de jouer \og \textbf{pierre} \fg{} à chaque partie. Mon adversaire joue au hasard. 

Construire l'arbre des possibles de l'adversaire pour ces deux parties. On notera P,\: F,\: C, pour pierre, feuille, ciseaux. 

\item  En déduire : 
	\begin{enumerate}
		\item La probabilité que je gagne les deux parties. 
		\item La probabilité que je ne perde aucune des deux parties. 
	\end{enumerate}
\end{enumerate}

\vspace{0,5cm}

