
\medskip

Le dépit de carburant de Koumourou, à  Ducos, dispose de trois sphères de stockage de butane.

\medskip

\begin{enumerate}
\item La plus grande sphère du dépôt a un diamètre de 19,7~m. Montrer que son volume de stockage est d'environ \np{4000}~m$^{3}$.

\emph{On rappelle que le volume d'une boule est donné par : $V = \dfrac{4}{3} \times \pi \times R^3$, où $R$ est le rayon de la boule.} 
\item Tous les deux mois, \np{1200} tonnes de butane sont importées sur le territoire. 

1 m$^3$ de butane pèse 580 kg. Quel est le volume, en m$^3$, correspondant aux \np{1200}~tonnes ? 

Arrondir le résultat à l'unité. 
\item  Les deux plus petites sphères ont des volumes de \np{1000} m$^3$ et 600 m$3$. Seront-elles suffisantes pour stocker les \np{1200}~tonnes de butane, ou bien aura-t-on besoin de la grande sphère ?

Justifier la réponse. 
\end{enumerate}

