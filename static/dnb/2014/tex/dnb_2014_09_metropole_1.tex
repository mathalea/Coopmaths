
\medskip
Cédric s’entraîne pour l’épreuve de vélo d’un triathlon.

La courbe ci-dessous représente la distance en kilomètres en fonction du temps écoulé en minutes.

\begin{center}
\psset{xunit=0.1cm,yunit=0.2cm,comma=true}
\begin{pspicture}(-8,-5)(90,49)
\multido{\n=0+10}{10}{\psline[linestyle=dotted,linecolor=orange](\n,0)(\n,45)}
\multido{\n=0+5}{10}{\psline[linestyle=dotted,linecolor=orange](0,\n)(90,\n)}
\psaxes[linewidth=1.25pt,Dx=10,Dy=10]{->}(0,0)(90,45)
\psaxes[linewidth=1.25pt,Dx=10,Dy=10](0,0)(90,45)
\psline[linewidth=1.5pt](0,0)(20,10)(28,18)(66,38.5)(88,43)
\uput[u](78.5,0){Durée (min)}
\uput[u](0,45){Distance (km)}
\end{pspicture}
\end{center}

Pour les trois premières questions, les réponses seront données grâce à des lectures graphiques. Aucune justification n'est attendue sur la copie.

\medskip
 
\begin{enumerate}
\item Quelle distance Cédric a-t-il parcourue au bout de 20 minutes? 
\item Combien de temps a mis Cédric pour faire les 30 premiers kilomètres? 
\item Le circuit de Cédric comprend une montée, une descente et deux portions plates. 
Reconstituer dans l'ordre le trajet parcouru par Cédric. 
\item Calculer la vitesse moyenne de Cédric (exprimée en km/h) sur la première des quatre parties du trajet. 
\end{enumerate}

\vspace{0,5cm}

