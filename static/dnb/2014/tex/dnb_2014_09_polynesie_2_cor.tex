
\bigskip

%Pour choisir un écran de télévision, d'ordinateur ou une tablette tactile, on peut s'intéresser : 
%\setlength\parindent{5mm}
%\begin{itemize}
%\item[$\bullet~~$] à  son format qui est le rapport longueur de l'écran largeur de l'écran 
%\item[$\bullet~~$] à  sa diagonale qui se mesure en pouces. Un pouce est égal à  2,54~cm.
%\end{itemize}
%\setlength\parindent{0mm}
%
%\medskip
 
\begin{enumerate}
\item %Un écran de télévision a une longueur de 80~cm et une largeur de 45~cm.
 
%S'agit-il d'un écran de format $\dfrac{4}{3}$ ou $\dfrac{16}{9}$ ?
Le format est égal à  $\dfrac{80}{45} = \dfrac{5\times 16}{5\times 9} = \dfrac{16}{9}$. 
\item %Un écran est vendu avec la mention \og 15 pouces \fg. On prend les mesures suivantes : la longueur est 30,5~cm et la largeur est 22,9~cm. 

%La mention \og 15 pouces \fg{} est-elle bien adaptée à  cet écran ?
La diagonale de longueur $d$ vérifie :

$d^2 = 30,5^2 + 22,9^2 = 930,25 + 524,41 = \np{1456,66}$, soit $d = \sqrt{\np{1456,66}} \approx 38,14$~(cm), soit en pouces $d \approx \dfrac{38,14}{2,54} \approx 15,02$.

La mention \og 15 pouces \fg{} est bien adaptée à  cet écran. 
\item %Une tablette tactile a un écran de diagonale 7 pouces et de format $\dfrac{4}{3}$ 
%Sa longueur étant égale à  14,3~cm, calculer sa largeur, arrondie au mm près.
Si la largeur est $l$, on a $\dfrac{14,3}{l} = \dfrac{4}{3}$, soit $l = \dfrac{3 \times 14,3}{4} = 10,725$, soit environ 10,7~cm au millimètre près.

\emph{Remarque} :  On pouvait également traduire la longueur de la diagonale en cm et utiliser le théorème de Pythagore pour trouver la largeur. Avec cette méthode on trouve $l \approx 10,56$ soit environ 10,6~cm !
\end{enumerate}
 
\bigskip
 
