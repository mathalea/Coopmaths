
\medskip
 
Il existe différentes unités de mesure de la température : en France on utilise le degré Celsius (\degres C), aux Etats-Unis on utilise le degré Fahrenheit (\degres~F).

\medskip
 
Pour passer des degrés Celsius aux degrés Fahrenheit, on multiplie le nombre de départ par $1,8$ et on ajoute 32 au résultat.

\medskip
 
\begin{enumerate}
\item Qu'indiquerait un thermomètre en degrés Fahrenheit si on le plonge dans une casserole d'eau qui gèle ? On rappelle que l'eau gèle à 0~\degres C. 
\item Qu'indiquerait un thermomètre en degrés Celsius si on le plonge dans une 
casserole d'eau portée à $212$~\degres F ? Que se passe t-il ? 
\item 
	\begin{enumerate}
		\item Si l'on note $x$ la température en degré Celsius et $f(x)$ la température en degré Fahrenheit, exprimer $f(x)$ en fonction de $x$. 
		\item Comment nomme-t-on ce type de fonction ? 
		\item Quelle est l'image de $5$ par la fonction $f$ ? 
		\item Quel est l'antécédent de $5$ par la fonction $f$ ? 
		\item Traduire en terme de conversion de température la relation $f(10) = 50$.
	\end{enumerate} 
\end{enumerate}
