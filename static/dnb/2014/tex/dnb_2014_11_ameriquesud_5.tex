
\medskip 

L'oncle de Pauline participe régulièrement à une régate* organisée tous les ans sur le même plan d'eau. 

* régate : course de voiliers

\medskip 

En 2012, il a réalisé le parcours constitué de deux boucles courtes et de trois boucles longues en 8 heures et 40~minutes. 

Lors de sa participation en 2013, il lui a fallu 8 heures et 25~minutes pour achever le parcours constitué, cette année-là, de trois boucles courtes et de deux boucles longues. 

Il se souvient qu'il n'a parcouru aucune boucle en moins de 75~minutes. Il sait aussi qu'il lui a fallu, pour parcourir la boucle longue, 15~minutes de plus que pour la boucle courte. 

Cependant il souhaite connaître la durée nécessaire pour parcourir sur son voilier la boucle courte et la boucle longue.

\medskip 

\begin{enumerate}
\item Convertir en minutes les temps réalisés pour ces parcours de 2012 et 2013. 
\item Pauline a décidé, en utilisant un tableur, d'aider son oncle à déterminer les durées pour la boucle courte ainsi que pour la boucle longue. 

Une copie de l'écran obtenu est donnée ci-dessous. 

\begin{center}
\begin{tabularx}{\linewidth}{|c|*{7}{>{\centering \arraybackslash}X|}}\hline
 &A 	&B	& C &D	&E	& F	& G\\ \hline 
1& $x$	&75	&80	&85	&90	&95	&100\\ \hline 
2&$f(x)$&	&	&	&	&	& \\ \hline
3&$f(x)$&	&	&	&	&	& \\ \hline
4&$f(x)$&	&	&	&	&	& \\ \hline
5&		&	&	&	&	&& \\ \hline
\end{tabularx}
\end{center}
 
Elle a noté $x$ la durée en minutes pour la boucle courte. 
	\begin{enumerate}
		\item Quelle formule permettant d'obtenir la durée en minutes nécessaire au parcours de la boucle longue va-t-elle saisir dans la cellule B2 ? 
		\item Elle va saisir dans la cellule B3 la formule \og =2*B1+3*B2 \fg. 

Que permet de calculer cette formule ? 
		\item Quelle formule va-t-elle saisir dans la cellule B4 pour calculer le temps de parcours lors de sa participation en 2013 ? 

Elle a ensuite recopié vers la droite les formules saisies en B2,\: B3 et B4 et obtenu l'écran suivant : 
	\end{enumerate}

\begin{center}
\begin{tabularx}{\linewidth}{|c|*{7}{>{\centering \arraybackslash}X|}}\hline
 &A 	&B		& C &D	&E	& F	& G\\ \hline 
1& $x$	&75		&80	&85	&90	&95	&100\\ \hline 
2&$f(x)$&90		&95	&100&105&110&115 \\ \hline
3&$f(x)$&420	&445&470&495&520&545 \\ \hline
4&$f(x)$&405	&430&455&480&505&530 \\ \hline
5&		&		&	&	&	&& \\ \hline
\end{tabularx}
\end{center}

\item Si elle saisit le nombre 105 dans la cellule H1, quelles valeurs obtiendra-t-elle dans les cellules H2, H3 et H4 ? 
\item À l'aide de la copie de l'écran obtenu avec le tableur préciser les durées nécessaires à son oncle pour parcourir la boucle courte ainsi que pour parcourir la boucle longue. 
\end{enumerate}
 

\vspace{0,5cm}

