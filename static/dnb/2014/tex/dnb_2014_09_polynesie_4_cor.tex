
\bigskip

%La figure ci-dessous, qui n'est pas dessinée en vraie grandeur, représente un cercle $(C)$ et plusieurs segments. On dispose des informations suivantes :
%
%\parbox{0.5\linewidth}{\begin{itemize}
%\item[$\bullet~~$] [AB] est un diamètre du cercle $(C)$ de centre O et de rayon 7,5 cm. 
%\item[$\bullet~~$]K et F sont deux points extérieurs au cercle $(C)$. 
%\item[$\bullet~~$]Les segments [AF] et [BK] se coupent en un point T situé sur le cercle $(C)$. 
%\item[$\bullet~~$]AT = 12 cm, BT = 9 cm, TF = 4 cm, TK = 3 cm.
%\end{itemize} 
% 
%\begin{enumerate}
%\item Démontrer que le triangle ATB est rectangle. 
%\item Calculer la mesure de l'angle $\widehat{\text{BAT}}$ arrondie au degré près. 
%\item Les droites (AB) et (KF) sont-elles parallèles ? 
%\item Calculer l'aire du triangle TKF.
%\end{enumerate}} \hfill
%\parbox{0.45\linewidth}{\psset{unit=0.6cm}
%\begin{pspicture}(9,7)
%\pscircle(3.4,3){3}
%\pspolygon(8.7,3.4)(1.6,5.4)(5.2,0.6)(6.7,6)%FABKF
%\uput[dl](3.4,3){O} \uput[ul](1.6,5.4){A} \uput[dr](5.2,0.6){B} 
%\uput[d](8.7,3.4){F} \uput[u](6.7,6){K} \uput[dr](6.2,4.1){T} 
%\uput[l](0.6,4.3){$(C)$}
%\psdots[dotstyle=+,dotangle=45](3.4,3) 
%\end{pspicture}}
\begin{enumerate}
\item AB$^2 = 15^2 = 225$ ; AT$^2 + \text{BT}^2 = 12^2 + 9^2 = 144 + 81 = 225$.

On a $225 = 144 + 81$, soit AB$^2 = \text{AT}^2 + \text{BT}^2$, donc d'après la réciproque du théorème de Pythagore : le triangle ABT est rectangle en T, d'hypoténuse [AB].
\item Dans le triangle ABT est rectangle en T, on a par exemple $\cos \widehat{\text{BAT}} = \dfrac{\text{AT}}{\text{AB}} = \dfrac{12}{15} = \dfrac{4}{5} = \dfrac{8}{10} = 0,8$. La calculatrice donne $\widehat{\text{BAT}} \approx 36,86$ soit 37\degres au degré près.
\item On a $\dfrac{\text{AT}}{\text{TF}} = \dfrac{12}{4} = 3$ et $\dfrac{\text{BT}}{\text{TK}} = \dfrac{9}{3} = 3$.

On a donc $\dfrac{\text{AT}}{\text{TF}} = \dfrac{\text{BT}}{\text{TK}}$, donc d'après la réciproque du théorème de Thalès les droites (AB) et (FK) sont parallèles.
\item L'aire du triangle BAT est égale à  $\dfrac{\text{AT} \times \text{BT}}{2} = \dfrac{12 \times 9}{2} = 6 \times 9 =  54$~cm$^2$.

Les dimensions de TKF sont 3 fois plus petites que celles du triangle BAT, donc son aire est $3^2$ fois plus petite. 

L'aire du triangle TKF est donc égale à  $\dfrac{57}{3^2} = \dfrac{54}{9} = 6$~cm$^2$.
\end{enumerate}
\bigskip
 
