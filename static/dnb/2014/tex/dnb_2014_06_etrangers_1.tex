
\medskip
 
Voici une feuille de calcul obtenue à l'aide d'un tableur.

\medskip
 
Dans cet exercice, on cherche à comprendre comment cette feuille a été remplie. 

\begin{center}
\begin{tabularx}{0.6\linewidth}{|*{4}{>{\centering \arraybackslash}X|}}\hline
&A&B&C\\ \hline
1&216	&126&90\\ \hline
2&126	&90	&36\\ \hline
3&90	&36	&54\\ \hline
4&54	&36	&18\\ \hline
5&36	&18	&18\\ \hline
6&18	&18	&0\\ \hline
\end{tabularx}
\end{center}
\medskip

\begin{enumerate}
\item En observant les valeurs du tableau, proposer une formule à entrer dans la 
cellule C1, puis à recopier vers le bas. 
\item \textbf{Dans cette question, on laissera sur la copie toutes les traces de 
recherche. Elles seront valorisées.}

\medskip
 
Le tableur fournit deux fonctions MAX et MIN. À partir de deux nombres, MAX renvoie la valeur la plus grande et MIN la plus petite. (exemple MAX(23~;~12) = 23)

Quelle formule a été entrée dans la cellule A2, puis recopiée vers le bas? 
\item Que représente le nombre figurant dans la cellule C5, par rapport aux 
nombres 216 et 126 ? 
\item La fraction $\dfrac{216}{126}$ est-elle irréductible ? Si ce n'est pas le cas, la rendre  irréductible en détaillant les calculs.
\end{enumerate}
 
\vspace{0,5cm}

