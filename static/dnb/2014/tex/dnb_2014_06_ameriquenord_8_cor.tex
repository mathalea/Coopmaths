
\medskip 

Le débit moyen $q$ d'un fluide dépend de la vitesse moyenne $v$ du fluide et de l'aire de la section d'écoulement d'aire $S$. Il est donné par la formule suivante : 

\[q = S \times v\]
 
où $q$ est exprimé en m$^3$.s$^{-1}$ ; $S$ est exprimé en m$^2$ ; $v$ est exprimé en m.s$^{-1}$. 

Pour cette partie, on considérera que la vitesse moyenne d'écoulement de l'eau à travers la vantelle durant le remplissage est $v = 2,8$ m.s$^{-1}$.
 
La vantelle a la forme d'un disque de rayon $R = 30$cm. 

\medskip

\begin{enumerate}
\item \textit{L'aire exacte $A$, en m$^2$, de la vantelle est :} $A=\pi\times(30\text{cm})^2=900\pi\text{cm}^2=$\fbox{$0,09\pi\text{m}^2$}
\item \textit{Le débit moyen arrondi au millième de cette vantelle durant le 
remplissage vaut :} 

$q=0,09\pi~\text{m}^2\times 2,8\text{m.s}^{-1}=0,252\pi~\text{m}^3\text{.s}^{-1}$\fbox{$\approx 0,792~\text{m}^3\text{.s}^{-1}$} arrondi au millième.

\item \textit{Il faudra patienter pour le remplissage d'une écluse de capacité 756 m$^3$ pendant :}

$t=\dfrac{756~\text{m}^3}{0,252\pi~\text{m}^3\text{.s}^{-1}}$\text{$\approx 955$~s}, arrondi à la seconde.

\textit{Or $\dfrac{955}{60}\approx 15,9> 15$ soit plus de $15$~minutes.}
\end{enumerate}
	 
\vspace{0.5cm}

