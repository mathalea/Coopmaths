
\medskip 

Une commune souhaite aménager des parcours de santé sur son territoire. On fait deux propositions au conseil municipal, schématisées ci-dessous :

\setlength\parindent{8mm}
\begin{itemize}
\item[$\bullet~~$] le parcours ACDA
\item[$\bullet~~$] le parcours AEFA
\end{itemize}
\setlength\parindent{0mm}
 
Ils souhaitent faire un parcours dont la longueur s'approche le plus possible de 4 km.
 
Peux-tu les aider \`{a} choisir le parcours ? Justifie.

\medskip
 
\emph{Attention: la figure proposée au conseil municipal n'est pas \`{a} l'échelle, mais les codages et les dimensions données sont correctes.}

\medskip

\begin{tabularx}{\linewidth}{m{8cm}X} 
\psset{unit=1cm}
\begin{pspicture}(7.8,6)
%\psgrid
\rput(1,1){Départ et arrivée.}
\pspolygon(2.5,2)(2.5,5.4)(4.6,5.4)
\uput[dr](2.5,2){A}\uput[u](2.5,5.4){C}\uput[ur](4.6,5.4){D}
\pspolygon(2.5,2)(6.5,2.5)(7,1.3)
\psframe(2.5,5.4)(2.7,5.2)
\uput[ur](6.5,2.5){E}\uput[dr](7,1.3){F}
\uput[ur](5,2.32){E$'$}\uput[dr](5.38,1.54){F$'$}
\psline(5,2.32)(5.38,1.54)
\psline{->}(1,1.15)(2.4,1.9)
\rput(3.9,1){(E$'$F$'$) // (EF)}
\rput(3.9,0.2){L'angle $\widehat{\text{A}}$ dans le triangle AEF vaut 30\,\degres}
\end{pspicture}& AC = 1,4 km 

CD = 1,05 km
 
AE$'$ = 0,5 km

AE = 1,3 km 

AF = 1,6 km 

E$'$F$'$ = 0,4 km
\end{tabularx}

\vspace{0,5cm}

