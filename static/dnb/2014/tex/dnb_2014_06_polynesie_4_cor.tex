
\medskip 

%Deux affirmations sont données ci-dessous. Pour chacune des affirmations, indiquer si elle est vraie ou fausse. On rappelle que toutes les réponses doivent être justifiées.

%\medskip
 
\textbf{Affirmation 1} : %Les diviseurs communs à $12$ et $18$ sont les mêmes que les diviseurs de $6$.
Le plus grand commun diviseur à 12 et 18 est 6, donc tous les diviseurs communs à $12$ et $18$ sont les mêmes que les diviseurs de $6$. L'affirmation est vraie. 

\textbf{Affirmation 2} : %$\left(\sqrt{2}\right)^{50}$ 	et $\left(\sqrt{2}\right)^{100}$ 	sont des nombres entiers. 
$\left(\sqrt{2}\right)^{50} = \left(\sqrt{2}\right)^{2 \times 25} = \left(\sqrt{2}^2\right)^{25} = 2^{25}$ qui est un entier (produit de 25 facteurs tous égaux à 2 ;

$\left(\sqrt{2}\right)^{100} = \left(\sqrt{2}\right)^{2 \times 50} = \left(\sqrt{2}^2\right)^{50} = 2^{50}$ qui est un entier (produit de 50 facteurs tous égaux à 2.

L'affirmation est vraie.
 

