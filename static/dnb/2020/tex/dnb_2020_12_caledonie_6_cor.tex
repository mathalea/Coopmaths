\textbf{EXERCICE 6 : Les étiquettes \hfill 14 points}

\medskip
 
\begin{enumerate}
\item %Justifier que le nombre 102 est divisible par 3.
$\bullet~~$Comme $1 + 0 + 2 = 3$, 102 est un multiple de 3 (critère de divisibilité par 3 ;

$\bullet~~$$102 = 90 + 12 = 3\times 30 + 3 \times 4 = 3\times (30 + 4) = 3\times 34$.

102 est un multiple de 3  : il est divisible par 3.
\item On donne la décomposition en produits de facteurs premiers de 85 : $85 = 5 \times 17$.

%Décomposer 102 en produits de facteurs premiers.
On a vu que $102 = 3 \times 34 = 3 \times 2 \times 17 = 2 \times 3 \times 17$.
\item Donner 3 diviseurs non premiers du nombre 102.

$2 \times 3 = 6 \:;\:  2 \times 17 = 34 \:;\: 3 \times 17 = 51$ sont trois diviseurs de 102 non premiers. 
%\end{enumerate}

%Un libraire dispose d'une feuille cartonnée de 85 cm sur 102 cm.
%
%Il souhaite découper dans celle-ci, en utilisant toute la feuille, des étiquettes carrées. 
%
%Les côtés de ces étiquettes ont tous la même mesure.
%\begin{enumerate}[resume]
\item %Les étiquettes peuvent-elles avoir $34$ cm de côté ? Justifier. 
Si toute la feuille est utilisée c'est que la longueur et la largeur sont des multiples des côtés du carré. Ces côtés ont donc une  longueur $c$ qui divise à la fois 102 et 85.

Or 34 ne divise pas 85 (car 2 divise 34 mais ne divise pas 85). les étiquettes ne peuvent pas faire 34cm de côté.
\item %Le libraire découpe des étiquettes de $17$ cm de côté.

%Combien d'étiquettes pourra-t-il découper dans ce cas ?
Par contre 17 divise 85 ($85 = 5 \times 17$) et 17 divise 102 ($102 = 17 \times 6$).

Les étiquettes rentrent 5 fois en largeur et 6 fois en longueur : il y en aura donc $5 \times 6 = 30$ par feuille.

\emph{Remarque } : on peut aussi utiliser les aires.

Une étiquette a une aire de $17 \times 17 = 289$ et la feuille une aire de $85 \times 102 = \np{8670}$.

On pourra donc faire $\dfrac{\np{8670}}{289} = 30$ étiquettes dans une feuille.
\end{enumerate}

\vspace{0,5cm}

