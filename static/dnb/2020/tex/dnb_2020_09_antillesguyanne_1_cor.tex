
\medskip

%\parbox{0.49\linewidth}{La figure ci-contre est dessinée à main levée. On donne les informations suivantes :
%
%\begin{itemize}[label=\textbullet]
%\item ABC est un triangle tel que :
%AC = 10,4 cm, AB =4 cm et BC = 9,6 cm ;
%\item les points A, L et C sont alignés ;
%\item les points B, K et C sont alignés ;
%\item la droite (KL) est parallèle à la droite (AB) ;
%\item CK = 3~cm.
%\end{itemize}}\hfill
%\parbox{0.49\linewidth}{\psset{unit=1cm}
%\begin{pspicture}(6,2.8)
%%\psgrid
%\pslineByHand(0.5,0.5)(5.5,1)(4.75,2.5)(0.5,0.5)%CBA
%\uput[ur](4.75,2.5){A} \uput[d](5.5,1){B} \uput[l](0.5,0.5){C} \uput[ul](1.9,1.1){L} \uput[dl](2.2,0.65){K} 
%\pslineByHand(2.5,0)(1.5,2)%LK
%\end{pspicture}
%}
%\medskip

\begin{enumerate}
\item ~
%À l'aide d'instruments de géométrie, construire la figure en vraie grandeur sur la copie en laissant apparents les traits de construction.

\begin{center}
\psset{unit=1cm}
\begin{pspicture}(-0.5,-0.5)(11,6)
\psline[linecolor=blue](0,0.5)(9,5.7)%CA
\uput[dl](0,0.5){C}\uput[ur](9,5.7){A}
%\pscircle[linecolor=blue](4.5,3.1){5.2}
\psarc[linecolor=blue](0,0.5){9.6}{0}{20}\psarc[linecolor=blue](9,5.7){4}{-100}{-70}
\pspolygon[linecolor=blue](0,0.5)(9,5.7)(9.55,1.73)
\uput[r](9.55,1.73){B}
\psarc[linecolor=blue](0,0.5){3}{-10}{25}
\uput[dr](3,0.9){K}
\psline(2.45,4.87)(3,0.9)
\psline[linestyle=dotted,linewidth=1.5pt](3,0.9)(9,5.7)
\psline[linestyle=dotted,linewidth=1.5pt](9.55,1.73)(2.45,4.87)
\uput[ul](2.8,2.13){L}
\end{pspicture}
\end{center}

\item On a AC$^2 = 10,4^2 = 108,16$ ;

$\text{AB}^2 + \text{CB}^2 = 4^2 + 9,6^2 = 16 + 92,16= 108,16$.

On a donc $\text{AC}^2 = \text{AB}^2 + \text{CB}^2$ ; d'après la réciproque du théorème de Pythagore cette égalité montre que le triangle ABC est rectangle en B.

%Prouver que le triangle ABC est rectangle en B.


\item %Calculer la longueur CL en cm.

Puisque les droites (BC) et (KL) sont parallèles on a une configuration de Thalès.

Donc $\dfrac{\text{CK}}{\text{CB}} = \dfrac{\text{CL}}{\text{CA}} $ ou $\dfrac{3}{9,6} = \dfrac{\text{CL}}{10,4}$ ; on en déduit que CL $ = 10,4 \times \dfrac{3}{9,6} = 10,4 \times \dfrac{1}{3,2} = \dfrac{10,4}{3,2} = \dfrac{104}{32} = \dfrac{26}{8} = \dfrac{13}{4} = 3,25$~cm.
\item %À l'aide de la calculatrice, calculer une valeur approchée de la mesure de l'angle $\widehat{\text{CAB}}$, au degré près.

On a en utilisant par exemple le cosinus :

$\cos \widehat{\text{CAB}} = \dfrac{\text{AB}}{\text{AC}} = \dfrac{4}{10,4} \approx 0,385$.

La calculatrice donne $\widehat{\text{CAB}} \approx 67,4$, soit 67\degres{} au degré près.
\end{enumerate}

\bigskip

