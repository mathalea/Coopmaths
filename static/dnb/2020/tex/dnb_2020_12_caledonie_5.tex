\textbf{EXERCICE 5 : La corde \hfill 7 points}

\medskip

Le triangle ABC rectangle en B ci-dessous est tel que AB $= 5$ m et AC $= 5,25$ m.

\medskip

\begin{enumerate}
\item ~

\parbox{0.47\linewidth}{Calculer, en m, la longueur BC.

 Arrondir au dixième.}\hfill \parbox{0.5\linewidth}{
\psset{unit=1cm}
\begin{pspicture}(4.5,2.4)
\pspolygon(0.5,0.5)(4.2,0.5)(4.2,2.2)%ABC
\uput[l](0.5,0.5){A} \uput[r](4.2,0.5){B} \uput[ur](4.2,2.2){C}
\psframe(4.2,0.5)(4,0.7) 
\end{pspicture}}
\end{enumerate}

Une corde non élastique de $10,5$ m de long est fixée au sol par ses deux extrémités entre deux poteaux distants de $10$~m.

\begin{enumerate}[resume]
\item ~

\parbox{0.47\linewidth}{Melvin qui mesure 1,55 m pourrait-il passer sous cette corde sans se baisser en la soulevant par le milieu ?}\hfill
\parbox{0.5\linewidth}{
\psset{unit=1cm,arrowsize=2pt 3}
\begin{pspicture}(7.5,4.2)
\psline{<->}(0.4,0.5)(7.4,0.5)
\rput(3.5,2.5){\includegraphics[width=1.5cm]{Melvin}}
\psline(0.4,1)(0.4,0.5)\psline(7.4,1)(7.4,0.5)
\qdisk(0.4,1){1mm}\qdisk(7.4,1){1mm}
\pscurve(0.4,1)(1,1.2)(2,1.9)(3.3,2.3)(4,2.1)(5,1.7)(6,1.4)(7.4,1)
\uput[d](3.9,0.5){10~m}
\end{pspicture}
}

\textbf{Toute trace de recherche même non aboutie sera prise en compte dans la notation.}
\end{enumerate}

\vspace{0,5cm}

