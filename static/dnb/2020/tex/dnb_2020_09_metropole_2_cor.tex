
\medskip
 
% On considère le programme de calcul suivant :
% 
%\begin{center}
%\begin{tabularx}{0.65\linewidth}{|X|}\hline
%$\bullet~~$ Choisir un nombre;
%
%$\bullet~~$ Ajouter $7$ à ce nombre;
%
%$\bullet~~$ Soustraire $7$ au nombre choisi au départ;
%
%$\bullet~~$ Multiplier les deux résultats précédents;
%
%$\bullet~~$ Ajouter $50$.\\ \hline
%\end{tabularx}
%\end{center}

\medskip

\begin{enumerate}
\item %Montrer que si le nombre choisi au départ est 2, alors le résultat obtenu est $5$.
On a la suite de nombres : $2 \to 9$ et d'autre part $2 - 7 = - 5$ : leur produit est $9 \times (- 5) = - 45$. Enfin $- 45 + 50 = 5$.
\item %Quel est le résultat obtenu avec ce programme si le nombre choisi au départ est $-10$ ?
De même $- 10 \to - 3$ et d'autre part $- 10 - 7 = - 17$ ; d'où $(- 3) \times (- 17) = 51$. Enfin $51 + 50 = 101$.
\item %Un élève s'aperçoit qu'en calculant le double de $2$ et en ajoutant $1$, il obtient $5$, le même résultat que celui qu'il a obtenu à la question 1.

%Il pense alors que le programme de calcul revient à calculer le double du nombre de départ et à ajouter 1.

%A-t-il raison ?
Il a tort puisque d'après la question 2 $- 10$ donne 101. or $2 \times (- 10) + 1 = - 20 + 1 = - 19$. 
\item %Si $x$ désigne le nombre choisi au départ, montrer que le résultat du programme de calcul est $x^2 + 1$.
$x$ donne d'une part le premier facteur $x + 7$ et le second facteur est $x - 7$, donc leur produit  est $(x + 7)(x - 7) = x^2 - 49$ (identité remarquable).

Le résultat final est $x^2 - 49 + 50 = x^2 + 1$.
\item %Quel(s) nombre(s) doit-on choisir au départ du programme de calcul pour obtenir $17$ comme résultat ?
Il faut trouver $x$ tel que :

$x^2 + 1 = 17$, soit en ajoutant $- 1$ à chaque membre : $x^2 = 16$ ou $x^2 - 16 = 0$ ou $(x + 4)(x - 4) = 0$ ; ce produit étant nul si l'in des facteurs est nul, il y a deux solutions : $- 4$ et $4$.
\end{enumerate}

\bigskip

