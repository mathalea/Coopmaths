
\medskip

%Un sac opaque contient $120$ boules toutes indiscernables au toucher, dont 30 sont bleues. Les autres boules sont rouges ou vertes.
%
%On considère l'expérience aléatoire suivante :
%
%On tire une boule au hasard, on regarde sa couleur, on repose la boule dans le sac et on mélange.

%\medskip

\begin{enumerate}
\item %Quelle est la probabilité de tirer une boule bleue ? Écrire le résultat sous la forme d'une fraction irréductible.
Il y a 30 boules bleues sur 120 boules : la probabilité est donc égale à  $\dfrac{30}{120} = \dfrac{30 \times 1}{30 \times 4} = \dfrac{1}{4}$.
\item %Cécile a effectué $20$ fois cette expérience aléatoire et elle a obtenu $8$ fois une boule verte. Choisir, parmi les réponses suivantes, le nombre de boules vertes contenues dans le sac (aucune justification n'est demandée) :

%\medskip
%\begin{tabularx}{\linewidth}{*{4}{X}}
%\textbf{a.~~} $48$ &\textbf{b.~~} $70$ &\textbf{c.~~} On ne peut pas savoir &\textbf{d.~~} $25$
%\end{tabularx}
%\medskip
On ne peut pas savoir.
\item  %La probabilité de tirer une boule rouge est égale à  $0,4$.
	\begin{enumerate}
		\item %Quel est le nombre de boules rouges dans le sac ?
Si $r$ est le nombre de boules rouges dans le sac, on a :
		
$0,4 = \dfrac{r}{120}$ soit $r = 120 \times 0,4 = 48$.

Il y a 48 boules rouges.
		\item %Quelle est la probabilité de tirer une boule verte ?
		D'après le résultat précédent, il reste :
		
		$120 - (30 + 48) = 120 - 78 = 42$~boules vertes.
		
		La probabilité de tirer une boule verte est donc égale à  :
		
		$\dfrac{42}{120} = \dfrac{7 \times 6}{20\times 6} = \dfrac{7}{20} = \dfrac{7 \times 5}{20\times 5} = \dfrac{35}{100} = 0,35$.
	\end{enumerate}
\end{enumerate}

\vspace{0,5cm}

