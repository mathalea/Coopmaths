
\medskip

%Aurel, Alexandra, Nathalie et Eli sont des fans de jeux de société. Ils possèdent $60$ jeux différents. Un après-midi ils décident de jouer à un de leurs jeux. N'arrivant pas à se mettre d'accord, ils le choisissent au hasard parmi l'ensemble de leurs jeux.
%
%\medskip
%
%Dans ce tableau sont présentés les jeux préférés de chacun d'eux :
%
%\begin{center}
%\begin{tabularx}{\linewidth}{|*{4}{X|}}\hline
%\multicolumn{1}{c|}{Aurel}& \multicolumn{1}{c|}{Alexandra}& \multicolumn{1}{c|}{Nathalie}& \multicolumn{1}{c|}{Eli}\\ \hline
%Kemet 			&Epix 			&Fourberies &Hyperborea\\ \hline
%Pitch car 		&Colt express 	&Happy pigs &Cyclades\\ \hline
%Miniville 		&Happy pigs 	&			&Happy pigs\\ \hline
%King of Tokyo	&				&			&\\ \hline
%Bruxelle		&				&			&\\ \hline
%\end{tabularx}
%\end{center}

Les joueurs tirent un jeu au hasard parmi les 60 jeux qu'ils possèdent.

\medskip

\begin{enumerate}
\item %Quelle est la probabilité que le jeu tiré soit un des jeux préférés d'Aurel ?
La probabilité que le jeu tiré soit un des jeux préférés d'Aurel est égale à $\dfrac{5}{60} = \dfrac{1}{12}$.
\item %Quelle est la probabilité que le jeu tiré soit un des jeux préférés d'Alexandra ou Nathalie ?
La probabilité que le jeu tiré soit un des jeux préférés d'Alexandra ou Nathalie est égale à $\dfrac{3 + 2 - 1}{60} = \dfrac{4}{60} = \dfrac{1}{15}$.
\item %Ces quatre amis ont noté la durée, en minutes, de chaque partie jouée ce mois ci :

%\[72~~;~~35~~;~~48~~;~~52~~;~~26~~;~~55~~;~~43~~;~~105.\]
	\begin{enumerate}
		\item %Calculer la durée moyenne d'une partie.
La durée moyenne d'une partie est $\dfrac{72+35+48+52+26+55+43+105}{8} = \frac{436}{8} = 54,5$~min soit 54~min 30~s.
		\item %Calculer la médiane de la série ci-dessus.
La série ordonnée des durées est : $26~~;~~35~~;~~43~~;~~48~~;~~52~~;~~55~~;~~72~~;~~105$.

On peut prendre pour médiane toute durée comprise entre 48 et 52, soit 49, 50 ou 51.
		\item %Interpréter le résultat obtenu à la question b.
Il y a autant de chances qu'une partie dure moins de 50 minutes que le contraire.
	\end{enumerate}
\end{enumerate}

\vspace{0,5cm}

