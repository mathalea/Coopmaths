
\medskip

%L'entraîneur d'un club d'athlétisme a relevé les performances de ses lanceuses de poids
%sur cinq lancers. Voici une partie des relevés qu'il a effectués (il manque trois
%performances pour une des lanceuses) :
%
%\begin{center}
%\begin{tabularx}{\linewidth}{|m{2cm}|*{6}{>{\centering \arraybackslash}X|}}\cline{3-7}
%\multicolumn{2}{c|}{~}						&\multicolumn{5}{c}{Lancers}\\ \cline{3-7}
%\multicolumn{2}{c|}{~}						&\no 1 		&\no 2 	&\no 3 	&\no 4 	&\no 5\\\hline
%\multirow{3}{2cm}{Performances (en mètre)}	&Solenne 	&17,8 	&17,9 	&18 	&19,9 	&17,4\\ \cline{2-7}
%											&Rachida 	&17,9 	&17,6 	&18,5 	&18 	&19\\ \cline{2-7}
%											&Sarah 		&18 	&$?$ 	&19,5 	&$?$ 	&$?$\\ \hline
%\end{tabularx}
%\end{center}
%
%\medskip
%
%On connaît des caractéristiques de la série d'une des lanceuses :
%
%\begin{center}
%\begin{tabular}{|c|}\hline
%\textbf{Caractéristiques des cinq lancers :}\\\hline
%Étendue : 2,5 m\\ \hline
%Moyenne : 18,2 m\\ \hline
%Médiane : 18 m\\ \hline
%\end{tabular}
%\end{center}

\begin{enumerate}
\item %Expliquer pourquoi ces caractéristiques ne concernent ni les résultats de Solenne, ni
ceux de Rachida.
Pour Solenne l'étendue est égale à 2,1~m et pour Rachida elle est égale à 1,4~m. Donc les caractéristiques sont celle de Sarah.
\item %Les caractéristiques données sont donc celles de Sarah. Son meilleur lancer est
%de 19,5 m.

%Indiquer sur la copie quels peuvent être les trois lancers manquants de Sarah ?
Avec une étendue de 2,5~m et un meilleur lancer de 19,5~m son moins bon lancer est de $19,5 - 2,5 = 17$~m.

Puisque la médiane est 18, les deux lancers manquants sont l'un inférieur ou égal à 18 et l'autre supérieur ou égal à 18.

Si $a$ et $b$ sont ces deux lancers on doit comme moyenne :

$18,2 = \dfrac{18 + 17 + 19,5 + a + b}{5}$, soit $18,2 = \dfrac{54,5 + a + b}{5}$ ou $54,5 + a + b = 5 \times 18,2$ ou encore $54,5 + a + b = 91$, donc enfin $a + b = 36,5$.

On peut prendre deux nombres dont la somme est 36,5, les deux nombres étant entre 17 et 19,5.

Exemple : 17,5 et 19.
\end{enumerate}

\vspace{0,5cm}

