
\medskip

Un condensateur est un composant électronique qui permet de stocker de l'énergie électrique pour la restituer plus tard.

Le graphique suivant montre l'évolution de la tension mesurée aux bornes d'un condensateur en fonction du
temps lorsqu'il est en charge.

\begin{center}
\psset{xunit=14cm,yunit=0.8cm,comma=true}
\begin{pspicture}(-0.1,-1)(0.6,6)
\multido{\n=0.00+0.01}{61}{\psline[linewidth=0.1pt](\n,0)(\n,6)}
\multido{\n=0.0+0.1}{7}{\psline[linewidth=1pt](\n,0)(\n,6)}
\multido{\n=0.0+0.2}{31}{\psline[linewidth=0.1pt](0,\n)(0.6,\n)}
\multido{\n=0+1}{7}{\psline[linewidth=1pt](0,\n)(0.6,\n)}
\psaxes[linewidth=1.25pt,Dx=0.1]{->}(0,0)(0,0)(0.6,6)
\psaxes[linewidth=1.25pt,Dx=0.1](0,0)(0,0)(0.61,6)
\uput[d](0.55,-0.6){Temps (s)}
\rput{90}(-0.075,3){Tension (V)}
\psplot[plotpoints=3000,linewidth=1.25pt,linecolor=blue]{0}{0.6}{5 5 2.71828 x 10 mul  exp div sub}
\end{pspicture}
\end{center}


\begin{enumerate}
\item S'agit-il d'une situation de proportionnalité ? Justifier.
\item Quelle est la tension mesurée au bout de $0,2$~s ?
\item Au bout de combien de temps la tension aux bornes du condensateur aura-t-elle atteint 60\,\% de la tension maximale qui est estimée à 5 V ?
\end{enumerate}

\vspace{0,5cm}

