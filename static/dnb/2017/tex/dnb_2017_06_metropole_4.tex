
\medskip

Les panneaux photovoltaïques permettent de produire de l'électricité à partir du rayonnement solaire.

Une unité courante pour mesurer l'énergie électrique est le kilowatt-heure, abrégé en kWh.

\medskip

\begin{enumerate}
\item Le plus souvent, l'électricité produite n'est pas utilisée directement, mais vendue pour être distribuée dans le réseau électrique collectif. Le prix d'achat du kWh, donné en \textbf{centimes d'euro}, dépend du type d'installation et de sa puissance totale, ainsi que de la date d'installation des panneaux photovoltaïques. 

Ce prix d'achat du kWh est donné dans le tableau ci-dessous.

\smallskip

\emph{Tarifs d'un kWh en \textbf{centimes d'euros}}

\smallskip

\begin{center}
\begin{tabularx}{\linewidth}{|m{1.8cm}|*{5}{>{\centering \arraybackslash}X|}}\cline{3-6}
\multicolumn{2}{c|}{~}&\multicolumn{4}{|c|}{Date d'installation}\\\hline
Type d'installation		&Puissance totale&\scriptsize Du 01/01/15 au 31/03/15&\scriptsize du 01/04/15 au 30/06/15&\scriptsize du 01/07/15  au 30/09/15 &\scriptsize du 01/10/15 au 31/12/15\\ \hline
Type A 					&\footnotesize 0 à 9 kW 		&26,57 &26,17 &25,78 	&25,39\\ \hline
\multirow{2}{*}{Type B}	&\footnotesize 0 à 36 kW 	&13,46 &13,95 &14,7 	&14,4\\ \cline{2-6}
						&\scriptsize 36 à 100 kW 	&12,79 &13,25 &13,96 	&13,68\\ \hline
\multicolumn{6}{r}{\emph{Source : http://www.developpement-durable.gouv.fr}}\\
\end{tabularx}
\end{center}

En mai 2015, on installe une centrale solaire du type B, d'une puissance de 28 kW.

Vérifier que le prix d'achat de \np{31420}~kWh est d'environ \np{4383}~\euro.
\item ~

\parbox{0.6\linewidth}{Une personne souhaite installer des panneaux photovoltaïques
sur la partie du toit de sa maison orientée au sud. Cette partie est
grisée sur la figure ci-contre. Elle est appelée pan sud du toit.

La production d'électricité des panneaux solaires dépend de
l'inclinaison du toit.

Déterminer, au degré près, l'angle $\widehat{\text{ABC}}$ que forme ce pan sud du
toit avec l'horizontale.}\hfill
\parbox{0.36\linewidth}{
\psset{unit=0.8cm,arrowsize=2pt 5}
\begin{pspicture}(5,5)
%\psgrid
\pspolygon(0,0.5)(3.6,0.5)(3.6,2.6)(1.8,3.7)(0,2.6)
\psline(3.6,0.5)(4.6,1.3)(4.6,3.4)(3.6,2.6)
\pspolygon[fillstyle=solid,fillcolor=lightgray](4.6,3.4)(3.6,2.6)(1.8,3.7)(2.8,4.5)
\pspolygon(1.8,3.7)(0,2.6)(1.4,3.8)(2.8,4.5)
\psline[linewidth=0.5pt]{<->}(1.8,0.3)(3.6,0.3)\uput[d](2.7,0.3){4,5 m}
\psline[linewidth=0.5pt]{<->}(3.7,0.3)(4.7,1.1)\uput[d](4.7,0.9){7,5 m}
\psline[linewidth=0.5pt]{<->}(3.7,0.5)(3.7,2.6)\uput[l](3.7,1.55){4,8 m}
\psline[linewidth=0.5pt]{<->}(1.7,0.5)(1.7,3.6)\uput[l](1.7,2.05){7 m}
\psline[linestyle=dashed](1.8,0.5)(1.8,3.6)
\psline[linestyle=dashed](3.6,2.6)(1.8,2.6)
\uput[ul](1.8,3.6){A}\uput[dl](3.6,2.6){B}\uput[dr](1.8,2.6){C}
\rput(3.5,3.6){\small pan sud}
\rput(3.5,3.3){\small du toit}
\psline(2.1,2.6)(2.1,2.9)(1.8,2.9)
\end{pspicture}}
\item ~ 

\parbox{0.6\linewidth}{
		\textbf{a.} Montrer que la longueur AB est environ égale à $5$~m.
		
		\textbf{b.} Les panneaux photovoltaïques ont la forme d'un carré de $1$~m de côté.
		
Le propriétaire prévoit d'installer 20 panneaux.
		
Quel pourcentage de la surface totale du pan sud du toit sera alors
couvert par les panneaux solaires ? On donnera une valeur approchée du
résultat à 1\,\% près.
		
\textbf{c.}La notice d'installation indique que les panneaux doivent être accolés
les uns aux autres et qu'une bordure d'au moins $30$~cm de large doit être
laissée libre pour le système de fixation tout autour de l'ensemble des
panneaux.
		
Le propriétaire peut-il installer les $20$~panneaux prévus ?

}
		\hfill
\parbox{0.38\linewidth}{\psset{unit=1cm}
\begin{pspicture}(4,3.5)
%\psgrid
\psline(0,0)(0,2.8)(3,2.8)
\psline(0.4,0)(0.4,2.4)(3,2.4)
\psframe(0.4,0.4)(1,1)\psframe(0.4,1)(1,1.6)
\psframe(1,1)(1.6,1.6)
\psline[linestyle=dotted](0.85,0)(0.85,0.4)
\psline[linestyle=dotted](1.3,0.85)(2.2,0.85)
\psline[linestyle=dotted](2.3,1.85)(3,1.85)
\rput(2.4,3.1){Bordure}\rput(3.3,1){Panneau}
\psline(2.4,2.9)(1.4,2.6) \psline(2.6,1)(1.4,1.3)
\end{pspicture}
}		
\end{enumerate}

\vspace{0,5cm}

