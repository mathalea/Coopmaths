
\medskip

\begin{enumerate}
\item On a $\dfrac{50}{24,07} \approx  2,08$~(m/s). Pernille Blume nage à environ 2,08~m par seconde.

$6~(\text{km/h} = \dfrac{\np{6000}~\text{m}}{\np{3600}~(s)} \approx  1,67$~(m/s).

Marcher à 6 km/h correspond à parcourir environ $1,67$ m/s.

Pernille Blume se déplace plus rapidement en nageant que le marcheur.
\item 
	\begin{enumerate}
		\item $E = (3x + 8)^2 - 64$
		
$E = (3x)^2 + 2 \times 3x \times 8 + 8^2 - 64$
		
$E = 9x^2 + 48x + 64 - 64$
		
$E = 9x^2 + 48x$
		\item ~
		
\begin{tabularx}{\linewidth}{X X}
Méthode 1 &Méthode 2\\
$E = (3x + 8)^2 - 64$				&\\
$E = (3x + 8)^2 - 8^2$				&$E = 9x^2 + 48x$\\
$E = [(3x + 8) - 8][ (3x + 8) + 8]$	&$E = 3x \times 3x + 3x \times 16$\\
$E = 3x(3x + 16)$					&$E = 3x(3x + 16)$\\
\end{tabularx}
		\item Résoudre l'équation $(3x + 8)^2 - 64 = 0$ revient à résoudre l'équation
		
$3x(3x + 16) = 0$.

Un produit de facteurs est nul si au moins l'un de ses facteurs est nul.

Soit $3x = 0$ donc $x = 0$, 

soit $3x+16 = 0$ ou $3x = -16$ ou $x = - \dfrac{16}{3}$.

Les solutions de l'équation $(3x + 8)^2 - 64 = 0$ sont $- \dfrac{16}{3}$ et $0$.
	\end{enumerate}
\item  Je cherche $V$ tel que : $15 = 0,14 \times  V^2$, c'est-à-dire $V^2 = \dfrac{15}{0,14}$.

Ainsi, $V = \sqrt{\dfrac{15}{0,14}} \approx 10,35$~(m/s).

La vitesse d'un véhicule dont la distance de freinage est de 15 m sur route
mouillée est d'environ $10,35$~m/s.

\end{enumerate}

\bigskip

