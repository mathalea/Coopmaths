
\medskip

%Monsieur Chapuis souhaite changer le carrelage et les plinthes\footnote{Une plinthe est un élément décoratif de faible hauteur fixé au bas des murs le long du sol.} dans le salon de son appartement. Pour cela il doit acheter des carreaux, de la colle et des plinthes en bois qui seront clouées. Il dispose des documents suivants :
%
%\begin{center}
%\psset{unit=1cm}
%\begin{pspicture}(0,-1)(12,6)
%\rput(3,5.8){Document 1 : \textbf{plan}, la pièce correspond à  la partie grisée}
%\pspolygon[fillstyle=solid,fillcolor=lightgray](0,0)(0,5)(4,5)(7,3)(7,0)
%\psline[linewidth=1.75pt](1,0)(0,0)(0,5)(4,5)(7,3)(7,0)(2,0)
%\uput[l](0,2.5){5 m}\uput[u](2,5){4 m}\uput[r](7,1.5){3 m}
%\uput[d](4.5,0){5 m}\uput[d](0.5,0){1 m}
%\psline[linestyle=dashed](4,5)(7,5)(7,3)
%\psframe(7,5)(6.8,4.8)
%\uput[ul](0,5){A} \uput[u](4,5){B}\uput[r](7,3){C}
%\uput[dr](7,0){D}\uput[d](2,0){E} \uput[d](1,0){F}\uput[dl](0,0){G}
%\rput(1.5,1){Porte de}
%\rput(1.5,0.5){largeur 1 m}
%\rput(9.3,2.6){Le schéma ci-contre n'est} \rput(9.3,2){pas
%réalisé à  l'échelle}
%\end{pspicture}
%\end{center}
%
%\begin{center}
%\begin{tabularx}{\linewidth}{|X|p{0.4cm}|X|}\cline{1-1}\cline{3-3}
%\multicolumn{1}{|c|}{Document 2}		&&\multicolumn{1}{|c|}{Document 3}\\
%\textbf{Carrelage}						&&\textbf{Colle pour le carrelage}\\
%Taille d'un carreau : 50 cm $\times$ 50 cm&&\\
%Epaisseur d'un carreau : 0,9 cm			&&Conditionnement: sac de 25 kg\\
%Conditionnement: 1,25 m$^2$ par boîte	&&Rendement (aire que l'on peut coller) : 4 m$^2$ par sac\\
%Prix : 19,95 \euro{} par boîte			&&Prix : 22 \euro{} le sac\\
%\textbf{Plinthe}						&&\textbf{Paquet de clous pour les plinthes}\\
%Forme: rectangulaire de longueur 1~m	&&Prix: 5,50 \euro{} le paquet\\
%Vendue à  l'unité						&&\\
%Prix: 2,95  \euro{} la plinthe en bois	&&\\ \cline{1-1}\cline{3-3}
%\end{tabularx}
%\end{center}

\begin{enumerate}
\item 
	\begin{enumerate}
		\item %En remarquant que la longueur GD est égale à  7~m, déterminer l'aire du triangle BCH.
		Soit I le point de [AG] tel que GI $= 3$~(m).
On a $\mathcal{A}(ABCDG) = \mathcal(ICDG) + \mathcal(IABC) = 7 \times 3 + \dfrac{7 + 4}{2}\times (5 - 3) = 21 + 11 = 32~\left(\text{m}^2\right)$.
		
Or $\mathcal(AHDG) = 7 \times 5 = 35~\left(\text{m}^2\right)$. Donc 
		
$\mathcal{A}(BCH) = 35 - 32 = 3~\left(\text{m}^2\right)$
		\item %Montrer que l'aire de la pièce est 32 m$^2$.
Déjà  fait.
	\end{enumerate}
\item %Pour ne pas manquer de carrelage ni de colle, le vendeur conseille à  monsieur Chapuis de prévoir une aire supérieure de 10\,\% à  l'aire calculée à  la question 1.
On a $32 \times \dfrac{10}{100} = 3,2$ : il faut donc prévoir $32 + 3,2 = 35,2~\left(\text{m}^2\right)$	
%Monsieur Chapuis doit acheter des boîtes entières et des sacs entiers.
	
%Déterminer le nombre de boîtes de carrelage et le nombre de sacs de colle à  acheter.

Monsieur Chapuis doit donc acheter $\dfrac{35,2}{1.25} = 28,16$ boîtes, donc 29 boîtes.

Il doit aussi acheter $\dfrac{35,2}{4} = 8,8$~sacs, donc 9 sacs de colle.
\item %Le vendeur recommande aussi de prendre une marge de 10\,\% sur la longueur des plinthes.
On a d'après le théorème de Pythagore appliqué au triangle BHC rectangle en H :

$\text{BC}^2 = \text{BH}^2 + \text{HC}^2 = 3^2 + 2^2 = 13$, d'où BC $ = \sqrt{13}$.

La longueur des plinthes est donc :

$3 + 6 + 5  + 4 + \sqrt{13} = 18 + \sqrt{13} \approx 21,61$~(m).

Avec une marge de 10\,\%, il lui faut donc acheter $22,61 \times 1,10 = 23,771$, soit en fait 24 plinthes de 1~m.
	
%Déterminer le nombre total de plinthes que monsieur Chapuis doit acheter pour faire le tour de la pièce. 
	
%On précise qu'il n'y a pas de plinthe sur la porte.
\item %Quel est le montant de la dépense de monsieur Chapuis, sachant qu'il peut se contenter d'un paquet de clous ? Arrondir la réponse à  l'euro près.
La dépense est égale à  : $29 \times 19,95 + 9 \times 22 + 24 \times 2,95 + 5,50 = 852,85$~\euro.
\end{enumerate}

\vspace{0,5cm}

