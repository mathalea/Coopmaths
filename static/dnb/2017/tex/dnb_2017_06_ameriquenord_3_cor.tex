
\medskip

%Il y a dans une urne 12 boules indiscernables au toucher, numérotées de 1 à 12. On veut
%tirer une boule au hasard.
%
%\medskip

\begin{enumerate}
\item %Est-il plus probable d'obtenir un numéro pair ou bien un multiple de $3$ ?
Il y a 6 numéros pairs et 4 multiple de 3. Il est donc plus probable d'obtenir un numéro pair qu'un multiple de $3$.
\item %Quelle est la probabilité d'obtenir un numéro inférieur à $20$ ?
Tous les numéros sont inférieurs à 20 : la probabilité est donc égale à 1.
\item %On enlève de l'urne toutes les boules dont le numéro est un diviseur de $6$. On veut à
%nouveau tirer une boule au hasard.

%Expliquer pourquoi la probabilité d'obtenir un numéro qui soit un nombre premier
%est alors $0,375$.
Les diviseurs de 6 sont 1 ; \:2, \:3,\: et 6.

Sur les huit numéros restants seuls 5, \: 7 et  11 sont premiers.

La probabilité d'obtenir un numéro qui soit un nombre premier est donc égale à : $\dfrac{3}{8} = \dfrac{3\times 125}{8 \times 125} = \dfrac{375}{\np{1000}} = 0,375$.
\end{enumerate}

\vspace{0,5cm}

