
\medskip

%Dans une station de ski, les responsables doivent enneiger la piste de slalom avec de la neige artificielle. La neige artificielle est produite à  l'aide de canons à  neige. La piste est modélisée par un rectangle dont la largeur est 25 m et la longueur est 480~m.
%
%Chaque canon à  neige utilise 1 m$^3$ d'eau pour produire 2 m$^3$ de neige.
%
%Débit de production de neige : 30 m$^3$ par heure
%et par canon.

%\medskip

\begin{enumerate}
\item %Pour préparer correctement la piste de slalom, on souhaite produire une couche de neige artificielle de $40$~cm d'épaisseur.

%Quel volume de neige doit-on produire ? Quel sera le volume d'eau utilisé ?
La neige peut être modélisée par un parrallélépipède rectangle de dimensions : 480~m, 25~m et 0,40~m, dont le volume est :

$480 \times 25 \times 0,4 = \np{12000} \times 0,4 = \np{4800}$~m$^3$.

1~m$^3$ d'eau produit 2m$^3$ de neige : il faudra donc $\dfrac{\np{4800}}{2} = \np{2400}$m$^3$ d'eau.
\item %Sur cette piste de ski, il y a 7 canons à  neige qui produisent tous le même volume de neige.

%Déterminer la durée nécessaire de fonctionnement des canons à  neige pour produire les \np{4800} m$^3$ de neige souhaités. Donner le résultat à  l'heure près.
Chaque heure les canons produisent $7 \times 30 = 210$~m$^3$ de neige.

Ils devront fonctionner pendant :

$\dfrac{\np{4800}}{210} = \dfrac{480}{21} = \dfrac{160}{7} \approx 22,857$~(h)   soit environ 23~h.
\end{enumerate}

\vspace{0,5cm}

