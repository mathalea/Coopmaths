
\medskip

On obtient la pente d'une route en calculant le quotient du dénivelé (c'est-à-dire du déplacement vertical) par
le déplacement horizontal correspondant. Une pente s'exprime sous forme d'un pourcentage.

\medskip

\parbox{0.45\linewidth}{Sur l'exemple ci-contre, la pente de la route est :

\medskip

$\dfrac{\text{dénivelé}}{\text{déplacement horizontal}} =  \dfrac{15}{120} = 0,125 = 12,5\,\%$.}
\hfill \parbox{0.52\linewidth}{\psset{unit=0.75cm}
\begin{pspicture}(9,5.5)
\pspolygon(0.5,0.5)(8,0.5)(0.5,5)
\psframe(0.5,0.5)(0.8,0.8)
\uput[d](4.25,0.5){Déplacement horizontal : 120~m}
\rput{90}(0.2,2.75){Dénivelé = 15~m}
\rput{-32}(4.25,3.36){Route}
\end{pspicture}
}

\bigskip

Classer les pentes suivantes dans l'ordre décroissant, c'est-à-dire de la pente la plus forte à la pente la moins forte.

\begin{center}
\begin{tabularx}{\linewidth}{|m{6cm}|X|}\hline
Route descendant du château des  Adhémar, à Montélimar.
\vspace{0,75cm}&\psset{unit=0.8cm}\begin{pspicture}(-2.75,-1.2)(2.75,2.1)
\pspolygon[fillstyle=solid,fillcolor=red](2;-30)(2;90)(2;210)
\pspolygon[fillstyle=solid,fillcolor=white](1.5;-30)(1.5;90)(1.5;210)
\pspolygon[fillstyle=solid,fillcolor=black](1.5;-30)(-0.75,0.2)(1.5;210)
\rput{-24}(0.3,0){24\,\%}
 \end{pspicture} \\ \hline
Tronçon d'une route descendant du col 
 du Grand Colombier (Ain).\vspace{0,75cm}&\psset{unit=0.6cm}
\begin{pspicture}(9,6)
\pspolygon(0.5,0.5)(8,0.5)(0.5,5.5)
\psframe(0.5,0.5)(0.8,0.8)
\rput{90}(0.2,2.75){Dénivelé = 280~m}
\rput{-32}(4.25,3.36){Route : 1,5~km}
\end{pspicture} \\ \hline
Tronçon d'une route descendant de l'Alto
de l'Angliru (région des Asturies,
Espagne).\vspace{0,75cm}&\psset{unit=0.6cm}
\begin{pspicture}(9,5.5)
%\psgrid
\pspolygon(0.5,0.5)(8,0.5)(0.5,5)
\psframe(0.5,0.5)(0.8,0.8)
\uput[d](4.25,0.5){Déplacement horizontal : 146~m}
\rput{-32}(4.25,3.36){Route}
\pswedge[fillstyle=solid,fillcolor=lightgray](8,0.5){0.6cm}{150}{180}
\rput(6.4,0.85){\small $12,4\degres$}
\end{pspicture}\\ \hline
\end{tabularx}
\end{center}

\bigskip
 
