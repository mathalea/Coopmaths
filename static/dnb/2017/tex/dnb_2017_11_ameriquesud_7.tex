
\medskip 

\parbox{0.49\linewidth}{Le tableau ci-contre indique l'apport énergétique en kilocalories par gramme (kcal/g) de quelques nutriments.}\hfill
\parbox{0.49\linewidth}{\begin{tabularx}{\linewidth}{|X|X|}\hline
\multicolumn{2}{|c|}{\small Apport énergétique pour quelques nutriments}\\ \hline   
Lipides   	&9 kcal/g \\ \hline  
Protéines  	&4 kcal/g\\ \hline   
Glucides   	&4 kcal/g \\ \hline
\end{tabularx}  
}

\medskip

\begin{enumerate}
\item Un œuf de 50 g est composé de:

\setlength\parindent{10mm}
\begin{itemize}
\item 5,3 g de lipides ; 
\item 6,4 g de protéines ; 
\item 0,6 g de glucides ; 
\item 37,7 g d'autres éléments non énergétiques. 
\end{itemize}
\setlength\parindent{0mm}

Calculer la valeur énergétique totale de cet œuf en kcal. 
\end{enumerate}

\medskip

\begin{minipage}{6.5cm}
\begin{enumerate}[resume]
\item On a retrouvé une partie de l'étiquette d'une tablette de chocolat. 

Dans cette tablette de 200 g de chocolat, quelle est la masse de glucides ?
\end{enumerate}
\end{minipage}
\hfill
\begin{minipage}{6cm}
\psset{unit=1cm}
\begin{pspicture}(-1,-3)(6,2.6)
%\psgrid
\psclip{\pscurve(-1,-2.4)(2,-2.)(2.5,-1.4)(3,-0.8)(4,-1)(4.3,1)(4.8,1.4)(4.4,2.2)(3.5,2.3)(2,2.5)(0,2.4)(-0.5,2.5)(-1,0)(-1,-2.4)}

\rput(2,0){\begin{tabularx}{\linewidth}{|m{3.cm}|X|}\hline
Valeurs nutritionnelles moyennes&Pour 100 g de chocolat\\ \hline 
Valeur énergétique				&520 kcal\\ \hline 
Lipides							&30 g \\ \hline 
Protéines						&4,5 g\\ \hline  
Glucides						& \\ \hline 
Autres éléments non énergétiques  	&\\ \hline
\end{tabularx}}
\endpsclip
\end{pspicture}
\end{minipage}
