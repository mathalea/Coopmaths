
\medskip

%On considère la figure ci-dessous qui n'est pas représentée en vraie grandeur.
%
%Les points A, B et E sont alignés ainsi que les points C, B et D.
%
%\begin{center}
%\psset{unit=1cm}
%\begin{pspicture}(7,3.8)
%%\psgrid
%\pspolygon(0.5,1.8)(6.5,1.8)(6.5,3.7)(0.5,0.5)%AEDC
%\uput[ul](0.5,1.8){A} \uput[ul](2.9,1.8){B} \uput[dl](0.5,0.5){C} \uput[ur](6.5,3.7){D} \uput[r](6.5,1.8){E} 
%\psframe(0.5,1.8)(0.8,1.5) \psframe(6.5,1.8)(6.2,2.1)
%\end{pspicture}
%\end{center}
%
\begin{enumerate}
\item %Dans chacun des cas suivants, indiquer sur la copie la réponse qui correspond à la
%longueur du segment [AB] parmi les réponses proposées. 
%
%Aucune justification n'est attendue.
%
%\begin{center}
%\begin{tabularx}{\linewidth}{|c|*{4}{>{\centering \arraybackslash}X|}}\cline{2-5}
%\multicolumn{1}{c|}{~}	&\textbf{Données :}&\textbf{Réponse A}&\textbf{Réponse B}&\textbf{Réponse C}\\ \hline
%						&AC = 51 cm		&					&					&\\  
%1 						&CB = 85 cm 	&68 cm 				&99,1 cm 			&67,7 cm\\ 
%						&DE = 64 cm		&					&					&\\ \hline
%						&$\widehat{\text{ACB}} = 62\degres$	&					&\\ 
%Cas 2 					&CB  = 9 cm							&\footnotesize Environ 10,2 cm	&\small Environ 4,2 cm		&\small Environ 7,9 cm\\ 
%						&BE = 5 cm		&					&					&\\ \hline     
%						&AC = 8 cm		&					&					&\\ 
%Cas 3 					&BE = 7 cm 		&11,2 cm 			&10,6 cm 			&4,3 cm\\ 
%						&DE = 5 cm		&					&					&\\ \hline
%\end{tabularx}
%\end{center}
$\bullet~~$D'après le théorème de Pythagore dans le triangle rectangle ABC rectangle en A , on peut écrire :

$\text{AB}^2 + \text{AC}^2 = \text{BC}^2$, soit $\text{AB}^2 = \text{BC}^2 - \text{AC}^2 = 85^2 - 51^2 = (85 - 51) \times (85 + 51) = 34 \times 136 = 34 \times 4 \times 34 = 2^2 \times 34^2 = (2 \times 34)^2 = 68^2$. Donc AB $ = 68$~(cm).

$\bullet~~$On a $\widehat{\text{ABC}} = 90 - \widehat{\text{ACB}} = 90 - 62 = 28$\degres.

Donc $\text{AB} = \text{BC} \times \cos \widehat{\text{ABC}} = 9 \times \cos 28 \approx 7,9$~(cm).

$\bullet~~$D'après la propriété de Thalès :
$\dfrac{\text{AB}}{\text{BE}} = \dfrac{\text{AC}}{\text{DE}}$ soit $\dfrac{\text{AB}}{7} = \dfrac{8}{5}$, d'où $\text{AB} = 7 \times \dfrac{8}{5} = \dfrac{56}{5} = 11,2$~(cm).
\item Voir la question \textbf{1.}
%Pour l'un des trois cas uniquement, au choix, justifier la réponse sur la copie en rédigeant.
\end{enumerate}

\vspace{0,5cm}

