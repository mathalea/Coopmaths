
\medskip 

Madame Duchemin a aménagé un studio dans les combles de sa maison, ces combles ayant la forme d'un prisme droit avec comme base le triangle ABC isocèle en C. 

Elle a pris quelques mesures, au cm près pour les longueurs et au degré près pour les angles. Elle les a reportées sur le dessin ci-dessous représentant les combles, ce dessin n'est pas à l'échelle. 

\begin{center}
\psset{unit=1cm,arrowsize=2pt 4}
\begin{pspicture}(0,-0.5)(8,4.6)
%\psgrid
\newgray{mongris}{0.95}
\pspolygon[fillstyle=solid,fillcolor=mongris](4,0.5)(6.5,2.5)(4.5,2.5)(1.8,0.5)
\pspolygon(0.5,0.5)(5.3,0.5)(2.9,2.3)%ABC
\psline(5.3,0.5)(7.8,2.5)(5.4,4.3)(2.9,2.3)%BxxC(pente droite)
\psline(5.4,4.3)(2.8,2.5)(0.5,0.5)
\psline{<->}(0.5,0)(2.9,0)\uput[d](1.7,0){5 m}
\psline{<->}(2.9,0)(5.3,0)\uput[d](4.1,0){5 m}
\psline{<->}(5.6,0.5)(8.1,2.5)\rput(7.25,1.5){\small 8~m}
\uput[d](2.9,0.5){K}\psline(2.9,0.5)(2.9,2.3)
\psline[linestyle=dashed](4,0.5)(4,1.45)
\psline[linestyle=dashed](1.8,0.5)(1.8,1.45)
\psline[linestyle=dashed](6.5,2.5)(6.5,3.5)
\psline[linestyle=dashed](4.5,2.5)(4.5,3.5)
\psline[linestyle=dashed](5.4,2.5)(5.4,4.3)
\psline[linestyle=dashed](4.5,2.5)(2.8,2.5)
\psline[linestyle=dashed](4.5,2.5)(7.8,2.5)
\psframe(4,0.5)(4.2,0.7)\psframe(2.9,0.5)(3.1,0.7)\psframe(1.8,0.5)(2,0.7)
\rput{90}(3.8,1){\small 1,80~m}\rput{90}(2.7,1.4){\small 2,90~m}
\rput(4.7,0.7){\small 30\degres}
\uput[dl](0.5,0.5){A}\uput[dr](5.3,0.5){B}\uput[ul](2.9,2.2){C}
\uput[ur](4,1.35){J}\uput[d](4,0.5){H}
\psarc(5.3,0.5){0.25}{150}{180}
\end{pspicture}
\end{center}

Madame Duchemin souhaite louer son studio. 

Les prix de loyer autorisés dans son quartier sont au maximum de 20~\euro{} par m$^2$ de surface habitable. 

Une surface est dite habitable si la hauteur sous plafond est de plus de 1,80~m (article R$111-2$ du code de construction) : cela correspond à la partie grisée sur la figure. 

Madame Duchemin souhaite fixer le prix du loyer à 700 \euro. 

Peut-elle louer son studio à ce prix ? 

\vspace{0,5cm}

