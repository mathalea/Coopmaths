
\medskip

%Sur une feuille de calcul, on a reporté le classement des dix premiers pays, par le
%nombre de médailles, aux Jeux Olympiques de Rio en 2016.
%\begin{center}
%\begin{tabularx}{\linewidth}{|>{\columncolor[gray]{0.85}}c|c|c|*{4}{>{\centering \arraybackslash}X|}}\hline
%\rowcolor[gray]{0.85}&A &B &C &D &E &F\\ \hline
%1&Rang 	&Pays 			&Or &Argent &Bronze &Total\\ \hline
%2&1 	&Etats-Unis 	&46 &37 &38 &121\\ \hline
%3&2		&Grande Bretagne&27 &23 &17 &67\\ \hline
%4&3		&Chine 			&26 &18 &26 &70\\ \hline
%5&4 	&Russie 		&19 &18 &19 &56\\ \hline
%6&5 	&Allemagne 		&17 &10 &15 &42\\ \hline
%7&6 	&Japon 			&12 &8 	&21 &41\\ \hline
%8&7 	&France 		&10 &18 &14 &42\\ \hline
%9&8 	&Corée du Sud 	&9 	&3 	&9 	&21\\ \hline
%10&9  	&Italie 		&8 	&12 &8 	&28\\ \hline
%11& 10 	&Australie 		&8 	&11 &10 &29\\ \hline
%\end{tabularx}
%\end{center}
%
%\medskip

\begin{enumerate}
\item %Quelle formule, parmi les trois proposées, a été saisie dans la cellule F2 de
%cette feuille de calcul, avant qu'elle soit étirée vers le bas ?
%\begin{center}
%\begin{tabularx}{\linewidth}{|*{3}{>{\centering \arraybackslash}X|}}\hline
%Formule A &Formule B &Formule C\\ \hline
%$=46+37+38$ &=SOMME(C2 :\:E2) &C2+D2+E2\\ \hline
%\end{tabularx}
%\end{center}
=SOMME(C2 :\:E2)
\item  %On observe la série des nombres de médailles d'or de ces dix pays.
	\begin{enumerate}
		\item %Quelle est l'étendue de cette série ?
L'étendue de la série des nombres de médailles d'or de ces dix pays est égale à $46 - 8 = 38$.
		\item %Quelle est la moyenne de cette série ?
La moyenne de la série des nombres de médailles d'or de ces dix pays est :
		
$\dfrac{46 + 27 + 26 + \cdots + 8}{10} = \dfrac{182}{10} = 18,2 \approx 18$.
 	\end{enumerate}
\item  %Quel est le pourcentage de médailles d'or remportées par la France par rapport
%à son nombre total de médailles ? Arrondir le résultat au dixième de \,\%.
Le rapport pour la France est égal à $\dfrac{10}{42} = \dfrac{5}{21} \approx 0,238$, soit 23,8\,\%.
\item  %Le classement aux Jeux Olympiques s'établit selon le nombre de médailles d'or
%obtenues et non selon le nombre total de médailles. Pour cette raison, la France
%avec 42 médailles se retrouve derrière le Japon qui n'en a que 41. En observant
%l'Italie et l'Australie, établir la règle de classement en cas d'égalité sur le nombre
%de médailles d'or.
À égalité pour les médailles d'or l'Italie est classée avant l'Australie car elle a eu plus de médailles d'argent.
%\item  Un journaliste sportif propose une nouvelle procédure pour classer les pays:
%chaque médaille d'or rapporte 3 points, chaque médaille d'argent rapporte 2
%points et chaque médaille de bronze rapporte 1 point. Dans ces conditions, la
%France dépasserait-elle le Japon ?
Avec ce nouveau barème le Japon aurait :

$12 \times 3 + 8 \times 2 + 21 \times 1 = 36 + 16 + 21 = 73$ (points) et la France :

$10 \times 3 + 18 \times 2 + 14 \times 1 = 30 + 36 + 14 = 80$ (points).

Avec ce barème la France devancerait le Japon.
\end{enumerate}

\vspace{0,5cm}

