
\medskip

Aurel, Alexandra, Nathalie et Eli sont des fans de jeux de société. Ils possèdent $60$ jeux différents. Un après-midi ils décident de jouer à un de leurs jeux. N'arrivant pas à se mettre d'accord, ils le choisissent au hasard parmi l'ensemble de leurs jeux.

\medskip

Dans ce tableau sont présentés les jeux préférés de chacun d'eux :

\begin{center}
\begin{tabularx}{\linewidth}{|*{4}{X|}}\hline
\multicolumn{1}{c|}{Aurel}& \multicolumn{1}{c|}{Alexandra}& \multicolumn{1}{c|}{Nathalie}& \multicolumn{1}{c|}{Eli}\\ \hline
Kemet 			&Epix 			&Fourberies &Hyperborea\\ \hline
Pitch car 		&Colt express 	&Happy pigs &Cyclades\\ \hline
Miniville 		&Happy pigs 	&			&Happy pigs\\ \hline
King of Tokyo	&				&			&\\ \hline
Bruxelle		&				&			&\\ \hline
\end{tabularx}
\end{center}

Les joueurs tirent un jeu au hasard parmi les 60 jeux qu'ils possèdent.

\medskip

\begin{enumerate}
\item Quelle est la probabilité que le jeu tiré soit un des jeux préférés d'Aurel ?
\item Quelle est la probabilité que le jeu tiré soit un des jeux préférés d'Alexandra ou Nathalie ?
\item Ces quatre amis ont noté la durée, en minutes, de chaque partie jouée ce mois ci :

\[72~~;~~35~~;~~48~~;~~52~~;~~26~~;~~55~~;~~43~~;~~105.\]
	\begin{enumerate}
		\item Calculer la durée moyenne d'une partie.
		\item Calculer la médiane de la série ci-dessus.
		\item Interpréter le résultat obtenu à la question b.
	\end{enumerate}
\end{enumerate}

\vspace{0,5cm}

