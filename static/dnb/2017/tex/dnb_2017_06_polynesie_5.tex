
\medskip

On considère le programme de calcul suivant :

\begin{center}
\begin{tabularx}{0.4\linewidth}{|lX|}\hline
$\bullet~~$& Choisir un nombre ;\\
$\bullet~~$& Le multiplier par - 4 ;\\
$\bullet~~$& Ajouter 5 au résultat.\\ \hline
\end{tabularx}
\end{center}

\begin{enumerate}
\item Vérifier que lorsque l'on choisit $- 2$ avec ce programme, on obtient 13.
\item Quel nombre faut-il choisir au départ pour obtenir $- 3$ ?
\item Salomé fait exécuter le script suivant:

\begin{center}
\textbf{Script}

\begin{scratch}
\blockinit{Quand \greenflag cliqué}
\blocksensing{demander \txtbox{Choisir un nombre} et attendre}
\blockifelse{si \boolsensing{- 4 * réponse + 5 < 0} alors}
{\blocklook{dire \txtbox{Bravo}}
}
{\blocklook{dire \txtbox{Essaie encore}}}
\end{scratch}

\end{center}

	\begin{enumerate}
		\item Quelle sera la réponse du lutin si elle choisit le nombre $12$ ?
		\item Quelle sera la réponse du lutin si elle choisit le nombre $- 5$ ?
 	\end{enumerate}
\item  Le programme de calcul ci-dessus peut se traduire par l'expression littérale
$- 4x + 5$ avec $x$ représentant le nombre choisi.

Résoudre l'inéquation suivante : $- 4x + 5 < 0$
\item  À quelle condition, portant sur le nombre choisi, est-on certain que la réponse
du lutin sera \og Bravo \fg{} ?
\end{enumerate}

\bigskip

