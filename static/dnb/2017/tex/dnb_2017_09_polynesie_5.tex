
\medskip

L'épreuve du marathon consiste à parcourir le plus rapidement possible la distance de
42,195 km en course à pied. Cette distance se réfère historiquement à l'exploit effectué
par le Grec Phillipidès, en 490 av. J-C, pour annoncer la victoire des Grecs contre
les Perses. Il s'agit de la distance entre Marathon et Athènes.

\medskip

\begin{enumerate}
\item En 2014, le kényan Dennis Kimetto a battu l'ancien record du monde en
parcourant cette distance en 2~h 2~min 57~s. Quel est alors l'ordre de grandeur
de sa vitesse moyenne : 5 km/h,\: 10 km/h ou 20 km/h ?
\item Lors de cette même course, le britannique Scott Overall a mis 2 h 15 min
pour réaliser son marathon. Calculer sa vitesse moyenne en km/h. Arrondir la
valeur obtenue au centième de km/h.
\item Dans cette question, on considérera que Scott Overall court à une vitesse
constante. Au moment où Dennis Kimetto franchit la ligne d'arrivée,
déterminer:
	\begin{enumerate}
		\item le temps qu'il reste à courir à Scott Overall ;
		\item la distance qu'il lui reste à parcourir. Arrondir le résultat au mètre près.
	\end{enumerate}
\end{enumerate}

\vspace{0,5cm}

