
\medskip

\parbox{0.65\linewidth}{Pour gagner le gros lot à une kermesse, il faut d'abord tirer une
boule rouge dans une urne, puis obtenir un multiple de 3 en
tournant une roue de loterie numérotée de 1 à 6.

L'urne contient 3 boules vertes, 2 boules bleues et 3 boules rouges.

\begin{enumerate}
\item Sur la roue de loterie, quelle est la probabilité d'obtenir un
multiple de $3$ ?
\item Quelle est la probabilité qu'un participant gagne le gros lot ?
\end{enumerate}}\hfill
\parbox{0.3\linewidth}{\psset{unit=0.85cm}
\begin{pspicture}(-2.5,-2.5)(2.8,2.5)
\pscircle(0,0){2.5}
\multido{\n=0+60,\na=30+60,\nb=1+1}{6}{\psline(2.5;\n)\rput(1.5;\na){\nb}}
\rput{-35}(2.6;-35){$\blacktriangleleft$}
\end{pspicture}
}

\begin{enumerate}[resume]
\item On voudrait modifier le contenu de l'urne en ne changeant que le nombre de boules rouges.

Combien faudra-t-il mettre en tout de boules rouges dans l'urne pour que la probabilité de
tirer une boule rouge soit de $0,5$. 

Expliquer votre démarche.
\end{enumerate}

\vspace{0,5cm}

