
\medskip

\begin{enumerate}
\item Cette expérience aléatoire n'a que deux issues : boule verte et boule bleue.

La somme des probabilités des issues d'une expérience aléatoire est égale à 1.

Donc, $p(\text{obtenir une boule bleue}) = 1 - p(\text{obtenir une boule verte}) = 1 - \dfrac{2}{5} = \dfrac{5}{5} - \dfrac{2}{5}  =\dfrac{3}{5} = 0,6$.
\item Chaque tirage est indépendant du précédent, les probabilités des différentes
issues ne sont pas modifiées, Paul aura toujours 3 chances sur 5 d'obtenir une
boule bleue.
\item
 
\textbf{Méthode 1 :}
 
$\dfrac{2}{5}$ du nombre total de boules représente 8 boules, je calcule donc
5
$\dfrac{8}{2} \times  3 = 4 \times 3 = 12$.
 
Il y a 12 boules bleues dans l'urne.
 
\textbf{Méthode 2 :}
 
$\dfrac{2}{5} = \dfrac{2\times 4}{5\times 4} = \dfrac{8}{20}$ et $20 - 8 = 12$\ldots 
\end{enumerate}

\bigskip

