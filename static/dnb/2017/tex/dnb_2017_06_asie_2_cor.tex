
%\medskip
%
%Dans une classe de 24 élèves, il y a 16 filles.

\medskip

\begin{enumerate}
\item %L'un des deux diagrammes ci-dessous peut-il représenter correctement la répartition des élèves de cette classe ?
Dans le diagramme 1 il y a autant de garçons que de filles : il n'est pas correct.

Dans le diagramme 2, le rectangle est partagé en 8 triangles de même aire.

La proportion de filles dans la classe est $\dfrac{16}{24}$.

Sur ce diagramme 2 les filles sont représentées par $\dfrac{5}{8} = \dfrac{15}{24}$.

$15\ne 16$, donc le diagramme 2 n'est pas correct.
\item %On a représenté la répartition des élèves de cette classe par un diagramme
%circulaire.

%\begin{center}
%\psset{unit=1cm}
%\begin{pspicture}(6,2.5)
%\psframe(0,1.3)(0.4,1.7)
%\psframe[fillstyle=solid,fillcolor=lightgray](0,0.7)(0.4,1.1)
%\uput[r](0.7,1.5){Garçons}
%\uput[r](0.7,0.9){Filles}
%\pscircle(4.5,1.25){1.25}
%\pswedge[fillstyle=solid,fillcolor=lightgray](4.5,1.25){1.25}{-150}{90}
%\end{pspicture}
%\end{center}
%
%Écrire le calcul permettant de déterminer la mesure de l'angle du secteur qui représente
%les garçons.
Le secteur plein a une mesure de 360 \degres.

Il y a dans la classe 8 garçons, soit une proportion de :

$\dfrac{8}{24} = \dfrac{1}{3} =  \dfrac{120}{360}$. L'angle au centre du secteur garçons a pour mesure 120\degres.
\end{enumerate}

\vspace{0,5cm}

