
\medskip

\emph{Pour illustrer l'exercice, la figure ci-dessous a été faite à main levée.}

\begin{center}
\psset{unit=1cm}
\begin{pspicture}(8,5)
%\psgrid
\pslineByHand(0.5,2.5)(7.5,4)
\pslineByHand(2,0.5)(6,4.8)
\pslineByHand(2,0.5)(0.5,2.5)
\pslineByHand(6,4.8)(7.5,4)
\pslineByHand(3,3)(4,2.6)
\uput[ul](4.75,3.4){A}\uput[r](7.5,4){B}\uput[u](6,4.8){C}
\uput[l](0.5,2.5){D} \uput[d](2,0.5){E}\uput[ul](3,3){F}\uput[dr](3.8,2.6){G}
\rput{-50}(1.2,1.2){8,1 cm}\rput{-27}(3.3,2.6){3 cm}
\rput{46}(3.2,1.4){6,8 cm}\rput{45}(4.4,2.78){4 cm}
\rput{15}(3.6,3.4){5 cm}\rput{45}(5.2,4.3){5 cm}
\rput{15}(6,3.4){6,25 cm}
\end{pspicture}
\end{center}

Les points D, F, A et B sont alignés, ainsi que les points E, G, A et C.

De plus, les droites (DE) et (FG) sont parallèles.

\medskip

\begin{enumerate}
\item Montrer que le triangle AFG est un triangle rectangle.
\item  Calculer la longueur du segment [AD]. En déduire la longueur du segment [FD].
\item  Les droites (FG) et (BC) sont-elles parallèles ? Justifier.
\end{enumerate}

\vspace{0,5cm}

