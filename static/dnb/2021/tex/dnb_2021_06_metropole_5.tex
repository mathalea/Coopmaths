
\medskip

La production annuelle de déchets par Français était de 5,2 tonnes par habitant en 2007.

Entre 2007 et 2017, elle a diminué de 6,5\,\%.

\medskip

\begin{enumerate}
\item De combien de tonnes la production annuelle de déchets par Français en 2017 a-t-elle diminué par rapport à l'année 2007 ?
\item Pour continuer à diminuer leur production de déchets de nombreuses familles utilisent désormais un composteur.

Une de ces familles a choisi le modèle ci-dessous, composé d'un pavé droit et d'un prisme droit

(la figure du composteur n'est pas à l'échelle). Le descriptif indique qu'il a une contenance d'environ 0,5 m$^3$,

On souhaite vérifier cette information.

\begin{center}
\psset{unit=0.5cm,arrowsize=2pt 3}
\begin{pspicture}(0,-0.75)(27,12)
%\psgrid
\psframe(0.5,0.5)(5.5,6.5)
\uput[dl](5.5,6.5){C}\uput[r](8.2,8.8){B}\uput[ur](8.2,10.5){A}
\uput[u](6.3,9.1){D}\uput[d](6.3,7.1){H}
\psline(5.5,0.5)(8.2,2.8)(8.2,8.8)(5.5,6.5)
\psline(8.2,8.8)(8.2,10.5)(6.3,9.1)(6.3,7.1)%BADH
\psline(6.3,9.1)(5.5,6.5)
\psline(0.5,6.5)(1.3,9.1)(6.3,9.1)
\psline(1.3,9.1)(3.2,10.5)(8.2,10.5)
\psline[linewidth=0.4pt]{<->}(9.1,2.8)(9.1,10.5)\uput[r](9.1,6.75){1,1 m}
\psline[linewidth=0.4pt]{<->}(1,9.1)(2.9,10.5)\uput[ul](2.4,9.8){39 cm}
\psline[linewidth=0.4pt]{<->}(6,0.5)(8.7,2.8)\uput[dr](7.35,1.65){67 cm}
\psframe(5.5,0.5)(5.1,0.9)
\psline(7.9,10.2)(7.9,9.8)(8.2,10)
\psline(7.9,8.6)(7.9,8.2)(8.2,8.4)
\psline(7.9,2.6)(7.9,3)(8.2,3.3)
\psline(5.5,0.9)(5.8,1.2)(5.8,0.7)
\pspolygon[fillstyle=solid,fillcolor=lightgray](1.2,7)(1.7,8.6)(5.7,8.6)(5.15,7)
%%%%%%
\pspolygon(13,0.5)(24,0.5)(24,7.5)(17.5,7.5)%CBAD
\uput[dl](13,0.5){C} \uput[dr](24,0.5){B} \uput[ur](24,7.5){A} \uput[ul](17.5,7.5){D} 
\psline(17.5,7.5)(17.5,0.5)\uput[d](17.5,0.6){H}
\psframe(17.5,0.5)(17.2,0.8)\psframe(24,7.5)(23.7,7.2)\psframe(24,0.5)(23.7,0.8)
\psline[linewidth=0.4pt]{<->}(25,0.5)(25,7.5)\rput{90}(25.3,4){Hauteur}
\psline[linewidth=0.4pt]{<->}(12.5,0.5)(17,7.5)\uput[ul](14.75,4){53 cm}
\psline[linewidth=0.4pt]{<->}(24,8)(17.5,8)\uput[u](20.75,8){Petit côté : 39 cm}
\rput(20,75,9.5){Trapèze ABCD}
\psline[linewidth=0.4pt]{<->}(13,-0.2)(24,-0.2)
\uput[d](18.5,-0.2){Grand côté : 67 cm}
\end{pspicture}
\end{center}
%39cm 
%
%
%
%Trapèze ASCO Petit côté: 39 cm
%
%l
%./
%c
%53 cm
%1 . Hauteur
%67 cm
%CHB
%
%
%Grand côté: 67 cm

	\begin{enumerate}
		\item Dans le trapèze ASCO, calculer la longueur CH.
		\item Montrer que la longueur DH est égale à 45~cm.
		\item Vérifier que l'aire du trapèze ASCO est de \np{2385}~cm$^2$.
		\item Calculer le volume du composteur.
		
L'affirmation \og il a une contenance d'environ 0,5 m$^3$ \fg{} est-elle vraie ? Justifier.
	\end{enumerate}
\end{enumerate}	
\medskip

\textbf{Rappels :}

\begin{tabularx}{\linewidth}{|X|}\hline
Aire du trapèze $ = \dfrac{(\text{Petit côté} + \text{Grand côté}) \times \text{Hauteur}}{2}$\rule[-3mm]{0mm}{9mm}\\
Volume du prisme droit $=$ Aire de la base $\times$  hauteur \\
Volume du pavé droit $= \text{Longueur} \times \text{largeur} \times  \text{hauteur}$.\\ \hline
\end{tabularx}

