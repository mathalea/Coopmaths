
\medskip

En cours d'éducation physique et sportive (EPS), les 24 élèves d'une classe de troisième pratiquent la course de fond.

Les élèves réalisent le test de demi-Cooper : ils doivent parcourir la plus grande distance possible en six minutes. 

Chaque élève calcule ensuite sa vitesse moyenne sur cette course. Le résultat obtenu est appelé VMA (Vitesse Maximale Aérobie).

\medskip

\begin{enumerate}
\item Après son échauffement, Chloé effectue ce test de demi-Cooper, Elle parcourt \np{1000} mètres en 6 minutes. 

Montrer que sa VMA est égale à 10~km/h.
\item L'enseignante a récolté les résultats et a obtenu les documents 1 et 2 ci-dessous : 

\psset{xunit=1cm,yunit=0.6cm}
\begin{center}
\scalebox{0.8}{
\begin{pspicture}(-1,-4)(11,17)
\multido{\n=.0+0.5}{31}{\psline[linewidth=0.1pt](0,\n)(11,\n)}
\psaxes[linewidth=1.25pt,labelFontSize=\scriptstyle,Dx=20](0,0)(0,0)(11,15)
\rput[rC]{45}(0.5,-0.2){Laura}\rput[rC]{45}(1.5,-0.2){Lisa}\rput[rC]{45}(2.5,-0.2){Claire}
\rput[rC]{45}(3.5,-0.2){Énora}\rput[rC]{45}(4.5,-0.2){Inès}\rput[rC]{45}(5.5,-0.2){Lou}
\rput[rC]{45}(6.5,-0.2){Chloé}\rput[rC]{45}(7.5,-0.2){Jeanne}\rput[rC]{45}(8.5,-0.2){Fatima}
\rput[rC]{45}(9.5,-0.2){Alexandra}\rput[rC]{45}(10.5,-0.2){Léonie}
\rput(6,16){Document 1 : VMA (en km/h) des filles}
\psframe[fillstyle=solid,fillcolor=gray](0.3,0)(0.7,10)
\psframe[fillstyle=solid,fillcolor=gray](1.3,0)(1.7,12.5)
\psframe[fillstyle=solid,fillcolor=gray](2.3,0)(2.7,13.5)
\psframe[fillstyle=solid,fillcolor=gray](3.3,0)(3.7,11.5)
\psframe[fillstyle=solid,fillcolor=gray](4.3,0)(4.7,13.5)
\psframe[fillstyle=solid,fillcolor=gray](5.3,0)(5.7,13.5)
\psframe[fillstyle=solid,fillcolor=gray](6.3,0)(6.7,10)
\psframe[fillstyle=solid,fillcolor=gray](7.3,0)(7.7,9)
\psframe[fillstyle=solid,fillcolor=gray](8.3,0)(8.7,11)
\psframe[fillstyle=solid,fillcolor=gray](9.3,0)(9.7,13.5)
\psframe[fillstyle=solid,fillcolor=gray](10.3,0)(10.7,11)
\end{pspicture}
}
\end{center}

\begin{center}
\begin{tabular}{|*{5}{l|}}\hline
\multicolumn{5}{|c|}{\textbf{Document 2: VMA(en km/h) des garçons}}\\ \hline
Nathan : 12	&Lucas: 11	&Jules: 14	&Abdel: 13,5	&Nicolas: 14\\ \hline
Thomas: 14,5& Martin: 11&Youssef: 14& Mathis : 12	&Léo : 15\\ \hline
Simon: 12	& José: 14	&Ilan: 14	&				&\\ \hline
\end{tabular}
\end{center}

Dire si les affirmations suivantes sont vraies ou fausses. On rappelle que toutes les réponses
doivent être justifiées.

	\begin{enumerate}
		\item \textbf{Affirmation 1}: l'étendue de la série statistique des VMA des filles de la classe est plus élevée que celle de la série statistique de VMA des garçons de la classe.
		\item \textbf{Affirmation 2} : plus de 25\,\% des élèves de la classe a une VMA inférieure ou égale à 11,5 km/h.
		\item L'enseignante souhaite que la moitié de la classe participe à une compétition. Elle sélectionne donc les douze élèves dont la VMA est la plus élevée.

\textbf{Affirmation 3} : Lisa participe à la compétition.
	\end{enumerate}
\end{enumerate}

\bigskip

