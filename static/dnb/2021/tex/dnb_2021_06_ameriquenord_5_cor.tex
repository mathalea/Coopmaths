

	\begin{enumerate}
		\item %Montrer que la surface à recouvrir de papier peint est de 26,4 m$^2$.
Aire de la surface  à recouvrir de papier peint :

$2 \times 3,5 \times 2,5 + 2 \times 2,5 \times 2,5 - 2,1 \times 0,8 - 1,6 \times 1,2 = 30 - 1,68 - 1,92 = 26,4$~m$^2$.		
		\item %Calculer le prix, en euro, d’un mètre carré de papier peint.
%Arrondir au centime d'euro.
16,95~\euro{} pour 5,3~m$^2$ donne un prix au m$^2$ de $\dfrac{16,95}{5,3} \approx 3,198$ soit 3,20~\euro{} au centime près.		
		\item %Si on suit les conseils du vendeur, combien coûtera la rénovation de la salle de bain ?
Il faut en principe $\dfrac{26,4}{5,3} \approx 4,98$ soit 5 rouleaux à l'unité près et avec 1 rouleau de plus pour les pertes, il faudra donc acheter 6 rouleaux.		
		\item %Le jour de l'achat, une remise de 8 \% est accordée.
		
%Quel est le prix à payer après remise ? Arrondir au centime d'euro.
Prix du papier peint : $6 \times 16,95 = 101,70$~\euro

Prix de la colle : $2 \times 5,70 = 11,40$~\euro pour un total de :

$101,70 + 11,40 = 113,10$~\euro.

Enlever 8\,\% revient à multiplier par $1 - \dfrac{8}{100} = 1 - 0,08 = 0,92$.

Le prix à payer après remise est donc :

$113,10 \times 0,92 = 104,052 \approx 104,05$~\euro.
	\end{enumerate}

