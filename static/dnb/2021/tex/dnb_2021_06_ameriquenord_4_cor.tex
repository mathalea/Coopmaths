
%\begin{minipage}{12cm}
\emph{Dans cet exercice, aucune justification n'est demandée.}
		
%		On dispose d’un tableau carré ci-contre partagé en neuf cases blanches de mêmes dimensions qui constituent un motif.
%	\end{minipage}
%	\hfill
%	\begin{tikzpicture}[x=6mm,y=6mm, baseline={(current bounding box.center)}]
%		\foreach \xy/\c in {(0,0)/white,(1,0)/white,(2,0)/white,
%							(0,1)/white,(1,1)/white,(2,1)/white,
%							(0,2)/white,(1,2)/white,(2,2)/white}
%			\draw[fill=\c, shift={\xy}] (0,0) rectangle (1,1);
%	\end{tikzpicture}
%		
%	\medskip
%	
%Quatre instructions A, B, C et E permettent de changer l'aspect de certaines cases, lorsqu'on applique ces instructions. Ainsi:
%	
%	\begin{tabularx}{\linewidth}{| >{\centering \arraybackslash}m{2cm} | m{8cm} | >{\centering \arraybackslash}X |} \hline
%		Instruction & Descriptif & Effet de l'instruction   \\ \hline
%		A &   La case centrale du motif est noircie.  & 
%		\begin{tikzpicture}[x=6mm,y=6mm, baseline={(current bounding box.center)}]
%			\clip (-0.2,-0.2) rectangle (3.2,3.2);
%			\foreach \xy/\c in {(0,0)/white,(1,0)/white,(2,0)/white,
%				(0,1)/white,(1,1)/gray!50,(2,1)/white,
%				(0,2)/white,(1,2)/white,(2,2)/white}
%			\draw[fill=\c, shift={\xy}] (0,0) rectangle (1,1);
%		\end{tikzpicture}  \\ \hline
%		B &   Dans le motif, la case en bas à gauche et la case en haut à droite sont noircies. & 
%				\begin{tikzpicture}[x=6mm,y=6mm, baseline={(current bounding box.center)}]
%			\clip (-0.2,-0.2) rectangle (3.2,3.2);
%			\foreach \xy/\c in {(0,0)/gray!50,(1,0)/white,(2,0)/white,
%				(0,1)/white,(1,1)/white,(2,1)/white,
%				(0,2)/white,(1,2)/white,(2,2)/gray!50}
%			\draw[fill=\c, shift={\xy}] (0,0) rectangle (1,1);
%		\end{tikzpicture}  \\ \hline
%		
%		C &   Dans le motif, la case médiane à gauche et la case médiane à droite sont noircies. & 
%				\begin{tikzpicture}[x=6mm,y=6mm, baseline={(current bounding box.center)}]
%			\clip (-0.2,-0.2) rectangle (3.2,3.2);
%			\foreach \xy/\c in {(0,0)/white,(1,0)/white,(2,0)/white,
%				(0,1)/gray!50,(1,1)/white,(2,1)/gray!50,
%				(0,2)/white,(1,2)/white,(2,2)/white}
%			\draw[fill=\c, shift={\xy}] (0,0) rectangle (1,1);
%		\end{tikzpicture}  \\ \hline
%		
%		
%		E &  Les couleurs du motif sont inversées: les cases blanches deviennent noires et les les cases noires deviennent blanches.&
%				\begin{tikzpicture}[x=6mm,y=6mm, baseline={(current bounding box.center)}]
%			\clip (-0.2,-0.2) rectangle (3.2,3.2);
%			\draw[] (0,0) rectangle (3,3) ;
%			\node (t) at (1.5,1.5) [text width=1.5cm, align=center] {Inverser\\ les\\ couleurs};
%		\end{tikzpicture}  \\ \hline
%
%	\end{tabularx}
%
%\smallskip
%
%\emph{Remarque} : si une case du motif est déjà noire et une instruction demande la noircir, alors cette case ne change pas de couleur et reste noire à la suite de cette instruction.
%	
%\smallskip
%	
%\emph{Exemples} : à partir d’un motif dont toutes les cases sont blanches:
%	
%		\begin{tabularx}{\linewidth}{X c @{\hspace{5mm}}|@{\hspace{5mm}} X c} 
%		la suite d'instructions A C permet d'obtenir ce motif &
%		\begin{tikzpicture}[x=6mm,y=6mm, baseline={(current bounding box.center)}]
%			\foreach \xy/\c in {(0,0)/white,(1,0)/white,(2,0)/white,
%								(0,1)/gray!50,(1,1)/gray!50,(2,1)/gray!50,
%								(0,2)/white,(1,2)/white,(2,2)/white}
%				\draw[fill=\c, shift={\xy}] (0,0) rectangle (1,1);
%
%		\end{tikzpicture}&
%		la suite d'instructions A C E permet d'obtenir ce motif &
%		\begin{tikzpicture}[x=6mm,y=6mm, baseline={(current bounding box.center)}]
%			\foreach \xy/\c in {(0,0)/gray!50,(1,0)/gray!50,(2,0)/gray!50,
%								(0,1)/white,(1,1)/white,(2,1)/white,
%								(0,2)/gray!50,(1,2)/gray!50,(2,2)/gray!50}
%					\draw[fill=\c, shift={\xy}] (0,0) rectangle (1,1);
%		\end{tikzpicture}\\		
%	\end{tabularx}
%
%\medskip
%
%Pour chacune des questions suivantes, on dispose au départ d’un motif dont toutes les cases sont blanches.
	
	\begin{enumerate}
		\item ~%Représenter le motif obtenu avec la suite d'instructions A B.
\begin{center}
\psset{unit=0.5cm,linewidth=0.1pt}
\begin{pspicture}(3,3)
\psframe[fillstyle=solid,fillcolor=lightgray](1,1)
\psframe[fillstyle=solid,fillcolor=lightgray](1,1)(2,2)
\psframe[fillstyle=solid,fillcolor=lightgray](2,2)(3,3)
\psframe(0,1)(1,2)\psframe(0,2)(1,3)\psframe(1,2)(2,3)\psframe(2,1)(3,2)
\psframe(1,0)(2,1)\psframe(2,0)(3,1)
\end{pspicture}
\end{center}
		
		\item %\begin{minipage}[t]{11 cm}
%			Parmi les quatre propositions suivantes, deux propositions permettent d'obtenir le motif ci-contre. Lesquelles ?
%			
%			\smallskip 
%			
%\begin{tabularx}{\linewidth}{*{2}{>{\bfseries Proposition \no }l <{ :} @{~} X}}
%1 & A B C   & 3 & B C E C\\ [6pt]
%2 & C E     & 4 & C A E A\\
%\end{tabularx}
%		\end{minipage}
%		\hfill
%		\begin{tikzpicture}[x=6mm,y=6mm, baseline={(0,1.6)}]
%			\foreach \xy/\c in {(0,0)/gray!50,(1,0)/gray!50,(2,0)/gray!50,
%								(0,1)/white,(1,1)/gray!50,(2,1)/white,
%								(0,2)/gray!50,(1,2)/gray!50,(2,2)/gray!50}
%				\draw[fill=\c, shift={\xy}] (0,0) rectangle (1,1);
%		\end{tikzpicture}
C  E et C A E A permettent d'obtenir le motif demandé.		
		\item %\begin{minipage}[t]{11 cm}
%			Donner une suite d'instructions qui permet d'obtenir le motif ci-contre.	
%		\end{minipage}
%		\hfill
%		\begin{tikzpicture}[x=6mm,y=6mm, baseline={(0,0.8)}]
%			\foreach \xy/\c in {(0,0)/gray!50,(1,0)/gray!50,(2,0)/white,
%								(0,1)/gray!50,(1,1)/white,(2,1)/gray!50,
%								(0,2)/white,(1,2)/gray!50,(2,2)/gray!50}
%			\draw[fill=\c, shift={\xy}] (0,0) rectangle (1,1);
%		\end{tikzpicture}
La suite A B E permet d'obtenir la diagonale montante blanche.
	\end{enumerate}

\bigskip

