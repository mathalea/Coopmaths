
\medskip

Le Futuroscope est un parc de loisirs situé dans la Vienne. L'année 2019 a enregistré 1,9 million de visiteurs.

\medskip

\begin{enumerate}
\item Combien aurait-il fallu de visiteurs en plus en 2019 pour atteindre 2 millions de visiteurs?
\item L'affirmation \og Il y a eu environ \np{5200} visiteurs par jour en 2019 \fg{} est-elle vraie? Justifier la
réponse.
\item Un professeur organise une sortie pédagogique au Futuroscope pour ses élèves de troisième. Il veut répartir les $126$ garçons et les $90$ filles par groupes. Il souhaite que chaque groupe comporte le même nombre de filles et le même nombre de garçons.
	\begin{enumerate}
		\item Décomposer en produit de facteurs premiers les nombres $126$ et $90$
		\item Trouver tous les entiers qui divisent à la fois les nombres $126$ et $90$.
		\item En déduire le plus grand nombre de groupes que le professeur pourra
constituer. 

Combien de filles et de garçons y aura-t-il alors dans chaque groupe?
	\end{enumerate}
\item Deux élèves de 3\up{e}, Marie et Adrien, se souviennent avoir vu en mathématiques que les hauteurs inaccessibles pouvaient être déterminées avec l'ombre. 

Ils souhaitent calculer la hauteur de la Gyrotour du Futuroscope.

Marie se place comme indiquée sur la figure ci-dessous, de telle sorte que son ombre coïncide avec celle de la tour. Après avoir effectué plusieurs mesures, Adrien effectue le schéma ci- dessous (le schéma n'est pas à l'échelle), sur lequel les points A, E et B ainsi que les points A, D et C sont alignés.

Calculer la hauteur BC de la Gyrotour.
\end{enumerate}

\begin{center}
\psset{unit=0.8cm,arrowsize=2pt 5}
\begin{pspicture}(14,6.25)
%\psgrid
\pspolygon[linewidth=1.2pt](0.5,0.5)(12.6,0.5)(12.6,6)
\psline(3.5,0.5)(3.5,1.84)
\psline[linewidth=0.3pt]{<->}(0.5,0.1)(3.5,0.1)\uput[d](2,0.1){2 m}
\psline[linewidth=0.3pt]{<->}(3.5,0.1)(12.6,0.1)\uput[d](8.05,0.1){54,25 m}
\rput(3.5,1.17){Marie}
\psline[linestyle=dashed](3.5,1.84)(4.,1.84)
\psline[linewidth=0.3pt]{<->}(4,0.5)(4,1.84)\uput[r](4,1.17){1,60 m}
\rput{90}(12.9,3.25){Gyrotour}
\psline[linewidth=0.5pt]{<-}(13.4,0.5)(13.4,2.8)
\psline[linewidth=0.5pt]{->}(13.4,3.2)(13.4,6)
\rput{90}(13.4,3){?}
\uput[d](0.5,0.5){A}\uput[d](3.5,0.5){D}\uput[u](3.5,1.84){E}
\uput[dr](12.6,0.5){C}\uput[u](12.6,6){B}
\psline[linestyle=dashed](3.5,1.84)(3.5,4)
\psframe(3.5,0.5)(3.2,0.8)\psframe(12.6,0.5)(12.3,0.8)
\end{pspicture}
\end{center}

\bigskip

