
%Une athlète a réalisé un triathlon d’une longueur totale de 12,9 kilomètres. Les trois épreuves se déroulent dans l’ordre suivant :
%	
%\begin{center}
%		\begin{tabularx}{14 cm}{|*{3}{>{\centering \arraybackslash}X|}} \hline
%		Épreuve \tikz[baseline=(x.base)]{\draw (0,0) circle (8pt) node (x) {1};}\rule{0pt}{14pt} :     & Épreuve \tikz[baseline=(x.base)]{\draw (0,0) circle (8pt) node (x) {2};} :     & Épreuve \tikz[baseline=(x.base)]{\draw (0,0) circle (8pt) node (x) {3};} : \\
%Natation          & Cyclisme          & Course à pied. \\
%Distance = 400 m  &                   & Distance = 2,5 km\\ \hline
%\end{tabularx}
%\end{center}
%
%Entre deux épreuves, l’athlète doit effectuer sur place un changement d'équipement.
%	
%Le graphique ci-dessous représente la distance parcourue (exprimée en kilomètre) par l’athlète, en fonction du temps de parcours (exprimé en minute) de l’athlète pendant son triathlon.
%	
%	\begin{center}
%		\begin{tikzpicture}[x=1.8mm,y=9mm,> = latex]
%			\draw [gray,xstep=5,ystep=1] (0,0) grid (65,15);
%			\draw[<->] (0,15.2) node [above right] {\textbf{Distance en kilomètre}}--(0,0)--(66,0) node[shift={(0,-1)}, left] {\textbf{Temps en minute}};
%			\foreach \x in {5,10,...,60}{
%				\draw (\x,2pt)--(\x,-2pt) node [below] {\x};}
%			\draw[line width=0.8pt] (0,0)--(14,0.4)--(15,0.4)--(42,10.4)--(44,10.4)--(56,12.9);
%			\foreach \c in {(14,0.4),(15,0.4),(42,10.4),(44,10.4),(56,12.9)}
%				\draw[shift=\c,line width=0.8pt] (-3pt,0)--(3pt,0) (0,3pt)--(0,-3pt);
%			\draw[fill = white] (7,.8) circle (8pt) node (x) {1};
%			\draw[fill = white] (27.2,6.1) circle (8pt) node (y) {2};
%			\draw[fill = white] (52.5,11.3) circle (8pt) node (z) {3};
%			\draw (42,10.4) node[above left] {M};
%			\draw[<-](14,0.6)--(14,5) node[text width=38 mm, align = center, fill=white, draw] {Premier changement d'équipement};
%			\draw[<-](43,10.6)--(43,14.2) node[text width=38 mm, align = center, fill=white, draw] {Deuxième changement d'équipement};
%		\end{tikzpicture}
%	\end{center}	
%
%Le point M a pour abscisse 42 et pour ordonnée 10,4.
%	
%À l’aide du tableau ci-dessus ou par lecture du graphique ci-dessus avec la précision qu’il permet, répondre aux questions suivantes, en justifiant la démarche.

\medskip
	
	\begin{enumerate}
		\item %Au bout de combien de temps l’athlète s’est-elle arrêtée pour effectuer son premier changement d'équipement ?
L'athlète a fait l'épreuve de natation en 14~min, début de son premier changement d'équipement.	
		\item %Quelle est la longueur, exprimée en kilomètre, du parcours de l'épreuve de cyclisme ?
Si $c$ est la longueur s parcours en vélo, on a :
		
$0,400 + c + 2,5 = 12,9$ soit $c + 2,9 = 12,9$, d'où $c = 10$~km.		
		\item %En combien de temps l’athlète a-t-elle effectué l'épreuve de course à pied ?
L'épreuve de course à pied s'est passée de la 44\up{e} à la 56\up{e} minute ; elle a donc couru pendant $56 - 44 = 12$~minutes.		
		\item %Parmi les trois épreuves, pendant laquelle l’athlète a été la moins rapide ?
Le segment ayant la plus faible pente est bien sûr celui de la natation.

Remarque : on peut calculer :

vitesse en natation : 400 m en 14 min soit $\dfrac{0,4}{14}\times 60 \approx 1,71$~km/h ;

vitesse en vélo : 10 km en 27 min soit $\dfrac{10}{27}\times 60 \approx 22,2$~km/h ;

vitesse à pied : 2,5 km en 12 min soit $\dfrac{2,5}{12}\times 60 = 12,5$~km/h.		
		\item %On considère que les changements d'équipement entre les épreuves font partie du triathlon.
		
%La vitesse moyenne de l’athlète sur l’ensemble du triathlon est-elle supérieure à 14 km/h ?
Elle a parcouru 12,9~km en  57 minutes, donc à une vitesse de $\dfrac{12,9}{57}\times 60 \approx 13,58 < 14$~km/h.
	\end{enumerate}
	
\bigskip
	
