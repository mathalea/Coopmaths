
%Pour chacune des six affirmations suivantes, indiquer sur la copie, si elle est vraie ou fausse.

\medskip

%\textbf{On rappelle que chaque réponse doit être justifiée.}

\begin{enumerate}[itemsep=5mm]
	\item On considère la fonction $f$ définie par $f(x) = 3x - 7$
	
	\textbf{Affirmation \no 1 :} \og{}L’image par $f$ du nombre $-1$ est 2 \fg{}.
	
	On a $f(- 1) = 3 \times (- 1) - 7 = - 3 - 7 = - 10$ : affirmation fausse.
	
	\item On considère l'expression $E = (x - 5)(x + 1)$.
	
	\textbf{Affirmation \no 2 :} \og{} L'expression $ \mathrm{E} $ a pour forme développée et réduite $x^2 - 4x - 5 $\fg.
	
$E = x^2  + x - 5x - 5 = x^2 - 4x - 5$ : affirmation vraie.
	
	\item $n$ est un nombre entier positif.
	
	\textbf{Affirmation \no 3 :} \og{}lorsque $n$ est égal à 5, le nombre $2^n +1$ est un nombre premier \fg{}.

$2^5 + 1 = 32 + 1 = 33$ ; or 33 est un multiple de 3 donc n'est pas premier : affirmation fausse.	
	\item On a lancé 15 fois un dé à six faces numérotées de 1 à 6 et on a noté les fréquences d'apparition dans le tableau ci-dessous :
	
%	\smallskip
%	\begin{tabularx}{\linewidth}{|m{3.5cm}|*{6}{>{\centering \arraybackslash}X|}} \hline
%Numéro de la face apparente & 1& 2& 3& 4& 5& 6 \\ \hline
%Fréquence d'apparition&$\dfrac{3}{15}$&$\dfrac{4}{15}$&$\dfrac{5}{15}$&$\dfrac{2}{15}$&$\dfrac{1}{15}$ &\dots\rule[-3mm]{0mm}{9mm}\\ \hline
%	\end{tabularx}
%	
%	\smallskip
	
	\textbf{Affirmation \no 4 :} \og{}la fréquence d'apparition du 6 est 0 \fg{}.
On sait que la somme des fréquences est égale à 1;, donc si $f_6$ est la fréquence d'apparition du 6, on a :

$\dfrac{3}{15} + \dfrac{4}{15} +\dfrac{5}{15} +\dfrac{2}{15} +\dfrac{1}{15} + f_6 = 1$, ou $\dfrac{15}{15} + f_6 = 1$, donc $f_6 = 0$ : affirmation vraie.
	
	\item \parbox{0.6\linewidth}{On considère un triangle RAS rectangle en S.
	
Le côté [AS] mesure 80 cm et l'angle $ \widehat{\mathrm{ARS}} $ mesure $26\degres$.
	
	\textbf{Affirmation \no 5 :} le segment [RS] mesure environ 164~cm.} \hfill \parbox{0.37\linewidth}{\psset{unit=1cm}
\begin{pspicture}(5,2.5)
\pspolygon(0.5,0.4)(4,0.4)(4,2)%ASR
\psframe(4,0.4)(3.75,0.65)
\uput[dl](0.5,0.4){A}\uput[dr](4,0.4){S}\uput[u](4,2){R}
\uput[d](2.25,0.4){80~cm}
\psarc(4,2){5mm}{207}{270}\rput(3.5,1.4){$26\degres$}
\end{pspicture}}

On a $\tan \widehat{\text{ARS}} = \dfrac{\text{AS}}{\text{RS}}$, soit $\tan 26 = \dfrac{80}{\text{RS}}$, ou $\text{RS} = \dfrac{80}{\tan 26} \approx 164,024$ : affirmation vraie.	
	\item Un rectangle ABCD a pour longueur 160 cm et pour largeur 95~cm.
	
	\textbf{Affirmation \no 6 :} les diagonales de ce rectangle mesurent exactement 186~cm.
	
Le demi-rectangle ABD est un triangle rectangle dont les côtés de l'angle droit mesurent 160~cm et 95~cm.

Le théorème de Pythagore appliqué à ce triangle s'écrit $\text{BD}^2 = 160^2 + 95^2 = \np{25600} + \np{9025} = \np{34625}$, d'où BD $ = \sqrt{\np{34625}} \approx 186,08$~cm, donc BD $\ne 186$ : affirmation fausse. 
\end{enumerate}

\bigskip

