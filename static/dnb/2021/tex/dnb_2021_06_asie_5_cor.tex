
\begin{enumerate}
\item Le premier motif comporte $1+2+3=6 cubes$. Quand on recule d'une rangée on ajoute $3 cubes$ au motif précédent. Il y a donc $6+9+12=27 cubes$. Or le pavé droit a pour volume:$3 \times 3 \times 6=54 cubes$.


Il manque donc $54-27=27 cubes$.

\remarque{ Si l'on est astucieux on peut se rendre compte que l'on a la moitié du pavé droit qui est remplie...}

\item %\includegraphics[scale=0.5]{ex5}

\item 
	\begin{enumerate}
		\item Le volume du cube est:
\begin{align*}
6 \times 4+3=27dm^3
\end{align*}
		\item $3^3=27$ l'arête de ce grand cube mesure donc $3dm$.

\remarque{sur votre calculatrice vous pouvez utiliser $\sqrt[3]{x}$.}
	\end{enumerate}
\end{enumerate}
