
\medskip

Voici un algorithme:

\begin{center}
\psset{unit=1cm,arrowsize=2pt 3}
\begin{pspicture}(-6,0)(6,6.5)
%\psgrid
\rput(0,6){Choisir un nombre $N$}
\psframe(-2,5.6)(2,6.3)
\psline{->}(0,5.8)(0,5)
\rput(0,4){A-t-on : $N > 15$?}
\pspolygon(-2.5,4)(0,5)(2.5,4)(0,3.)
\psline{->}(-2.5,4)(-4,4)(-4,2)\psline{->}(2.5,4)(4,4)(4,2)
\rput(-4,1.5){Calculer $100 - N \times 4$}\rput(4,1.5){Calculer $2 \times (N + 10)$}
\psframe(-5.6,1)(-2.3,2)\psframe(2.4,1)(5.6,2)
\uput[u](-3.2,4){OUI}\uput[u](3.2,4){NON}
\end{pspicture}
\end{center}

\begin{enumerate}
\item Justifier que si on choisit le nombre $N$ de départ égal à 18, le résultat final de cet algorithme est $28$.
\item Quel résultat final obtient-on si on choisit $14$ comme nombre $N$ de départ ?
\item En appliquant cet algorithme, deux nombres de départ différents permettent d'obtenir $32$ comme résultat final. Quels sont ces deux nombres ?
\item On programme l'algorithme précédent:

\begin{center}
\begin{tabular}{r l}
\multicolumn{1}{l}{\textbf{Numéros}}&\\
\multicolumn{1}{l}{\textbf{de ligne}}&\\
\psline{->}(-1,0.2)(0.5,0.2)(0.5,0)&\\
&\begin{scratch}
\setscratch{num blocks=true}
\blockinit{quand \greenflag est cliqué}
\blocksensing{demander \txtbox{Choisir un nombre} et attendre}
\blockifelse{si \booloperator{réponse > \txtbox{\phantom{de}} alors}}
{\blocksound{dire \ovalnum{100} - réponse * \ovalnum{4} pendant \ovalnum{2} secondes}}
{\blocksound{dire \ovalnum{} * \ovalnum{} + \ovalnum{} pendant \ovalnum{2} secondes}}
\end{scratch}
\\
\end{tabular}
\end{center}

	\begin{enumerate}
		\item Recopier la ligne 3 en complétant les pointillés: 
		
		ligne 3 : si réponse $> \ldots$  alors
		\item Recopier la ligne 6 en complétant les pointillés:
		
		ligne 6 : dire \ldots $*(\ldots  + \ldots)$ pendant 2 secondes
	\end{enumerate}
\item On choisit au hasard un nombre premier entre 10 et 25 comme nombre $N$ de départ. 

Quelle est la probabilité que l'algorithme renvoie un multiple de 4 comme résultat final ?
\end{enumerate}

\bigskip

