
\medskip
\begin{enumerate}[itemsep=1em]
    \item Si on choisit $4$ comme nombre de départ :
    \begin{list}{$\leadsto$}{}
    \item Choisir un nombre : $4$
    \item Prendre le carré du nombre de départ : $4^2=16$
    \item Ajouter le triple du nombre de départ : $16 + 3\times 4 = 16 + 12 = 28$
    \item Soustraire 10 au résultat : $28 -10 = 18$
    \end{list}
    \textbf{On trouve bien 18}
    \item Si on choisit $-3$ comme nombre de départ :
    \begin{list}{$\leadsto$}{}
    \item Choisir un nombre : $-3$
    \item Prendre le carré du nombre de départ : $(-3)^2=9$
    \item Ajouter le triple du nombre de départ : $9 + 3\times (-3) = 9 - 9 = 0$
    \item Soustraire 10 au résultat : $0 -10 = -10$
    \end{list}
    \textbf{On trouve -10}
    \item 
    	\begin{scratch}[scale=0.75,print]
		\blockinit{quand \greenflag est cliqué}
		\blocksensing{demander \ovalnum{Choisir un nombre}~ et attendre}
		\blockvariable{mettre \selectmenu{x} à \ovalsensing{réponse}}
		\blockvariable{mettre \selectmenu{y} à \ovaloperator{\ovalvariable{x} * \ovalvariable{x}}}
		\blockvariable{mettre \selectmenu{z} à \ovaloperator{\ovalvariable{y} + \ovaloperator{\ovalnum{3} * \ovalvariable{x}}}}
		\blockvariable{mettre \selectmenu{Résultat} à \ovaloperator{\ovalvariable{z} - \ovalnum{10}}}
		\blocklook{dire \ovaloperator{regroupe \ovalnum{Le nombre final est} et \ovalvariable{Résultat}} pendant \ovalnum{2} secondes}
	\end{scratch}
	\item 
	\begin{enumerate}[itemsep=1em]
	   \item Si on choisit $x$ comme nombre de départ :
        \begin{list}{$\leadsto$}{}
        \item Choisir un nombre : $x$
        \item Prendre le carré du nombre de départ : $x^2$
        \item Ajouter le triple du nombre de départ : $x^2 + 3\times x$
        \item Soustraire 10 au résultat : $x^2+3x-10$
        \end{list}
        \textbf{On trouve : $\mathbf{x^2+3x-10}$}
        \item Développons $(x+5)(x-2)$.\par
        $(x+5)(x-2) = x^2 +5x-2x-10 = x^2+3x-10$\par
        \textbf{On retrouve bien : $\mathbf{x^2+3x-10}$}
        \item Résolvons l'équation $(x+5)(x-2)=0$\par
        \textbf{Un produit de facteurs est nul si l'un au moins de ses facteurs est nul.}\par
        Donc $(x+5)(x-2)=0$ si $(x+5)=0$ ou $(x-2)=0$\par
        \textbf{C'est à dire pour $\mathbf{x=-5}$ ou pour $\mathbf{x=2}$}
	\end{enumerate}
\end{enumerate}

\medskip

