
%Aurélie fait du vélo en Angleterre au col de Hardknott.
%
%Elle est partie d'une altitude de 251 mètres et arrivera au sommet à une altitude de 393 mètres.
%
%Sur le schéma ci-dessous, qui n'est pas en vraie grandeur, le point de départ est représenté par le point A et le sommet par le point E. Aurélie est actuellement au point D. 
%
%\begin{tikzpicture}
%	\draw (0,0) node [below left] {A} -- (10,0) node[below right] {C} -- (10,4) node[above right]{E} -- cycle;
%	
%	\draw[dash pattern = on 1mm off 0.5mm, <-,> = latex] (-0.1,0)-- (-1.1,0) node [left]{Altitude : 251 m};
%	
%	\draw[dash pattern = on 1mm off 0.5mm, <-,> = latex] (9.9,4)-- (-1.1,4) node [left]{Altitude : 393 m};
%	
%	\draw (0,0) -- (4,1.6) node[pos=0.5, sloped, below] {51,25 m} node [below right]{D} -- (4,0) node[pos = 0.5, right]{11,25 m} node[below] {B};
%	
%	\draw (3.7,0)--(3.7,0.3)--(4,0.3) (9.7,0)--(9.7,0.3)--(10,0.3);
%		
%%	\draw[shift={(4,1.6)}, line width = 0.8 mm, line cap = round, line join = angle] (-0.09,0.06) -- (-0.2,0.37)--(0.05,0.28)--(-0.12,0.2);
%
%\draw [shift={(4,1.6)},line width=3.2pt,fill=black,fill opacity=1.0] (0.22,1.1) circle (0.1);
%\draw [shift={(4,1.6)}]  (-0.63,0.11) circle (0.33) (0.31,0.49) circle (0.33);
%
%	\draw [shift={(4,1.6)}, line width = 0.8 mm, line cap = round, line join = angle] (-0.18,0.12)--(-0.39,0.73) -- (-0.09,0.55) (-0.09,0.55) -- (-0.25,0.39);
%
%	\draw [fill=black, shift={(4,1.6)}, line width = 0.8 mm, line cap = round, line join = angle] (-0.37,0.87) -- (0,1) arc (109:-71:0.09) -- (-0.17,.75)-- cycle;
%	
%	\draw [fill=black, shift={(4.4,1.7)}, line width = 0.8 mm, line cap = round, line join = angle]	(-0.39,0.73) -- (-0.09,0.55);
%	
%	%(0.21,0.62) -- (-0.28,0.82) ;
%\end{tikzpicture}

%Les droites (AB) et (DB) sont perpendiculaires. Les droites (AC) et (CE) sont perpendiculaires. Les points A, D et E sont alignés. Les points A, B et C sont alignés.
%
%AD = 51,25 m et DB = 11,25 m.

\begin{enumerate}
	\item %Justifier que le dénivelé qu'Aurélie aura effectué, c'est-à-dire la hauteur EC, est égal à $142$~m.
On a CE $ = 393 - 251 = 142$~(m).	
	\item \begin{enumerate}
		\item %Prouver que les droites (DB) et (EC) sont parallèles.
Les droites (DB) et (EC) étant toutes les deux perpendiculaires à la droite (AC) sont parallèles.
		\item %Montrer que la distance qu'Aurélie doit encore parcourir, c'est-à-dire la longueur DE, est d'environ 596 m.
A, D, E sont alignés dans cet ordre,

A, B et C sont alignés dans cet ordre,

et les droites (DB) et (EC) sont parallèles : on est donc une situation où l'on peut appliquer le théorème de Thalès, soit :

$\dfrac{\text{BD}}{\text{EC}} = \dfrac{\text{AD}}{\text{AE}}$,

soit $\dfrac{11,25}{142} = \dfrac{51,25}{\text{AE}}$ ;

on en déduit $11,25\text{AE} = 142 \times 51,25$ puis $\text{AE} = \dfrac{142 \times 51,25}{11,25}\approx 646,8$.

Donc DE $= \text{AE} - \text{AD} \approx 646,8 - 51,25 \approx 595,6$ soit 596~(m) au mètre près. 
	\end{enumerate}
\item %On utilisera pour la longueur DE la valeur 596 m.
	
%Sachant qu'Aurélie roule à une vitesse moyenne de 8 km/h, si elle part à 9~h~55 du point D, à quelle heure arrivera-t-elle au point E ? Arrondir à la minute.
Aurélie parcourt donc \np{8000}~m en 60 minutes ou 800~m en 6~min ou 400~m en 3 minutes.

Elle mettra donc pour parcourir $596$~(m) un temps $t$ tel que $\dfrac{3}{400} = \dfrac{t}{596}$, soit en multipliant chaque membre par 596 :

$t = \dfrac{3 \times 596}{400} = 4,47$~(min), donc $t \approx 4$~(m) : elle arrivera donc à 9~h~59~min à la minute près.	
\item %La pente d'une route est obtenue par le calcul suivant :
	
%	\begin{tabularx}{\linewidth}[t]{@{}X@{\qquad}| l|} \cline{2-2}
%	
%$\text{pente} =	\dfrac{\text{dénivelé}}{\text{longueur horizontale parcourue}}$.
%	
%La pente s'exprime en pourcentage.
%	
%Démontrer que la pente de la route parcourue par Aurélie est de 22,5\,\%.&
%\begin{tikzpicture}[baseline={(t)}]
%	 \node (t) at (0,3.5) [right] {Exemple d'une pente à 13\,\%};
%	 \node (d) at (0,3) [below right, text width = 3 cm] {\emph{La figure n'est pas en vraie grandeur}};
%	 \draw (0,0)--(4.5,0) node [pos = 0.5, below] {200 m} node [pos = 0.5, below=3mm] {longueur horizontale parcourue}--(4.5,2.5) node [pos = 0.4, right]{dénivelé} node [pos = 0.6,right]{26 m}--cycle node[sloped, pos = 0.5, above] {Route}
%	 (4.2,0) -- (4.2,0.3)--(4.5,0.3);
%\end{tikzpicture} \\ \cline{2-2}
%\end{tabularx}
On a par définition dans le triangle rectangle ABD : $\sin \widehat{\text{CAE}} = \dfrac{\text{BD}}{\text{AD}} = \dfrac{11,25}{51,25}$. La calculatrice donne $\widehat{\text{CAE}} \approx 12,68\degres$.

Dabs le triangle ABC on a $\tan \widehat{\text{CAE}} = \dfrac{\text{CE}}{\text{AC}}$ d'où AC $ = \dfrac{\text{CE}}{\tan \widehat{\text{CAE}}} \approx \dfrac{142}{0,225} \approx 631,1$~(m).

Finalement la pente  est  $\approx \dfrac{142}{631,1} \approx 0,225$, donc $\dfrac{22,5}{100} = 22,5\,\%$. 
\end{enumerate}

\vspace{0,5cm}

