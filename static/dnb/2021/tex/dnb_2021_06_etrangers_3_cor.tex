
%Un professeur propose à ses élèves trois programmes de calculs, dont deux sont réalisés avec un logiciel de programmation.

%\begin{tabularx}{\linewidth}{|*{2}{>{\centering \arraybackslash} X|}} \hline
%	Programme A & Programme B\\
%	\begin{scratch}[scale=0.75]
%		\blockinit{quand \greenflag est cliqué}
%		\blocksensing{demander \ovalnum{choisir un nombre}~ et attendre}
%		\blockvariable{mettre \selectmenu{nombre choisi} à \ovalsensing{réponse}}
%		\blockvariable{mettre \selectmenu{Valeur 1} à \ovaloperator{\ovalnum{1} + \ovalvariable{nombre choisi}}}
%		\blockvariable{mettre \selectmenu{Valeur 2} à \ovaloperator{\ovalnum{3} * \ovalvariable{Valeur 1}}}
%		\blockvariable{mettre \selectmenu{résultat} à \ovaloperator{\ovalvariable{Valeur 2} - \ovalnum{3}}}
%		\blocklook{dire \ovaloperator{regrouper \ovalnum{On obtient} et \ovalvariable{résultat}} pendant \ovalnum{2} secondes}
%	\end{scratch}&\begin{scratch}[scale=0.75]
%	\blockinit{quand \greenflag est cliqué}
%	\blocksensing{demander \ovalnum{choisir un nombre}~ et attendre}
%	\blockvariable{mettre \selectmenu{nombre choisi} à \ovalsensing{réponse}}
%	\blockvariable{mettre \selectmenu{Valeur 1} à \ovaloperator{\ovalvariable{nombre choisi} + \ovalnum{3}}}
%	\blockvariable{mettre \selectmenu{Valeur 2} à \ovaloperator{\ovalvariable{nombre choisi} - \ovalnum{5}}}
%	\blockvariable{mettre \selectmenu{résultat} à \ovaloperator{\ovalvariable{Valeur 1} * \ovalvariable{Valeur 2}}}
%	\blocklook{dire \ovaloperator{regrouper \ovalnum{On obtient} et \ovalvariable{résultat}} pendant \ovalnum{2} secondes}
%	\end{scratch}\\ \hline
%	\multicolumn{2}{|c|}{Programme C}\\
%	\multicolumn{2}{|c|}{\begin{minipage}{6cm}
%			\textbullet~~ Choisir un nombre
%			
%			\textbullet~~ Multiplier par 7
%			
%			\textbullet~~ Ajouter 3
%			
%			\textbullet~~ Soustraire le nombre de départ
%			
%		\end{minipage}}\\ \hline
%\end{tabularx}

\medskip

\begin{enumerate}
	\item \begin{enumerate}
		\item %Montrer que si on choisit 1 comme nombre de départ alors le programme A affiche pendant 2 secondes « On obtient 3 ».
On obtient successivement : $1 \to 1 + 1 = 2 \to 3 \times 2  = 6 \to 6 - 3 = 3$.	
		\item %Montrer que si on choisit 2 comme nombre de départ alors le programme B affiche pendant 2 secondes « On obtient $-15$ ».
On obtient successivement : $2 \to 2 + 3 = 5 \to 2 - 5 = - 3 \to 5 \times - 3 = - 15$.
	\end{enumerate}
	\item Soit $x$ le nombre de départ, quelle expression littérale obtient-on à la fin de l'exécution du programme C ?
On obtient successivement : $x \to x \times 7 \to 7x + 3 \to 7x + 3 - x = 6x + 3$.	
	\item %Un élève affirme qu'avec un des trois programmes on obtient toujours le triple du nombre choisi. A-t-il raison ?
On vient de voir que le programme C donne $6x + 3 \ne 3x$ ;

Le programme A donne à partir de $x$ : $x \to 1 + x \to 3(1 + x) = 3 + 3x \to 3 + 3x -3 = 3x$ : on 
obtient bien le triple.

Le programme B donne à partir de $x$ : $x \to x + 3 \to x - 5 \to (x + 3)(x - 5) = x^2 - 5x + 3x - 15 = x^2 - 2x - 15 \ne 3x$.

L'élève a raison.
	\item \begin{enumerate}
		\item %Résoudre l'équation $(x + 3)(x - 5) = 0$.
U produit de deux facteurs est nul si l'un des facteurs est nul, donc :

$(x + 3)(x - 5) = 0$ si $\left\{\begin{array}{l c l}
x + 3&=&0\\
x - 5&=&0
\end{array}\right.$ ou encore $\left\{\begin{array}{l c l}
x &=&- 3\\
x &=&5
\end{array}\right.$

L'ensemble des solutions est $S = \{- 3~;~5\}$.

		\item %Pour quelles valeurs de départ le programme B affiche-t-il « On obtient 0 » ?
On a vu que le programme B donne à partir de $x$ le produit $(x + 3)(x - 5)$ ry on a vu dans la question précédente que $- 3$ et $5$ annulaient ce produit.

Donc le programme B donne à partir de $- 3$ et à partir de 5 le nombre $0$.
	\end{enumerate}
	\item %Pour quelle(s) valeur(s) de départ le programme C affiche-t-il le même résultat que le programme A ?
Il faut trouver $x$ tel que $6x + 3 = 3x$ soit en ajoutant à chaque membre $- 3x$ : $3x + 3 = 0$ ou $3x = - 3$, soit $3\times x = 3 \times (- 1)$ et finalement $x = - 1$

Le nombre $- 1$ donne par A ou C le même résultat $- 3$.
\end{enumerate}

\vspace{0,5cm}

