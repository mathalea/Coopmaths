
\begin{enumerate}
\item $18\longrightarrow 18>10\longrightarrow 100-18 \times 4=100-72=28$
\item $14\longrightarrow 14<10\longrightarrow 2 \times (14+10)=2 \times 24=48$. On obtient $48$.
\item \begin{multicols}{2}
Appelons $x$ le premier nombre et supposons que $x>15$. On a:

\begin{align*}
100-x \times 4&=32\\
100-x \times 4-100&=32-100\\
-x \times 4 &=-68\\
\frac{-x \times 4}{-4} &=\frac{-68}{-4}\\
x &=17
\end{align*}

On vérifie que $17>15$ c'est le seul nombre strictement supérieur à quinze solution du problème.
\columnbreak

Appelons $y$ le premier nombre et supposons que $y\geq15$. On a:

\begin{align*}
2 \times (y+10)&=32\\
\frac{2 \times (y+10)}{2}&=\frac{32}{2}\\
y+10 &=16\\
y+10-10&=16-10\\
y&=6
\end{align*}

On vérifie que $6\geq 15$. C'est le seul nombre inférieur à quinze solution du problème.
\end{multicols}

Les deux nombres solutions sont donc $6$ et $17$.
\item 
\begin{enumerate}

\item Si réponse >15 alors...
\item dire 2 *(réponse+10)pendant 2 secondes
\end{enumerate}
\item Les nombres premiers compris entre $10$ et $25$ sont:
\begin{itemize}[label=$\bullet$]
\item $11<15$ dont l'image par l'algorithme est: $2 \times (11+10)=42=4 \times 10,5$
\item $13<15$ dont l'image par l'algorithme est: $2 \times (13+10)=46=4 \times 11,5$
\item $17>15$ dont l'image par l'algorithme est: $100-4 \times 17=4 \times 25-4 \times 17=4 \times (25-17)$
\item $23>15$ dont l'image par l'algorithme est: $100-4 \times 23=4 \times (25-23)$
\end{itemize}

Il y a donc quatre issues dont deux sont des multiples de quatre.  Il y a équiprobabilité donc la probabilité d'obtenir un multiple de quatre est: $\frac{2}{4}\left(=\frac{1}{2}\right)$
\end{enumerate}

