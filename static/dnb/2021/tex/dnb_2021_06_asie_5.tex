
\bigskip

\textbf{Première partie}

\medskip

En plaçant plusieurs cubes unités, on construit ce solide:

\begin{center}
\psset{unit=0.7cm}
\begin{pspicture}(-1.5,-1.5)(4,5)
%\psgrid
\def\cubed{\psframe(1,1)
\psline(1,0)(1.414,0.4)(1.414,1.4)(1,1)
\psline(1.414,1.4)(0.414,1.4)(0,1)}
\rput(0.5,3.5){\cubed}\rput(1.5,2.5){\cubed}\rput(2.5,1.5){\cubed}
\rput(2.086,0.1){\cubed}\rput(1.675,-1.3){\cubed}
\rput(0.086,2.1){\cubed}\rput(-0.328,0.71){\cubed}\rput(0.672,-0.29){\cubed}
\psline(-0.328,0.71)(-0.328,-1.3)(1.7,-1.3)
\psline(-0.328,-0.29)(0.672,-0.29)(0.672,-1.29)
\psline(3.1,-0.9)(3.514,-0.5)(3.514,0.5) 
\psline(3.514,0.5)(3.928,0.9)(3.928,-0.1)(3.5,-0.5)
\psline(3.928,0.9)(3.928,1.9)
\rput(1.1,1.1){\cubed}
\end{pspicture}
\end{center}

\textbf{Question }: Combien de cubes unités au minimum manque-t-il pour compléter ce solide et obtenir un pavé droit ?

\bigskip

\textbf{Deuxième partie}

\medskip

Un jeu en 3D contient les sept pièces représentées ci-dessous. Chaque pièce est constituée de cubes identiques d'arête 1dm.

\begin{center}
\psset{unit=0.7cm}
\begin{tabularx}{\linewidth}{*{4}{>{\centering \arraybackslash}X}}
Pièce \no 1 (3 cubes) &Pièce \no 2 (4 cubes)&Pièce \no 3 (4 cubes)& Pièce \no (4 cubes)\\
\begin{pspicture}(3,3)
\def\cubed{\psframe(1,1)
\psline(1,0)(1.414,0.4)(1.414,1.4)(1,1)
\psline(1.414,1.4)(0.414,1.4)(0,1)}
\psframe(1,1)\rput(0,1){\cubed}\rput(1,0){\cubed}
\end{pspicture}&
\begin{pspicture}(3,3)
\def\cubed{\psframe(1,1)
\psline(1,0)(1.414,0.4)(1.414,1.4)(1,1)
\psline(1.414,1.4)(0.414,1.4)(0,1)}
\rput(0.414,1.4){\cubed}\rput(1.414,0.4){\cubed}\rput(0,0){\cubed}
\end{pspicture}&
\begin{pspicture}(3,3)
\def\cubed{\psframe(1,1)
\psline(1,0)(1.414,0.4)(1.414,1.4)(1,1)
\psline(1.414,1.4)(0.414,1.4)(0,1)}
\psframe(1,1)\rput(1,0){\cubed}\rput(2,0){\cubed}
\rput(0,1){\cubed}
\end{pspicture}&
\begin{pspicture}(3,3)
\def\cubed{\psframe(1,1)
\psline(1,0)(1.414,0.4)(1.414,1.4)(1,1)
\psline(1.414,1.4)(0.414,1.4)(0,1)}
\rput(0.6,0){\cubed}\rput(0,1.4){\cubed}
\psline(0,1.4)(0,0.4)(0.6,0.4)
\psline(2,0.4)(2.414,0.8)(2.414,1.8)(1.414,1.8)
\psline(2,1.4)(2.414,1.8)
\end{pspicture}\\
Pièce \no 5 (4 cubes) &Pièce \no 6 (4 cubes)&Pièce \no 7 (4 cubes)& \\
\begin{pspicture}(3,3)
\def\cubed{\psframe(1,1)
\psline(1,0)(1.414,0.4)(1.414,1.4)(1,1)
\psline(1.414,1.4)(0.414,1.4)(0,1)}
\psline(1,0)(0,0)(0,1)(1,1)\psline(0,1)(0.414,1.4)(1,1.4)\psline(1,1)(2,1)\psline(1,0)(2,0)\rput(2,0){\cubed}\psline(1,0)(1,1)
\rput(1,1){\cubed}
\end{pspicture}
&\begin{pspicture}(3,3)
\def\cubed{\psframe(1,1)
\psline(1,0)(1.414,0.4)(1.414,1.4)(1,1)
\psline(1.414,1.4)(0.414,1.4)(0,1)}
\rput(0,0){\cubed}
\rput(1.414,1.4){\cubed}\psline(1.414,1.4)(2.414,1.4)(2.414,0.4)(1.414,0.4)
\psline(2.414,0.4)(2.828,0.8)(2.828,1.8)
\psline(0.414,1.4)(0.828,1.8)(1.414,1.8)
\end{pspicture} &
\begin{pspicture}(3,3)
\def\cubed{\psframe(1,1)
\psline(1,0)(1.414,0.4)(1.414,1.4)(1,1)
\psline(1.414,1.4)(0.414,1.4)(0,1)}
\psframe(2,1)(3,2)\psframe(1,1)
\rput(2,1){\cubed}\psframe(1,1)\psframe(1,0)(2,1)
\psline(1,1)(1,2)(2,2)
\psline(1,2)(1.414,2.4)(2.414,2.4)
\psline(0,1)(0.414,1.4)(1,1.4)
\psline(2,0)(2.414,0.4)(2.414,1)
\end{pspicture}&
\end{tabularx}
\end{center}

\begin{enumerate}
\item Dessiner une vue de dessus de la pièce \no 4 (en prenant 2 cm sur le dessin pour représenter
1~dm dans la réalité).
\item À l'aide de la totalité de ces sept pièces, il est possible de construire un grand cube sans espace vide.
	\begin{enumerate}
		\item Quel sera alors le volume $\left(\text{en dm}^3\right)$ de ce grand cube ?
		\item Quelle est la longueur d'une arête (en dm) de ce grand cube ?
	\end{enumerate}
\end{enumerate}
