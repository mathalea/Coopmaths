
\begin{enumerate}[itemsep=1em]
    \item Il aurait fallu $0,1$ million de visiteurs en plus, soit $100~000$  visiteurs en plus en 2019 pour atteindre les 2 millions d'entrées.
    \item On peut calculer le nombre moyen de visiteurs sur l'année. \\
    $\dfrac{1~900~000}{365}\approx 5~205$\\
    On peut donc dire qu'en moyenne, le parc a accueilli environ $5~200$ spectateurs par jour.\\
    Mais il est faux de dire qu'il en a accueilli $5~200$ chaque jour, car ce résultat n'est qu'une moyenne. Il y a des variations selon les jours et périodes de l'année.
    \item \begin{enumerate}[itemsep=1em]
        \item $126=2\times 3^{2}\times7$ et $90=2\times 3^{2}\times5$.
        \item Les diviseurs de $126$ sont : 1; 2; 3; 6; 7; 9; 14; 18; 21; 42; 63; 126.\\ Les diviseurs de $90$ sont 1; 2; 3; 5; 6; 9; 10; 15; 18; 30; 45; 90.
        \item Pour faire un groupe qui respecte les conditions de l'énoncé, le nombre de groupes doit être un diviseur commun à $126$ et à $90$.\\
        On déduit de la question précédente que les diviseurs communs de $90$ et $126$ sont : 1; 2; 3; 6; 9 et 18.\\
        Le plus grand nombre de groupes est donc le plus grand diviseur commun à 126 et 90 à savoir 18  .\\
        On peut noter ainsi : $PGCD(126;90)=18$\\
        \textbf{Le professeur pourra donc faire 18 groupes de 7 garçons et 5 filles chacun.}
    \end{enumerate}
    \item Dans le triangle $ABC$, on observe que $(ED)\perp (AC)$ et $(BC)\perp (AC)$.\\ On en déduit que $(ED)\ //\ (BC)$.\\
    Comme les points $A$, $E$ et $B$ sont alignés, ainsi que les points $A$, $ D$ et $C$, on peut appliquer le théorème de Thalès :
    
    \medskip
        $\dfrac{AD}{AC}=\dfrac{AE}{AB}=\dfrac{ED}{BC}$\\
            \medskip
    avec les valeurs de l'énnoncé, cela donne :\\
      $\dfrac{2}{56,25}=\dfrac{AE}{AB}=\dfrac{1,6}{BC}$.\\
          \medskip
    On extrait :   $\dfrac{2}{56,25}=\dfrac{1,6}{BC}$\\ qui donne : $BC=\dfrac{56,25\times 1,6}{2}=45$\\
        \medskip
    La hauteur de la Gyrotour est de $45$ m.
\end{enumerate}

\medskip 

