	
%	\begin{minipage}[t]{7 cm}
%		\emph{Dans cet exercice, aucune justification n'est demandée.}
%		
%		On a construit un carré ABCD. 
%		
%		\smallskip
%		
%		On a construit le point O sur la droite (DB), à
%		l'extérieur du segment [DB] et tel que : OB = AB.
%		
%		\smallskip
%		
%		Le point H est le symétrique de D par rapport à O.
%		
%		\bigskip
%		
%		On a obtenu la figure ci-contre en utilisant plusieurs
%		fois la même rotation de centre O et d'angle 45°.
%		
%		\bigskip
%		
%		La figure obtenue est symétrique par rapport à l’axe
%		(DB) et par rapport au point O.
%	\end{minipage}
%	\hfill
%	\begin{tikzpicture}[baseline={(D)}]
%		\foreach \a in {0,45,...,315}
%			\draw [line width=0.8pt] (0+\a:1.414)--++(-45+\a:1.414)--++(45+\a:1.414)--++(135+\a:1.414)--cycle;
%		\foreach \a/\l in {2/E,3/F,4/G,6/I,7/J,8/K}
%			\draw (135-\a*45:2.414) circle (7pt) node {\a}
%			(135-\a*45:3.7) node{\l};
%		\draw (0.4,2.414) circle (7pt) node {1};
%		\draw (-0.4,-2.414) circle (7pt) node {5};
%		\draw[line width=0.8pt] (0,-4)--(0,4) 
%		(-3pt,-3pt)--(3pt,3pt) (-3pt,3pt)--(3pt,-3pt) (1pt,0)node[right] {O};
%		\node (A) at (67.5:2.6) [above right]{A};
%		\node (B) at (90:1.414) [below right] {B};
%		\node (C) at (112.5:2.6) [above left] {C};
%		\node (D) at (90:3.7) [right] {D};
%		\node (H) at (-90:3.7) [left] {H};
%	\end{tikzpicture}
\medskip

	\begin{enumerate}
		\item %Donner deux carrés différents, images l’un de l’autre par la symétrie axiale d’axe (DB).
Les carrés 8 et 2,  les carrés 6 et 4, les carrés 7 et 3 sont symétriques autour de l'axe (DB).	
		\item %Le carré \tikz[baseline=(x.base)]{\draw (0,0) circle (7pt) node (x) {3};} est-il l’image du carré \tikz[baseline=(x.base)]{\draw (0,0) circle (7pt) node (x) {8};} par la symétrie centrale de centre O ?
Les carrés 8 et 3 ne sont pas symétriques autour de O (leurs centres ne sont pas alignés avec O).		
		\item %On considère la rotation de centre O qui transforme le carré \tikz[baseline=(x.base)]{\draw (0,0) circle (7pt) node (x) {1};} en le carré \tikz[baseline=(x.base)]{\draw (0,0) circle (7pt) node (x) {2};}.
		
%Quelle est l’image du carré \tikz[baseline=(x.base)]{\draw (0,0) circle (7pt) node (x) {8};} par cette rotation ?
L'image du carré 8 par la rotation de centre O et d'angle $45\degres$ est le carré 1.
		
		\item %On considère la rotation de centre O qui transforme le carré \tikz[baseline=(x.base)]{\draw (0,0) circle (7pt) node (x) {2};} en le carré \tikz[baseline=(x.base)]{\draw (0,0) circle (7pt) node (x) {5};}.
		
%Préciser l’image du segment [EF] par cette rotation.
La rotation est la rotation de centre O et d'angle $135\degres$. E donne H et F donne I, donc l'image de [EF] est le segment [HI].
	\end{enumerate}

\bigskip

