
\begin{enumerate}
	\item  $126 = 6 \times 21$ \textbf{réponse C.}
	
	\remarque{$256 >126$ donc $252$ ne divise pas $126$.
	
	$126$ n'est pas divisible par $5$ donc il n'est pas divisible par $20$.}
	
	\item $f(0) = 0^2 - 2= - 2$ \textbf{réponse C.}
	\remarque{$f(2)= 2^2-2=4-2=2$ et $f(-2)=(-2)^2-2=4-2=2$}
	
	\item On remplace \texttt{A1} par \texttt{B1} et donc on évalue la formule pour la valeur $ -3$.
	
	$-5\times (-3) \times (-3)+2 \times (-3)-14=-5 \times 9+(-6)-14=- 45 - 6-14=-65$
	
	\textbf{Réponse A}
	
	\item \begin{align*}
	x^2&=16\\
	x^2-16&=0\\
	x^2-4^2&=0\\
	(x-4)(x+4)&=0
	\end{align*}
	\textbf{Un produit de facteurs est nul si et seulement si l'un des facteurs est nul:}
	
	Soit $x-4=0$, soit $x+4=0$ donc $x=4$ ou $x=-4$
	
	On vérifie que $4^2=(-4)^2=16$.
	
	\textbf{Réponse B.}
	
	\item \begin{align*}
	2 \times 2^401&=2^1 \times 2^401\\
	&=2^{1+400}\\
	&=2^{401}
	\end{align*}
	\textbf{Réponse A.}
	
	\item 
	\begin{minipage}{0.5 \linewidth}
	On a le tableau de proportionnalité:
	
	\begin{tabular}{|c|c|c|}
	\hline 
	anciennes longueurs & $16 cm$ & $9 cm$ \\ 
	\hline 
	Nouvelles longueurs & $x$ & $54cm$ \\ 
	\hline 
	\end{tabular} 
	\end{minipage}
	\begin{minipage}{0.5 \linewidth}
	\begin{align*}
	x&=\frac{16 \times 54}{9}\\
	&=96 cm
	\end{align*}
	
	\textbf{Réponse B.}
	\end{minipage}
	
	\remarque{Ou directement la notion de ratio $x$ et $54$ sont dans le ratio $16:9$ si et seulement si $\frac{x}{54}=\frac{16}{9}$.}
	\end{enumerate}


