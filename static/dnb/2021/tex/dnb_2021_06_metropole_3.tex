
\medskip

Cet exercice est un questionnaire à choix multiples (QCM). Aucune justification n'est demandée.

Pour chaque question trois réponses (A, B et C) sont proposées. 

\emph{Une seule réponse est exacte}. 

Recopier sur la copie le numéro de la question et la réponse.

\bigskip

\textbf{PARTIE A :}

\medskip

Une urne contient 7 jetons verts, 4 jetons rouges, 3 jetons bleus et 2 jetons jaunes. Les jetons sont indiscernables au toucher. 

On pioche un jeton au hasard dans cette urne.

\begin{center}
\begin{tabularx}{\linewidth}{|m{2.25cm}|*{3}{X|}}\hline
\textbf{Questions}&\textbf{Réponse A}&\textbf{Réponse B}&\textbf{Réponse C}\\ \hline
\textbf{1.~} À quel événement correspond une probabilité de $\dfrac{7}{16}$ ?&Obtenir un jeton de couleur rouge ou jaune.&Obtenir un jeton qui n'est pas vert.&Obtenir un jeton vert.\\ \hline
\textbf{2.~} Quelle est la probabilité de ne pas tirer un jeton
bleu ?& \qquad \qquad$\dfrac{13}{16}$ &\qquad  \qquad$\dfrac{3}{16}$ &\qquad  \qquad$\dfrac{3}{4}$\\ \hline
\end{tabularx}
\end{center}

\bigskip

\textbf{PARTIE B :}

\medskip

On considère la figure suivante, composée de vingt motifs numérotés de 1 à 20, dans laquelle : 

\setlength\parindent{9mm}
\begin{itemize}
\item[$\bullet~~$] $\widehat{\text{AOB}} =  36\degres$
\item[$\bullet~~$] le motif 11 est l'image du motif 1 par l'homothétie de centre O et de rapport 2.
\end{itemize}
\setlength\parindent{0mm}

\begin{center}
\psset{unit=0.6cm}
\begin{pspicture}(-8,-8)(11,8)
%\psgrid
\def\penta{\pspolygon(1.1;18)(1.1;90)(1.1;162)(1.1;234)(1.1;306)}
\def\pentab{\pspolygon(2.2;18)(2.2;90)(2.2;162)(2.2;234)(2.2;306)}
\rput(2.8;18){\penta}\rput{-36}(2.8;54){\penta} \rput{-72}(2.8;90){\penta}
\rput{-108}(2.8;126){\penta}\rput{-144}(2.8;162){\penta}\rput{-180}(2.8;198){\penta}
\rput{-216}(2.8;234){\penta}\rput{-250}(2.8;270){\penta}\rput{-286}(2.8;306){\penta}
\rput{-322}(2.8;342){\penta}
\pspolygon(4.12;0)(4.12;36)(4.12;72)(4.12;108)(4.12;144)(4.12;180)(4.12;216)(4.12;252)(4.12;288)(4.12;324)
\rput(5.7;18){\pentab}\rput{-36}(5.7;54){\pentab}\rput{-72}(5.7;90){\pentab}
\rput{-108}(5.7;126){\pentab}\rput{-144}(5.7;162){\pentab}\rput{-180}(5.7;198){\pentab}
\rput{-216}(5.7;234){\pentab}\rput{-250}(5.7;270){\pentab}\rput{-286}(5.7;306){\pentab}
\rput{-322}(5.7;342){\pentab}
\multido{\n=90+-36,\na=1+1}{10}{\rput(2.9;\n){\na}}
\multido{\n=90+-36,\na=11+1}{10}{\rput(5.8;\n){\na}}
\psline[linestyle=dashed](0.62,1.95)(0,0)(-0.62,1.95)
\psline[linestyle=dashed,linewidth=1.25pt](-8,0)(10,0)
\uput[u](9,0){droite $(d)$}
\rput(0,1.2){\small 36\degres}\uput[d](0,0){O}
\psarc(0,0){5mm}{72}{108}
\uput[r](0.62,1.95){A}\uput[l](-0.62,1.95){B}
\end{pspicture}
\end{center}

\begin{center}
\begin{tabularx}{\linewidth}{|m{2.25cm}|*{3}{X|}}\hline
\textbf{Questions}&\textbf{Réponse A}&\textbf{Réponse B}&\textbf{Réponse C}\\ \hline
\textbf{3.~} Quelle est l'image du motif 20 par la symétrie d'axe la droite $(d)$ ?&Le motif 17&Le motif 15&Le motif 12\\ \hline
\textbf{4.~} Par quelle rotation le motif 3 est-il l'image du motif
1?&Une rotation de centre O, et d'angle 36\degres.&Une rotation de centre O, et d'angle 72\degres& Une rotation de  centre 0, et d'angle 90\degres\\ \hline
\textbf{5.~} L'aire du motif 11 est-elle
égale:&au double de l'aire du motif 1 ?&à 4 fois l'aire du motif 1.&
à la moitié de l'aire du motif 1.\\ \hline
\end{tabularx}
\end{center}

\bigskip

