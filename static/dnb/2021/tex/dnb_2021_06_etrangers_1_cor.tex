
%	\emph{Dans cet exercice, chaque question est indépendante. Aucune justification n'est demandée.}
	
\begin{enumerate}
\item ~%Décomposer 360 en produit de facteurs premiers.

\begin{center}
$\begin{array}{r|l}
360&9\\
40&8\\
5&5\\
\end{array}$

Donc $360 = 9 \times 8 \times 5 = 2^3 \times 3^2 \times 5$.
\end{center}
		
\item %\begin{minipage}[t]{9 cm}
%À partir du triangle BEJ, rectangle isocèle en J, on a obtenu par pavage la figure ci-contre.
	\begin{enumerate}
		\item %Quelle est l'image du triangle BEJ par la symétrie d'axe (BD) ?
Le point B a pour image B et le point J appartient  (BD), il est aussi égal à son image.
				
Enfin l'image de E est le point F.
				
Donc l'image du triangle BEJ par la symétrie d'axe (BD) est le triangle BJF.
				
\item %Quelle est l'image du triangle AMH par la translation qui transforme le point E en B ?
La translation qui transforme le point E en B  transforme A en E, M en F et H en M.

Donc le triangle AMH a pour image EFM.
\item %Par quelle transformation passe-t-on du triangle AIH au triangle AMD ?
Le triangle AMD contient 4 triangles identiques au triangle initial BEJ ; l'aire étant le quadruple de celle du triangle initial ses dimensions sont le double de celle de AIH.

Le point A étant commun aux deux triangles le triangle AMD est l'image du triangle AIH par l'homothétie de centre A et de rapport 2.
			\end{enumerate}
%		\end{minipage}\hfill
%		\begin{tikzpicture}[baseline={(A)}]
%			\draw (0,2) node (A)[above]{A}-- (-2,0) node[left]{B} -- (0,-2) node[below] {C} -- (2,0) node[right]{D} -- cycle;
%			\draw (-1,1) node[above left ]{E} -- (-1,-1) node[below left] {F} --(1,-1) node [below right] {G} -- (1,1) node [above right] {H} -- cycle;
%			\draw (-2,0)--(-1,0) node[above left] {J} -- (0,0) node[shift={(20:.4)}] {M} -- (1,0) node [above right] {L}--(2,0);
%			\draw (0,2)--(0,1) node[above right]{I} -- (0,-1) node[below right] {K}--(0,-2);
%			\draw (-1,1)--(1,-1) (-1,-1)--(1,1);
%		\end{tikzpicture}
		
\item %Calculer en détaillant les étapes : 

%\[\dfrac{7}{2} + \dfrac{15}{6} \times \dfrac{7}{25}\]
$\dfrac{7}{2} + \dfrac{15}{6} \times \dfrac{7}{25} = \dfrac{7}{2} + \dfrac{15\times 7 }{6\times25} = 
 \dfrac{7}{2} + \dfrac{5 \times 3 \times 7 }{2 \times 3 \times 5\times 5} =  \dfrac{7}{2} + \dfrac{7}{10} = \dfrac{35}{10} + \dfrac{7}{10} = \dfrac{42}{10} = \dfrac{2 \times 21}{2 \times 5} = \dfrac{21}{5}$
%On donnera le résultat sous la forme d'une fraction irréductible.
		
		\item %Pour cette question, on indiquera sur la copie l'unique bonne réponse. Sachant que le diamètre de la Lune est d'environ \np[km]{3474}, la valeur qui approche le mieux son volume est :
		
%		\renewcommand{\arraystretch}{1.5}
%		\begin{tabularx}{\linewidth}{|*{4}{>{\centering \arraybackslash}X|}} \hline
%		Réponse A     & Réponse B     &	Réponse C     &	Réponse D   \\ \hline
%		$12,3 \times 10^{17}\mathrm{~km}^3$&  $\np{1456610}\mathrm{~km}^3$	&
%		$1,8 \times 10^{11}\mathrm{~km}^3$ &  $2,2 \times 10^{10}\mathrm{~km}^3$ \\ \hline
%		\end{tabularx}
Une boule de rayon $R$ a un volume de $V = \dfrac{4}{3}\times \pi R^3$.

Donc le volume de la Lune est environ :

$V_{\text{Lune}} \approx \dfrac{4}{3}\times \pi \times \np{1737}^3 \approx \np{2,195} \times 10^{10}$ ; donc réponse D : $2,2 \times 10^{10}$.	
		
		\item %On considère un triangle RST rectangle en S. Compléter le tableau donné en ANNEXE à rendre avec la copie.
%On arrondira la valeur des angles à l'unité. 
Pour les angles, on peut utiliser le cosinus, le sinus ou la tangente.

Avec le cosinus : $\cos \widehat{\text{STR}} = \dfrac{\text{ST}}{\text{RT}} = \dfrac{24}{26} = \dfrac{12}{13}$.

La calculatrice donne $\widehat{\text{STR}} \approx 22,6$, soit 23\degres au degré près.

L'angle complémentaire $\widehat{\text{SRT}}$ mesure donc $67\degres$ au degré près.

Voir le tableau à la fin.
	\end{enumerate}

\vspace{0,5cm}

\textbf{Annexe à rendre avec la copie - question 5 :}

\medskip

\begin{tabularx}{\linewidth}{|*{4}{>{}X|}} \hline
\textbf{Longueurs} & \textbf{Angles} & \textbf{Périmètre du triangle RST} & \textbf{Aire du triangle RST}\\ \hline
	
\rule[-3mm]{0mm}{10mm}RS = 10 mm & $\widehat{\mathrm{RST}} = 90\degres$ & \multirow{3}{*}{\rule{0mm}{12mm}$ \mathcal{P} = 10 + 24 + 36 = 60$~(mm)}
	& \multirow{3}{*}{\rule{0mm}{12mm}$ \mathcal{A} = \dfrac{10 \times 24}{2} = 120$~mm$^2$}
	\\ \cline{1-2}
	\rule[-3mm]{0mm}{10mm}ST = 24 mm & $\widehat{\mathrm{STR}} \approx $23\degres&&\\ \cline{1-2}
	\rule[-3mm]{0mm}{10mm}RT = 26 mm & $\widehat{\mathrm{SRT}} \approx $67\degres&&\\ \hline
	
\end{tabularx}


