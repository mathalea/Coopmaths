
	\textbf{Partie 1}
	
	Dans cette première partie, on lance un dé bien équilibré à six faces numérotées de 1 à 6, puis on note le numéro de la face du dessus.
	
	\begin{enumerate}
		\item %Donner sans justification les issues possibles.
Les issues sont 1,\,2,\,\,3,\,\,4,\,\,5,\,\,6.
		\item %Quelle est la probabilité de l'évènement A : « On obtient 2 » ?
La probabilité d'obtenir le 6 (comme les autres nombres) est $\dfrac{1}{6}$.
		
		\item %Quelle est la probabilité de l'évènement B : « On obtient un nombre impair » ?
Il y a 3 nombres impairs (ou pairs). la probabilité est donc égale à $\dfrac{3}{6} = \dfrac{1}{2}$.
	\end{enumerate}

	\textbf{Partie 2}
	
Dans cette deuxième partie, on lance simultanément deux dés bien équilibrés à six faces, un rouge et un vert. On appelle « score » la somme des nombres correspondants aux issues  de chaque dé.
	
	\begin{enumerate}
		\item %Quelle est la probabilité de l'évènement C : « le score est 13 » ? Comment appelle-t-on un tel évènement ?
La plus grande somme possible étant 12, l'évènement est impossible de probabilité nulle.
		
		\item %Dans le tableau à double entrée donné en ANNEXE, on remplit chaque case avec la somme des numéros obtenus sur chaque dé.
		
		\begin{enumerate}
			\item %Compléter, sans justifier, le tableau donné en ANNEXE à rendre avec la copie.
Voir à la fin 
			\item %Donner la liste des scores possibles.
			Les scores possibles sont : 2,\,3,\,4,\,5,\,6,\,7,\,8,\,9,\,10,\,11,\,12, soit 11 scores différents possibles 
		\end{enumerate}
\item 
	\begin{enumerate}
		\item %Déterminer la probabilité de l'évènement D : « le score est 10 ».
		Il y a $6 \times 6 = 36$ issues possibles.
		
On a $10 = 4 + 6 = 5 + 5  = 6 + 4$ : 3 issues, donc $p(D) = \dfrac{3}{36} = \dfrac{1}{12}$.
		\item %Déterminer la probabilité de l'évènement E : « le score est un multiple de 4 ».
On a $p(E) = \dfrac{9}{36} = \dfrac{1}{4}$.
		\item %Démontrer que le score obtenu a autant de chance d'être un nombre premier qu'un nombre strictement plus grand que 7.
Il y a 15 scores premiers et 15 scores supérieurs à 15.
		\end{enumerate}
	\end{enumerate}

\vspace{0,5cm}

\textbf{Annexe à rendre avec la copie - Partie 2, question 2. a.}

\begin{center}
	\begin{tabularx}{0.8\linewidth}{|>{\centering \arraybackslash}p{3cm}|*{6}{>{\centering \arraybackslash \rule[-3mm]{0mm}{10mm}}X|}}  \hline
	\backslashbox{\rule{0mm}{6mm}Dé rouge}{Dé vert}&1&2&3&4&5&6\\ \hline
	1& 2& 3& 4& 5& 6& 7\\ \hline
	2& 3& 4& 5& 6& 7& 8\\ \hline
	3& 4& 5& 6& 7& 8& 9\\ \hline
	4& 5&6& 7& 8& 9& 10\\ \hline
	5& 6& 7& 8& 9& 10& 11\\ \hline
	6& 7& 8& 9& 10& 11& 12\\ \hline
\end{tabularx}
\end{center}

