
\begin{enumerate}
\item Dans le triangle $ABC$ rectangle en $B$, l'hypoténuse est $[AC]$. D'après le théorème de Pythagore, on a:
\begin{align*}
AC^2&=AB^2+BC^2\\
&=1^2+1^2\\
&=2\\
AC=\sqrt{2} cm
\end{align*}
\item
	\begin{enumerate}
		\item Les longueurs sont multipliées par $2$.
\remarque{\texttt{"en doublant les longueurs", c'est écrit dans l'énoncé.}}
		\item C'est une homothétie.
\remarque{ de centre $A$ et de rapport $2$... donc les surfaces seront multipliées par $2^2=4$.}
	\end{enumerate}
\item Les longueurs du carré 3 sont celles du carrés 2 multipliées par $2$ qui elles-mêmes sont celles du carré 1 multipliées par $2$. Les longueurs du carré 3 sont donc égales à celles du carré 1 multipliées par $4$. L'affirmation est donc fausse.
\item Dans le triangle $AJB$ rectangle en $A$, $AJ=4cm$ et $AB=1cm$. On utilise la trigonométrie:

\begin{center}
C\textbf{A}H \hspace{20 pt} S\textbf{O}H \hspace{20 pt} T\textbf{O}\textbf{A}
\end{center}

\begin{align*}
\tan (\widehat{AJB})&=\frac{\texttt{côté opposé }}{côté adjacent}\\
&=\frac{AB}{AJ}\\
&=\frac{1}{4}\\
\widehat{AJB}&=\arctan\left(\frac{1}{4}\right)\\
&\simeq 14^\circ
\end{align*}

\end{enumerate}

