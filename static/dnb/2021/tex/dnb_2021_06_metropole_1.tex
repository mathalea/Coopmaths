
\medskip

Cette feuille de calcul présente les températures moyennes mensuelles à Tours en 2019.

\begin{center}
\begin{tabularx}{\linewidth}{|c|m{1.6cm}|*{12}{>{\centering \arraybackslash}X|}m{1.5cm}|}\hline
	&A		&B&C&D&E&F&G&H&I&J&K&L&M&N\\ \hline
1	&Mois 	&J&F&M&A&M&J&J&A&S&O&N&D&Moyenne sur l'année\\ \hline
2	&Températu\-re en \degres C& 4,4& 7,8& 9,6& 11,2& 13,4& 19,4& 22,6& 20,5& 17,9& 14,4& 8,2& 7,8& \\ \hline
\end{tabularx}
\end{center}

\smallskip

\begin{enumerate}
\item D'après le tableau ci-dessus, quelle a été la température moyenne à Tours en novembre 2019 ?
\item Déterminer l'étendue de cette série.
\item Quelle formule doit-on saisir en cellule N2 pour calculer la température moyenne annuelle ?
\item Vérifier que la température moyenne annuelle est $13,1~\degres$C.
\item La température moyenne annuelle à Tours en 2009 était de $11,9~\degres$C.

Le pourcentage d'augmentation entre 2009 et 2019, arrondi à l'unité, est-il de : 
7\,\% ; 10\,\%  ou 13\,\%  ? Justifier la réponse.
\end{enumerate}


\bigskip

