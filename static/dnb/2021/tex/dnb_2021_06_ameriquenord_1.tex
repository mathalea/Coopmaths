
Pour chacune des six affirmations suivantes, indiquer sur la copie, si elle est vraie ou fausse.

\medskip

\textbf{On rappelle que chaque réponse doit être justifiée.}

\begin{enumerate}[itemsep=5mm]
	\item On considère la fonction $f$ définie par $f(x) = 3x - 7$
	
	\textbf{Affirmation \no 1 :} \og{}L’image par $f$ du nombre $- 1$ est 2 \fg{}.
	
	\item On considère l'expression $\mathrm{E} = (x - 5)(x + 1)$.
	
	\textbf{Affirmation \no 2 :} \og{} L'expression $ \mathrm{E} $ a pour forme développée et réduite $x^2 - 4x - 5 $\fg{}.
	
	\item $n$ est un nombre entier positif.
	
	\textbf{Affirmation \no 3 :} \og{}lorsque $n$ est égal à 5, le nombre $2^n +1$ est un nombre premier \fg{}.
	
	\item On a lancé 15 fois un dé à six faces numérotées de 1 à 6 et on a noté les fréquences d'apparition dans le tableau ci-dessous :
	
	\smallskip
	\begin{tabularx}{\linewidth}{|m{3.5cm}|*{6}{>{\centering \arraybackslash}X|}} \hline
Numéro de la face apparente & 1& 2& 3& 4& 5& 6 \\ \hline
Fréquence d'apparition&$\dfrac{3}{15}$&$\dfrac{4}{15}$&$\dfrac{5}{15}$&$\dfrac{2}{15}$&$\dfrac{1}{15}$ &\dots\rule[-3mm]{0mm}{9mm}\\ \hline
	\end{tabularx}
	
	\smallskip
	
	\textbf{Affirmation \no 4 :} \og{}la fréquence d'apparition du 6 est 0 \fg{}.
	
	\item On considère un triangle RAS rectangle en S.
	
Le côté [AS] mesure 80 cm et l'angle $ \widehat{\mathrm{ARS}} $ mesure $26\degres$.
	
	\textbf{Affirmation \no 5 :} le segment [RS] mesure environ 164~cm.
	
	\item Un rectangle ABCD a pour longueur 160 cm et pour largeur 95~cm.
	
	\textbf{Affirmation \no 6 :} les diagonales de ce rectangle mesurent exactement 186~cm.
\end{enumerate}

\bigskip

