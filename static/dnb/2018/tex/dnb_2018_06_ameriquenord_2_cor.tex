
\medskip
%\begin{minipage}[t]{0.5\linewidth}
%	La figure ci-contre n'est pas en vraie grandeur.
%	On donne les informations suivantes :
%	\begin{itemize}
%		\item Le triangle ADE a pour dimensions :
%		
%			AD = 7 cm, AE = 4,2 cm et DE = 5,6 cm.
%		
%		\item F est le point de [AD] tel que AF = 2,5 cm.
%		\item B est le point de [AD) et C est le point de [AE) tels que : AB = AC = 9 cm.
%		\item La droite (FG) est parallèle à la droite (DE).
%	\end{itemize}
%
%\smallskip
%
%	\begin{enumerate}
%		\item Réaliser une figure en vraie grandeur.
%		\item Prouver que ADE est un triangle rectangle en E.
%		\item Calculer la longueur FG.
%	\end{enumerate}
%\end{minipage}
\parbox{0.55\linewidth}{
\begin{enumerate}
\item Voir ci-contre
\item On calcule :

$\text{AD}^2 = 7^2 = 49$, \: $\text{AE}^2 = 4,2^2 = 17,64$ et 

$\text{DE}^2 = 5,6^2 = 31,36$.

Or $17,64 + 31,36 = 49$ ou encore $\text{AE}^2 + \text{DE}^2 = \text{AD}^2$, ce qui montre d'après la réciproque de Pythagore que le triangle ADE est rectangle en E car d'hypoténuse [AD].
\item Dans le triangle ADE on a (FG) parallèle à (DE) ; on a donc une configuration de Thalès et par conséquent l'égalité de quotients :

$\dfrac{\text{FG}}{\text{DE}} = \dfrac{\text{AF}}{\text{AD}}$, soit $\dfrac{\text{FG}}{5,6} = \dfrac{2,5}{7}$.

On a donc FG $ = \dfrac{2,5}{7} \times 5,6 = \dfrac{14}{7} = 2$~cm.
\end{enumerate}}\hfill
\parbox{0.45\linewidth}{\psset{unit=0.75cm}
\begin{pspicture}(-4,-4)(5,5.6)
%\psgrid
\pscircle(0,0){2.8}
\psline(2.8;30)(2.8;210)
\psarc(-2.4,-1.4){7}{40}{80}
\psarc(2.8;30){4.2}{100}{130}
\psline(-2.4,-1.4)(0.34,5.03)(2.4,1.4)
\psarc(0.34,5.03){4.2}{240}{260}
\psline(-1.3,1.18)(3.5,3.58)
\uput[l](2.8;210){D} \uput[r](2.8;30){E} \uput[u](0.34,5.03){A} 
\uput[ul](-1.3,1.18){F} \uput[u](1.66,2.64){G} 
\uput[dl](-3.22,-3.22){B} \uput[dr](4.8,-2.8){C} 
\psline(2.8;30)(-1.3,1.18)
\psarc(0.34,5.03){9}{240}{300}
\psline(2.8;210)(-3.22,-3.22)
\psline(2.8;30)(4.8,-2.8)
\end{pspicture}
}

\vspace{0,5cm}

