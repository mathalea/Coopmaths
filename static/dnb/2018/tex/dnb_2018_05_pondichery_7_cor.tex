
\medskip

%\parbox{0.4\linewidth}{La figure ci-contre n'est pas
%à l'échelle}\hfill \parbox{0.57\linewidth}{\psset{unit=1cm}
%\begin{pspicture}(0,-0.75)(7.5,5)
%%\psgrid
%\pspolygon(0.5,0.5)(7.2,0.5)(0.5,4.4)%ABC
%\uput[dl](0.5,0.5){A} \uput[ur](7.2,0.5){B}\uput[u](0.5,4.4){C}
%\uput[ur](2.2,3.4){H}
%\psdots[dotstyle=+,dotangle=45,dotscale=1.3](0.5,0.5)(7.2,0.5)(0.5,4.4)(2.2,3.4)
%\psline(0.5,0.5)(2.2,3.4)
%\psline[linewidth=0.6pt]{<->}(0.5,0.2)(7.2,0.2)
%\uput[d](3.85,0.2){7 cm}
%\psframe(0.5,0.5)(1,1)\rput{-120}(2.2,3.4){\psframe(0,0)(0.5,0.5)}
%\psarc(7.2,0.5){0.5cm}{-210}{-180}
%\rput(6.3,0.7){30~\degres}
%\end{pspicture}}
%\medskip
%
%On  considère ci-dessus un triangle ABC rectangle en A tel que $\widehat{\text{ABC}} = 30~\degres$ et AB = 7 cm. H est le pied de la hauteur issue de A.
%
%\medskip

\begin{enumerate}
\item %Tracer la figure en vraie grandeur sur la copie. Laisser les traits de construction apparents sur la copie.
$\bullet~~$On trace le demi-cercle de diamètre [AB] ; 

$\bullet~~$Le cercle de centre A et de rayon 3,5 coupe le demi-cercle précédent en H ;

$\bullet~~$La perpendiculaire à [AB] en A coupe la droite (BH) en C.
\item %Démontrer que AH $= 3,5$~cm.
Dans le triangle ABH rectangle en H : $\widehat{\text{BAH}} = 90 - \widehat{\text{ABC}} = 90 - 30 = 60$\degres.

Donc AH = $ \text{AB} \times \cos 60 = 7 \times \dfrac{1}{2} = \dfrac{7}{2} = 3,5$~(cm).
\item %Démontrer que les triangles ABC et HAC sont semblables.
Les triangles ABC et HAC sont rectangles, ont en commun l'angle en C de mesure 60\degres, donc leurs troisièmes angles ont pour mesure 30\degres : ils sont donc semblables
\item %Déterminer le coefficient de réduction permettant de passer du triangle ABC au triangle HAC.
En comparant les côtés adjacents aux angles de mesure 30\degres, on a un coefficient de réduction de :

$\dfrac{\text{AH}}{\text{AB}} = \dfrac{3,5}{7} = \dfrac{1}{2} = 0,5$.

Les dimensions de HAC sont deux fois plus petites que celles du triangle ABC.
\end{enumerate}


