
\medskip

%Un garçon et une fille pratiquent le roller. Ils décident de faire une course en
%empruntant deux parcours différents. 
%
%La fille, qui part du point F et arrive au point A, met 28,5 secondes. 
%
%Le garçon, qui part du point G et arrive aussi au point A, met 28
%secondes.
%
%Le dessin ci-après, qui n'est pas à l'échelle, représente les deux parcours; celui de la
%fille comporte deux demi-cercles de $5$~m de rayon.

\begin{center}
\psset{unit=1cm}
\begin{pspicture}(12,3)
%\psgrid
\psline(0,2.5)(10,2.5)
\psarc(10,2){0.5}{-90}{90}
\psline(10,1.5)(8,1.5)
\psarc(8,1){0.5}{90}{270}
\psline(8,0.5)(10,0.5)
\uput[u](4,2.5){200 m} \uput[l](0,2.5){G}\uput[u](8,2.5){A} 
\uput[r](8,1.25){5 m}\uput[u](10,2.5){B}\uput[d](10,1.5){C}
\uput[u](8,1.5){D}\uput[d](8,0.5){E}\uput[r](10,0.5){F}
\uput[u](8,2.5){A}\uput[u](9,2.5){60 m}
\psdots[dotstyle=+,dotangle=45](8,2.5)(10,2.5)(10,1.5)(8,1.5)(8,0.5)(10,0.5)(0,2.5)
\rput(9,2.5){|||}\rput(9,1.5){|||}\rput(9,0.5){|||}
\psline{->}(8,1)(8,1.5)\psline{->}(10,2)(10,2.5)
\uput[r](10,2.25){5 m}
\end{pspicture}
\end{center}

\begin{enumerate}
\item %Quel est le parcours le plus long ?
Le trajet de la fille a une longueur de $3 \times 60 + 2 \times 5 \times \pi  = 180 + 10\pi \approx 211,4$~(m) contre 200~(m) pour le garçon.

La fille a le trajet le plus long.
\item %Qui se déplace le plus vite, le garçon ou la fille ?
$\bullet~~$Vitesse du garçon : $\dfrac{200}{28} \approx 7,14$~(m/s) ;

$\bullet~~$Vitesse de la fille : $\dfrac{180 + 10\pi}{28,5} \approx 7,41$~(m/s).

La fille est la plus rapide.
\end{enumerate}
%\smallskip
%
%\emph{On rappelle que si $p$ est le périmètre d'un cercle de rayon $r$, alors} $p = 2 \times \pi \times r$.
%
%\bigskip

