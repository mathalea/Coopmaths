
\medskip

Une personne pratique le vélo de piscine depuis plusieurs années dans un centre
aquatique à raison de deux séances par semaine. Possédant une piscine depuis peu,
elle envisage d'acheter un vélo de piscine pour pouvoir l'utiliser exclusivement chez
elle et ainsi ne plus se rendre au centre aquatique.

\medskip

\setlength\parindent{6mm}
\begin{itemize}
\item[$\bullet~~$] Prix de la séance au centre aquatique: 15~\euro.
\item[$\bullet~~$] Prix d'achat d'un vélo de piscine pour une pratique à la maison: 999~\euro.
\end{itemize}
\setlength\parindent{0mm}

\medskip

\begin{enumerate}
\item Montrer que 10 semaines de séances au centre aquatique lui coûtent 300~\euro.
\item Que représente la solution affichée par le programme ci-après?

\begin{center}
\begin{scratch}
\blockinit{quand \greenflag est cliqué}
\blockvariable{mettre \selectmenu{x} à \txtbox{0}}
\blockinfloop{répéter jusqu'à \booloperator{\ovalvariable{x} * 2 * 15 > \txtbox{999}}}
{\blockvariable{ajouter à \selectmenu{x} \ovalnum{1}}
}
\blocklook{dire \ovaloperator{regroupe}{\txtbox{La solution est :}}\ovalvariable{x}}

\end{scratch}
\end{center}

\item  Combien de semaines faudrait-il pour que l'achat du vélo de piscine soit
rentabilisé?
\end{enumerate}

\bigskip

