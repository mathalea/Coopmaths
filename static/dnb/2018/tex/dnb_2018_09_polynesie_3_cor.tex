
\medskip

%Une personne s'intéresse à un magazine sportif qui parait une fois par semaine. Elle
%étudie plusieurs formules d'achat de ces magazines qui sont détaillées ci-après.
%
%\begin{center}
%\begin{tabularx}{\linewidth}{|X|}\hline
%$\bullet~~$ Formule A - Prix du magazine à l'unité: 3,75~\euro{} ;\\
%$\bullet~~$ Formule B - Abonnement pour l'année: 130~\euro{} ;\\
%$\bullet~~$ Formule C - Forfait de 30~\euro{} pour l'année et 2,25~\euro{} par magazine.\\ \hline
%\end{tabularx}
%\end{center}
%
%On donne ci-dessous les représentations graphiques qui correspondent à ces trois
%formules.
%
\begin{center}
\psset{xunit=0.225cm,yunit=0.05cm,arrowsize=2pt 3}
\begin{pspicture}(-1,-5)(49,150)
\multido{\n=0+2}{25}{\psline[linewidth=0.3pt](\n,0)(\n,150)}
\multido{\n=0+20}{8}{\psline[linewidth=0.3pt](0,\n)(49,\n)}
\psaxes[linewidth=1.25pt,Dx=2,Dy=20]{->}(0,0)(0,0)(49,150)
\psaxes[linewidth=1.25pt,Dx=2,Dy=20](0,0)(0,0)(49,150)
\uput[u](41,0){Nombre de magazines}
\uput[r](0,145){Coût en euros}
\psline(40,150)
\psline(0,130)(49,130)
\psplot[plotpoints=3000,linewidth=1.25pt]{0}{48}{2.25 x mul 30 add}
\uput[d](30,95){(D$_1$)} \uput[u](30,130){(D$_2$)} \uput[u](30,112){(D$_3$)}
\psline{->}(16,0)(16,60)(0,60) \uput[l](0,60){60}
\psline{->}(0,120)(40,120)(40,0)\uput[d](40,2){40}
\psline{->}(0,100)(31.1,100)(31.1,0)\uput[d](31.1,2){31.1}
\end{pspicture}
\end{center}

\begin{enumerate}
\item %Sur votre copie, recopier le contenu du cadre ci-dessous et relier par un trait
%chaque formule d'achat avec sa représentation graphique.
%
%\begin{center}
%\newcolumntype{Y}{>{\raggedleft \arraybackslash}X}
%\begin{tabularx}{0.6\linewidth}{|X Y|}\hline
%Formule A  \psdots[dotstyle=+,dotangle=45](0.1,0.1)&\psdots[dotstyle=+,dotangle=45](0,0.1)~(D1)
%\\
%Formule B \psdots[dotstyle=+,dotangle=45](0.1,0.1)&\psdots[dotstyle=+,dotangle=45](0,0.1)~(D2)\\
%Formule C \psdots[dotstyle=+,dotangle=45](0.1,0.1)&\psdots[dotstyle=+,dotangle=45](0,0.1)~(D3)\\ \hline
%\end{tabularx}
%\end{center}
La formule A est représentée par la droite $\left(\text{D}_3\right)$.

La formule B est représentée par la droite $\left(\text{D}_2\right)$.

La formule C est représentée par la droite $\left(\text{D}_1\right)$.
\item  %En utilisant le graphique, répondre aux questions suivantes.

\emph{Les traits de construction devront apparaitre sur le graphique en ANNEXE qui est à rendre avec la copie.}
	\begin{enumerate}
		\item %En choisissant la formule A, quelle somme dépense-t-on pour acheter 16 magazines dans l'année ?
On lit à peu près 60 \euro.
		\item %Avec $120$~\euro, combien peut-on acheter de magazines au maximum dans une année avec la formule C ?
On lit à peu près 40 numéros.
		\item %Si on décide de ne pas dépasser un budget de $100$~\euro{} pour l'année, quelle est alors la formule qui permet d'acheter le plus grand nombre de magazines ?
		On lit un peu plus de 31 magazines mais pas 32 avec la formule C.
	\end{enumerate}
\item  %Indiquer la formule la plus avantageuse selon le nombre de magazines achetés dans l'année.
Suivant le nombre de magazines en abscisse on regarde en suivant la verticale laquelle des trois droites est rencontrée la première.

De 0 à 19 magazines la formule la plus avantageuse est la formule A.

Pour 20 magazines les formules A et C sont les plus avantageuses.

De 21 à 44 magazines la formule la plus avantageuse est la formule C.

De 45 à 52 magazines le forfait de la formule B est le plus avantageux.
\end{enumerate}


\begin{center}

	\textbf{\large ANNEXE}
	
	\medskip
	
	\textbf{À détacher du sujet et à joindre avec la copie}
	
	\vspace{1cm}
	
	\textbf{question 2}
	
	\medskip
	
	\begin{center}
	\psset{xunit=0.225cm,yunit=0.05cm}
	\begin{pspicture}(-1,-5)(49,150)
	\multido{\n=0+2}{25}{\psline[linewidth=0.3pt](\n,0)(\n,150)}
	\multido{\n=0+20}{8}{\psline[linewidth=0.3pt](0,\n)(49,\n)}
	\psaxes[linewidth=1.25pt,Dx=2,Dy=20]{->}(0,0)(0,0)(49,150)
	\psaxes[linewidth=1.25pt,Dx=2,Dy=20](0,0)(0,0)(49,150)
	\uput[u](41,0){Nombre de magazines}
	\uput[r](0,145){Coût en euros}
	\psline(40,150)
	\psline(0,130)(49,130)
	\psplot[plotpoints=3000,linewidth=1.25pt]{0}{48}{2.25 x mul 30 add}
	\uput[d](30,95){(D$_1$)} \uput[u](30,130){(D$_2$)} \uput[u](30,112){(D$_3$)} 
	\psline{->}(16,0)(16,68)(0,68) \uput[u](0,68){68}
	\psline{->}(0,120)(48,120)(48,0)\uput[d](48,0){48}
	\psline(0,100)(31,1,100)(31,1,0)\uput[d](31.1,0){31.1}
	\end{pspicture}
	\end{center}
\end{center}
\bigskip

