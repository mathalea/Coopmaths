
\medskip

\parbox{0.55\linewidth}{Après un de ses entraînements de course à pied, Bob reçoit
de la part de son entraineur le récapitulatif de sa course,
reproduit ci-contre.

L'\textbf{allure} moyenne du coureur est le quotient de la durée de
la course par la distance parcourue et s'exprime en min/km.

\smallskip

Exemple : si Bob met 18~min pour parcourir 3~km, son allure est de 6 min/km.}
\hfill
\parbox{0.4\linewidth}{\psset{unit=0.6cm}
\begin{pspicture}(9.5,3.8)
\psframe(9.5,3.8)
\rput(3.6,3.2){Entrainement course à pied}
\rput(1.5,2.6){\textbf{10,5 km}} \rput(4.2,2.6){\textbf{1 h 03 min}} \rput(7.3,2.6){\textbf{6 min/km}}
\rput(1.5,2.1){Distance} \rput(4.2,2.1){Durée} \rput(7.3,2.1){Allure moyenne}
\rput(3.2,1){\textbf{851}}\rput(6.4,1){\textbf{35 m}}
\rput(3.2,0.5){Calories}
\rput(6.4,0.5){Gain altitude}
\end{pspicture}}

\begin{enumerate}
\item Bob s'étonne de ne pas voir apparaître sa vitesse moyenne. Calculer cette vitesse moyenne en km/h.
\item Soit $f$ la fonction définie pour tout $x > 0$ par $f(x) = \dfrac{60}{x}$, où $x$ est l'allure en min/km et $f(x)$ est la vitesse en km/h.

Cette fonction permet donc de connaître la vitesse (en km/h) en fonction de l'allure (en min/km).
	\begin{enumerate}
		\item La fonction $f$ est-elle une fonction linéaire ? Justifier.
		\item Lors de sa dernière course, l'allure moyenne de Bob était de 5 min/km.
		
Calculer l'image de 5 par $f$. Que représente le résultat obtenu ?
	\end{enumerate}
\item Répondre aux questions suivantes en utilisant la représentation graphique de la fonction $f$ ci-dessous:
	\begin{enumerate}
		\item Donner un antécédent de $10$ par la fonction $f$.
		\item Un piéton se déplace à environ $14$~min/km. Donner une valeur approchée de sa vitesse en km/h.

\begin{center}
\psset{unit=0.3cm}
\begin{pspicture}(-1,-3)(36,34)
\psgrid[gridlabels=0pt,subgriddiv=1,gridwidth=0.2pt](0,0)(36,32)
\psaxes[linewidth=1.25pt,Dx=5,Dy=5]{->}(0,0)(0,0)(36,32)
\uput[u](35.5,0){$x$}\uput[r](0,31.5){$y$}
\uput[r](0,33){Vitesse en km/h}
\uput[d](32,-2){Allure en min/km}
\psplot[plotpoints=3000,linewidth=1.25pt,linecolor=blue]{1.9}{32}{60 x div}
\end{pspicture}
\end{center}
\end{enumerate}
\end{enumerate}

\vspace{0,5cm}

