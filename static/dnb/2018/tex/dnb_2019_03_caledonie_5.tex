
\medskip

Voici le classement des 21 pays ayant obtenu des médailles d'or lors des jeux olympiques d'hiver de
Pyeongchang 2018 en Corée.


\begin{tabularx}{\linewidth}{c*{21}{>{\centering \arraybackslash}X}}
\rotatebox{45}{Pays}&\rotatebox{30}{Norvège}	&\rotatebox{30}{Allemagne}	&\rotatebox{30}{Canada}	&\rotatebox{30}{États-Unis}	&\rotatebox{30}{Pays-Bas}	&\rotatebox{30}{Suède}	&\rotatebox{30}{Rép. de Corée}	&\rotatebox{30}{Suisse}	&\rotatebox{30}{France}	&\rotatebox{30}{Autriche}	&\rotatebox{30}{Japon}	&\rotatebox{30}{Italie}	&\rotatebox{30}{Russie}	&\rotatebox{30}{Rép. Tchèque}	&\rotatebox{30}{Bélarus}	&\rotatebox{30}{Chine}	&\rotatebox{30}{Slovaquie}	&\rotatebox{30}{Finlande}	&\rotatebox{30}{Grande Bretagne}	&\rotatebox{30}{Pologne}	 &\rotatebox{30}{Hongrie}\\
\end{tabularx}
\begin{tabularx}{\linewidth}{|c|*{21}{>{\centering \arraybackslash}X|}}\hline
Or					&14 &14	&11 &9 	&8 	&7 	&5 	&5 	&5 	&5 	&4 	&3 	&2 	&2 	&2 	&1 	&1 	&1 	&1 	&1 &1\\ \hline
\end{tabularx}
\medskip

On considère la série constituée des nombres de médailles d'or obtenues par chaque pays.

Le classement est résumé dans la feuille de calcul ci-dessous:

\begin{center}
\begin{tabularx}{\linewidth}{|c|l|*{11}{>{\centering \arraybackslash}X|}}\hline
	&A 					&B &C &D &E &F 	&G 	&H 	&I	& J	& K &L\\ \hline
1 	&Nombre de médailles&1 &2 &3 &4	&5	&7	&8	&9	&11 &14	&\\ \hline
2 	&Effectif			&6 &3 &1 &1 &4 	&1 &1 	&1 	&1 	&2 	&21\\ \hline
\end{tabularx}
\end{center}

\smallskip

\begin{enumerate}
\item 
	\begin{enumerate}
		\item Calculer le nombre moyen de médailles d'or par pays (arrondir le résultat au dixième).
		\item Déterminer la médiane des nombres de médailles d'or par pays.
		\item Interpréter le résultat de la question \textbf{1. b.}
	\end{enumerate}
\item Quelle formule a-t-on saisie dans la cellule L2 pour obtenir le nombre total de pays ayant eu au moins une médaille d'or ?
\item On prend un pays au hasard parmi les pays qui ont au moins une médaille d'or.
	\begin{enumerate}
		\item Quelle est la probabilité qu'il ait une seule médaille d'or? Donner la réponse sous forme fractionnaire.
		\item Quelle est la probabilité qu'il ait au moins 5 médailles d'or? Donner la réponse sous forme fractionnaire.
	\end{enumerate}
\end{enumerate}

\vspace{0,5cm}

