
\medskip 

\emph{Les réponses aux questions de cet exercice seront lues sur la graphique de l’annexe $1$, située à la fin de ce sujet.}

\medskip 

Celui-ci représente le profil d'une course à pied qui se déroule sur l'île de la Réunion (ce graphique exprime l'altitude en fonction de la distance parcourue par les coureurs), 

\smallskip

\emph{Aucune justification n’est attendue pour les questions $1$ à $4$.}

\medskip 

\begin{enumerate}
\item Quelle est la distance parcourue par un coureur, en kilomètres, lorsqu'il arrive au sommet de la plaine des Merles ? 
\item Quelle est l’altitude atteinte, en mètres, au gîte du Piton des Neiges ?
\item Quel est le nom du sommet situé à 900 mètres d’altitude ? 
\item À quelle(s) distance(s) du départ un coureur atteindra-t-il \np{1900}~m d’altitude ?
\item Le dénivelé positif se calcule uniquement dans les montées ; pour chaque montée, il est égal à la différence entre l’altitude la plus haute et l’altitude la plus basse.
	\begin{enumerate}
		\item Calculer le dénivelé positif entre Cilaos et le gîte du Piton des Neiges. 
		\item Montrer que le dénivelé positif total de cette course est \np{4000}~m. 
	\end{enumerate}
\item  Maëlle a effectué sa course à une vitesse moyenne de 7 km/h et Line a mis 13~h~20~min pour passer la ligne d'arrivée. 

Laquelle de ces deux sportives est arrivée en premier ? 
\end{enumerate}

\begin{center}
	\textbf{\large Annexe 1}
	
	\begin{flushleft}
	\textbf{profil de la course}
	\end{flushleft}
	
	\bigskip
	
	\psset{xunit=0.12cm,yunit=0.005cm}
	\begin{pspicture}(-4,-600)(105,2900)
	\multido{\n=0+1}{101}{\psline[linewidth=0.2pt](\n,0)(\n,2600)}
	\multido{\n=0+100}{27}{\psline[linewidth=0.2pt](0,\n)(100,\n)}
	\psaxes[linewidth=1.25pt,Dx=5,Dy=100]{->}(0,0)(0,0)(100,2700)
	\psline[linewidth=1.25pt,linecolor=blue](0,1200)(13,2500)(22,1000)(27,700)(37,1800)(48,700)(56,300)(60,900)(71,0)(74,300)(78,0)(88,700)(93,0)
	
	\rput{-45}(10,-400){{Départ : Cilaos : 0 km}}
	\rput{-45}(23,-490){{Gîte : Piton des neiges : 13 km}}
	\rput{-45}(37,-370){{Trou blanc :  27 km}}
	\rput{-45}(47,-370){{Plaine des merles}}
	\rput{-45}(66,-370){{Deux bras :  56 km}}
	\rput{-45}(70,-370){{Dos d’âne : 60 km}}
	\rput{-45}(79,-370){{Possession : 70 km}}
	\rput{-45}(85,-450){{Chemin des anglais : 74 km}}
	\rput{-45}(88,-370){{La Chaloupe : 78 km}}
	\rput{-45}(96,-370){{Colorado : 86 km}}
	\rput{-45}(103,-460){{Arrivée :La Redoute : 93 km}}
	\end{pspicture}
	\end{center}
\vspace{0.5cm}

