
\medskip

\textbf{Partie A. Parcours du robot}

\medskip

On sait que (CE) et (BD) se coupent en F et que (BC) // (DE).

D'après la propriété de Thalès, on a :

$\dfrac{\text{FD}}{\text{FB}} = \dfrac{\text{DE}}{\text{BC}} = \dfrac{\text{FE}}{\text{FC}}$. Soit 

$\dfrac{4}{5} = \dfrac{\text{DE}}{80} = \dfrac{\text{FE}}{\text{FC}}$.

Donc DE $= \dfrac{80 \times 4}{5}= 64$~(m).

\bigskip

\textbf{Partie B. Programme de déplacement du robot}

\medskip

\begin{enumerate}
\item ~
\begin{center}
\begin{scratch}
\initmoreblocks{définir \namemoreblocks{Motif montant}}
\blockmove{avancer de \ovalnum{80}}
\blockmove{tourner \turnright{} de \ovalnum{90} degr\' es}
\blockmove{avancer de \ovalnum{1}}
\blockmove{tourner \turnright{} de \ovalnum{90} degr\' es}
\end{scratch}
\end{center}

\item Il suffit de tourner dans l'autre sens aux lignes 3 et 5.
\item $\dfrac{48}{2} = 24$ donc $x = 24$.

$y = 64$ (dernière longueur).
\end{enumerate}
