
\medskip

On étudie les performances de deux nageurs (nageur 1 et nageur 2).

La distance parcourue par le nageur 1 en fonction du temps est donnée par le graphique ci-dessous.

\medskip

\begin{center}
\psset{xunit=0.25cm,yunit=0.0035cm}
\begin{pspicture}(-1,-50)(50,2100)
\multido{\n=0+5}{10}{\psline[linestyle=dashed,linewidth=0.25pt](\n,0)(\n,2100)}
\multido{\n=0+200}{11}{\psline[linestyle=dashed,linewidth=0.2pt](0,\n)(50,\n)}
\psaxes[linewidth=1.25pt,Dx=5,Dy=200](0,0)(0,0)(50,2100)
\uput[r](0,2100){Distance parcourue (en mètres)}
\uput[u](44,0){Temps (en minutes)}
\psline[linewidth=1.2pt](0,0)(10,400)(30,1600)(45,2000)
\end{pspicture}
\end{center}

\smallskip

\begin{enumerate}
\item Répondre aux questions suivantes par lecture graphique. Aucune justification n'est demandée.
	\begin{enumerate}
		\item Quelle est la distance totale parcourue lors de cette course par le nageur 1 ?
		\item En combien de temps le nageur 1 a-t-il parcouru les $200$ premiers mètres ?
	\end{enumerate}
\item Y a-t-il proportionnalité entre la distance parcourue et le temps sur l'ensemble de la course ? 
	
Justifier.
\item Montrer que la vitesse moyenne du nageur 1 sur l'ensemble de la course est d'environ $44$ m/min.
\item On suppose maintenant que le nageur 2 progresse à vitesse constante.
La fonction $f$ définie par $f(x) = 50x$ représente la distance qu'il parcourt en fonction du temps $x$.
	\begin{enumerate}
		\item Calculer l'image de $10$ par $f$.
		\item Calculer $f(30)$.
	\end{enumerate}
\item Les nageurs 1 et 2 sont partis en même temps,
	\begin{enumerate}
		\item Lequel est en tête au bout de $10$~min ? Justifier.
		\item Lequel est en tête au bout de $30$~min ? Justifier.
	\end{enumerate}
\end{enumerate}

\vspace{0,5cm}

