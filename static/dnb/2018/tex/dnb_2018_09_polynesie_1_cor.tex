
\medskip

Indiquer si les affirmations suivantes sont vraies ou fausses. Justifier vos réponses.

\medskip

\textbf{Affirmation 1}

\smallskip

%On lance un dé équilibré à six faces numérotées de 1 à 6.

%Un élève affirme qu'il a deux chances sur trois d'obtenir un diviseur de 6.

%A-t-il raison ?
Les diviseurs de 6 sont : 1~;~2~;~3~;~6 ; il y a donc 4 chances sur 6 d'avoir un diviseur de 6, soit deux chances sur trois d'obtenir un diviseur de 6. l'élève a raison.
\medskip

\textbf{Affirmation 2}

\smallskip

%On considère le nombre $a = 3^4 \times 7$.

%Un élève affirme que le nombre $b = 2 \times 3^5 \times 7^2$ est un multiple du nombre $a$.

%A-t-il raison ?
On a $b = 2 \times 3^5 \times 7^2  = 2 \times 3 \times 7 \times \left(3^4 \times 7 \right) = 42\left(3^4 \times 7 \right) = 42a $, donc $b$ est un multiple de $a$.

\medskip

\textbf{Affirmation 3}

\smallskip

%En 2016, le football féminin comptait en France \np{98800} licenciées alors qu'il y en avait \np{76000} en 2014.

%Un journaliste affirme que le nombre de licenciées a augmenté de $30$\,\% de 2014 à 2016.

%A-t-il raison ?
De 2014 à 2016 le nombre de licenciés a augmenté de $\np{98800} - \np{76000} = \np{22800}$ sur un total initial de \np{76000}.

L'augmentation est donc  égale à : $\dfrac{\np{22800}}{\np{76000}} = \dfrac{228}{760} = \dfrac{57}{190} = \dfrac{3 \times 19}{19 \times 10} = \dfrac{3}{10}  = \dfrac{30}{100} = 30\,\%$. Le journaliste a raison.
\medskip

\textbf{Affirmation 4}

\smallskip

%Une personne A a acheté un pull et un pantalon de jogging dans un magasin.
%
%Le pantalon de jogging coûtait 54~\euro. Dans ce magasin, une personne B a acheté le
%même pull en trois exemplaires; elle a dépensé plus d'argent que la personne A.
%
%La personne B affirme qu'un pull coûte $25$~\euro.
%
%A-t-elle raison ?
La personne B a dépensé $3p$, $p$ étant le prix d'un pull. On sait que $3p > 54 + p$ ou $2 \times p > 54$, soit $p > 27$.

Le pull coûte plus de 27~\euro donc ne peur coûter 25~\euro.
\bigskip

