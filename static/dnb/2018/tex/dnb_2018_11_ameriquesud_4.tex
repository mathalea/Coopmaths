
\medskip
 
Valentin souhaite acheter un écran de télévision ultra HD (haute définition).

Pour un confort optimal, la taille de l'écran doit être adaptée aux dimensions de son salon.

Voici les caractéristiques du téléviseur que Valentin pense acheter:

\begin{center}
\begin{tabularx}{0.4\linewidth}{|X m{1.5cm}|}\hline
Hauteur de l'écran	&60 cm\\
Format de l'écran	&16/9\\
Ultra HD			&Oui\\ \hline
\end{tabularx}
\end{center}

\textbf{Question :} Valentin a-t-il fait un choix adapté ?

\medskip

Utiliser les informations ci-dessous et les caractéristiques du téléviseur pour répondre.

Toute trace de recherche, même incomplète, pourra être prise en compte dans la notation.

\textbf{Information 1.} 

Distance écran-téléspectateur du salon de Valentin : 3,20~m. 

\medskip

\textbf{Information 2.} Format 16/9

Pour un écran au format 16/9, on a : Largeur $= \dfrac{16}{9} \times $ Hauteur

\medskip

\textbf{Information 3.} Graphique pour aider au choix de la taille de l'écran

\begin{center}
\psset{xunit=0.06cm,yunit=0.015cm,arrowsize=2pt 4}
\begin{pspicture}(-10,-20)(200,550)
\multido{\n=0+10}{16}{\psline[linewidth=0.2pt](\n,0)(\n,500)}
\multido{\n=0+50}{4}{\psline[linewidth=1.2pt](\n,0)(\n,500)}
\multido{\n=0+20}{26}{\psline[linewidth=0.2pt](0,\n)(150,\n)}
\multido{\n=0+100}{6}{\psline[linewidth=1.2pt](0,\n)(150,\n)}
\psaxes[linewidth=1.25pt,Dx=50,Dy=100]{->}(0,0)(0,0)(150,500)
\psaxes[linewidth=1.25pt,Dx=50,Dy=100](0,0)(0,0)(150,500)
\uput[r](0,520){\small Distance écran-téléspectateur (en cm)}
\uput[r](150,460){\small  Distance maximale}\psline{->}(150,460)(132,460)
\uput[r](150,230){\small Distance minimale}\psline{->}(150,230)(135,230)
\uput[u](140,0){\small Longueur de la diagonale de l'écran (en cm)}
\psline[linewidth=1.25pt](30,100)(140,485)
\psline[linewidth=1.25pt](30,48)(140,240)
\end{pspicture}
\end{center}
\medskip

