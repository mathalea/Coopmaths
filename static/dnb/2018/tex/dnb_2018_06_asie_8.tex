
\medskip

Lorsqu'on fait geler de l'eau, le volume de glace obtenu est proportionnel au volume d'eau
utilisé.

En faisant geler $1,5$ L d'eau on obtient $1,62$ L de glace.

\medskip

\begin{enumerate}
\item Montrer qu'en faisant geler $1$ L d'eau, on obtient $1,08$ L de glace.
\item On souhaite compléter le tableau ci-dessous à l'aide d'un tableur.

Quelle formule peut-on saisir dans la cellule B2 avant de la recopier vers la
droite jusqu'à la cellule G2 ?

\begin{center}
\begin{tabularx}{\linewidth}{|c|l|*{6}{>{\centering \arraybackslash}X|}}\hline
&\multicolumn{1}{|c|}{A}&B&C&D&E&F&G\\ \hline
1 &Volume d'eau initial (en L)	&0,5 	&1 	&1,5 	&2 	&2,5& 3\\ \hline
2 &Volume de glace obtenu (en L)&		&	&		&	&	&\\ \hline
\end{tabularx}
\end{center}

\item Quel graphique représente le volume de glace obtenu (en L) en fonction du volume
d'eau contenu dans la bouteille au départ (en L) ? 

\emph{On rappelle que toute réponse doit être justifiée.}

\begin{center}
\begin{tabularx}{\linewidth}{|*{3}{>{\centering \arraybackslash}X|}}
\psset{unit=1.5cm,comma=true}
\begin{pspicture*}(-0.5,-0.4)(2.15,2.5)
\psgrid[gridlabels=0pt,subgriddiv=2,gridwidth=0.4pt,subgridwidth=0.2pt](0,0)(2,2.5)
\psaxes[linewidth=1pt,Dx=0.5,Dy=0.5](0,0)(0,0)(2,2.4)
\psaxes[linewidth=1.5pt]{->}(0,0)(1,1)
\psplot[plotpoints=3000,linewidth=1.25pt]{0}{2}{x dup mul 0.625 mul}
\end{pspicture*}&
\psset{unit=1.5cm,comma=true}
\begin{pspicture*}(-0.5,-0.4)(2.15,2.5)
\psgrid[gridlabels=0pt,subgriddiv=2,gridwidth=0.4pt,subgridwidth=0.2pt](0,0)(2,2.5)
\psaxes[linewidth=1pt,Dx=0.5,Dy=0.5](0,0)(0,0)(2,2.4)
\psaxes[linewidth=1.5pt]{->}(0,0)(1,1)
\psplot[plotpoints=3000,linewidth=1.25pt]{0}{2}{x 1.08 mul}
\end{pspicture*}&
\psset{unit=1.5cm,comma=true}
\begin{pspicture*}(-0.5,-0.4)(2.15,2.5)
\psgrid[gridlabels=0pt,subgriddiv=2,gridwidth=0.4pt,subgridwidth=0.2pt](0,0)(2,2.5)
\psaxes[linewidth=1pt,Dx=0.5,Dy=0.5](0,0)(0,0)(2,2.4)
\psaxes[linewidth=1.5pt]{->}(0,0)(1,1)
\psline[linewidth=1.25pt](0,0.55)(2,1.85)
\end{pspicture*}\\ 
Graphique \no 1 &Graphique \no 2 &Graphique \no 3\\ \hline
\end{tabularx}
\end{center}
\end{enumerate}
