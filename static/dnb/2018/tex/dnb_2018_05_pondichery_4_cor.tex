
\bigskip

%\begin{tabularx}{\linewidth}{|X|X|}\hline
%Programme A							&Programme B\\
%\qquad$\bullet~~$ Choisir un nombre	&\qquad$\bullet~~$Choisir un nombre\\
%\qquad$\bullet~~$ Soustraire 3		&\qquad$\bullet~~$ Calculer le carré de ce nombre\\
%\qquad$\bullet~~$ Calculer le carré du résultat obtenu&\qquad$\bullet~~$Ajouter le triple du nombre de départ\\
%&\qquad$\bullet~~$Ajouter 7\\ \hline
%\end{tabularx} 
%
%\medskip

\begin{enumerate}
\item %Corinne choisit le nombre 1 et applique le programme A. 

%Expliquer en détaillant les calculs que le résultat du programme de calcul est 4.
Corinne obtient : $1 \to 1 - 3 = - 2 \to (- 2)^2 = 4$.
\item %Tidjane choisit le nombre $- 5$ et applique le programme B. Quel résultat obtient-il ?
Tidjane obtient : $- 5 \to (- 5)^2 = 25 \to 25 + 3 \times (- 5) = 25 - 15 = 10 \to 10 + 7 = 17$.
\item %Lina souhaite regrouper le résultat de chaque programme à l'aide d'un tableur. Elle crée la feuille de
%calcul ci-dessous. Quelle formule, copiée ensuite à droite dans les cellules C3 à H3, a-t-elle saisie dans
%la cellule B3 ?

%\begin{center}
%\begin{tabularx}{\linewidth}{|c|l|*{7}{>{\centering \arraybackslash}X|}}\hline
%\multicolumn{2}{|l|}{B2}&\multicolumn{7}{l|}{=(B1$-3$) \verb+^+ 
%2}\\ \hline
%&A&B&C&D&E&F&G&H\\ \hline
%1&Nombre de départ			&$- 3$	&$- 2$	&$- 1$	&0	&1	&2	&3\\ \hline
%2&Résultat du programme A	&36 	&25 	&16 	&9 	&4 	&1 	&0\\ \hline
%3&Résultat du programme B	&7 		&5 		&5 		&7 	&11 &17 &25\\ \hline
%\end{tabularx}
%\end{center}
Lina a saisi en B3 : $=\text{B}1\verb+^+2+3*\text{B}1+7$.
\item  %Zoé cherche à trouver un nombre de départ pour lequel les deux programmes de calcul donnent le
%même résultat. Pour cela, elle appelle $x$ le nombre choisi au départ et exprime le résultat de chaque programme de calcul en fonction de $x$.
	\begin{enumerate}
		\item Montrer que le résultat du programme A en fonction de $x$ peut s'écrire sous forme développée et réduite: $x^2 - 6x + 9$.
Le programme A donne à partir de $x$ : $(x - 3)^2 = x^2 + 9 - 6x = x^2 - 6x + 9$.
		\item %Écrire le résultat du programme B
				Le programme B donne $x^2 + 3x + 7$.
		\item %Existe-t-il un nombre de départ pour lequel les deux programmes donnent le même résultat ?
		
%Si oui, lequel ?
Les résultats sont égaux si $x^2 - 6x + 9 = x^2 + 3x + 7$ soit en simplifiant par $x^2$ : $- 6x + 9 = 3x + 7$ ou $9 - 7 = 3x + 6x$ ou $2 = 9x$ et enfin $x = \dfrac{2}{9}$.

Le résultat commun est $\left(\dfrac{2}{9} - 3 \right)^2  = \left(\dfrac{2}{9} - \dfrac{27}{9} \right)^2 = \left(- \dfrac{25}{9}\right)^2 = \dfrac{25^2}{9^2} = \dfrac{625}{81}$.
	\end{enumerate} 
\end{enumerate}


