
\medskip

\parbox{0.58\linewidth}{
On considère un jeu composé d'un plateau tournant et d'une boule. Représenté ci-contre, ce plateau comporte 13 cases numérotées de 0 à 12.

\smallskip

On lance la boule sur le plateau, La boule finit par s'arrêter
au hasard sur une case numérotée.

\smallskip

La boule a la même probabilité de s'arrêter sur  chaque case.

\smallskip

\begin{enumerate}
\item Quelle est la probabilité que la boule s'arrête sur la case
numérotée 8 ?
\item  Quelle est la probabilité que le numéro de la case sur
lequel la boule s'arrête soit un nombre impair ?
\end{enumerate}}\hfill
\parbox{0.38\linewidth}{\psset{unit=0.75cm}
\begin{pspicture}(-3.5,-3.5)(3.5,3.5)
\pscircle(0,0){3.5}
\pscircle*(0,0){0.8}
\multido{\n=1+27.692,\na=0+27.692,\nb=10+27.692}{13}{\pspolygon*(0.8;\na)(3.5;\n)(0.8;\nb)}
\multido{\n=264+-27.692,\na=0+1}{13}{\rput(2.2;\n){\na}}
\end{pspicture}
}
\begin{enumerate} \setcounter{enumi}{2}
\item Quelle est la probabilité que le numéro de la case sur
laquelle la boule s'arrête soit un nombre premier ?
\item Lors des deux derniers lancers, la boule s'est arrêtée à chaque fois sur la case numérotée 9.

A-t-on maintenant plus de chances que la boule s'arrête sur la case numérotée 9 plutôt que sur la case
numérotée 7 ? Argumenter à l'aide d'un calcul de probabilités.
\end{enumerate}

\vspace{0,5cm}

