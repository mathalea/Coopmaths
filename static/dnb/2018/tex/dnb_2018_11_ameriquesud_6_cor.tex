
\medskip 

%Léna et Youri travaillent sur un programme. Ils ont obtenu le dessin suivant :
%
%\begin{center}
%\psset{unit=1cm}
%\begin{pspicture}(8,1)
%\multido{\n=0.0+1.6}{5}{\rput(\n,0){\psframe(1,1)}}
%\end{pspicture}
%\end{center}
%
%Ils ont ensuite effacé une donnée par erreur dans le script principal.
%
%Voici les copies d'écran de leur travail :
%
%\begin{center}
%\begin{tabularx}{\linewidth}{|*{3}{>{\centering \arraybackslash }X|}}\hline
%\multicolumn{2}{|c|}{Programme}&Pour information\\ \hline
%Script principal & Bloc du motif& \\ 
%\begin{scratch}
%\blockinit{quand \greenflag est cliqué}
%\blockmove{s’orienter à \ovalnum{90\selectarrownum}}
%\blockmove{aller à x: \ovalnum{-200} y: \ovalnum{0}}
%\blockpen{effacer tout}
%\blockrepeat{répéter \ovalnum{5} fois}
%{
%\blockmoreblocks{Motif}
%\blockmove{avancer de \ovalnum{}}
%}
%\end{scratch}&\begin{scratch}
%\initmoreblocks{définir \namemoreblocks{Motif}}
%\blockpen{stylo en position d’écriture}
%\blockrepeat{répéter \ovalnum{4} fois}
%{
%\blockmove{avancer de \ovalnum{40}}
%\blockmove{tourner \turnright{} de \ovalnum{90} degrés}
%}
%\blockpen{relever le stylo}
%\end{scratch}&L'instruction \begin{scratch}\blockmove{s’orienter à \ovalnum{90\selectarrownum}} \end{scratch}
%
%signifie qu'on se dirige vers
%la droite.\\ \hline
%\multicolumn{1}{r}{Valeur effacée}&\multicolumn{1}{r}{}&\multicolumn{1}{r}{}\\
%\end{tabularx}
%\psset{arrowsize=2pt 4}\psline{->}(-3.5,0.4)(-4,1.6)
%
%\end{center}
%
%\emph{Dans cet exercice, aucune justification n'est demandée.}

\medskip

\begin{enumerate}
\item 
	\begin{enumerate}
		\item %La valeur effacée dans le script principal était-elle $40$ ou bien $60$ ?
La valeur effacée est 60 sinon les carrés seraient jointifs.
		\item ~%Dessiner sur la copie ce qu'on aurait obtenu avec l'autre valeur.
		
%On représentera l'instruction \og avancer de $20$ \fg par un segment de longueur $1$ cm.
\begin{center}
\psset{unit=2cm}
\begin{pspicture}(5,1)
\multido{\n=0+1}{4}{\rput(\n,0){\psframe(0,0)(1,1)}}
\end{pspicture}
\end{center}
	\end{enumerate}
\item %Léna et Youri souhaitent maintenant obtenir un triangle équilatéral comme motif.

%\parbox{0.6\linewidth}{Afin d'obtenir un triangle équilatéral :
%
%\medskip
%
%$\bullet~~$ %par quelle valeur peut-on remplacer $a$ ? 
%
%
%\medskip
%
%$\bullet~~$ par quelle valeur peut-on remplacer $b$ ?
%
%\medskip
%
%$\bullet~~$ %par quelle valeur peut-on remplacer $c$ ?}\hfill
%\parbox{0.38\linewidth}{\begin{scratch}
%\initmoreblocks{définir \namemoreblocks{Motif}}
%\blockpen{stylo en position d’écriture}
%\blockrepeat{répéter \ovalnum{$a$} fois}
%{
%\blockmove{avancer de \ovalnum{$b$}}
%\blockmove{tourner \turnright{} de \ovalnum{$c$} degrés}
%}
%\blockpen{relever le stylo}
%\end{scratch}}
\end{enumerate}

$a = 3$

$b = 40$ par exemple

$c = 120$.
\medskip

