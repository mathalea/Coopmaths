
\medskip

\begin{enumerate}
\item %Décomposer les nombres $162$ et $108$ en produits de facteurs premiers.
$162 = 2 \times 81 = 2 \times 9 \times 9 = 2 \times 3^2 \times 3^2 = 2 \times 3^4$.

$108 = 2 \times 54 = 2 \times 2 \times 27 = 2^2 \times 3^3$.
\item %Déterminer deux diviseurs communs aux nombres $162$ et $108$ plus grands que $10$.
Les diviseurs communs à 162 et 108 sont : 1 ; 2 ; 3 ;  6 ; 9 ; 18 ; 27 et 54.
\item %Un snack vend des barquettes composées de nems et de samossas.

%Le cuisinier a préparé $162$ nems et $108$ samossas.

%Dans chaque barquette :
%
%\setlength\parindent{9mm}
%\begin{itemize}
%\item le nombre de nems doit être le même.
%\item le nombre de samossas doit être le même,
%\end{itemize}
%\setlength\parindent{0mm}
%
%Tous les nems et tous les samossas doivent être utilisés.
	\begin{enumerate}
		\item %Le cuisiner peut-il réaliser $36$ barquettes ?
Le cuisiner ne peut pas  réaliser $36$ barquettes car 36 ne divise pas 162.
		\item %Quel nombre maximal de barquettes pourra-t-il réaliser ?
		Le plus grand commun diviseur à 162 et 108 est 54 ; le cuisinier peut donc préparer 54 barquettes. 
		\item %Dans ce cas, combien y aura-t-il de nems et de samossas dans chaque barquette ?
Chaque barquette contiendra alors  3 nems et 2 samoussas.
	\end{enumerate}
\end{enumerate}

\vspace{0,5cm}

