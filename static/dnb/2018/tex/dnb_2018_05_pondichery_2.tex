
\medskip

Le pavage représenté sur la figure 1 est réalisé à partir d'un motif appelé pied-de-coq qui est présent sur de
nombreux tissus utilisés pour la fabrication de vêtements.

Le motif pied-de-coq est représenté par le polygone ci-dessous à droite (figure 2) qui peut être réalisé à
l'aide d'un quadrillage régulier.

\begin{center}
\begin{tabularx}{\linewidth}{>{\centering \arraybackslash}X c}
\def\pied{\psset{unit=0.5cm}\begin{pspicture}(-3,-2)(2,3)
\pspolygon*(0,0)(-1,-1)(-1,-2)(1,0)(1,1)(2,2)(1,2)(1,3)(0,2)(-1,2)(-3,0)(-2,0)(-1,1)(-1,0)
\end{pspicture}
}
\psset{unit=1cm}
\begin{pspicture}(-1,-1.5)(5,5)
%\psgrid
\multido{\n=0.0+2.}{3}{\multido{\na=0.0+2.}{3}{\rput(\n,\na){\pied}}}
\rput(0.2,0.2){\pscircle[fillstyle=solid,fillcolor=white](0,0){0.25} }
\rput(0.2,0.2){1}
\rput(2.2,2.2){\pscircle[fillstyle=solid,fillcolor=white](0,0){0.25} }
\rput(2.2,2.2){2}
\end{pspicture}
&\psset{unit=0.5cm}
\begin{pspicture*}(-3.5,-3.5)(3,4)
\psgrid[gridlabels=0pt,subgriddiv=1,gridwidth=0.1pt]
\psdots[dotstyle=+,dotangle=45](0,0)(-1,-1)(-1,-2)(1,0)(1,1)(2,2)(1,2)(1,3)(0,2)(-1,2)(-3,0)(-2,0)(-1,1)(-1,0)%BCDEFGHIJKLMN
\uput[ur](0,0){\footnotesize B}\uput[dl](-1,-1){\footnotesize C}\uput[dl](-1,-2){\footnotesize D}\uput[dr](1,0){\footnotesize E}\uput[dr](1,1){\footnotesize F}\uput[ur](2,2){\footnotesize G}\uput[dl](1,2){\footnotesize H}\uput[ur](1,3){\footnotesize I}\uput[ul](0,2){\footnotesize J}\uput[ul](-1,2){\footnotesize K}\uput[dl](-3,0){\footnotesize L}\uput[dl](-2,0){\footnotesize M}\uput[ur](-1,1){\footnotesize N}\uput[dl](-1,0){\footnotesize A}
\pspolygon(0,0)(-1,-1)(-1,-2)(1,0)(1,1)(2,2)(1,2)(1,3)(0,2)(-1,2)(-3,0)(-2,0)(-1,1)(-1,0)
\end{pspicture*}
\\
Figure 1&Figure 2
\end{tabularx}
\end{center}

\medskip
\begin{enumerate}
\item Sur la figure 1, quel type de transformation géométrique permet d'obtenir le motif 2 à partir du motif 1 ?
\item Dans celte question, on considère que : AB = 1 cm (figure 2).

Déterminer l'aire d'un motif pied-de-coq.
\item  Marie affirme \og si je divise par 2 les longueurs d'un motif, son aire sera aussi divisée par 2 \fg.

A-t-elle raison ? Expliquer pourquoi.
\end{enumerate}

\vspace{0,5cm}

