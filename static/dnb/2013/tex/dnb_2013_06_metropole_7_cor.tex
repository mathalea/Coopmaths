 
Chacune des trois affirmations suivantes est-elle vraie ou fausse ? On rappelle que les réponses doivent être justifiées.
 
\subsection*{Affirmation 1 : VRAIE}
 
Dans un club sportif les trois quarts des adhérents sont mineurs et le tiers des adhérents majeurs a plus de 25 ans. Un adhérent sur six a donc entre 18 ans et 25~ans.

\begin{itemize}
\item les trois quarts des adhérents sont mineurs, donc un quart sont majeurs,
\item le tiers des adhérents majeurs a plus de 25 ans, donc le tiers d'un quart a plus de 25 ans. Ainsi les deux tiers de un quart ont entre 18 et 25 ans, soit
\[
\frac{2}{3}\times\frac{1}{4}=\frac{1}{6}
\]
\end{itemize}

\subsection*{Affirmation 2 : FAUX}
 
Durant les soldes si on baisse le prix d'un article de 30\,\% puis de 20\,\%, au final le prix de l'article a baissé de 50\,\%.

\begin{itemize}
\item une baisse de 30\%{} revient à multiplier le prix de départ par $1-0,30=0,70$.
\item une baisse de 20\%{} revient à multiplier par $1-0,20=0,80$
\end{itemize}
Donc, si on baisse le prix d'un article de 30\,\% puis de 20\,\%, cela revient à multiplier le prix de départ par $0,70\times 0,80=0,56$, soit une baisse de 44\%.

\subsection*{Affirmation 3 : VRAI}
Pour tout entier $n$,
 \[
 (n + 1)^2 - (n - 1)^2=\left((n+1)-(n-1)\right)\left((n+1)+(n-1)\right)=2(2n)=4n
 \]
Pour n'importe quel nombre entier $n,\: (n + 1)^2 - (n - 1)^2$ est bien un multiple de $4$. 


% \tkzDefPoints{1/5.5/A,9/5.5/B,1/.5/C}
% \tkzDrawSegments(A,B B,C C,A)
% \tkzLabelPoint[below](C){$C$} \tkzLabelPoint[right](B){$B$} \tkzLabelPoint[left](A){$A$}
% \tkzLabelSegment[sloped](C,A){$AC=3$~cm}
% \tkzLabelSegment[sloped,below](C,B){$BC=6$~cm}
% \tkzMarkRightAngle(B,A,C)
% \tkzMarkAngle(A,B,C)\tkzLabelAngle[pos=-1.5,color=green](A,B,C){$30^{\circ}$}
% \tkzText(-4,5.5){Figure 1}

