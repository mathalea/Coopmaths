
\medskip

Tom lance cinquante fois deux dés à six faces parfaitement équilibrés. Il note dans une feuille de calcul les sommes obtenues à chaque lancer. Il obtient le tableau suivant : 

\begin{center}
\begin{tabularx}{\linewidth}{|c|l|c|*{10}{>{\centering \arraybackslash}X|}c|X|}\hline
\multicolumn{2}{|c|}{B3}&\multicolumn{13}{l|}{=B2/M2}\\ \hline
&A&B&C&D&E&F&G&H&I&J&K&L&M&N\\ \hline
1 &somme obtenue 			&2 		&3 	&4 	&5 	&6 	&7 	&8 	&9 	&10 &11 &12 &total	&\\ \hline 
2 &nombre d'apparitions 	&3 		&1 	&4 	&6 	&9 	&9 	&7 	&3 	&5 	&3 	&0 	&50		&\\ \hline 
3 &fréquence d'apparition	&0,06	&	&	&	&	&	&	&	&	&	&	&		&\\ \hline 
\end{tabularx}
\end{center}

\begin{enumerate}
\item Quelle formule a-t-il saisie dans la cellule M2 pour vérifier qu'il a bien relevé 50 résultats? 
\item Tom a saisi dans la cellule B3 la formule $\fbox{=B2/M2}$. Il obtient un message d'erreur quand il la tire dans la cellule C3. Pourquoi ? 
\item Tom déduit de la lecture de ce tableau que s'il lance ces deux dés, il n'a aucune chance d'obtenir la somme 12. A-t-il tort ou raison ?
\end{enumerate}
 
\bigskip  

