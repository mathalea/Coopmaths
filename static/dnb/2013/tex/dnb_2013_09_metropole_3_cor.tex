
\medskip

%ABCD est un rectangle tel que AB = 30 cm et BC = 24 cm.
% 
%On colorie aux quatre coins du rectangle quatre carrés identiques en gris. On délimite ainsi un rectangle central que l'on colorie en noir. 
%
%\begin{center}
%\psset{unit=1cm}
%\begin{pspicture}(0,-0.2)(4.8,4)
%\psframe(4.8,3.8)
%\psframe[fillstyle=solid,fillcolor=lightgray](0,3.8)(1.1,2.7)
%\psframe[fillstyle=solid,fillcolor=lightgray](0,1.1)(1.1,0)
%\psframe[fillstyle=solid,fillcolor=lightgray](3.7,3.8)(4.8,2.7)
%\psframe[fillstyle=solid,fillcolor=lightgray](3.7,1.1)(4.8,0)
%\psframe*(1.1,2.7)(3.7,1.1)
%\uput[ul](0,3.8){A}\uput[ur](4.8,3.8){B}
%\uput[ur](4.8,0){C}\uput[dl](0,0){B}
%\end{pspicture}
%\end{center}
 
\begin{enumerate}
\item %Dans cette question, les quatre carrés gris ont tous 7 cm de côté. Dans ce cas : 
	\begin{enumerate}
		\item %quel est le périmètre d'un carré gris ?
		Périmètre d'un carré gris : $4 \times 7 = 28$~cm. 
		\item %quel est le périmètre du rectangle noir ? 
Longueur du rectangle noir : $30 - 2\times 7 = 30 - 14 = 16$ ;

Largeur du rectangle noir : $24 - 2\times 7 = 24 - 14 = 10$.

Le périmètre du rectangle noir est donc : $2 \times (16 + 10) = 2 \times 26 = 52$~cm.
	\end{enumerate}
\item %Dans cette question, la longueur du côté des quatre carrés gris peut varier. Par conséquent, les dimensions du rectangle noir varient aussi.
 
%Est-il possible que le périmètre du rectangle noir soit égal à la somme des périmètres des quatre carrés gris ?
Si $x$ est la longueur des côtés du carré gris, le périmètre des quatre carrés gris est égal à $4 \times 4 \times x = 16x$.

Le rectangle noir a pour longueur $30 - 2x$ et pour largeur $24 - 2x$. Le périmètre du rectangle noir est donc égal à $2[(30 - 2x) + (24 - 2x)] = 108 - 8x$.

Il y a égalité de ces deux périmètres si :

$16x = 108 - 8x$  soit $24x = 108$ ou $8x = 36$ ou $2x = 9$ et enfin $x = 4,5$~cm(les périmètres valent alors 72~cm).
\end{enumerate} 

\vspace{0,5cm} 

