
\medskip
 
%Indiquer si les affirmations suivantes sont vraies ou fausses. 
%
%\textbf{Rappel : toutes les réponses doivent être justifiées.}
%
%\medskip
 
Affirmation 1 : %\og La vitesse moyenne d'un coureur qui parcourt 18 km en une heure est strictement supérieure à celle d'une voiture télécommandée qui parcourt 5 m par seconde. \fg
Le coureur parcourt \np{18000}~m en \np{3600} soit $\dfrac{\np{18000}}{\np{3600}} = 5$~(m/s). Affirmation fausse.

Affirmation 2 : %\og Pour tout nombre $x$, on a l'égalité : $(3x - 5)^2 = 9x^2 - 25$. \fg
 $(3x - 5)^2 = 9x^2 + 25 - 30x $. Affirmation fausse.
 
Affirmation 3 : %\og Dans une série de données numériques, la médiane de la série est toujours strictement supérieure à la moyenne. \fg 
Soit la série 1 ; 2; 3 ; 9 ; 10.

La médiane est 3 et la moyenne 5. Affirmation fausse.
\bigskip  

