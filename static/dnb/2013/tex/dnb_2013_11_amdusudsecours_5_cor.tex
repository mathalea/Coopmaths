
\medskip  

%Un éleveur a acheté $40$ m de grillage; il veut adosser un enclos rectangulaire à sa grange, contre un mur de $28$~m de long. 
% 
%\begin{center}
%\psset{unit=1cm}
%\begin{pspicture}(8,5.5)
%\psframe[fillstyle=vlines](0,4.5)(8,5) \uput[u](4,5){mur}
%\psline(0.5,4.5)(0.5,0.5)(7.5,0.5)(7.5,4.5)
%\rput{90}(0.1,2.5){grillage}\rput{90}(7.6,2.5){grillage}
%\rput(4,0.3){grillage}
%\uput[d](4,0.1){$x$}\uput[r](7.7,2.5){$y$}
%\psline{<->}(0.5,0.1)(7.5,0.1)
%\psline{<->}(7.75,0.5)(7.75,4.5)
%\end{pspicture}
%\end{center}
%
%Il souhaite offrir ainsi le maximum de place à ses brebis en utilisant le grillage. 

\begin{enumerate}
\item 
	\begin{enumerate}
		\item %Pour $x = 4$~m , calculer la longueur $y$, puis l'aire $A$ de l'enclos en m$^2$.
On a $4 + 2y = 48$ soit $2y = 44$ et donc $y = 22$. D'où $A = xy = 4 \times 22 = 88$. 
		\item %Recopier et compléter le tableau ci-dessous : 
~
\begin{center}
\begin{tabularx}{0.6\linewidth}{|c|*{4}{>{\centering \arraybackslash}X|}}\hline
$x$ (en m)						&4	&10		&20		&28\\ \hline   
$y$ (en m)						&18	&15		&10		&6\\ \hline             
$A \left(\text{en m}^2\right)$	&72	&150	&200	&168\rule[-2mm]{0mm}{7mm}\\ \hline
\end{tabularx}
\end{center}

	\end{enumerate}   
\item %Déterminer $y$ en fonction de $x$.
On a $x + 2y = 40$ soit $2y = 40 - x$ et enfin $y = - \dfrac{1}{2}x + 20$. (fonction affine)

%En déduire que $A = 20x - 0,5x^2$.
$A = xy = x\left(- \dfrac{1}{2}x + 20 \right) = - \dfrac{1}{2}x^2 + 20x$. 
\item %Voici la plage de cellules réalisées dans un tableur-grapheur qui permettra de calculer la valeur de $A$. 

%\begin{center}    
%\begin{tabularx}{0.4\linewidth}{|c|*{2}{>{\centering \arraybackslash}X|}}\hline    
%				&Valeur de $x$	&   Valeur de $A$\\ \hline     
%2				&4				&\\ \hline
%3				&6				&\\ \hline
%4				&8				&\\ \hline
%5				&10				&\\ \hline
%6				&12				&\\ \hline
%7				&14				&\\ \hline
%8				&16				&\\ \hline
%9				&18				&\\ \hline            
%11				&22				&\\ \hline         
%12				&24				&\\ \hline         
%13				&26				&\\ \hline      
%14				&28				&\\ \hline       
%\end{tabularx}
%\end{center}

%Quelle formule doit-il saisir dans la cellule B2 et qui pourra être étendue sur toute la colonne B ? 
=20*A2$-$0,5*A2*A2 ou =20*A2$-$A$2^2$.
\item %Le graphique ci-dessous représente l'aire $A$ en fonction de la longueur $x$ compris entre 4~m et 28m. 

%\begin{center}
%\psset{xunit=0.25cm,yunit=0.025cm}
%\begin{pspicture}(-2,-15)(30,250)
%\multido{\n=0+5}{7}{\psline[linewidth=0.5pt](\n,0)(\n,250)}
%\multido{\n=0+10}{26}{\psline[linewidth=0.5pt](0,\n)(30,\n)}
%\multido{\n=0+50}{6}{\psline[linewidth=1.25pt](0,\n)(30,\n)}
%\psaxes[linewidth=1.5pt,Dx=5,Dy=50](0,0)(30,250)
%\psplot[plotpoints=5000,linewidth=1.25pt,linecolor=blue]{4}{28}{24 x mul x dup mul 0.5 mul sub}
%\end{pspicture}
%\end{center}
%
%À l'aide de ce graphique répondre aux questions suivantes en donnant des valeurs approchées : 

	\begin{enumerate}
		\item %Quelle est l'aire de cet enclos pour $x = 14$~m ?
La verticale passant par le point de coordonnées (14~;~0) coupe la courbe en un point dont l'ordonnée est à peu près égale à 180. 
		\item %Pour quelle(s) valeur(s) de $x$ l'aire de l'enclos est égale à $192$~m$^2$ ? 
L'horizontale contenant le point de coordonnées (0,192) coupe la courbe en deux points dont les abscisses sont environ 16 et 24.
		\item %Pour quelle(s) valeur(s) de $x$ l'aire de l'enclos est maximale ? 
Il semble que le maximum de l'aire 200 est atteint pour $x = 20$.
%En déduire les dimensions de l'enclos pour que les brebis aient le maximum de place.

On a déjà vu qu'alors $y = 10$ et donc $A = xy = 20 \times 10 = 200$.
	\end{enumerate} 
\end{enumerate}

\bigskip

