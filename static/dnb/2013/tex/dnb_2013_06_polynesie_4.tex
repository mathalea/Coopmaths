
\medskip

\begin{enumerate}
\item Construis un triangle ABC rectangle en C tel que AB = 10 cm et AC = 8 cm. 
\item Calcule la longueur BC (en justifiant précisément). 
\item 
	\begin{enumerate}
		\item Place le point M de l'hypoténuse [AB] tel que AM = $2$~cm. 
		\item Trace la perpendiculaire à [AC] passant par M. Elle coupe [AC] en E. 
		\item Trace la perpendiculaire à [BC] passant par M. Elle coupe [BC] en F. 
		\item À l'aide des données de l'exercice, \textbf{recopie sur ta copie} la proposition que l'on peut directement utiliser pour prouver que le quadrilatère MFCE est un rectangle.
	\end{enumerate}

\medskip
	 
\textbf{Proposition 1 :} Si un quadrilatère a 4 angles droits alors c'est un rectangle. 

\textbf{Proposition 2 :} Si un quadrilatère est un rectangle alors ses diagonales ont la même longueur. 

\textbf{Proposition 3 :} Si un quadrilatère a 3 angles droits alors c'est un rectangle. 

\end{enumerate}

\bigskip

