
\medskip 

%\parbox{0.6\linewidth}{Un laboratoire pharmaceutique produit des gélules de paracétamol.}\hfill
%\parbox{0.35\linewidth}{\psset{unit=1cm}\begin{center}
%\begin{pspicture}(1.2,2.8)
%
%\def\haut{\pscircle[fillstyle=solid,fillcolor=lightgray,linecolor=lightgray](0.5,1.4){0.55}\psframe[fillstyle=solid,fillcolor=lightgray,linecolor=lightgray](-0.025,1.2)(1.025,0) \scalebox{.99}[0.45]{\psarc[fillstyle=solid,fillcolor=lightgray](0.5,0){0.5}{180}{0}}\scalebox{.99}[0.45]{\psarc[linestyle=dashed](0.5,0){0.5}{0}{180}} }
%\rput(0.3,1.3){\haut}
%\psset{linecolor=black}
%\psline(0.25,1.35)(0.25,0.3)\psline(1.25,1.35)(1.25,0.3)
%\scalebox{.99}[0.45]{\psarc(0.75,0.6){0.5}{180}{0}}
%\scalebox{.99}[0.45]{\psarc[linestyle=dashed](0.7,0.6){0.5}{0}{180}}
%\scalebox{.99}[0.45]{\psarc(0.61,6.2){0.5}{180}{0}}
%\scalebox{.99}[0.45]{\psarc[linestyle=dashed](0.56,6.2){0.5}{0}{180}}
%\psarc(0.54,0.25){0.5}{180}{0}
%\end{pspicture}
%\end{center}}
%
%Chaque gélule contient 500~mg de produit. 
%
%Une gélule est constituée de deux demi-sphères de $7$ mm de diamètre et d'un cylindre de hauteur $14$ mm. 
%
%\medskip

\begin{enumerate}
\item %L'usine de fabrication produit $5$ tonnes de paracétamol. (1 tonne = \np{1000}~kg) 

%Combien de gélules de $500$ mg peut-on produire ?
$500$ mg $= 0,5$ g.

5 t = \np{5000} kg = \np{5000000}.

On peut donc fabriquer $\dfrac{\np{5000000}}{0,5} = \dfrac{\np{50000000}}{5} = \np{10000000}$~gélules (10 millions). 
\item %Sachant qu'une boîte contient deux plaquettes de $8$ gélules chacune, combien de boîtes peuvent être produites avec ces $5$~tonnes ?
On a $\dfrac{\np{10000000}}{16} = \np{625000}$~boîtes. 
\item %Calculer le volume d'une gélule. On arrondira à 1 mm$^3$ près. 

%{\footnotesize \emph{On rappelle que le volume d'une boule de rayon R est donné par la formule $V = \dfrac{4}{3}\pi R^3$ et le volume d'un cylindre de hauteur $h$ et dont la base a pour rayon $R$ est $V = \pi R^2h$.}}
La gélule se compose d'une boule de rayon 3,5~mm et d'un cylindre de même rayon et de hauteur 14~mm. Donc le volume d'une gélule est :

$\dfrac{4}{3}\pi \times 3,5^3 + \pi\times 3,5^2 \times 14 \approx 718,3$ soit 718~mm$^3$ au mm$^3$ près. 
\end{enumerate}

\bigskip

