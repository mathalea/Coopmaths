
\medskip

\parbox{0.6\linewidth}{Dans cet exercice, on considère le rectangle ABCD ci-contre tel que son périmètre soit égal à 31 cm.}\hfill
\parbox{0.38\linewidth}{ \psset{unit=2cm}
\begin{pspicture}(2.7,1.7)
\psframe(.3,0.2)(2.1,1.3)
\uput[ul](0.3,1.3){A} \uput[ur](2.1,1.3){B} \uput[dr](2.1,0.2){C} \uput[dl](.3,0.2){D} 
\end{pspicture}}

\bigskip

\begin{enumerate}
\item 
	\begin{enumerate}
		\item Si un tel rectangle a pour longueur 10 cm, quelle est sa largeur ? 
		\item Proposer une autre longueur et trouver la largeur correspondante. 
		\item On appelle $x$ la longueur AB. 

En utilisant le fait que le périmètre de ABCD est de $31$ cm, exprimer la longueur BC en fonction de $x$. 
		\item En déduire l'aire du rectangle ABCD en fonction de $x$. 
	\end{enumerate}
\item On considère la fonction $f$ définie par $f(x) = x (15,5 - x)$.
	\begin{enumerate}
		\item Calculer $f(4)$. 
		\item Vérifiez qu'un antécédent de $52,5$ est $5$. 
	\end{enumerate}
\item Sur le graphique ci-dessous, on a représenté l'aire du rectangle ABCD en fonction de la valeur de $x$. 

\begin{center}
\psset{xunit=0.65cm,yunit=0.1cm}
\begin{pspicture}(-0.5,-10)(16,65)
\psaxes[linewidth=1.5pt,Dy=10]{->}(0,0)(0,0)(16,65)
\psplot[plotpoints=5000,linewidth=1.25pt,linecolor=blue]{0}{15.5}{15.5 x sub x mul}
\rput{90}(-1.5,30){Aire de ABCD}
\uput[d](14,-5){Valeur de $x$} 
\end{pspicture}
\end{center}
 
À l'aide de ce graphique, répondre aux questions suivantes en donnant des valeurs approchées: 

\medskip

	\begin{enumerate}
		\item Quelle est l'aire du rectangle ABCD lorsque $x$ vaut 3 cm ? 
		\item Pour quelles valeurs de $x$ obtient-on une aire égale à 40 cm$^2$ ? 
		\item Quelle est l'aire maximale de ce rectangle ? Pour quelle valeur de $x$ est-elle obtenue ? 
	\end{enumerate}
\item Que peut-on dire du rectangle ABCD lorsque AB vaut $7,75$~cm ? 
\end{enumerate}
