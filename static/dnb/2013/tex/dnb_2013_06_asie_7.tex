
\medskip

\textbf{Dans cet exercice, si le travail n'est pas terminé, laisser tout de même une trace de la recherche. Elle sera prise en compte dans l'évaluation. }

\bigskip

\parbox{0.6\linewidth}{Un moule à muffins(2) est constitué de 9 cavités.

Toutes les cavités sont identiques.
  
Chaque cavité a la forme d'un tronc de cône (cône 
coupé par un plan parallèle à sa base) représenté ci-contre. 

Les dimensions sont indiquées sur la figure.

(2) un muffin est une pâtisserie}\hfill
\parbox{0.38\linewidth}{\psset{unit=0.8cm}
\begin{pspicture}(6,7)
\psline(0,4.6)(0.5,3)\psline(2.8,4.6)(2.3,3)
\psellipse(1.4,4.6)(1.4,0.375)
\psline[linestyle=dashed](0.5,3)(1.4,0)(2.3,3.1)
\psline[linestyle=dashed](2.8,4.6)(4.3,4.6)
\psline[linestyle=dashed](2.3,3.1)(3.6,3.1)
\psline[linestyle=dashed](1.4,0)(4.3,0)
\psline[linestyle=dashed](2.8,4.6)(2.8,5.4)
\psline[linestyle=dashed](0,4.6)(0,5.4)
\psline[linewidth=0.5pt,arrowsize=2pt 3]{<->}(0,5.4)(2.8,5.4)
\psline[linewidth=0.5pt,arrowsize=2pt 3]{<->}(3,3.1)(3,4.6)
\psline[linewidth=0.5pt,arrowsize=2pt 3]{<->}(4.1,0)(4.1,4.6)
\uput[u](1.4,5.4){7,5 cm}\uput[r](3,3.85){4 cm}\uput[r](4.1,2.3){12 cm}
\psdots[dotstyle=+,dotangle=45,dotsize=0.2](1.4,4.6)(1.4,3.1)
\psline[linestyle=dashed](1.4,4.6)(1.4,3.1)
\scalebox{.99}[0.3]{\psarc(1.4,10.4){0.92}{180}{0}}%
\scalebox{.99}[0.3]{\psarc[linestyle=dashed](1.4,10.4){0.92}{0}{180}}%
\end{pspicture}}

\medskip
 
\begin{tabularx}{\linewidth}{|l X|}\hline
Rappels :& Volume d'un cône de rayon de base $r$ et de hauteur $h$ :\\ 
&$\dfrac{1}{3}\pi r^2 h$\\ 
&1 L = 1 dm$^3$\\ \hline
\end{tabularx} 

\bigskip
 
\begin{enumerate}
\item Montrer que le volume d'une cavité est d'environ 125~cm$^3$. 
\item Léa a préparé 1 litre de pâte. Elle veut remplir chaque cavité du moule au $\dfrac{3}{4}$ de son volume.
 
A-t-elle suffisamment de pâte pour les 9 cavités du moule ? Justifier la réponse.
\end{enumerate}
 

