
\medskip 

%Un pâtissier a préparé 840 financiers* et \np{1176} macarons*. Il souhaite faire des lots, tous identiques, en mélangeant financiers et macarons. Il veut utiliser tous les financiers et tous les macarons. 

\medskip

\begin{enumerate}
\item 
	\begin{enumerate}
		\item %Sans faire de calcul, expliquer pourquoi les nombres $840$ et \np{1176} ne sont pas premiers entre eux.
$840$ et \np{1176} sont pairs ayant tous les deux pour diviseur 2, ils ne sont pas premiers entre eux.  
		\item %Le pâtissier peut-il faire $21$ lots ? Si oui, calculer le nombre de financiers et le nombre de macarons dans chaque lot.
On a $\dfrac{840}{21} = 40$ et $\dfrac{\np{1176}}{21} = 46$.

On peut faire 21 lots de 40 financiers et 46 macarons. 
		\item %Quel est le nombre maximum de lots qu'il peut faire? Quelle sera alors la composition de chacun des lots ? 
D'après la question précédente 40 et 46 sont divisibles par 2, donc $\dfrac{840}{42} = 20$ et $\dfrac{\np{1176}}{42} = 23$ et 20 et 23 sont premiers entre eux.

On peut donc faire 42 lots de 20 financiers et 23 macarons.

On aurait pu calculer le PGCD de 840 et \np{1176} : on aurait trouvé 42.
	\end{enumerate}
\item %Cette année, chaque lot de $5$ financiers et $7$ macarons est vendu $22,40$~\euro. 

%L'année dernière, les lots, composés de $8$ financiers et de $14$ macarons étaient vendus $42$~\euro. 

%Sachant qu'aucun prix n'a  changé entre les deux années, calculer le prix d'un financier et d'un macaron. 
Avec des notations évidentes :

$\left\{\begin{array}{l c l}
5f + 7m&=&22,4\\
8f + 14m&=&42
\end{array}\right.$ ou encore $\left\{\begin{array}{l c l}
5f + 7m&=&22,4\\
4f + 7m&=&21
\end{array}\right.$ d'où par différence $f = 1,4$ puis $m = \dfrac{22,4 - 5\times 1,4}{7} = 2,2$.

Un financier est vendu 1,40~\euro{} et un macaron 2,20~\euro.
%\medskip
\end{enumerate}
%* Les financiers et les macarons sont des pâtisseries. 

\bigskip

