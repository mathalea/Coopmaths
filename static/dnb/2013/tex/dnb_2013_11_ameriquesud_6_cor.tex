
\medskip

\begin{enumerate}
\item 
	\begin{enumerate}
		\item %Si un tel rectangle a pour longueur 10 cm, quelle est sa largeur ?
Si $\ell$ est la largeur on a :
		
$2\times 10 + 2\ell = 31$, d'où $\ell = 15,5 - 10 = 5,5$~cm. 
		\item %Proposer une autre longueur et trouver la largeur correspondante.
Si la longueur a pour mesure 13~cm, on a :
		
$2\times 13 + 2\ell = 31$, d'où $\ell = 15,5 - 13 = 2,5$~cm.		 
		\item %On appelle $x$ la longueur AB. 

%En utilisant le fait que le périmètre de ABCD est de $31$ cm, exprimer la longueur BC en fonction de $x$. 
On a $2x + 2\ell = 31$ soit $\ell = 15,5 - x$.
		\item %En déduire l'aire du rectangle ABCD en fonction de $x$.
On a $\mathcal{A}(x) = x \times \ell = x(15,5 - x) = 15,5x - x^2$. 
	\end{enumerate}
\item %On considère la fonction $f$ définie par $f(x) = x (15,5 - x)$.
	\begin{enumerate}
		\item %Calculer $f(4)$.
$f(4) = 4 \times (15,5 - 4) = 4 \times 11,5 = 46$~cm$^2$. 
		\item %Vérifiez qu'un antécédent de $52,5$ est $5$.
Si  un antécédent de $52,5$ est $5$, l'image de 5 est 52,5.

$f(5) = 5 \times (15,5 - 5) = 5 \times 10,5 = 52,5$~cm$^2$.
	\end{enumerate}
\item %Sur le graphique ci-dessous, on a représenté l'aire du rectangle ABCD en fonction de la valeur de $x$. 

%\begin{center}
%\psset{xunit=0.65cm,yunit=0.1cm}
%\begin{pspicture}(-0.5,-10)(16,65)
%\psaxes[linewidth=1.5pt,Dy=10]{->}(0,0)(-0.5,-5)(16,65)
%\psplot[plotpoints=5000,linewidth=1.25pt,linecolor=blue]{0}{15.5}{15.5 x sub x mul}
%\rput{90}(-1.5,30){Aire de ABCD}
%\uput[d](14,-5){Valeur de $x$} 
%\end{pspicture}
%\end{center}
% 
%À l'aide de ce graphique, répondre aux questions suivantes en donnant des valeurs approchées: 

\medskip

	\begin{enumerate}
		\item %Quelle est l'aire du rectangle ABCD lorsque $x$ vaut 3 cm ?
La verticale partant du point de coordonnées (3~;~0) coupe la courbe en un point dont l'ordonnée est à peu près 38. 
		\item %Pour quelles valeurs de $x$ obtient-on une aire égale à 40 cm$^2$ ?
L'horizontale partant du point de coordonnées (0~;~40) coupe la courbe en deux points dont les abscisses sont  peu près 3,3 et 12,2.		 
		\item %Quelle est l'aire maximale de ce rectangle ? Pour quelle valeur de $x$ est-elle obtenue ? 
On lit un peu plus de 60~cm$^2$ pour $x \approx 7,75$.
	\end{enumerate}
\item %Que peut-on dire du rectangle ABCD lorsque AB vaut $7,75$~cm ?
Si $x = 7,75$ alors l'autre côté mesure $15,5 - 7,75 = 7,75$ donc la même valeur : le rectangle est donc un carré. 
\end{enumerate}
