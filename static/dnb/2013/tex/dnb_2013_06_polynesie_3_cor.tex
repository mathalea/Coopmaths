
\medskip

%Dans l'Océan Pacifique Nord, des déchets plastiques qui flottent se sont accumulés pour constituer une poubelle géante qui est, aujourd'hui, grande comme 6 fois la France.
%
%\medskip
 
\begin{enumerate}
\item %Sachant que la superficie de la France est environ \np{550000}~km$^2$, quelle est la superficie actuelle de cette poubelle géante ?
La poubelle a une superficie de $6 \times \np{550000} = \np{3300000}$ (3,3 millions de kilomètres carrés) 
\item %Sachant que la superficie de cette poubelle géante augmente chaque année de 10 \,\%, quelle sera sa superficie dans un an ?
Augmenter de 10\,\%, c'est multiplier par 1,1.

Dans un an la superficie sera égale à $\np{3300000} \times 1,1 = \np{3630000}~\left(\text{km}^2\right)$.
\item %Que penses-tu de l'affirmation \og dans 4 ans, la superficie de cette poubelle aura doublé \fg{} ? Justifie ta réponse.
Chaque année on multiplie la superficie par 1,1, donc au bout de quatre ans celle-ci sera égale à :

$\np{3300000} \times 1,1^4 = \np{4831530}$, soit beaucoup moins que le double de la superficie de départ. $1,1^4 = \np{1,4641}$ qui correspond à une augmentation de 46,41\,\%.
\end{enumerate}
 
\bigskip

