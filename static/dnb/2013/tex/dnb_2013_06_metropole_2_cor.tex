
On a utilisé un tableur pour calculer les images de différentes valeurs de $x$ par une fonction affine $f$ et par une autre fonction $g$. Une copie de l'écran obtenu est donnée ci-dessous. 

\begin{tabularx}{\linewidth}{|c|*{8}{>{\centering \arraybackslash}X|}}\hline
\multicolumn{3}{|c|}{\texttt{C2}}&$\texttt{fx}$&\multicolumn{5}{|l|}{\textcolor{blue}{$=-5\star\text{\texttt{C}}1+7$}}\\ \hline
&A&B&C&D&R&F&G&H\\ \hline
1&$x$&\textcolor{red}{$- 3$}&$- 2$&$- 1$&$0$&1&2&3\\ \hline 
2&$f(x)$&\textcolor{red}{22}&\textcolor{blue}{\fbox{~~~~~~17~~~~~}}&12&7&2&$- 3$&$- 8$\\ \hline 
3&$g(x)$&13	&8	&5	&4	&5	&8	&13\\ \hline
4&		&	&	&	&	&	&	&\\ \hline
\end{tabularx}
\begin{enumerate}
\item L'image de $- 3$ par  $f$ est \textcolor{red}{$f(-3)=22$}.
\item Dans la case \textcolor{blue}{\texttt{C2}} se trouve la formule \textcolor{blue}{$=-5\star\text{\texttt{C}}1+7$}, ce qui signifie que la valeur de \textcolor{blue}{\texttt{C2}} est obtenue en multipliant le contenu de la case \texttt{C1} par $-5$ et en ajoutant $7$ au résultat.

En \og tirant sur la formule\fg, on obtient pour la case \texttt{L2}: $=-5\star\text{\texttt{L}}1+7$.

\texttt{L1} contient $7$, donc \texttt{L2} contient $-5\times 7+7=-28$

Ainsi $f(7)=-28$. 
\item $f(x)=-5x+7$. 
\item On sait que $g(x) = x^2 + 4$. La formule saisie dans la cellule \texttt{B3} et recopiée ensuite vers la droite pour compléter la plage de cellules \texttt{C3:H3} est: \texttt{B1$\star$B1+4}
\end{enumerate}

