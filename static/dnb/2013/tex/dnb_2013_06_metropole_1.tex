
\medskip

\parbox{0.5\linewidth}{Avec un logiciel :
 
\begin{itemize}
\item on a construit un carré ABCD, de côté 4 cm.
\item on a placé un point M mobile sur [AB] et construit le carré MNPQ comme visualisé sur la copie d'écran ci-contre. 
\item on a représenté l'aire du carré MNPQ en 
fonction de la longueur AM.
\end{itemize}}\hfill  \parbox{0.4\linewidth}{\psset{unit=0.8cm}
\begin{pspicture}(6.5,6.5)
\psframe(0.5,0.5)(5.7,5.7)
\pspolygon(4.2,0.5)(5.7,4.2)(2,5.7)(0.5,2)
\uput[ul](0.5,5.7){A} \uput[ur](5.7,5.7){B} \uput[dr](5.7,0.5){C} \uput[dl](0.5,0.5){D} 
\uput[u](2,5.7){M} \uput[r](5.7,4.2){N} \uput[d](4.2,0.5){P} \uput[l](0.5,2){Q}
\uput[u](2,5.7){M} \uput[r](5.7,4.2){N} \uput[d](4.2,0.5){P} \uput[l](0.5,2){Q}
\psline(1.25,5.6)(1.25,5.8)\psline(1.35,5.6)(1.35,5.8)
\psline(4.90,0.4)(4.90,0.6)\psline(5.00,0.4)(5.00,0.6)
\psline(0.4,1.2)(0.6,1.2)\psline(0.4,1.3)(0.6,1.3)
\psline(5.6,4.9)(5.8,4.9)\psline(5.6,5)(5.8,5)
\end{pspicture}}

On a obtenu le graphique ci-dessous.

\begin{center}
\psset{xunit=1.25cm,yunit=0.5cm}
\begin{pspicture}(-1,-0.5)(6,18)
\psgrid[gridlabels=0pt,subgriddiv=1,gridwidth=1pt,griddots=10,gridcolor=orange](0,0)(6,18)
\psaxes[linewidth=1pt](0,0)(0,0)(6,18)
\psaxes[linewidth=1.5pt]{->}(0,0)(1,1)
\psplot[plotpoints=5000,linewidth=1.25pt,linecolor=blue]{0}{4}{x dup mul 2 mul 8 x mul sub 16 add}
\uput[r](0,17.5){Aire de MNPQ $\left(\text{en cm}^2\right)$}
\uput[u](5,0){Longueur AM (en cm)}\uput[dl](0,0){O}
\end{pspicture} 
\end{center}
 
En utilisant ce graphique répondre aux questions suivantes. \textbf{Aucune justification n'est attendue.} 

\begin{enumerate}
\item Déterminer pour quelle(s) valeur(s) de AM, l'aire de MNPQ est égale à $10$~cm$^2$.
\item Déterminer l'aire de MNPQ lorsque AM est égale à 0,5cm.
\item Pour quelle valeur de AM l'aire de MNPQ est-elle minimale ? Quelle est alors cette aire ? 
\end{enumerate}

\bigskip

