
\medskip

%\textbf{Dans cet exercice, si le travail n'est pas terminé, laisser tout de même une trace de la recherche. Elle sera prise en compte dans l'évaluation.}
%
%\medskip
% 
%La ville BONVIVRE possède une plaine de jeux bordée d'une piste cyclable. La piste cyclable a la forme d'un rectangle ABCD dont on a \og enlevé trois des coins \fg.
% 
%Le chemin de G à H est un arc de cercle ; les chemins de E à F et de I à J sont des segments.
% 
%Les droites (EF) et (AC) sont parallèles. 
%
%\begin{center}
%\psset{unit=0.7cm}
%\begin{pspicture}(16.5,8)
%\psline[linewidth=1.8pt](16,3.6)(16,6)(13.4,7)(1.3,7)(1.3,4.6)(2.8,1)(13.4,1)%GFEAJIH
%\uput[ul](1.3,7){A} \uput[ur](16,7){B} \uput[dr](16,1){C} \uput[dl](1.3,1){D} 
%\uput[ul](13.4,7){E} \uput[ur](16,6){F} \uput[dr](16,3.6){G} \uput[d](13.4,1){H} 
%\uput[dr](2.8,1){I} \uput[ul](1.3,4.6){J} 
%\psarc[linewidth=1.8pt](13.4,3.6){2.6}{-90}{0}
%\psline[linewidth=0.6pt,arrowsize=3pt 3]{<->}(1.3,7)(16,1)
%\rput(6,5.5){Rugby}\psframe(3.3,4.3)(8.4,6.6)%rugby\rput(6,5.5){Rugby}
%\rput(6,2.5){Rugby}\psframe(3.3,1.6)(8.4,3.9)%rugby
%\rput(11.8,4.3){Foot}\psframe(10.7,1.6)(13,6.6)%foot
%\psframe[linestyle=dashed](1.3,1)(16,7)
%\rput{-23}(9.5,4){312 m}
%\psline[linestyle=dashed](13.4,1)(13.4,3.6)(16,3.6)
%\psframe(13.4,3.4)(13.6,3.6)
%\psframe(1.3,1)(1.5,1.2)
%\psframe(16,1)(15.8,1.2)
%\psline[linewidth=0.6pt,arrowsize=3pt 3]{<->}(13.4,7.2)(16,7.2)\uput[u](14.7,7.2){48 m}
%\psline[linewidth=0.6pt,arrowsize=3pt 3]{<->}(1.3,7.8)(16,7.8)\uput[u](8.65,7.8){288 m}
%\psline[linewidth=0.6pt,arrowsize=3pt 3]{<->}(16.4,6)(16.4,3.6)\uput[r](16.4,4.1){52 m}
%\psline[linewidth=0.6pt,arrowsize=3pt 3]{<->}(1.3,0.7)(3,0.7)\uput[d](2.15,0.7){29 m}
%\psline[linewidth=0.6pt,arrowsize=3pt 3]{<->}(1.1,1)(1.1,4.6)\uput[l](1.1,2.8){72 m}
%\psframe(16,7)(15.8,6.8)
%\psline(14.6,6.8)(14.7,7.2)\psline(14.7,6.8)(14.8,7.2)
%\psline(14.6,3.4)(14.7,3.8)\psline(14.7,3.4)(14.8,3.8)
%\rput{90}(13.4,1.8){\psline(-0.1,-0.2)(-0,0.2)}
%\rput{90}(13.4,1.8){\psline(-0.2,-0.2)(-0.1,0.2)}
%\psline(1.1,5.8)(1.5,5.7)
%\psline(1.1,5.7)(1.5,5.6)
%\psframe(1.3,7)(1.5,6.8)
%\end{pspicture}
%\end{center}

%Quelle est la longueur de la piste cyclable ? Justifier la réponse.
Calcul de EF : dans le triangle ABC rectangle en B, les droites (EF) et (AC) sont parallèles, les points B, E, A d'une part, B, F, C de l'autre sont alignés dans cet ordre ; d'après le théorème de Thalès :

$\dfrac{\text{BE}}{\text{BA}} =  \dfrac{\text{EF}}{\text{AC}}$ soit $\dfrac{48}{288} =  \dfrac{\text{EF}}{312}$, d'où EF $ = \dfrac{48 \times 312}{288} = 52$~(m).

Calcul de l'arc GE : cet arc est un quart de cercle de rayon 48 ; sa longueur est donc : $\pi \times \dfrac{2\times 48}{4} = 24\pi$.

Calcul de IJ : dans le triangle IGJ rectangle en D, le théorème de PYthagore permet de calculer :

IJ$^2 = 29^2 + 72^2 = \np{6025}$, d'où IJ $ = \sqrt{\np{6025}} \approx 77,62$~(m)

La longueur de la piste cyclable est donc égale à :

$(288 - 48) + 52 + 52 + 24\pi + (288 - 48 - 29) + \sqrt{\np{6025}} + 48 \approx 756$~(m).
