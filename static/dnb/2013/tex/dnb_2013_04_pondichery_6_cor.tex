
\medskip

%Lancé le 26 novembre 2011, le Rover Curiosity de la NASA est chargé d'analyser la planète Mars, appelée aussi planète rouge.
% 
%Il a atterri sur la planète rouge le 6 août 2012, parcourant ainsi une distance d'environ 560 millions de km en 255 jours.
%
%\medskip
 
\begin{enumerate}
\item %Quelle a été la durée en heures du vol ?
On a 255 jours $ = 255 \times 24 = \np{6120}$~(h). 
\item %Calculer la vitesse moyenne du Rover en km/h. Arrondir à la centaine près.
$v = \dfrac{d}{t}  =  \dfrac{560 \times 10^6}{255 \times 24} \approx \np{91503,3}$~(km/h) soit environ \np{91500}~km/h. 
%\emph{Pour cette question toute trace de recherche, même incomplète, sera prise en compte dans l'évaluation} 
\item %\emph{Pour cette question toute trace de recherche, même incomplète, sera prise en compte dans l'évaluation}

%Via le satellite Mars Odyssey, des images prises et envoyées par le Rover ont été retransmises au centre de la NASA.
% 
%Les premières images ont été émises de Mars à 7~h 48~min le 6~août~2012.
% 
%La distance parcourue par le signal a été de $248 \times 10^6$~km à une vitesse moyenne de \np{300000}~km/s environ (vitesse de la lumière).
% 
%À quelle heure ces premières images sont-elles parvenues au centre de la NASA ? (On donnera l'arrondi à la minute près).
Le temps de transfert est égal à  $t = \dfrac{d}{v} = \dfrac{248 \times 10^6}{\np{300000}} \approx 826,667$~(s) soit 13~min et 47~s environ.

Les premières images sont arrivées à 7~h 48~min $+$ 13 min 47~s soit à 

8~h 01~min 47~s (8 h 02 min à la minute près).
\end{enumerate}

%\medskip
%
%\textbf{Maîtrise de la langue : 4 points}
