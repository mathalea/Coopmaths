
\medskip

Flavien veut répartir la totalité de 760~dragées au chocolat et \np{1045}~dragées aux amandes dans des sachets dans des sachets ayant la même répartition de dragées au chocolat et aux amandes.

\medskip
 
\begin{enumerate}
\item %Peut-il faire 76 sachets ? Justifier la réponse.
On a $760 = 76 \times 10$ mais  $\np{1045}$ impair ne peut être multiple de 76 qui est pair. On ne peut donc répartir ces dragées dans 76 sachets.
\item 
	\begin{enumerate}
		\item %Quel nombre maximal de sachets peut-il réaliser ?
		On cherche avec l'algorithme d'Euclide le PGCD à 760 et \np{1045} :
		
$\np{1045} = 760 \times 1 + 285$ ;

$760 = 285 \times 2 + 190$ ;

$285 = 190 \times 1 + 95$ ;

$190 = 95 \times 2 + 0$.

On a donc $PGCD(760~;~\np{1045}) = 95$.

On peut faire au maximum 95 sachets. 
		\item %Combien de dragées de chaque sorte y aura-t-il dans chaque sachet ?
On  a $760 = 95 \times 8$ et $\np{1045} = 95 \times 11$.

Il y a dans chacun des 95 sachets, 8 dragées au chocolat et 11 dragées aux amandes.
	\end{enumerate}
\end{enumerate}
 
\vspace{0,5cm}

