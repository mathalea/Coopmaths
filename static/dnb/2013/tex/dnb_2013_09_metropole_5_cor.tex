
\medskip

%\parbox{0.5\linewidth}{Voici une figure codée réalisée à main levée :
% 
%On sait que
%\begin{itemize}
%\item La droite (AC) est perpendiculaire à la droite (AB).
%\item La droite (EB) est perpendiculaire à la droite (AB).
%\item Les droites (AE) et (BC) se coupent en D.
%\item AC = 2,4 cm ; AB = 3,2 cm ; \mbox{BD = 2,5~cm} et DC = 1,5 cm.\end{itemize}} \hfill
%\parbox{0.45\linewidth}{\psset{unit=1cm}
%\begin{pspicture}(5,3) 
%\pscurve(0.2,2.6)(2.55,1.43)(4.9,0.2)
%\pscurve(4.9,0.2)(2.55,0.24)(0.2,0.2)
%\pscurve(0.2,0.2)(0.17,1.4)(0.2,2.6)
%\pscurve(0.2,2.6)(1.65,2.57)(3.1,2.6)
%\pscurve(3.1,2.6)(1.65,1.37)(0.2,0.2)%AEBACDB
%\uput[ul](0.2,2.6){A} \uput[dl](0.2,0.2){B} \uput[ur](3.1,2.6){C}\uput[d](2,1.5){D}\uput[dr](4.9,0.2){E}
%\psframe(0.2,0.2)(0.4,0.4) \psframe(0.2,2.6)(0.4,2.4)
%\end{pspicture}} 
%
%\medskip
 
\begin{enumerate}
\item %Réaliser la figure en vraie grandeur sur la copie.
~
\begin{center}\psset{unit=1cm}
\begin{pspicture}(5,3)
%\psgrid 
\psline(0.2,2.6)(3.1,2.6)(0.2,0.2)
(4.9,0.2)(4.9,0.2)(0.2,2.6)(0.2,0.2)%ACBEAB
\uput[ul](0.2,2.6){A} \uput[dl](0.2,0.2){B} \uput[ur](3.1,2.6){C}\uput[d](2,1.5){D}\uput[dr](4.9,0.2){E}
\psframe(0.2,0.2)(0.4,0.4) \psframe(0.2,2.6)(0.4,2.4)
\end{pspicture} 
\end{center}
\item %Déterminer l'aire du triangle ABE. 
Les droites (AC) et (BE) sont parallèles car perpendiculaires à la même droite (AB).

Les points A, D, E d’une part, C, D, B de l’autre sont alignés dans cet ordre ; le théorème de Thalès permet d’écrire :

$\dfrac{\text{CD}}{\text{DB}} = \dfrac{\text{AC}}{\text{BE}}$ soit BE $ = \dfrac{\text{AC} \times \text{DB}}{\text{CD}} = \dfrac{2,4 \times 2,5}{1,5} = 4$~(cm).

L’aire du triangle ABE rectangle en B est égale à $\dfrac{1}{2} \times \text{AB} \times \text{BE} = \dfrac{3,2 \times 4}{2} = 6,4$~cm$^2$.
\end{enumerate} 

\bigskip  

