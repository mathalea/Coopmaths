
Dans les marais salants, le sel récolté est stocké sur une surface plane. On admet qu'un tas de sel a toujours la forme d'un cône de révolution. 

\begin{enumerate}
\item 
	\begin{enumerate}
		\item Pascal souhaite déterminer la hauteur d'un cône de sel de diamètre 5 mètres. Il possède un bâton de longueur $1$~mètre. Il effectue des mesures et réalise les deux schémas ci-dessous :
		
%\begin{center}
%\begin{tikzpicture}[scale=.9]
%    \draw (-1,0) arc (180:360:2cm and 0.5cm) -- (1,3.2) -- cycle;
%    \draw[dashed] (-1,0) arc (180:0:2cm and 0.5cm);
%   \shade[left color=blue!5!white,right color=blue!40!white,opacity=0.3] (-1,0) arc (180:360:2cm and 0.5cm) -- (1,3.2) -- cycle;
%   \tkzDefPoints{1/0/O,1/3.2/A,-6.2/0/D,-2.92/0/B,-2.92/3/B'}
%   \tkzDrawSegments[dashed](O,A O,D D,A)
%   \tkzInterLL(A,D)(B,B') \tkzGetPoint{C}
%   \tkzDrawSegment(B,C)
%   \tkzLabelSegment[right](B,C){{\scriptsize Bâton}}\tkzLabelSegment(A,O){{\scriptsize Cône de sel}}
%\end{tikzpicture}
%
%\begin{tikzpicture}[scale=.9]
%\tkzDefPoints{1/0/O,1/3.2/S,-1/0/E,3/0/L,-2.92/0/B,-6.2/0/A,-2.92/3/B'}
%\tkzInterLL(A,S)(B,B') \tkzGetPoint{C}
%\tkzDrawSegments(E,S S,L L,E B,C)
%\tkzDrawSegments[dashed](A,E A,S S,O)
%\tkzLabelPoints[below](A,B,E,L) \tkzLabelPoints[above](C,S)
%\tkzLabelPoint[above left](O){$O$}
%\tkzMarkRightAngle(L,O,S) \tkzMarkRightAngle(L,B,C)
%\tkzLabelSegment[below](A,B){3,20~m}
%\tkzLabelSegment[below](B,E){2,30~m}
%\tkzLabelSegment[below](E,L){5~m}
%\tkzLabelSegment[sloped](B,C){1~m}
%\end{tikzpicture}
%\end{center}
\begin{center}
\psset{unit=0.7cm} 
\begin{pspicture}(0,-0.3)(15,12)
\rput(11.5,9.5){Cône de sel} 
\pspolygon[linestyle=dashed](11.5,8)(0.3,8)(11.5,11.7)
\psline[linewidth=1.8pt](4.5,8)(4.5,9.4)
\uput[r](4.5,8.7){Bâton}
\scalebox{.99}[0.3]{\psarc(11.5,26.8){2.5}{180}{0}}%
\scalebox{.99}[0.3]{\psarc[linestyle=dashed](11.5,26.8){2.5}{0}{180}}%
\psline(8.9,8)(11.5,11.7)(13.9,8)
\pspolygon[linestyle=dashed](11.5,0.5)(0.3,0.5)(11.5,4.2)
\pspolygon(8.9,0.5)(11.5,4.2)(13.9,0.5)
\psline[linewidth=1.8pt](4.5,0.5)(4.5,1.9)
\psframe(11.5,0.5)(11.8,0.8)
\psframe(4.5,0.5)(4.2,0.8)
\psline[linewidth=0.6pt,arrowsize=3pt 3]{<->}(0.2,0.2)(4.5,0.2)
\psline[linewidth=0.6pt,arrowsize=3pt 3]{<->}(4.5,0.2)(8.9,0.2)
\psline[linewidth=0.6pt,arrowsize=3pt 3]{<->}(8.9,0.2)(13.9,0.2)
\uput[d](2.35,0.2){3,20 m} \uput[d](6.7,0.2){2,30 m} \uput[d](11.4,0.2){5 m}
\uput[r](4.5,1.2){1 m}\uput[ul](0.2,0.5){A}\uput[u](4.5,1.9){C}
\uput[u](11.5,4.2){S}\uput[ul](11.5,0.5){O}
\uput[ul](8.9,0.5){E}\uput[ur](13.9,0.5){L} \uput[ur](4.5,0.5){B}
\end{pspicture}
\end{center}

La hauteur de ce cône de sel est $h=2,50$~mètres. On utilise le théorème de Thalès:
\[
\frac{AB}{BC}=\frac{AO}{OS}\Longleftrightarrow \frac{3,2}{1}=\frac{3,2+2,3+2,5}{h}\Longleftrightarrow h=\frac{8}{3,2}=2,5
\]
 
\item Volume du cône $V$:
\[
V=\frac{\pi\times 2,5^2\times 2,5}{3}\simeq 16,3541666667\simeq 16~\text{m}^3
\]
\end{enumerate} 
\item Le sel est ensuite stocké dans un entrepôt sous la forme de cônes de volume \np{1000}~m$^3 $. Par mesure de sécurité, la hauteur d'un tel cône de sel ne doit pas dépasser $6$~mètres. 

\[
h = 6 \Longrightarrow 1000 = \frac{6\pi R^2}{3}=2\pi R^2\Longrightarrow \frac{500}{\pi}= R^2 \Longrightarrow R =  \sqrt{\frac{500}{\pi}} \approx 12,61 (R\ \text{est positif})
\]
Ainsi au dixième près $R = 12,7$~m.
\end{enumerate}
 
