
\medskip

%\emph{On considère l'expérience aléatoire suivante: on tire au hasard une carte dans un jeu bien mélangé de $32$ cartes (il y a $4$ \og familles \fg{} cœur, trèfle, carreau et pique et on a $8$ cœurs, $8$ trèfles, $8$ carreaux et $8$ piques).\\
%On relève pour la carte tirée la \og famille \fg{} (trèfle, carreau, cœur ou pique) puis on remet la carte dans le jeu et on mélange.}
% 
%On note $A$ l'évènement : \og la carte tirée est un trèfle \fg}.
%
%\medskip

\begin{enumerate}
\item %Quelle est la probabilité de l'évènement A ? 
Il y a 8 trèfles sur 32 cartes. La probabilité est donc égale à $\dfrac{8}{32} = \dfrac{1}{4} = 0,25$.
\item %On répète 24 fois l'expérience aléatoire ci-dessus. La représentation graphique ci-dessous donne la répartition des couleurs obtenues lors des vingt-quatre premiers tirages: 

%\begin{center}
%\psset{xunit=1.2cm,yunit=0.4cm}
%\begin{pspicture}(-3,-2)(5,11)
%\psframe(-3,-2)(5,11)
%\psline(4,0)(0,0)(0,10)
%\multido{\n=0+2}{6}{\psline[linewidth=0.2pt](0,\n)(4,\n) \uput[l](0,\n){\n}}
%\psframe[fillstyle=solid,fillcolor=lightgray](0.333,0)(0.666,6)
%\psframe[fillstyle=solid,fillcolor=lightgray](1.333,0)(1.666,8)
%\psframe[fillstyle=solid,fillcolor=lightgray](2.333,0)(2.666,3)
%\psframe[fillstyle=solid,fillcolor=lightgray](3.333,0)(3.666,7)
%\uput[d](0.5,0){cœur}\uput[d](1.5,0){trèfle} 
%\uput[d](2.5,0){carreau}\uput[d](3.5,0){pique}
%\rput(-1.5,8){nombre}
%\rput(-1.5,6){de fois où}
%\rput(-1.5,4){la carte}
%\rput(-1.5,2){est tirée}
%\end{pspicture}
%\end{center} 
% 
%Calculer la fréquence d'une carte de la \og famille \fg{} cœur et d'une carte de la \og famille \fg{} trèfle. 
Fréquence des cœur : $\dfrac{6}{24} = \dfrac{1}{4} = 0,25$.

Fréquence des trèfles : $\dfrac{8}{24} = \dfrac{1}{3}$.
\item %On reproduit la même expérience qu'à la question 2. Arthur mise sur une carte de la \og famille \fg{} cœur et Julie mise sur d'une carte de la \og famille \fg{} trèfle. 

%Est-ce que l'un d'entre deux a plus de chance que l'autre de gagner ? 

En théorie la fréquence d’apparition de chaque couleur est égale à $\dfrac{1}{4} = 0,25$. Les deux ont la même probabilité de gagner.
\end{enumerate}
 
\bigskip

