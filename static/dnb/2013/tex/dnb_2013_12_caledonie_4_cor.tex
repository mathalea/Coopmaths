
\bigskip

%\parbox{0.55\linewidth}{Sur le dessin ci-contre, les points A, B et E sont alignés, et C le milieu de [BD].
% 
%\begin{enumerate}
%\item Quelle est la nature du triangle ABC ? Justifier. 
%\item En déduire la nature du triangle BDE. 
%\item Calculer ED. Arrondir le résultat au dixième. 
%\end{enumerate}}\hfill
%\parbox{0.42\linewidth}{\psset{unit=0.8cm}
%\begin{pspicture}(6.25,4)
%\pspolygon(2.4,1.8)(0.4,0.3)(2.4,0.3)(2.4,3.3)(6,0.3)(2.4,0.3)%CABDE
%\uput[l](2.4,2.55){3}\uput[ul](1.6,1){5}\uput[d](1.4,0.3){4}
%\uput[d](4.6,0.3){7}
%\psdots(2.4,1.8)(0.4,0.3)(2.4,0.3)(2.4,3.3)(6,0.3)(2.4,0.3)
%\uput[dl](0.4,0.3){A} \uput[d](2.4,0.3){B} \uput[ul](2.4,1.8){C} \uput[u](2.4,3.3){D} \uput[dr](6,0.3){E} 
%\end{pspicture}}
\begin{enumerate}
\item Puisque C est le milieu de [BD], BC = 3.

Dans le triangle ABC on a le célèbre triplet de Pythagore, c'est-à-dire que :

$3^2 + 4^2 = 5^2$, donc d'après la réciproque du théorème de Pythagore, le triangle ABC est rectangle en B.
\item D'après la question précédente le triangle BDE est lui aussi rectangle en B mais n'est pas isocèle car BD = 6 et BE = 7.
\item D'après le théorème de Pythagore dans DBE rectangle en B :

$\text{DE}^2 = \text{DB}^2 + \text{BE}^2 = 6^2 + 7^2 = 36 + 49 = 85$.

Donc DE $ = \sqrt{85} \approx 9,22$ soit environ 9,2 au dixième.
\end{enumerate}

\bigskip 

