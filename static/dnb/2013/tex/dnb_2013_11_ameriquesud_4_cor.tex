
\medskip 

%\textbf{Dans cet exercice, si le travail n'est pas terminé, laisser tout de même une trace de la recherche, elle sera prise en compte dans l'évaluation.}
%
%\medskip 
%
%Le fleuve Amazone est celui qui possède le débit moyen le plus important au monde. Il est d'environ \np{190000}~m$^3$/s. 
%
%En France, un foyer de 3 personnes consomme en moyenne \np{10000}~L d'eau par mois. 
%
%Donner un ordre de grandeur du nombre de ces foyers que pourrait alimenter ce fleuve en un an. 
%
%Rappel : 1 L = 1 dm$^3$ et 1 m$^3$ = \np{1000}~L 
En un an l'Amazone débite :

$\np{190000} \times 60 \times 60 \times 24 \times 365 = \np{5991840000000}$~m$^3$ soit \np{5991840000000000}~L.

Cela permet d'alimenter $\dfrac{\np{5991840000000000}}{12 \times \np{10000}} = \dfrac{\np{599184000000}}{12} =  \np{49932000000}$~foyers de trois personnes vivant en France soit largement plus que le nombre total de ménages français.

\bigskip

