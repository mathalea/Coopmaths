
\medskip

Pour cet exercice, on utilise uniquement la courbe donnée ci-dessous qui représente une fonction $f$.

\medskip

En laissant apparaître les tracés utiles sur le graphique ci-dessous :
 
\begin{enumerate}
\item Donne une valeur approchée de $f(2)$. 
\item Donne l'(ou les) antécédent(s) de $5$ par la fonction $f$. 
\item Place, sur la courbe de la fonction $f$ un point S qui te semble avoir la plus petite ordonnée. 
\item Par lecture graphique, donne des valeurs approchées des coordonnées de ton point S. 
\end{enumerate}
\begin{center}
\psset{unit=1cm}
\begin{pspicture}(10,10)
\psgrid[gridlabels=0pt,subgriddiv=2,gridwidth=0.4pt,gridcolor=orange,subgridwidth=0.4pt,subgridcolor=orange]
\psaxes[linewidth=1pt](0,0)(10,10)
\psaxes[linewidth=1.5pt]{->}(0,0)(1,1)
\psplot[plotpoints=8000,linewidth=1.25pt,linecolor=blue]{0}{10}{x dup mul  0.08 mul  x   sub 8 add}
\rput(5,-1){Cette feuille est à rendre avec votre copie}
\uput[u](9.75,0){$x$}\uput[r](0,9.75){$y$}
\end{pspicture}
\end{center}

\vspace{1cm} 
 
\bigskip

