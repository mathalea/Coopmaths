
\medskip
 
Le jeu de fléchettes consiste à lancer $3$ fléchettes sur une cible. La position des 
fléchettes sur la cible détermine le nombre de points obtenus.
 
La cible est installée de sorte que son centre se trouve à $1,73$~m du sol. Les pieds du joueur ne doit pas s'approcher à moins de $2,37$~m lorsqu'il lance les fléchettes. Pour cela, un dispositif électronique est installé qui en mesurant l'angle calcule automatiquement la distance du joueur au mûr. Il sonne si la distance n'est pas réglementaire. 
1,73 m 

\bigskip

\parbox{0.5\linewidth}{\hspace{0,6cm}\textbf{1.}
 Un joueur s'apprête à lancer une fléchette. La droite passant par le centre de la cible et son pied fait un angle de $36,1$\degres{} avec le sol.\\
Le mur est perpendiculaire au sol.\\
Est-ce que la sonnerie va se déclencher ?\\ 
Justifier la réponse.}\hfill \parbox{0.48\linewidth}{\psset{unit=0.9cm}
\begin{pspicture}(7,4.8)
%\psgrid
\psline(1.5,4.4)(1.5,0.6)(6.6,0.6)
\psline[linestyle=dashed](1.5,3.6)(5.8,0.6)
\psline[linewidth=0.4pt,arrowsize=2pt 3]{<->}(1.2,0.6)(1.2,3.6)
\psarc(5.8,0.6){0.8cm}{143.9}{180}
\psline(5.8,0.6)(6,0.7)(5.85,1.3)(6.2,2)(6.1,1.3)(6.3,0.8)(6.2,0.7)
\psline(6.2,2)(6.2,3)(5.7,2.6)(5.4,2.75)
\psline(5.3,2.7)(5.4,2.8)
\pspolygon*(5.4,2.8)(5.3,2.9)(5.45,2.85)
\psline(6.2,3)(6.35,2.5)(6,2.2)%bras droit
\psellipse(6.2,3.35)(0.25,0.35) \psline(6,3.15)(6.1,3.25)
\qdisk(6.05,3.5){1pt}\qdisk(5.9,3.4){1.5pt}
\rput(4.4,0.8){36,1\degres} \uput[l](1.5,2.1){1,73~m}
\uput[d](5.8,0.6){P}\uput[dl](1.5,0.6){M}\uput[ul](1.5,4.4){C}
\psframe*(1.5,3.2)(1.6,4)
\end{pspicture}} 

\begin{enumerate} 
\item[2.]  On a relevé dans le tableau ci-dessous les points obtenus par Rémi et Nadia lors de sept parties de fléchettes. Le résultat de Nadia lors la partie 6 a été égaré.

\medskip
\begin{tabularx}{\linewidth}{|l|*{7}{>{\centering \arraybackslash}X|}c|c|}\hline 
Partie &1&2 &3 &4 &5 &6 &7 &Moyenne &Médiane\\ \hline 
Rémi &40 &35 &85 &67 &28 &74 &28&&\\ \hline  
Nadia &12 &62 &7 &100 &81& &30 &51&\\ \hline 
\end{tabularx}
\medskip 

	\begin{enumerate}
		\item Calculer le nombre moyen de points obtenus par Rémi. 
		\item Sachant que Nadia a obtenu en moyenne $51$ points par partie, calculer le nombre de points qu'elle a obtenus à la 6\up{e} partie. 
		\item Déterminer la médiane de la série de points obtenus par Rémi, puis par Nadia. 
	\end{enumerate}
\end{enumerate}

\bigskip

