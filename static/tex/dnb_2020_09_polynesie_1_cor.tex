
\medskip

\emph{Dans cet exercice, toutes les questions sont indépendantes}

\medskip

\begin{enumerate}
\item On obtient $- 7 \to - 5 \to (- 5)^2 = 25$.

%\parbox{0.6\linewidth}{Quel nombre obtient-on avec le programme de calcul ci- contre, si l'on choisit comme nombre de départ $-7$ ?}\hfill
%\parbox{0.38\linewidth}{
%\begin{tabular}{|l|}\hline
%\multicolumn{1}{|c|}{\textbf{Programme de calcul}}\\
%Choisir un nombre de départ.\\
%Ajouter 2 au nombre de départ.\\
%Élever au carré le résultat.\\ \hline
%\end{tabular}}

\item %Développer et réduire l'expression $(2x - 3)(4x + 1)$.

$(2x - 3)(4x + 1) = 8x^2 + 2x  - 12x  - 3 = 8x^2 - 10x  - 3$.
\item Les droites (AB) et (DE) sont parallèles, d'après le théorème de Thalès, on peut écrire :

$\dfrac{\text{CB}}{\text{CE}} = \dfrac{\text{CA}}{\text{CD}}$, soit ici $\dfrac{\text{CB}}{1,5} = \dfrac{3,5}{1}$, d'où $\text{CB} = 3,5 \times 1,5 = 5,25$~(cm).

\item %Un article coûte 22~\euro. Son prix baisse de 15\,\%. Quel est son nouveau prix ?

Enlever 15\,\%, c'est multiplier par $1 - \dfrac{15}{100} = 1 - 0,15 = 0,85$.

Le nouveau prix est donc : $22 \times 0,85 = 18,70$~(\euro).
\item %Les salaires mensuels des employés d'une entreprise sont présentés dans le tableau suivant.

%\begin{center}
%\begin{tabularx}{\linewidth}{|m{2.7cm}|*{7}{>{\centering \arraybackslash}X|}}\hline
%Salaire mensuel (en euro)&\np{1300} &\np{1400} &\np{1500} &\np{1900} &\np{2000} &\np{2700} &\np{3500}\\
% \hline
%Effectif				 & 11 		&6 		&5 		&3 		&3 		&1 		&1\\ \hline
%\end{tabularx}
%\end{center}
%
%Déterminer le salaire médian et l'étendue des salaires dans cette entreprise.
IL y a $11 + 6 + 5 + 3 + 3 + 1 + 1 = 30$ salariés. Le 15\up{e} et le 16\up{e} salaire sont de \np{1400}~\euro{} qui est le salaire médian.

L'étendue est $\np{3500} - \np{1300} = \np{2200}$.
\item Quel est le plus grand nombre premier qui divise \np{41895} ?

\np{41895} est multiple de 5 : $\np{41895} = 5 \times \np{8379}$ et $\np{8379}$ est un multiple de 9 : $\np{8379} = 9 \times \np{931}$ qui est  multiple de 7 : $\np{931} = 7 \times 133$.

Enfin 133 est multiple de 7  : $133 = 7 \times 19$.

Avec $9 = 3^2$, on a donc :

$\np{41895} = 3^2 \times 5 \times 7^2 \times 19$.

Le plus grand diviseur premier de \np{41895} est donc 19.
\end{enumerate}

\vspace{0.5cm}

