
\medskip
\begin{enumerate}
	\item Entre la station 1 et la station 4, il y a $(4 - 1) = 3$ distances inter-stations, donc la distance entre la station 1 et la station 4 est donc d'environ $3\times 450 = \np[m]{1350}$
	
	\item Il y a 60 minutes dans une heure, donc 24 minutes correspondent à : $\dfrac{24}{60} = \dfrac{4}{10}=0,4$\,h.
	
Pendant cette durée, le bus parcourt 9,9\,km, cela donne une vitesse moyenne de $ \dfrac{9,9}{0,4} = \np[km/h]{24,75}$.
	
	\item Une augmentation de 40\,\%, cela se traduit par un coefficient multiplicateur de $1 + \dfrac{40}{100} = 1,4$.
	
Le nouveau prix du bus Calédorail serait donc de : $190\times 1,4 = \np[F]{266}$.
\end{enumerate}


