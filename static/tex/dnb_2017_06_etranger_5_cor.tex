\documentclass[10pt]{article}
\usepackage[T1]{fontenc}
\usepackage[utf8]{inputenc}%ATTENTION codage UTF8
\usepackage{fourier}
\usepackage[scaled=0.875]{helvet}
\renewcommand{\ttdefault}{lmtt}
\usepackage{amsmath,amssymb,makeidx}
\usepackage[normalem]{ulem}
\usepackage{diagbox}
\usepackage{fancybox}
\usepackage{tabularx,booktabs}
\usepackage{colortbl}
\usepackage{pifont}
\usepackage{multirow}
\usepackage{dcolumn}
\usepackage{enumitem}
\usepackage{textcomp}
\usepackage{lscape}
\newcommand{\euro}{\eurologo{}}
\usepackage{graphics,graphicx}
\usepackage{pstricks,pst-plot,pst-tree,pstricks-add}
\usepackage[left=3.5cm, right=3.5cm, top=3cm, bottom=3cm]{geometry}
\newcommand{\R}{\mathbb{R}}
\newcommand{\N}{\mathbb{N}}
\newcommand{\D}{\mathbb{D}}
\newcommand{\Z}{\mathbb{Z}}
\newcommand{\Q}{\mathbb{Q}}
\newcommand{\C}{\mathbb{C}}
\usepackage{scratch}
\renewcommand{\theenumi}{\textbf{\arabic{enumi}}}
\renewcommand{\labelenumi}{\textbf{\theenumi.}}
\renewcommand{\theenumii}{\textbf{\alph{enumii}}}
\renewcommand{\labelenumii}{\textbf{\theenumii.}}
\newcommand{\vect}[1]{\overrightarrow{\,\mathstrut#1\,}}
\def\Oij{$\left(\text{O}~;~\vect{\imath},~\vect{\jmath}\right)$}
\def\Oijk{$\left(\text{O}~;~\vect{\imath},~\vect{\jmath},~\vect{k}\right)$}
\def\Ouv{$\left(\text{O}~;~\vect{u},~\vect{v}\right)$}
\usepackage{fancyhdr}
\usepackage[french]{babel}
\usepackage[dvips]{hyperref}
\usepackage[np]{numprint}
%Tapuscrit : Denis Vergès
%\frenchbsetup{StandardLists=true}

\begin{document}
\setlength\parindent{0mm}
% \rhead{\textbf{A. P{}. M. E. P{}.}}
% \lhead{\small Brevet des collèges}
% \lfoot{\small{Polynésie}}
% \rfoot{\small{7 septembre 2020}}
\pagestyle{fancy}
\thispagestyle{empty}
% \begin{center}
    
% {\Large \textbf{\decofourleft~Brevet des collèges Polynésie 7 septembre 2020~\decofourright}}
    
% \bigskip
    
% \textbf{Durée : 2 heures} \end{center}

% \bigskip

% \textbf{\begin{tabularx}{\linewidth}{|X|}\hline
%  L'évaluation prend en compte la clarté et la précision des raisonnements ainsi que, plus largement, la qualité de la rédaction. Elle prend en compte les essais et les démarches engagées même non abouties. Toutes les réponses doivent être justifiées, sauf mention contraire.\\ \hline
% \end{tabularx}}

% \vspace{0.5cm}\textbf{\textsc{Exercice 5} \hfill 6 points}

\medskip

%Sarah vient de faire construire une piscine dont la forme est un pavé droit de 8 m de longueur, 4 m de largeur et 1,80 m de profondeur. Elle souhaite maintenant remplir sa piscine. Elle y installe donc son tuyau d'arrosage. 
%
%Sarah a remarqué qu'avec son tuyau d'arrosage, elle peut remplir un seau de 10 litres en 18 secondes.
%
%Pour remplir sa piscine, un espace de 20 cm doit être laissé entre la surface de l'eau et le haut de la piscine. 
%
%Faut-il plus ou moins d'une journée pour remplir la piscine ? Justifier votre réponse.
Le débit du tuyau est égal à $\dfrac{10}{18} = \dfrac{5}{9}$~l/s.

Le volume à remplir est celui d'un pavé de 8 m sur 4 m d'une hauteur de 1,6~m, donc égal à :

$8 \times 4 \times 1,6 = 51,2$~m$^3$ ou $\np{51200}$~dm$^3$ ou $\np{51200}$~l.

Le temps nécessaire est donc égal à :

$\dfrac{\np{51200}}{\frac{5}{9}} = \dfrac{\np{51200} \times 9}{5} = \np{92160}$~s soit $\dfrac{\np{92160}}{60} = \np{1536}$~min ou $\dfrac{\np{1536}}{60} = 25,6$~heures, donc plus d'une journée.
\bigskip

\end{document}