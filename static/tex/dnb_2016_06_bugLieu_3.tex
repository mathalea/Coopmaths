\documentclass[10pt]{article}
\usepackage[T1]{fontenc}
\usepackage[utf8]{inputenc}%ATTENTION codage UTF8
\usepackage{fourier}
\usepackage[scaled=0.875]{helvet}
\renewcommand{\ttdefault}{lmtt}
\usepackage{amsmath,amssymb,makeidx}
\usepackage[normalem]{ulem}
\usepackage{diagbox}
\usepackage{fancybox}
\usepackage{tabularx,booktabs}
\usepackage{colortbl}
\usepackage{pifont}
\usepackage{multirow}
\usepackage{dcolumn}
\usepackage{enumitem}
\usepackage{textcomp}
\usepackage{lscape}
\newcommand{\euro}{\eurologo{}}
\usepackage{graphics,graphicx}
\usepackage{pstricks,pst-plot,pst-tree,pstricks-add}
\usepackage[left=3.5cm, right=3.5cm, top=3cm, bottom=3cm]{geometry}
\newcommand{\R}{\mathbb{R}}
\newcommand{\N}{\mathbb{N}}
\newcommand{\D}{\mathbb{D}}
\newcommand{\Z}{\mathbb{Z}}
\newcommand{\Q}{\mathbb{Q}}
\newcommand{\C}{\mathbb{C}}
\usepackage{scratch}
\renewcommand{\theenumi}{\textbf{\arabic{enumi}}}
\renewcommand{\labelenumi}{\textbf{\theenumi.}}
\renewcommand{\theenumii}{\textbf{\alph{enumii}}}
\renewcommand{\labelenumii}{\textbf{\theenumii.}}
\newcommand{\vect}[1]{\overrightarrow{\,\mathstrut#1\,}}
\def\Oij{$\left(\text{O}~;~\vect{\imath},~\vect{\jmath}\right)$}
\def\Oijk{$\left(\text{O}~;~\vect{\imath},~\vect{\jmath},~\vect{k}\right)$}
\def\Ouv{$\left(\text{O}~;~\vect{u},~\vect{v}\right)$}
\usepackage{fancyhdr}
\usepackage[french]{babel}
\usepackage[dvips]{hyperref}
\usepackage[np]{numprint}
%Tapuscrit : Denis Vergès
%\frenchbsetup{StandardLists=true}

\begin{document}
\setlength\parindent{0mm}
% \rhead{\textbf{A. P{}. M. E. P{}.}}
% \lhead{\small Brevet des collèges}
% \lfoot{\small{Polynésie}}
% \rfoot{\small{7 septembre 2020}}
\pagestyle{fancy}
\thispagestyle{empty}
% \begin{center}
    
% {\Large \textbf{\decofourleft~Brevet des collèges Polynésie 7 septembre 2020~\decofourright}}
    
% \bigskip
    
% \textbf{Durée : 2 heures} \end{center}

% \bigskip

% \textbf{\begin{tabularx}{\linewidth}{|X|}\hline
%  L'évaluation prend en compte la clarté et la précision des raisonnements ainsi que, plus largement, la qualité de la rédaction. Elle prend en compte les essais et les démarches engagées même non abouties. Toutes les réponses doivent être justifiées, sauf mention contraire.\\ \hline
% \end{tabularx}}

% \vspace{0.5cm}\textbf{\textsc{Exercice 3} \hfill 5 points}

\medskip

Trois figures codées sont données ci-dessous. Elles ne sont pas dessinées en vraie grandeur. 

Pour chacune d'elles, déterminer la longueur AB au millimètre près. 

\textbf{Dans cet exercice, on n'attend pas de démonstration rédigée. Il suffit d'expliquer brièvement le raisonnement suivi et de présenter clairement les calculs.}

\begin{center}
\begin{tabularx}{\linewidth}{|*{2}{>{\centering \arraybackslash}X|}}\hline 
\textbf{Figure 1} &\textbf{Figure 2}\\
\psset{unit=1cm}
\begin{pspicture}(5.75,3.)
%\psgrid
\pspolygon(0.5,0.5)(4,0.75)(1.6,2.3)%CAB
\uput[l](0.5,0.5){C} \uput[r](4,0.75){A} \uput[u](1.6,2.3){B} \uput[d](2.25,0.625){J}
\rput{-120}(1.6,2.3){\psframe(.3,.3)}
%\pscircle(2.25,0.625){1.78}
\psdots(2.25,0.625)
\psdots[dotstyle=+,dotangle=45](1.05,1.4)(1.375,0.575)(3.125,0.675) 
\rput(4.75,2.5){BC = 6~cm}
\end{pspicture}& \psset{unit=1cm}
\begin{pspicture}(5.75,3.)
\pspolygon(0.5,0.5)(4,0.5)(0.5,3)%ABC
\uput[ul](0.5,3){C} \uput[l](0.5,0.5){A} \uput[dr](4,0.5){B}
\uput[ur](2.25,1.75){36~cm}
\psarc(0.5,3){4mm}{-90}{-40}
\psframe(0.5,0.5)(0.8,0.8)
\rput(0.8,2.3){53\degres}
\end{pspicture}\\ \hline
\multicolumn{1}{|c}{\textbf{Figure 3}}&\multicolumn{1}{c|}{~}\\
\multicolumn{2}{|c|}{\psset{unit=1cm}\begin{pspicture}(-3,-1.5)(8,1.5)
%\psgrid
\pscircle(0,0){1.5}\psline(1.5;40)(1.5;220)
\uput[l](1.5;220){A} \uput[ur](1.5;40){B}
\uput[ul](0,0){O}
\psdots[dotstyle=+,dotangle=45](0.75;220)(0.75;40)
\psdots(0,0)
\uput[r](2.,0.5){[AB] est un diamètre du cercle de centre O.}
\uput[r](2.,-0.5){ La longueur du cercle est 154 cm.}
\end{pspicture}}\\ \hline
\end{tabularx}
\end{center}

\bigskip

\end{document}