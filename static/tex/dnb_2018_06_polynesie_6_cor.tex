\documentclass[10pt]{article}
\usepackage[T1]{fontenc}
\usepackage[utf8]{inputenc}%ATTENTION codage UTF8
\usepackage{fourier}
\usepackage[scaled=0.875]{helvet}
\renewcommand{\ttdefault}{lmtt}
\usepackage{amsmath,amssymb,makeidx}
\usepackage[normalem]{ulem}
\usepackage{diagbox}
\usepackage{fancybox}
\usepackage{tabularx,booktabs}
\usepackage{colortbl}
\usepackage{pifont}
\usepackage{multirow}
\usepackage{dcolumn}
\usepackage{enumitem}
\usepackage{textcomp}
\usepackage{lscape}
\newcommand{\euro}{\eurologo{}}
\usepackage{graphics,graphicx}
\usepackage{pstricks,pst-plot,pst-tree,pstricks-add}
\usepackage[left=3.5cm, right=3.5cm, top=3cm, bottom=3cm]{geometry}
\newcommand{\R}{\mathbb{R}}
\newcommand{\N}{\mathbb{N}}
\newcommand{\D}{\mathbb{D}}
\newcommand{\Z}{\mathbb{Z}}
\newcommand{\Q}{\mathbb{Q}}
\newcommand{\C}{\mathbb{C}}
\usepackage{scratch}
\renewcommand{\theenumi}{\textbf{\arabic{enumi}}}
\renewcommand{\labelenumi}{\textbf{\theenumi.}}
\renewcommand{\theenumii}{\textbf{\alph{enumii}}}
\renewcommand{\labelenumii}{\textbf{\theenumii.}}
\newcommand{\vect}[1]{\overrightarrow{\,\mathstrut#1\,}}
\def\Oij{$\left(\text{O}~;~\vect{\imath},~\vect{\jmath}\right)$}
\def\Oijk{$\left(\text{O}~;~\vect{\imath},~\vect{\jmath},~\vect{k}\right)$}
\def\Ouv{$\left(\text{O}~;~\vect{u},~\vect{v}\right)$}
\usepackage{fancyhdr}
\usepackage[french]{babel}
\usepackage[dvips]{hyperref}
\usepackage[np]{numprint}
%Tapuscrit : Denis Vergès
%\frenchbsetup{StandardLists=true}

\begin{document}
\setlength\parindent{0mm}
% \rhead{\textbf{A. P{}. M. E. P{}.}}
% \lhead{\small Brevet des collèges}
% \lfoot{\small{Polynésie}}
% \rfoot{\small{7 septembre 2020}}
\pagestyle{fancy}
\thispagestyle{empty}
% \begin{center}
    
% {\Large \textbf{\decofourleft~Brevet des collèges Polynésie 7 septembre 2020~\decofourright}}
    
% \bigskip
    
% \textbf{Durée : 2 heures} \end{center}

% \bigskip

% \textbf{\begin{tabularx}{\linewidth}{|X|}\hline
%  L'évaluation prend en compte la clarté et la précision des raisonnements ainsi que, plus largement, la qualité de la rédaction. Elle prend en compte les essais et les démarches engagées même non abouties. Toutes les réponses doivent être justifiées, sauf mention contraire.\\ \hline
% \end{tabularx}}

% \vspace{0.5cm}\textbf{Exercice 6 \hfill 14 points}

\medskip

\begin{enumerate}
\item Résultat 1 prend la valeur : $2 \times 3 + 3 = 6 + 3 = 9$, puis Résultat 1 prend la valeur : $9 \times 9 = 81$.

Résultat 2 prend la valeur $3 \times 3 = 9$, puis la valeur $9 \times 4 = 36$, puis la valeur $36 + 12 \times 3 = 36 + 36 = 72$ et enfin la valeur $72 + 9 = 81$.
\item
	\begin{enumerate}
		\item En remplaçant 3 par $x$, Résultat 1 prend la valeur : $2 \times x + 3 = 2x + 3$, puis Résultat 1 prend la valeur : $(2x + 3)\times (2x + 3) = (2x + 3)^2$.
		\item Résultat 2 prend la valeur $x \times x = x^2$, puis la valeur $x^2 \times 4 = 4x^2$, puis la valeur $4x^2 + 12 \times x = 4x^2 + 12x$ et enfin la valeur $4x^2 + 12x + 9$.
		\item On a vu dans la question précédente que pour un nombre choisi $x$, le Résultat 2 est 
		
$4x^2 + 12x + 9$.
		
Il faut donc trouver $x$ tel que :
		
$4x^2 + 12x + 9 = 9$, soit $4x^2 + 12x = 0$ ou en factorisant :
		
$4x(x + 3) = 0$ : il y a donc deux possibilités :
		
$x = 0$ ou $x + 3 = 0$, soit  $x = - 3$.
		
Conclusion : Alice a introduit $0$ ou $- 3$.
	\end{enumerate}
\end{enumerate}
\end{document}
\end{document}