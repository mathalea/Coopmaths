
\medskip

\begin{minipage}{9cm}
Pour gagner le gros lot à  une kermesse, il faut d'abord tirer une boule rouge dans une urne, puis obtenir un multiple de 3 en tournant une roue de loterie numérotée de 1 à  6.

L'urne contient 3 boules vertes, 2 boules bleues et 3 boules rouges.

\begin{enumerate}
\item Sur la roue de loterie, il y a deux issues (3 et 6) sur 6 issues qui réalisent l’évènement \og obtenir un multiple de $3$ \fg.

La probabilité d'obtenir un multiple de $3$ est donc égale à $\dfrac{2}{6}$ $\left( \text{ou~} \dfrac{1}{3}\right)$
\item Dans l’urne, la probabilité de tirer une boule rouge est égale à $\dfrac{3}{8}$.

la probabilité de tirer une boule rouge dans une urne, puis d’obtenir un multiple de 3 sur la roue de loterie est égale à $\dfrac{3}{8}\times\dfrac{1}{3}$, soit $\dfrac{1}{8}$.

La probabilité qu'un participant gagne le gros lot est égale à $\dfrac{1}{8}$.
\end{enumerate}
\end{minipage}
\hspace{0.5cm}\begin{minipage}{5cm}
\psset{unit=0.85cm}
\begin{pspicture}(-2.5,-2.5)(2.8,2.5)
\pscircle(0,0){2.5}
\multido{\n=0+60,\na=30+60,\nb=1+1}{6}{\psline(2.5;\n)\rput(1.5;\na){\nb}}
\rput{-35}(2.6;-35){$\blacktriangleleft$}
\end{pspicture}
\end{minipage}

\begin{enumerate}
\item[\textbf{3.}] Comme on ne change pas le nombre de boules vertes et de boules bleues, il y a 5 boules vertes ou bleues.

Il faut que la moitié des boules soient rouges, donc il faut mettre en tout 5 boules rouges dans l'urne pour que la probabilité de tirer une boule rouge soit de $0,5$.
\end{enumerate}


