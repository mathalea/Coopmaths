\textbf{Exercice 4 \hfill 5 points}

\medskip

Adèle et Mathéo souhaitent participer au marathon de Paris. Après s'être entraînés
pendant des mois, ils souhaitent évaluer leur état de forme avant de s'engager. Pour
cela, ils ont réalisé un test dit \og de Cooper\fg{} : l'objectif est de courir, sur une piste
d'athlétisme, la plus grande distance possible en 12 minutes. La distance parcourue
détermine la forme physique de la personne.

\medskip

\textbf{Document 1 : Indice de forme selon le test de Cooper}

\medskip

L'indice de forme d'un sportif dépend du sexe, de l'âge et de la distance parcourue
pendant les 12~min.

\begin{center}
Pour les hommes

\smallskip

\begin{tabularx}{\linewidth}{|*{5}{>{\footnotesize\centering \arraybackslash}X|}}\hline
Indice de Forme &Moins de 30 ans& De 30 à 39 ans& De 40 à 49 ans &Plus de 50 ans\\ \hline
Très faible& moins de \np{1600} m& moins de \np{1500} m& moins de \np{1350} m &moins de \np{1250} m\\ \hline
Faible& \np{1601} à \np{2000} m& \np{1501} à \np{1850} m& \np{1351} à \np{1700} m& \np{1251} à \np{1600} m\\ \hline
Moyen& \np{2001} à \np{2400} m& \np{1851} à \np{2250} m& \np{1701} à \np{2100} m& \np{1601} à \np{2000} m\\ \hline
Bon& \np{2401} à \np{2800} m& \np{2251} à \np{2650} m &\np{2101} à \np{2500} m &\np{2001} à \np{2400} m\\ \hline
Très bon& plus de \np{2800} m &plus de \np{2650} m &plus de \np{2500} m &plus de \np{2400} m\\ \hline
\end{tabularx}
\end{center}
\begin{center}
Pour les femmes

\smallskip
\begin{tabularx}{\linewidth}{|*{5}{>{\footnotesize\centering \arraybackslash}X|}}\hline
Indice de Forme &Moins de 30 ans &De 30 à 39 ans &De 40 à 49 ans &Plus de 50 ans\\ \hline
Très faible &moins de \np{1500} m &moins de \np{1350} m &moins de \np{1200} m &moins de \np{1100} m\\ \hline
Faible &\np{1501} à \np{1850} m &\np{1351} à \np{1700} m &\np{1201} à \np{1500} m &\np{1101} à \np{1350} m\\ \hline
Moyen &\np{1851} à \np{2150} m &\np{1701} à \np{2000} m &\np{1501} à \np{1850} m &\np{1351} à \np{1700} m\\ \hline
Bon &\np{2151} à \np{2650} m &\np{2001} à \np{2500} m &\np{1851} à \np{2350} m &\np{1701} à \np{2200} m\\ \hline
Très bon &plus de \np{2650} m &plus de \np{2500} m &plus de \np{2350} m &plus de \np{2200} m\\ \hline
\end{tabularx}
\end{center}
\begin{center}
\begin{tabularx}{\linewidth}{|m{7cm}|X|}\hline
Document 2 : Plan de la piste&Document 3 : Données du test\\
\psset{unit=0.8cm}
\begin{pspicture}(8,4)
\psline(1.5,0.5)(6.5,0.5)
\psline(1.5,3.5)(6.5,3.5)
\psline[linestyle=dashed](1.5,0.5)(1.5,3.5)
\psarc(6.5,2){1.5}{-90}{90}
\psarc(1.5,2){1.5}{90}{270}
\rput{90}(1.7,2){58 m}\uput[d](4,0.5){109 m}
\psdots[dotstyle=+,dotangle=45,dotscale=1.5](4,0.5)(4,3.5)
\end{pspicture}
Cette piste est composée de deux parties rectilignes et
de deux demi-cercles.&$\bullet~~$Adèle a 31 ans.

$\bullet~~$Mathéo a 27 ans.

$\bullet~~$Adèle a réalisé 6 tours de piste et 150 mètres.

$\bullet~~$Mathéo a réalisé le test avec une vitesse moyenne de
13,5 km/h.\\ \hline
\end{tabularx}
\end{center}

\smallskip

\begin{enumerate}
\item Vérifier que la longueur de la piste est d'environ $400$ mètres.
\item Adèle et Mathéo ont décidé de participer au marathon uniquement si leur
indice de forme est au moins au niveau \og moyen\fg.

Déterminer si Adèle et Mathéo participeront à la course.
\end{enumerate}

\bigskip

