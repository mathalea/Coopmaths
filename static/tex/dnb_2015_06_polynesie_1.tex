\textbf{Exercice 1 \hfill 3 points}

\medskip

Djamel et Sarah ont un jeu de société : pour y jouer, il faut tirer au hasard des jetons dans un sac. Tous les jetons ont la même probabilité d'être tirés. Sur chaque jeton un nombre entier est inscrit.

\smallskip

Djamel et Sarah ont commencé une partie. Il reste dans le sac les huit jetons suivants :

\[\fbox{5}\quad \fbox{14}\quad\fbox{26}\quad\fbox{18}\quad\fbox{5}\quad\fbox{9}\quad\fbox{18}\quad\fbox{20}\]

\begin{enumerate}
\item C'est à Sarah de jouer.
	\begin{enumerate}
		\item Quelle est la probabilité qu'elle tire un jeton \og 18 \fg{}?
		\item Quelle est la probabilité qu'elle tire un jeton multiple de 5 ?
	\end{enumerate}
\item  Finalement, Sarah a tiré le jeton \og 26 \fg{} qu'elle garde. C'est au tour de Djamel de jouer.
	
La probabilité qu'il tire un jeton multiple de 5 est-elle la même que celle trouvée à la question 1. b. ?

\end{enumerate}

\vspace{0,5cm}

