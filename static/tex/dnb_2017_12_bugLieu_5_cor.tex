\documentclass[10pt]{article}
\usepackage[T1]{fontenc}
\usepackage[utf8]{inputenc}%ATTENTION codage UTF8
\usepackage{fourier}
\usepackage[scaled=0.875]{helvet}
\renewcommand{\ttdefault}{lmtt}
\usepackage{amsmath,amssymb,makeidx}
\usepackage[normalem]{ulem}
\usepackage{diagbox}
\usepackage{fancybox}
\usepackage{tabularx,booktabs}
\usepackage{colortbl}
\usepackage{pifont}
\usepackage{multirow}
\usepackage{dcolumn}
\usepackage{enumitem}
\usepackage{textcomp}
\usepackage{lscape}
\newcommand{\euro}{\eurologo{}}
\usepackage{graphics,graphicx}
\usepackage{pstricks,pst-plot,pst-tree,pstricks-add}
\usepackage[left=3.5cm, right=3.5cm, top=3cm, bottom=3cm]{geometry}
\newcommand{\R}{\mathbb{R}}
\newcommand{\N}{\mathbb{N}}
\newcommand{\D}{\mathbb{D}}
\newcommand{\Z}{\mathbb{Z}}
\newcommand{\Q}{\mathbb{Q}}
\newcommand{\C}{\mathbb{C}}
\usepackage{scratch}
\renewcommand{\theenumi}{\textbf{\arabic{enumi}}}
\renewcommand{\labelenumi}{\textbf{\theenumi.}}
\renewcommand{\theenumii}{\textbf{\alph{enumii}}}
\renewcommand{\labelenumii}{\textbf{\theenumii.}}
\newcommand{\vect}[1]{\overrightarrow{\,\mathstrut#1\,}}
\def\Oij{$\left(\text{O}~;~\vect{\imath},~\vect{\jmath}\right)$}
\def\Oijk{$\left(\text{O}~;~\vect{\imath},~\vect{\jmath},~\vect{k}\right)$}
\def\Ouv{$\left(\text{O}~;~\vect{u},~\vect{v}\right)$}
\usepackage{fancyhdr}
\usepackage[french]{babel}
\usepackage[dvips]{hyperref}
\usepackage[np]{numprint}
%Tapuscrit : Denis Vergès
%\frenchbsetup{StandardLists=true}

\begin{document}
\setlength\parindent{0mm}
% \rhead{\textbf{A. P{}. M. E. P{}.}}
% \lhead{\small Brevet des collèges}
% \lfoot{\small{Polynésie}}
% \rfoot{\small{7 septembre 2020}}
\pagestyle{fancy}
\thispagestyle{empty}
% \begin{center}
    
% {\Large \textbf{\decofourleft~Brevet des collèges Polynésie 7 septembre 2020~\decofourright}}
    
% \bigskip
    
% \textbf{Durée : 2 heures} \end{center}

% \bigskip

% \textbf{\begin{tabularx}{\linewidth}{|X|}\hline
%  L'évaluation prend en compte la clarté et la précision des raisonnements ainsi que, plus largement, la qualité de la rédaction. Elle prend en compte les essais et les démarches engagées même non abouties. Toutes les réponses doivent être justifiées, sauf mention contraire.\\ \hline
% \end{tabularx}}

% \vspace{0.5cm}\textbf{Exercice 5 :  \hfill 9 points}

\medskip

\begin{enumerate}
\item 
	\begin{enumerate}
		\item Avec la formule $f(x) = 220 - x$, on remplace $x$ par 5.
		
$220-5 = 215$. La fréquence cardiaque maximale recommandée pour un enfant de $5$ ans est de 215~pulsations/minute.
		\item Avec la formule $g(x) = 208 - 0,7x$, on remplace $x$ par 5. 
		
$208- 0,7\times 5 = 208 - 3,5 = 204,5$. La fréquence cardiaque maximale recommandée pour un enfant de $5$ ans est de 204~pulsations/minute (on ne compte pas de demi-pulsation !).
	\end{enumerate}
\item  
	\begin{enumerate}
		\item Sur l'annexe 2, on complète le tableau de valeurs comme ci-dessous : \\[2mm]
		\begin{tabularx}{\linewidth}{|*{12}{>{\centering \arraybackslash}X|}}\hline
$x$		&5		&10		&20		&30		&40	&50		&60		&70		&80		&90		&100\\ \hline
$f(x)$	&215 	&210 	&200 	&190 	&180&170 	&160 	&150 	&140	&130 	&120 \\ \hline
$g(x)$	&204,5 	&201 	&194 	&187 	&180&173 	&166 	&159 	&152 	&145 	&138\\ \hline
\end{tabularx}
		\item Sur l'annexe 2, on a tracé en rouge la droite $d$ représentant la fonction $f$ dans le repère tracé.
		\item Sur le même repère, on a tracé en violet la droite $d'$ représentant la fonction $g$.
	\end{enumerate}
\item  Selon la nouvelle formule, à  partir de 40~ans la fréquence cardiaque maximale
recommandée est supérieure ou égale à  celle calculée avec l'ancienne formule. Ceci se voit dans le tableau : avant la colonne correspondant à 40~ans, $f(x)$ est supérieur à $g(x)$ et après cette colonne,  $f(x)$ est inférieur à $g(x)$.

Ceci se voit aussi sur la représentation graphique : avant le point d’intersection de $d$ et $d’$ correspondant à 40~ans, $d$ est au-dessus de $d’$ et après ce point,  $d$ est en-dessous de $d’$.
\item  L’exercice physique, pour une personne de $30$~ans, est le plus efficace lorsque la
fréquence cardiaque atteint 80\,\% de  187~pulsations/minute.

$\dfrac{80}{100}\times187=149,6$

Pour que l'exercice physique soit le plus efficace pour une personne de $30$~ans, la fréquence cardiaque doit être de 149 pulsations/minute (on ne compte pas 6 dixièmes de pulsation !).
\end{enumerate}

\vspace{0,5cm}

\end{document}