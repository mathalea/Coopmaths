\textbf{Exercice 3 \hfill 3 points}

\medskip

%Dans une classe de collège, après la visite médicale, on a dressé le tableau suivant : 
%
%\begin{center}
%\begin{tabularx}{0.8\linewidth}{|*{3}{>{\centering \arraybackslash}X|}}\hline
%&Porte des lunettes& Ne porte pas de lunettes\\ \hline 
%Fille &3 &15\\ \hline 
%Garçon &7 &5\\ \hline 
%\end{tabularx}
%\end{center} 
%
%Les fiches individuelles de renseignements tombent par terre et s'éparpillent.
%
%\medskip
 
\begin{enumerate}
\item %Si l'infirmière en ramasse une au hasard, quelle est la probabilité que cette fiche soit :
	\begin{enumerate}
		\item %celle d'une fille qui porte des lunettes ?
Il y a $3 + 15 + 7 + 5 = 30$ élèves  et parmi ceux-ci 3 filles qui portent des lunettes ; la probabilité est donc égale à $\dfrac{3}{30} = \dfrac{1}{10} = 0,1 = 10\,\%$. 
		\item %celle d'un garçon ? 
		Il y a 12 garçons, donc la probabilité est égale à $\dfrac{12}{30} = \dfrac{4}{10} = \dfrac{2}{5} = 0,4 = 40$\,\%.
	\end{enumerate} 
\item %Les élèves qui portent des lunettes dans cette classe représentent 12,5\,\% de ceux qui en portent dans tout le collège. Combien y a-t-il d'élèves qui portent des lunettes dans le collège ?
12,5\:\% correspondent à 10 élèves, donc 1\,\% correspond à $\dfrac{10}{12,5}$ et 100\,\% correspondent à $\dfrac{10}{12,5} \times 100 = 80$.

Il y a 80 élèves dans le collège qui portent des lunettes. 
\end{enumerate} 

\vspace{0,5cm}

