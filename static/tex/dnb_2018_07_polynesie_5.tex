
\medskip

\textbf{Les deux parties de cet exercice sont indépendantes.}

\medskip

Mario, qui dirige un centre de plongée sous-marine en pleine expansion, décide de
construire un bâtiment pour accueillir ses clients lors de la pause déjeuner. Celui-ci sera
constitué d'un rez-de-chaussée climatisé servant de réfectoire et d'un étage non climatisé
qui pourra être utilisé pour le stockage du matériel de plongée.

Pour finir d'établir son budget, il ne lui reste plus qu'à choisir un modèle de climatisation adapté et à calculer la quantité nécessaire de tuiles pour couvrir le toit de sa construction qu'il a schématisé ci-dessous.

\begin{center}
\begin{tabularx}{\linewidth}{|X|}\hline
\textbf{Document 1 :} Croquis réalisé par Mario.\\

\begin{center}
\psset{unit=0.9cm}
\begin{pspicture}(16,10)
%\psgrid
\psframe[linewidth=1.25pt](4.5,1)(9.5,4)%IKDF
\psframe(4.5,1)(4.7,1.2)\psframe(9.5,1)(9.3,1.2)
\psframe(4.5,4)(4.7,3.8)\psframe(9.5,4)(9.3,3.8)
\psline[linewidth=1.25pt](9.5,1)(15.5,3.1)(15.5,6.1)(9.5,4)%KLCD
%\psline[linewidth=1.25pt](15.5,6.1)
\psline[linewidth=1.25pt](7,4)(7,7.9)(13,10)(15.5,6.1)%EABC
\psline[linewidth=1.25pt](4.5,4)(7,7.9)(9.5,4)%FAD
\psframe(6,1)(8,2.8) \psline(7,1)(7,2.8)
\psline(12.5,2.05)(12.5,3.85)
\psarc(7,7.9){0.5}{-90}{-60}\rput(7.3,7){38\degres}
\psarc(7,7.9){0.5}{-120}{-90}\rput(6.7,7){38\degres}
\rput(2,2.4){FI = DK = CL = 2,70~m}
\rput(7,0.3){IK = FD = 5,06~m}
\psline{<->}(4.5,0.8)(9.5,0.8)
\psline{<->}(4.2,1)(4.2,4)
\psline{<->}(9.7,0.8)(15.6,2.9)
\psline(11.8,1.8)(11.8,3.6)(13.2,4.1)(13.2,2.3)
\rput{19}(12.5,1.5){KL = DC = AB = 13~m}
\uput[u](7,7.9){A} \uput[u](13,10){B} \uput[ur](15.5,6.1){C} \uput[dr](9.5,4){D} 
\uput[ul](4.5,4){F} \uput[dl](4.5,1){I} \uput[dr](9.45,1){K} \uput[r](15.5,3.1){L}
\uput[d](7,4){E}\psframe(7,4)(7.2,4.2)
\psline(5.7,3.9)(5.8,4.1)\psline(5.8,3.9)(5.9,4.1)
\psline(8.2,3.9)(8.3,4.1)\psline(8.3,3.9)(8.4,4.1)
\psline(7.15,7.7)(7.35,7.8)(7.25,8)
\end{pspicture}
\end{center}
\smallskip

Le croquis n'est pas réalisé à l'échelle.

Les deux pentes (ou versants) de la toiture forment un angle $\widehat{\text{FAD}}$ de mesure $76$\degres{}  qui est partagé en deux parties égales de $38$\degres.\\ \hline
\end{tabularx}
\end{center}
\smallskip

\begin{center}
\begin{tabularx}{\linewidth}{|X|}\hline
\textbf{Document 2 :} Tuiles plates choisies par Mario pour recouvrir son toit.\\

\parbox{0.5\linewidth}{\psset{unit=1cm}
\begin{center}\begin{pspicture}(4,3.9)
\pspolygon[fillstyle=solid,fillcolor=lightgray](0,0)(0,3.8)(2.8,3.9)(3.7,2.2)(3.15,2.18)(3.5,1.4)(3.05,1.35)(3.4,0.5)(0.4,0)
\psline[linewidth=1.25pt](2,3.9)(2.8,2.1)
\psline[linewidth=1.25pt](1,3.85)(1.8,2.)
\psline[linewidth=1.25pt](0,3.4)(0.65,1.9)
\psline[linewidth=1.25pt](0,1.9)(3.7,2.2)
\psline[linewidth=1.25pt](2.3,2.1)(2.6,1.3)
\psline[linewidth=1.25pt](1.15,1.95)(1.5,1.1)
\psline[linewidth=1.25pt](0,1.65)(0.3,0.9)
\psline[linewidth=1.25pt](2,1.2)(2.35,0.4)
\psline[linewidth=1.25pt](0.9,1)(1.2,0.2)
\psline[linewidth=1.25pt](3.5,1.4)(0,0.9)
\psline[linewidth=1.25pt](3.4,0.5)(0.35,0)
%\psgrid
\end{pspicture}
\end{center}}\hfill
\parbox{0.45\linewidth}{
Prévoir $26$ tuiles par m$^2$

\medskip

Prix : $0,65$ euro l'unité.}\\ \hline
\end{tabularx}
\end{center}


\begin{enumerate}
\item \textbf{PARTIE 1 :} Calcul du budget correspondant aux tuiles.
\smallskip

	\begin{enumerate}
		\item Calculer AD. \emph{Vous donnerez le résultat arrondi au centimètre près}.
		\item Calculer AE. \emph{Vous donnerez le résultat arrondi au centimètre près}.
		\item En déduire le prix des tuiles nécessaires à la couverture des deux pentes du toit.
 	\end{enumerate}
\item  \textbf{PARTIE 2 :} Choix d'un climatiseur adapté.
\smallskip

À l'aide des documents, faire un choix de climatiseur raisonné, adapté et le moins cher
possible pour climatiser le rez-de-chaussée du bâtiment, c'est dire à dire le réfectoire.
	
\begin{center}
\begin{tabularx}{\linewidth}{|X|}\hline	
\textbf{Document 3 :} Comment choisir un climatiseur ?\\[4pt]
Étape 1 : Connaître la puissance frigorifique nécessaire.\\[4pt]

Celle-ci dépend du volume des pièces à refroidir.

La puissance de froid s'exprime en BTU qui est une unité de mesure frigorifique.

Le tableau ci-dessous fait la correspondance entre le volume du bâtiment à refroidir et la
puissance en BTU nécessaire.

\begin{center}
\begin{tabular}{|m{2cm}|c|}\hline
Volume 		&Puissance frigorifique\\ \hline
100 m$^3$ 	&\np{12000} BTU\\ \hline
150 m$^3$ 	&\np{18000} BTU\\ \hline
250 m$^3$ 	&\np{25000} BTU\\ \hline
300 m$^3$ 	&\np{33000} BTU\\ \hline
350 m$^3$ 	&\np{41000} BTU\\ \hline
400 m$^3$ 	&\np{49000} BTU\\ \hline
450 m$^3$ 	&\np{56000} BTU\\ \hline
500 m$^3$ 	&\np{62000} BTU\\ \hline
\multicolumn{2}{r}{\hfill \footnotesize \emph{BTU : British Thermal Unit}}
\end{tabular}
\end{center}

\medskip

Étape 2 : Choisir le climatiseur le plus adapté.\\[4pt]

\begin{center}
\begin{tabular}{|m{3cm}|*{3}{>{\centering \arraybackslash}c|}}\hline
Modèle de différentes 
marques				&Type			& Puissance frigorifique&Prix T.T.C. en Euros\\ \hline
Freez 4000 			&monobloc 		&\np{15000} BTU &880\\ \hline
Freez 8000 			&monobloc 		&\np{22000} BTU &\np{1050}\\ \hline
Air 10 pingouin 	&Bi-split 		&\np{27000} BTU &990\\ \hline
Air 100 phoque 		&Bi-split 		&\np{39000} BTU &\np{1390}\\ \hline
Pôle Nord 500 		&Quadri-split	&\np{48000} BTU &\np{1180}\\ \hline
Laponglace			&Quadri-split 	&\np{50000} BTU &\np{2300}\\ \hline
Maxi Everest $+$	&Quadri-split 	&\np{53000} BTU &\np{1990}\\ \hline
Froid Extrême 2000	&Inverter 		&\np{55000} BTU &\np{2650}\\ \hline
\end{tabular}
\end{center}\\ \hline
\end{tabularx}
\end{center}
\end{enumerate}


