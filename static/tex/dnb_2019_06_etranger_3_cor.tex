\documentclass[10pt]{article}
\usepackage[T1]{fontenc}
\usepackage[utf8]{inputenc}%ATTENTION codage UTF8
\usepackage{fourier}
\usepackage[scaled=0.875]{helvet}
\renewcommand{\ttdefault}{lmtt}
\usepackage{amsmath,amssymb,makeidx}
\usepackage[normalem]{ulem}
\usepackage{diagbox}
\usepackage{fancybox}
\usepackage{tabularx,booktabs}
\usepackage{colortbl}
\usepackage{pifont}
\usepackage{multirow}
\usepackage{dcolumn}
\usepackage{enumitem}
\usepackage{textcomp}
\usepackage{lscape}
\newcommand{\euro}{\eurologo{}}
\usepackage{graphics,graphicx}
\usepackage{pstricks,pst-plot,pst-tree,pstricks-add}
\usepackage[left=3.5cm, right=3.5cm, top=3cm, bottom=3cm]{geometry}
\newcommand{\R}{\mathbb{R}}
\newcommand{\N}{\mathbb{N}}
\newcommand{\D}{\mathbb{D}}
\newcommand{\Z}{\mathbb{Z}}
\newcommand{\Q}{\mathbb{Q}}
\newcommand{\C}{\mathbb{C}}
\usepackage{scratch}
\renewcommand{\theenumi}{\textbf{\arabic{enumi}}}
\renewcommand{\labelenumi}{\textbf{\theenumi.}}
\renewcommand{\theenumii}{\textbf{\alph{enumii}}}
\renewcommand{\labelenumii}{\textbf{\theenumii.}}
\newcommand{\vect}[1]{\overrightarrow{\,\mathstrut#1\,}}
\def\Oij{$\left(\text{O}~;~\vect{\imath},~\vect{\jmath}\right)$}
\def\Oijk{$\left(\text{O}~;~\vect{\imath},~\vect{\jmath},~\vect{k}\right)$}
\def\Ouv{$\left(\text{O}~;~\vect{u},~\vect{v}\right)$}
\usepackage{fancyhdr}
\usepackage[french]{babel}
\usepackage[dvips]{hyperref}
\usepackage[np]{numprint}
%Tapuscrit : Denis Vergès
%\frenchbsetup{StandardLists=true}

\begin{document}
\setlength\parindent{0mm}
% \rhead{\textbf{A. P{}. M. E. P{}.}}
% \lhead{\small Brevet des collèges}
% \lfoot{\small{Polynésie}}
% \rfoot{\small{7 septembre 2020}}
\pagestyle{fancy}
\thispagestyle{empty}
% \begin{center}
    
% {\Large \textbf{\decofourleft~Brevet des collèges Polynésie 7 septembre 2020~\decofourright}}
    
% \bigskip
    
% \textbf{Durée : 2 heures} \end{center}

% \bigskip

% \textbf{\begin{tabularx}{\linewidth}{|X|}\hline
%  L'évaluation prend en compte la clarté et la précision des raisonnements ainsi que, plus largement, la qualité de la rédaction. Elle prend en compte les essais et les démarches engagées même non abouties. Toutes les réponses doivent être justifiées, sauf mention contraire.\\ \hline
% \end{tabularx}}

% \vspace{0.5cm}\textbf{\textsc{Exercice 3 \hfill 16 points}}

\medskip

\textbf{Partie I}

\begin{enumerate}
\item On trace un segment de longueur $4 \times 2 + 1 = 8 + 1 = 9$~cm. Par les deux extrémités de ce segment on trace deux arcs de cercle de rayon 9 (cm) qui se coupent au troisième sommet du triangle équilatéral.
\item
	\begin{enumerate}
		\item Le périmètre du rectangle est égal à :
		
		$2(L + l) = 2(4x + 1,5 + 2x) = 2(6x + 1,5) = 12x + 3$.
		\item Il faut résoudre l'équation :
		
		$12x + 3 = 18$ ou en ajoutant à chaque membre $- 3$ :
		
		$12x = 15$ soit $3 \times 4x = 3 \times 5$ et en simplifiant par 3 :
		
		$4x = 5$ et enfin en multipliant chaque membre par l'inverse de 4 :
		
		$\dfrac{1}{4} \times 4x = \dfrac{1}{4} \times 5$, d'où finalement :
		
		$x = \dfrac{5}{4}$
	\end{enumerate}
\item Le périmètre du triangle équilatéral est égal à : 

$3 \times (4x + 1) = 3 \times 4x + 3 \times 1 = 12x + 3$.

Quel que soit le nombre positif $x$, le triangle équilatéral et le rectangle ont le même périmètre.
\end{enumerate}

\textbf{Partie II}

A = 2 (on trace deux fois la longueur puis la largeur)

B = 90 (mesures des  angles d'un rectangle)

C = 3 (tracé des trois côtés)

D = 120 (mesure en degré des trois angles d'un triangle équilatéral : 60).

Le premier script trace le rectangle et le second le triangle équilatéral.

\vspace{0,5cm}

\end{document}