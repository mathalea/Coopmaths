\textbf{Exercice 1 \hfill 3 points}

\medskip

%Djamel et Sarah ont un jeu de société : pour y jouer, il faut tirer au hasard des jetons dans un sac. Tous les jetons ont la même probabilité d'être tirés. Sur chaque jeton un nombre entier est inscrit.

%\smallskip
%
%Djamel et Sarah ont commencé une partie. Il reste dans le sac les huit jetons suivants :
%
%\[\fbox{5}\quad \fbox{14}\quad\fbox{26}\quad\fbox{18}\quad\fbox{5}\quad\fbox{9}\quad\fbox{18}\quad\fbox{20}\]

\begin{enumerate}
\item %C'est à Sarah de jouer.
	\begin{enumerate}
		\item %Quelle est la probabilité qu'elle tire un jeton \og 18 \fg{}?
La probabilité que Sarah tire un jeton \og 18 \fg{} est de $\dfrac{2}{8}
 = \dfrac{1}{4} =  0,25$.
		\item %Quelle est la probabilité qu'elle tire un jeton multiple de 5 ?
Il y a 3 jetons multiples de 5, la probabilité que Sarah tire un jeton multiple de 5 est
donc de $\dfrac{3}{8} = 0,375$.
	\end{enumerate}
\item  %Finalement, Sarah a tiré le jeton \og 26 \fg{} qu'elle garde. C'est au tour de Djamel de jouer.
	
%La probabilité qu'il tire un jeton multiple de 5 est-elle la même que celle trouvée à la question 1. b. ?
Si Sarah garde le jeton tiré, il n’y a plus que $7$ jetons dans le sac dont $3$ multiples de
5, la probabilité que Djamel tire un jeton multiple de 5 est de
$\dfrac{3}{7} \ne \dfrac{3}{8}$.
\end{enumerate}

\vspace{0,5cm}

