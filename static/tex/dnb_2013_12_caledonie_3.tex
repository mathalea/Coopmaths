\textbf{Exercice 3 : Pizzeria FinBon \hfill 5 points}

\bigskip
 
Un restaurant propose cinq variétés de pizzas, voici leur carte :

\begin{center}
\renewcommand\arraystretch{1.9}
\begin{tabularx}{0.75\linewidth}{|p{3cm}X|}\hline 
\textbf{CLASSIQUE :}& tomate, jambon, oeuf, champignons\\ \hline 
\textbf{MONTAGNARDE :}& crème, jambon, pomme de terre, champignons\\ \hline 
\textbf{LAGON :}& crème, crevettes, fromage\\ \hline 
\textbf{BROUSSARDE :}& crème, chorizo, champignons, salami\\ \hline 
\textbf{PLAGE :}& tomate, poivrons, chorizo\\ \hline
\end{tabularx}
\renewcommand\arraystretch{1.9}
\end{center}
 
\begin{enumerate}
\item Je commande une pizza au hasard, quelle est la probabilité qu'il y ait des champignons dedans ? 
\item J'ai commandé une pizza à la crème, quelle est la probabilité d'avoir du jambon? 
\item Il est possible de commander une grande pizza composée à moitié d'une variété et à moitié d'une autre. Quelle est la probabilité d'avoir des champignons sur toute la pizza ? On pourra s'aider d'un arbre des possibles. 
\item On suppose que les pizzas sont de forme circulaire. La pizzeria propose deux tailles :

\setlength\parindent{8mm} 
\begin{itemize}
\item[$\bullet~~$] moyenne : 30~cm de diamètre 
\item[$\bullet~~$] grande  : 44~cm de diamètre.
\end{itemize}
\setlength\parindent{0mm} 
 
Si je commande deux pizzas moyennes, aurai-je plus à manger que si j'en commande une grande ? Justifier la réponse. 
\end{enumerate}

\bigskip

