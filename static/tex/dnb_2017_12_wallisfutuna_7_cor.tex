
\begin{minipage}{7.5cm}
\begin{enumerate}
\item Le triangle $ABC$ est rectangle en $A$, donc d’après le théorème de Pythagore :

$BC^2 = BA^2 + AC^2$

$BC^2 = 300^2 + 400^2$

$BC^2 = \np{90000} + \np{160000}$

$BC^2 = \np{250000}$ 

$BC=500$~m.
\end{enumerate}
\end{minipage}
\begin{minipage}{6cm}
\psset{unit=0.65cm,arrowsize=2pt 4}
\begin{pspicture}(10.5,5)
%\psgrid
\psline[ArrowInside=->](1.7,4)(0.5,2)(10.2,3.8)(8.5,0.5)
\psline[linestyle=dashed](1.7,4)(8.5,0.5)
\rput{-118}(1.7,4){\psframe(0.25,0.25)}
\rput{62}(8.5,0.5){\psframe(0.25,0.25)}
\uput[u](1.7,4){$A$ (départ)}\uput[dl](0.7,2){$B$}\uput[u](4.3,2.8){$C$}\uput[ur](10.2,3.8){$D$}
\uput[r](8.5,0.5){$E$ (arrivée)}
\rput(0.5,3.3){300~m}\rput(3.5,3.6){400~m}\rput(7.3,1.7){\np{1000}~m}
\end{pspicture}
\end{minipage}

\begin{enumerate}
\item[\textbf{2.}] Les triangles $ABC$ et $CDE$ ont deux angles de même mesure : l’angle droit et l’angle au sommet $C$, ils sont donc semblables.

Le triangle $CDE$ est un agrandissement du triangle $ABC$.

Si $k$ est le coefficient d’agrandissement, alors on a :

$\np{1000}=k\times 400$ \hspace{1cm} ; \hspace{1cm} $ED=k\times 300$ \hspace{1cm} et \hspace{1cm} $CD=k\times 500$

Avec la première égalité, on obtient $k=\dfrac{1~000}{400}$, soit $k=2,5$.

Avec la deuxième égalité, on obtient $ED=2,5\times 300$, soit $ED=750$~m. 
\item[\textbf{3.}] Avec la troisième égalité, on obtient $CD=2,5\times 500$, soit $CD=1~250$~m. 

$300+500+1~250+750=2~800$.

La longueur réelle du parcours $ABCDE$ est égale à \np{28000}~m.
\end{enumerate}



