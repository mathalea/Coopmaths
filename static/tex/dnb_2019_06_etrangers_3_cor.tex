
\medskip

\textbf{Partie I}

\begin{enumerate}
\item On trace un segment de longueur $4 \times 2 + 1 = 8 + 1 = 9$~cm. Par les deux extrémités de ce segment on trace deux arcs de cercle de rayon 9 (cm) qui se coupent au troisième sommet du triangle équilatéral.
\item
	\begin{enumerate}
		\item Le périmètre du rectangle est égal à :
		
		$2(L + l) = 2(4x + 1,5 + 2x) = 2(6x + 1,5) = 12x + 3$.
		\item Il faut résoudre l'équation :
		
		$12x + 3 = 18$ ou en ajoutant à chaque membre $- 3$ :
		
		$12x = 15$ soit $3 \times 4x = 3 \times 5$ et en simplifiant par 3 :
		
		$4x = 5$ et enfin en multipliant chaque membre par l'inverse de 4 :
		
		$\dfrac{1}{4} \times 4x = \dfrac{1}{4} \times 5$, d'où finalement :
		
		$x = \dfrac{5}{4}$
	\end{enumerate}
\item Le périmètre du triangle équilatéral est égal à : 

$3 \times (4x + 1) = 3 \times 4x + 3 \times 1 = 12x + 3$.

Quel que soit le nombre positif $x$, le triangle équilatéral et le rectangle ont le même périmètre.
\end{enumerate}

\textbf{Partie II}

A = 2 (on trace deux fois la longueur puis la largeur)

B = 90 (mesures des  angles d'un rectangle)

C = 3 (tracé des trois côtés)

D = 120 (mesure en degré des trois angles d'un triangle équilatéral : 60).

Le premier script trace le rectangle et le second le triangle équilatéral.

\vspace{0,5cm}

