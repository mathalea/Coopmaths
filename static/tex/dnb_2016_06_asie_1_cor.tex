\textbf{Exercice 1 \hfill 4 points}

\medskip

%\emph{Cet exercice est un questionnaire à choix multiple (QCM). Pour chaque ligne du
%tableau, trois réponses sont proposées, mais une seule est exacte.\\
%Toute réponse exacte vaut $1$ point.\\
%Toute réponse inexacte ou toute absence de réponse n'enlève pas de point.}
%
%Indiquez sur votre copie le numéro de la question et, sans justifier, recopier la
%réponse exacte (A ou B ou C).
%
%\begin{center}
%\begin{tabularx}{\linewidth}{|c|m{4cm}|*{3}{>{\centering \arraybackslash}X|}}\hline
%&&A &B &C\\ \hline
%\textbf{1.}&Dans une urne, il y a 10 boules rouges et 20 boules noires. La probabilité de tirer une boule rouge est :& $\dfrac{1}{2}$&$\dfrac{1}{3}$&$\dfrac{2}{3}$\\ \hline
%\textbf{2.}&\rule[-3mm]{0mm}{9mm} $(3x+2)^2 = \ldots$& $9x^2 + 4$& $3x^2 + 6x + 4$& $4 + 3x(3x + 4)$\\ \hline
%\textbf{3.}& Une solution de l'équation $x^2 - 2x - 8 = 0$ est :& 0& 3 &4\\ \hline
%\textbf{4.}& Si on double toutes les dimensions d'un aquarium, alors son volume est multiplié par:&2 &6 &8\\ \hline
%\end{tabularx}
%\end{center}
\begin{enumerate}
\item Il y a 10 boules rouges sur un total de $10 + 20 = 30$ boules ; la probabilité de tirer une rouge est donc de $\dfrac{10}{30} = \dfrac{1}{3}$. Réponse B.
\item $(3x + 2)^2 = (3x)^2 + 2^2 + 2 \times 3x\times 2 = 9x^2 + 4 + 12x = 4 + 9x^2 + 12x = 4 + 3x(3x + 4)$. Réponse C.
\item $0$ n'est pas solution ; 3 non plus car $3^2 - 2\times 3 - 8 = 3 - 8 = - 5 \ne 0$ ; reste 4. Or  $4^2 - 2 \times 4 - 8 = 16 - 8 - 8 = 0$ est vraie; Réponse C.
\item Chaque dimension est multipliée par 2, donc le volume est multiplié par $2 \times 2 \times 2 = 2^3 = 8$. Réponse C. 
\end{enumerate}

\bigskip

