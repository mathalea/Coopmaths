\documentclass[10pt]{article}
\usepackage[T1]{fontenc}
\usepackage[utf8]{inputenc}%ATTENTION codage UTF8
\usepackage{fourier}
\usepackage[scaled=0.875]{helvet}
\renewcommand{\ttdefault}{lmtt}
\usepackage{amsmath,amssymb,makeidx}
\usepackage[normalem]{ulem}
\usepackage{diagbox}
\usepackage{fancybox}
\usepackage{tabularx,booktabs}
\usepackage{colortbl}
\usepackage{pifont}
\usepackage{multirow}
\usepackage{dcolumn}
\usepackage{enumitem}
\usepackage{textcomp}
\usepackage{lscape}
\newcommand{\euro}{\eurologo{}}
\usepackage{graphics,graphicx}
\usepackage{pstricks,pst-plot,pst-tree,pstricks-add}
\usepackage[left=3.5cm, right=3.5cm, top=3cm, bottom=3cm]{geometry}
\newcommand{\R}{\mathbb{R}}
\newcommand{\N}{\mathbb{N}}
\newcommand{\D}{\mathbb{D}}
\newcommand{\Z}{\mathbb{Z}}
\newcommand{\Q}{\mathbb{Q}}
\newcommand{\C}{\mathbb{C}}
\usepackage{scratch}
\renewcommand{\theenumi}{\textbf{\arabic{enumi}}}
\renewcommand{\labelenumi}{\textbf{\theenumi.}}
\renewcommand{\theenumii}{\textbf{\alph{enumii}}}
\renewcommand{\labelenumii}{\textbf{\theenumii.}}
\newcommand{\vect}[1]{\overrightarrow{\,\mathstrut#1\,}}
\def\Oij{$\left(\text{O}~;~\vect{\imath},~\vect{\jmath}\right)$}
\def\Oijk{$\left(\text{O}~;~\vect{\imath},~\vect{\jmath},~\vect{k}\right)$}
\def\Ouv{$\left(\text{O}~;~\vect{u},~\vect{v}\right)$}
\usepackage{fancyhdr}
\usepackage[french]{babel}
\usepackage[dvips]{hyperref}
\usepackage[np]{numprint}
%Tapuscrit : Denis Vergès
%\frenchbsetup{StandardLists=true}

\begin{document}
\setlength\parindent{0mm}
% \rhead{\textbf{A. P{}. M. E. P{}.}}
% \lhead{\small Brevet des collèges}
% \lfoot{\small{Polynésie}}
% \rfoot{\small{7 septembre 2020}}
\pagestyle{fancy}
\thispagestyle{empty}
% \begin{center}
    
% {\Large \textbf{\decofourleft~Brevet des collèges Polynésie 7 septembre 2020~\decofourright}}
    
% \bigskip
    
% \textbf{Durée : 2 heures} \end{center}

% \bigskip

% \textbf{\begin{tabularx}{\linewidth}{|X|}\hline
%  L'évaluation prend en compte la clarté et la précision des raisonnements ainsi que, plus largement, la qualité de la rédaction. Elle prend en compte les essais et les démarches engagées même non abouties. Toutes les réponses doivent être justifiées, sauf mention contraire.\\ \hline
% \end{tabularx}}

% \vspace{0.5cm}\textbf{Exercice 7 :  \hfill 9 points}

\medskip

Pour soutenir la lutte contre l'obésité, un collège décide d'organiser une course.

\parbox{0.48\linewidth}{Un plan est remis aux élèves participant à l'épreuve.

Les élèves doivent partir du point A et se rendre
au point E en passant par les points B, C et D.

C est le point d'intersection des droites (AE) et
(BD)

La figure ci-contre résume le plan, elle n'est pas à
l'échelle.}\hfill
\parbox{0.47\linewidth}{\psset{unit=0.65cm,arrowsize=2pt 4}
\begin{pspicture}(10.5,5)
%\psgrid
\psline[ArrowInside=->](1.7,4)(0.5,2)(10.2,3.8)(8.5,0.5)
\psline[linestyle=dashed](1.7,4)(8.5,0.5)
\rput{-118}(1.7,4){\psframe(0.25,0.25)}
\rput{62}(8.5,0.5){\psframe(0.25,0.25)}
\uput[u](1.7,4){A (départ)}\uput[dl](0.7,2){B}\uput[u](4.3,2.8){C}\uput[ur](10.2,3.8){D}
\uput[r](8.5,0.5){E (arrivée)}
\rput(0.5,3.3){300~m}\rput(3.5,3.6){400~m}\rput(7.3,1.7){\np{1000}~m}
\end{pspicture}
}

\smallskip

On donne AC $= 400$~m, EC $= \np{1000}$~m et AB $= 300$~m.

\medskip

\begin{enumerate}
\item Calculer BC.
\item Montrer que ED $= 750$m.
\item Déterminer la longueur réelle du parcours ABCDE.
\end{enumerate}

\newpage

\begin{center}
{\Large \textbf{Annexe 1 : exercice 4}}

\vspace{3cm}

\begin{scratch}
\initmoreblocks{définir \namemoreblocks{Losange}}
\blockpen{stylo en position d'écriture}
\blockmove{avancer de \ovalnum{}}
\blockmove{tourner \turnleft{} de \ovalnum{30} degrés}
\blockmove{avancer de \ovalnum{}}
\blockmove{tourner \turnleft{} de \ovalnum{150} degrés}
\blockmove{avancer de \ovalnum{}}
\blockmove{tourner \turnleft{} de \ovalnum{} degrés}
\blockmove{avancer de \ovalnum{}}
\blockmove{tourner \turnleft{} de \ovalnum{} degrés}
\blockpen{relever le stylo}
\end{scratch}
\end{center}

\newpage

\begin{center}
{\Large \textbf{Annexe 2 : exercice 5}}

\bigskip

\begin{tabularx}{\linewidth}{|*{12}{>{\centering \arraybackslash}X|}}\hline
$x$		&5	&10	&20	&30	&40	&50	&60	&70	&80	&90	&100\\ \hline
$f(x)$	&	&	&	&	&	&	&	&	&	&	&\\ \hline
$g(x)$	&	&	&	&	&	&	&	&	&	&	&\\ \hline
\end{tabularx}

\bigskip

\psset{xunit =0.08cm,yunit=0.08cm}
\begin{pspicture}(-5,-5)(100,220)
\multido{\n=0+5}{21}{\psline[linecolor=cyan,linewidth=0.3pt](\n,0)(\n,220)}
\multido{\n=0+5}{45}{\psline[linecolor=cyan,linewidth=0.3pt](0,\n)(100,\n)}
\psaxes[linewidth=1.25pt,Dx=10,Dy=10,labelFontSize=\scriptstyle]{->}(0,0)(0,0)(100,220)
\psaxes[linewidth=1.25pt,Dx=10,Dy=10,labelFontSize=\scriptstyle](0,0)(0,0)(100,220)
\uput[r](0,217.5){fréquence cardiaque}
\uput[u](97,0){$x$}
\end{pspicture}
\end{center}
\end{document}\end{document}