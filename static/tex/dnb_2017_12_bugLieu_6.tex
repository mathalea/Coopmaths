\documentclass[10pt]{article}
\usepackage[T1]{fontenc}
\usepackage[utf8]{inputenc}%ATTENTION codage UTF8
\usepackage{fourier}
\usepackage[scaled=0.875]{helvet}
\renewcommand{\ttdefault}{lmtt}
\usepackage{amsmath,amssymb,makeidx}
\usepackage[normalem]{ulem}
\usepackage{diagbox}
\usepackage{fancybox}
\usepackage{tabularx,booktabs}
\usepackage{colortbl}
\usepackage{pifont}
\usepackage{multirow}
\usepackage{dcolumn}
\usepackage{enumitem}
\usepackage{textcomp}
\usepackage{lscape}
\newcommand{\euro}{\eurologo{}}
\usepackage{graphics,graphicx}
\usepackage{pstricks,pst-plot,pst-tree,pstricks-add}
\usepackage[left=3.5cm, right=3.5cm, top=3cm, bottom=3cm]{geometry}
\newcommand{\R}{\mathbb{R}}
\newcommand{\N}{\mathbb{N}}
\newcommand{\D}{\mathbb{D}}
\newcommand{\Z}{\mathbb{Z}}
\newcommand{\Q}{\mathbb{Q}}
\newcommand{\C}{\mathbb{C}}
\usepackage{scratch}
\renewcommand{\theenumi}{\textbf{\arabic{enumi}}}
\renewcommand{\labelenumi}{\textbf{\theenumi.}}
\renewcommand{\theenumii}{\textbf{\alph{enumii}}}
\renewcommand{\labelenumii}{\textbf{\theenumii.}}
\newcommand{\vect}[1]{\overrightarrow{\,\mathstrut#1\,}}
\def\Oij{$\left(\text{O}~;~\vect{\imath},~\vect{\jmath}\right)$}
\def\Oijk{$\left(\text{O}~;~\vect{\imath},~\vect{\jmath},~\vect{k}\right)$}
\def\Ouv{$\left(\text{O}~;~\vect{u},~\vect{v}\right)$}
\usepackage{fancyhdr}
\usepackage[french]{babel}
\usepackage[dvips]{hyperref}
\usepackage[np]{numprint}
%Tapuscrit : Denis Vergès
%\frenchbsetup{StandardLists=true}

\begin{document}
\setlength\parindent{0mm}
% \rhead{\textbf{A. P{}. M. E. P{}.}}
% \lhead{\small Brevet des collèges}
% \lfoot{\small{Polynésie}}
% \rfoot{\small{7 septembre 2020}}
\pagestyle{fancy}
\thispagestyle{empty}
% \begin{center}
    
% {\Large \textbf{\decofourleft~Brevet des collèges Polynésie 7 septembre 2020~\decofourright}}
    
% \bigskip
    
% \textbf{Durée : 2 heures} \end{center}

% \bigskip

% \textbf{\begin{tabularx}{\linewidth}{|X|}\hline
%  L'évaluation prend en compte la clarté et la précision des raisonnements ainsi que, plus largement, la qualité de la rédaction. Elle prend en compte les essais et les démarches engagées même non abouties. Toutes les réponses doivent être justifiées, sauf mention contraire.\\ \hline
% \end{tabularx}}

% \vspace{0.5cm}\textbf{Exercice 6 :  \hfill 7 points}

\medskip

Dans un laboratoire A, pour tester le vaccin contre la grippe de la saison hivernale prochaine, on a injecté la même souche de virus à 5 groupes comportant 29 souris chacun.

3 de ces groupes avaient été préalablement vaccinés contre ce virus.

Quelques jours plus tard, on remarque que :

\setlength\parindent{10mm}
\begin{itemize}
\item[$\bullet~~$] dans les $3$ groupes de souris vaccinées, aucune souris n'est malade ;
\item[$\bullet~~$] dans chacun des groupes de souris non vaccinées, $23$ souris ont développé la maladie.
\end{itemize}
\setlength\parindent{0mm} 

\medskip
 
\begin{enumerate}
\item 
	\begin{enumerate}
		\item En détaillant les calculs, montrer que la proportion de souris malades lors de ce test est $\dfrac{46}{145}$.
		\item Justifier sans utiliser la calculatrice pourquoi on ne peut pas simplifier cette fraction.
	\end{enumerate}	
\end{enumerate}
		
\textbf{Donnée utile} Le début de la liste ordonnée des nombres premiers est : 
		
		2,\: 3,\: 5,\: 7,\: 11,\: 13,\: 17,\: 19,\: 23,\:29.
		
Dans un laboratoire B on informe que $\dfrac{140}{870}$ des souris ont été malades.

\begin{enumerate}[resume]		
\item  
	\begin{enumerate}
		\item Décomposer $140$ et $870$ en produit de nombres premiers.
		\item En déduire la forme irréductible de la proportion de souris malades dans le laboratoire B.
	\end{enumerate}
\end{enumerate}

\vspace{0,5cm}

\end{document}