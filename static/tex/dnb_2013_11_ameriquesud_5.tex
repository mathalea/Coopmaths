\textbf{Exercice 5 \hfill 7 points}

\medskip

Un jeu\up{1} est constitué des dix étiquettes suivantes toutes identiques au toucher qui sont mélangées dans un sac totalement opaque.

\bigskip

{\footnotesize \begin{tabularx}{\linewidth} {|>{\centering \arraybackslash}X|m{0.5cm}|>{\centering \arraybackslash}X|}\cline{1-1}\cline{3-3}
Deux angles droits seulement &&Quatre angles droits\\\cline{1-1}\cline{3-3}
\multicolumn{3}{c}{~}\\\cline{1-1}\cline{3-3}
Côtés égaux deux à deux &&Deux côtés égaux seulement \\\cline{1-1}\cline{3-3}
\multicolumn{3}{c}{~}\\\cline{1-1}\cline{3-3}
Quatre côtés égaux &&Côtés opposés parallèles\\\cline{1-1}\cline{3-3} 
\multicolumn{3}{c}{~}\\\cline{1-1}\cline{3-3}
Deux côtés parallèles seulement &&Diagonales égales\\\cline{1-1}\cline{3-3} 
\multicolumn{3}{c}{~}\\\cline{1-1}\cline{3-3}
Diagonales qui se coupent en leur milieu &&Diagonales perpendiculaires\\\cline{1-1}\cline{3-3}
\end{tabularx}} 

\bigskip

\begin{enumerate}
\item On choisit au hasard une étiquette parmi les dix. 
	\begin{enumerate}
		\item Quelle est la probabilité de tirer l'étiquette \og Diagonales égales\fg ? 
		\item Quelle est la probabilité de tirer une étiquette sur laquelle est inscrit le mot \og diagonales \fg{} ? 
		\item Quelle est la probabilité de tirer une étiquette qui porte à la fois le mot \og côtés\fg et le mot \og diagonales \fg{} ? 
	\end{enumerate}
\item On choisit cette fois au hasard deux étiquettes parmi les dix et on doit essayer de dessiner un quadrilatère qui a ces deux propriétés. 
	\begin{enumerate}
		\item Madjid tire les deux étiquettes suivantes : 
		
\smallskip
{\footnotesize\begin{tabularx}{\linewidth} {|>{\centering \arraybackslash}X|m{0.5cm}|>{\centering \arraybackslash}X|}\cline{1-1}\cline{3-3}
Diagonales perpendiculaires &&Diagonales égales\\ \cline{1-1}\cline{3-3}
\end{tabularx}}
\smallskip

Julie affirme que la figure obtenue est toujours un carré. Madjid a des doutes. Qui a raison ? Justifier la réponse. 
		\item Julie tire les deux étiquettes suivantes :
		 
\smallskip
{\footnotesize\begin{tabularx}{\linewidth} {|>{\centering \arraybackslash}X|m{0.5cm}|>{\centering \arraybackslash}X|}\cline{1-1}\cline{3-3}
Côtés opposés parallèles &&Quatre côtés égaux \\ \cline{1-1}\cline{3-3}
\end{tabularx}}
\smallskip

Quel type de figure Julie est-elle sûre d'obtenir ? 
	\end{enumerate}
		\item Lionel tire les deux étiquettes suivantes :
		 
\smallskip
{\footnotesize \begin{tabularx}{\linewidth} {|>{\centering \arraybackslash}X|m{1.cm}|>{\centering \arraybackslash}X|}\cline{1-1}\cline{3-3}
Deux côtés égaux seulement &&Quatre angles droits\\ \cline{1-1}\cline{3-3}
\end{tabularx}}
\smallskip

Lionel est déçu. Expliquer pourquoi. 

\medskip

1 {\scriptsize D'après \og Géométrie à l'Ecole\fg de François Boule. Savoir dire et savoir-faire, IREM de Bourgogne.} 
\end{enumerate}

\bigskip

