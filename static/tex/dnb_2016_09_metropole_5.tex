\textbf{\textsc{Exercice 5} \hfill 5 points} 

\medskip

Sur un blog de couture, Archibald a trouvé une fiche technique pour tracer un
pentagramme (étoile à cinq branches).

Cette fiche technique est donnée en  \textbf{annexe} qui sera à rendre avec la copie.

\medskip

\begin{enumerate}
\item Compléter et terminer sur la \textbf{feuille annexe} la construction de l'étoile à cinq branches débutée par Archibald. On fera apparaître les points B,  D, J, M,  E, F,  G, H  et I.
\item Réécrire la troisième consigne sur la copie en utilisant le vocabulaire mathématique adapté.
\item En utilisant cette fiche technique,  Anaïs a obtenu la construction ci-dessous.
\begin{center}
\psset{unit=1cm}
\begin{pspicture}(-3,-3)(3,3)
\pscircle(0,0){3}
\pspolygon[fillstyle=solid,fillcolor=lightgray](3;45)(3;189)(3;333)(3;477)(3;621)
\end{pspicture}
\end{center}

Elle mesure les angles $\widehat{\text{EGI}}$ et $\widehat{\text{EHI}}$ et constate qu'ils sont égaux. Est-ce le cas pour tous les pentagrammes construits avec cette méthode ?
\end{enumerate}

\vspace{0,5cm}

