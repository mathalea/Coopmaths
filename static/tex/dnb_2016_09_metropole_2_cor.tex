\textbf{\textsc{Exercice 2} \hfill 6 points}

\medskip

\begin{enumerate}
\item $\text{IJ}^2 = 4^2 = 16$. D'autre part $\text{IK}^2 + \text{KJ}^2 = 3,2^2 + 2,4^2 = 16$.

Donc $\text{IJ}^2 = \text{IK}^2 + \text{KJ}^2$. L'égalité de Pythagore est vérifiée, donc IKJ est un triangle rectangle.
\item Les droites (KJ) et (LM) sont toutes les deux perpendiculaires à la même droite (IL), donc elles sont parallèles.

De plus, le point J appartient au segment [LM] et le point K appartient au segment [IL].

D'après la propriété de Thalès, on a :

$\dfrac{\text{IK}}{\text{IL}} =  \dfrac{\text{IJ}}{\text{IM}} =  \dfrac{\text{KJ}}{\text{LM}}$ soit  $\dfrac{3,2}{5} = \dfrac{4}{\text{IM}} =  \dfrac{2,4}{\text{LM}}$.

et donc LM $= \dfrac{2,4 \times 5}{3,2} = 3,75$~m.
\item On sait que le triangle KLM est rectangle en L.

D'après la propriété de Pythagore, on a : 

$\text{KM}^2 = \text{KL}^2 + \text{LM}^2 = 1,8^2 + 3,75^2 = \np{17,3025}$
et donc KM $=  \sqrt{\np{17,3025}} \approx  4,16$~m.
\end{enumerate}
 
\vspace{0,5cm}

