
\medskip

Les longueurs sont en pixels.

L'expression \og s'orienter à 90 \fg{} signifie que l'on s'oriente vers la droite.

\smallskip

On donne le programme suivant :

\begin{center}
\parbox{0.46\linewidth}{
%\begin{tabular}{m{7cm} m{7cm}}
\begin{scratch}[num blocks]
\blockinit{quand \greenflag est cliqué}
\blockmove{aller à x: \ovalnum 0 y: \ovalnum 0}
\blockpen{stylo en position d'écriture}
\blockmove{s'orienter à \ovalnum{90\selectarrownum} degrés}
\blockvariable{mettre \ovalvariable{Longueur} à 300}
\blockevent{Carré}
\blockevent{Triangle}
\blockmove{avancer de \ovalnum{Longueur} / \ovalnum{6}}
\blockvariable{mettre \ovalvariable{Longueur\selectarrownum} à \ovaloperator{}}
\blockevent{Carré}
\blockevent{Triangle}
\end{scratch}}
\hfill
\parbox{0.46\linewidth}{\begin{scratch}
\initmoreblocks{définir \namemoreblocks{Carré}}
   \blockrepeat{répéter \ovalnum{4} fois}
     {\blockmove{avancer de \ovalnum{Longueur}}
     \blockmove{tourner \turnleft{} de \ovalnum{90} degrés}
     }
\end{scratch}

\begin{scratch}
\initmoreblocks{définir \namemoreblocks{Triangle}}
   \blockrepeat{répéter \ovalnum{3} fois}
     {\blockmove{avancer de \ovalnum{Longueur}}
     \blockmove{tourner \turnleft{} de \ovalnum{120} degrés}
     }
\end{scratch}}
%\end{tabular}
\end{center}

\medskip

\begin{enumerate}
\item On prend comme échelle 1~cm pour $50$ pixels.
	\begin{enumerate}
		\item Représenter sur votre copie la figure obtenue si le programme est exécuté jusqu'à la ligne 7 comprise.
		\item Quelles sont les coordonnées du stylo après l'exécution de la ligne 8 ?
 	\end{enumerate}
\item  On exécute le programme complet et on obtient la figure ci-dessous qui possède un axe de symétrie vertical.

\begin{center}
\psset{unit=1cm}
\begin{pspicture}(4.6,4.6)
\psframe(4.6,4.6)
\psframe(0.75,0)(3.85,3.1)
\psline(0,0)(2.3,4)(4.6,0)
\psline(0.75,0)(2.3,2.65)(3.85,0)
\end{pspicture}
\end{center}
	
Recopier et compléter la ligne 9 du programme pour obtenir cette figure.
\item  
	\begin{enumerate}
		\item Parmi les transformations suivantes, translation, homothétie, rotation, symétrie axiale, quelle est la transformation géométrique qui permet d'obtenir le petit carré à partir du grand carré ? Préciser le rapport de réduction.
		\item Quel est le rapport des aires entre les deux carrés dessinés ?
	\end{enumerate}
\end{enumerate}

\bigskip

