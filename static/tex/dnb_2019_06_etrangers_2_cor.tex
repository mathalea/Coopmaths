
\medskip

\begin{enumerate}
\item On obtient successivement :

$1 \to 1^2 = 1 \to 1 + 3 \times 1 = 1 + 3 = 4 \to 4 + 2 = 6$. 
\item De même en partant de $- 5$ : 

$- 2 \to (- 5)^2 = 25 \to 25 + 3 \times (- 5) = 25 - 15 = 10 \to 10 + 2 = 12$.
\item En partant de $x$, on obtient :

$x \to x^2 \to x^2 + 3x \to x^2 + 3x + 2$.
\item On a quel que soit le nombre $x$ : 

$(x + 2)(x + 1) = x^2 + x + 2x + 2 = x^2 + 3x + 2$, donc inversement, quel que soit le nombre $x$ :

$x^2 + 3x + 2 = (x + 1)(x + 2)$.
\item
	\begin{enumerate}
		\item La formule est =(B1 + 2)*(B1 + 1)
		\item Il faut trouver les nombres $x$ tels que $(x + 2)(x + 1) = 0$ ; or un produit est nul si l'un de ses facteurs est nul, soit :
		
		$\left\{\begin{array}{l c l}
		x + 2&=&0 \:\text{ou}\\
		x + 1&=&0
		\end{array}\right.$ ou encore $\left\{\begin{array}{l c l}
		x &=&- 2\:\text{ou}\\
		x &=&- 1
		\end{array}\right.$
		
Si l'on part de $- 1$ ou de $- 2$, le programme donne $0$.
	\end{enumerate}
\end{enumerate}

\vspace{0,5cm}

