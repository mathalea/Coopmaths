\textbf{Exercice 1 \hfill 6 points}

\medskip

%Ceci est un questionnaire à choix multiples (Q.C.M).
%
%Pour chaque énoncé, une seule réponse est exacte. Aucune justification n'est demandée.
%
%Entourer la bonne réponse.

\begin{center}
\renewcommand\arraystretch{1.8}
\begin{tabularx}{\linewidth}{|m{4cm}|*{4}{>{\centering \arraybackslash}X|}}\hline
$\dfrac{2}{9} + \dfrac{5}{9}$ est égal à&$\dfrac{10}{18}$&\fbox{$\dfrac{7}{9}$}&$\dfrac{7}{18}$&
$\dfrac{10}{81}$\rule[-4mm]{0mm}{9mm}\\ \hline
$\dfrac{2}{3}\times \dfrac{8}{5}$ est égal à&$\dfrac{10}{8}$&$\dfrac{16}{8}$&\fbox{$\dfrac{16}{15}$}&$\dfrac{2}{3}$\rule[-4mm]{0mm}{9mm}\\ \hline
Si $a = 10,\: b = - 3$ et $c = 2$ alors $a + b \times c$ est égal à&14&16&\fbox{4}&$- 28$\\ \hline
6\,\% de \np{1900} est égal à& \np{11400} &\fbox{114} &0,06 &11,4\\ \hline
\end{tabularx}
\renewcommand\arraystretch{1}
\end{center}

\bigskip

