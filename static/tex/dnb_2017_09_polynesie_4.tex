
\medskip

La figure ci-dessous représente le plan de coupe d'une tribune d'un gymnase. Pour
voir le déroulement du jeu, un spectateur du dernier rang assis en C doit regarder au-dessus
du spectateur placé devant lui et assis en D. Une partie du terrain devant la
tribune lui est alors masquée. On considèrera que la hauteur moyenne d'un spectateur
assis est de $80$~cm ($\text{CT} = \text{DS} = 80$~cm).

\begin{center}
\psset{unit=0.9cm}
\begin{pspicture}(12,7)
%\psgrid
\pspolygon(3,0)(11.6,0)(11.6,4.5)
\psarc(3,0){8mm}{0}{28}\rput(4.2,0.2){30\degres}
\uput[u](7.3,0){11 m}
\psline[linestyle=dashed](11.6,4.5)(11.6,6.1)(0,0)(3,0)
\psline[linestyle=dashed](10.3,4.5)(10.3,5.4)
\psarc(0,0){8mm}{0}{28}
\psline(3,0)(3,0.7)(4.4,0.7)
\psline(7.5,2.4)(7.5,3.1)(8.9,3.1)
\psline(8.9,3.1)(8.9,3.8)(10.3,3.8)
\psline(10.3,3.8)(10.3,4.5)(11.6,4.5)
\uput[d](3,0){A} \uput[d](11.6,0){B}\uput[r](11.6,4.5){C}
\uput[d](10.3,3.8){D}\uput[d](0,0){R}\uput[r](11.6,6.1){T}
\uput[u](10.1,5.35){S}
\rput(11,4.7){80 cm}\uput[r](11.6,5.2){80 cm}
\psframe(11.6,0)(11.3,0.3)
\psline[linestyle=dotted](4.5,1.1)(6.8,2.3)
\end{pspicture}
\end{center}

Sur ce plan de coupe de la tribune :

\setlength\parindent{9mm}
\begin{itemize}
\item[$\bullet~~$] les points R, A et B sont alignés horizontalement et les points B, C et T sont alignés verticalement ;
\item[$\bullet~~$] les points R, S et T sont alignés parallèlement à l'inclinaison (AC) de la tribune ;
\item[$\bullet~~$] on considérera que la zone représentée par le segment [RA] n'est pas visible par le spectateur du dernier rang ;
\item[$\bullet~~$] la largeur au sol AB de la tribune est de 11~m et l'angle 
$\widehat{\text{BAC}}$ d'inclinaison de la tribune mesure 30\degres.
\end{itemize}

\medskip

\begin{enumerate}
\item Montrer que la hauteur BC de la tribune mesure $6,35$~m, arrondie au centième
de mètre près.
\item Quelle est la mesure de l'angle $\widehat{\text{BRT}}$ ?
\item Calculer la longueur RA en centimètres. Arrondir le résultat au centimètre près.
\end{enumerate}

\vspace{0,5cm}

