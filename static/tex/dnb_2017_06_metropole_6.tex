
\medskip

\textbf{Document \no 1}

\begin{tabularx}{\linewidth}{|X|}\hline
Le surpoids est devenu un problème majeur de santé, celui-ci prédispose à beaucoup de maladies et diminue l'espérance de vie.\\
L' indice le plus couramment utilisé est celui de masse corporelle (IMC).\\ \hline
\end{tabularx}

\medskip

\textbf{Document \no 2}

\begin{tabularx}{\linewidth}{|X|}\hline
L'IMC est une grandeur internationale permettant de déterminer la corpulence d'une personne adulte entre
18 ans et 65 ans.\\
Il se calcule avec la formule suivante : IMC $= \dfrac{\text{masse}}{\text{taille}^2}$
 avec \og masse \fg{} en kg et \og taille \fg{} en m.\\
Normes : $18,5 \leqslant  \text{IMC} < 25$ corpulence normale\\
$25 \leqslant IMC < 30$  surpoids\\
$\text{IMC} > 30$ obésité\\ \hline
\end{tabularx}

\medskip

\begin{enumerate}
\item  Dans une entreprise, lors d'une visite médicale, un médecin calcule l'IMC de six des employés.

Il utilise pour cela une feuille de tableur dont voici un extrait :

\begin{center}
\begin{tabularx}{\linewidth}{|c|c|*{6}{>{\centering \arraybackslash}X|}}\hline
	&A				&B 		&C 		&D 		&E 		&F 		&G\\ \hline
1	&Taille (en m) 	&1,69 	&1,72 	&1,75 	&1,78 	&1,86 	&1,88\\ \hline
2	&Masse (en kg) 	&72 	&85 	&74 	&70 	&115 	&85\\ \hline
3	&IMC (*)		&25,2 	&28,7 	&24,2 	&22,1 	&33,2 	&24,0\\ \hline
4	& \multicolumn{7}{|l|}{(*)  \emph{valeur approchée au dixième}}\\ \hline
\end{tabularx}
\end{center}

	\begin{enumerate}
		\item Combien d'employés sont en situation de surpoids ou d'obésité dans cette entreprise ?
		\item Laquelle de ces formules a-t-on écrite dans la cellule B3, puis recopiée à droite, pour calculer l'IMC ?
		
Recopier la formule correcte sur la copie.

\medskip
\begin{tabularx}{\linewidth}{*{4}{X}}		
\fbox{$=72/1,69 \:\:\hat\:\: 2$}& \fbox{= B1/ (B2 * B2)}& \fbox{ = B2/ (B1 * B1) }&\fbox{= \$B2/ (\$B1*\$B1)}
\end{tabularx}
\medskip

 	\end{enumerate}
\item Le médecin a fait le bilan de l'IMC de chacun des 41 employés de cette entreprise. Il a reporté les
informations recueillies dans le tableau suivant dans lequel les IMC ont été arrondis à l'unité près.
\begin{center}
\begin{tabularx}{\linewidth}{|c|*{9}{>{\centering \arraybackslash}X|}}\hline
IMC &20 &22 &23 &24 &25 &29 &30 &33 &Total\\ \hline
Effectif &9 &12&6 &8 &2 &1 &1 &2 &41\\ \hline
\end{tabularx}
\end{center}
	\begin{enumerate}
		\item Calculer une valeur approchée, arrondie à l'entier près, de l'IMC moyen des employés de cette
entreprise.
		\item Quel est l'IMC médian ? Interpréter ce résultat.
		\item On lit sur certains magazines : \og On estime qu'au moins 5\,\% de la population mondiale est en surpoids ou est obèse \fg. Est-ce le cas pour les employés de cette entreprise ?
	\end{enumerate}
\end{enumerate}

\vspace{0,5cm}

