\textbf{\textsc{Exercice 7} \hfill 5 points}

\medskip

On étudie plus précisément le remplissage d'une écluse pour faire passer une péniche de l'amont vers l'aval. 

\textbf{Principe :} Il s'agit de faire monter le niveau de l'eau dans l'écluse jusqu'au niveau du canal en amont afin que l'on puisse ensuite faire passer la péniche dans l'écluse. 

Ensuite, l'écluse se vide et le niveau descend à celui du canal en aval. La péniche peut sortir de l'écluse et poursuivre dans le canal en aval. 

\begin{center}
\psset{unit=0.75cm}
\begin{pspicture}(15,9)
%\psgrid
\def\vague{\pscurve(0,0.2)(0.25,0.25)(0.5,0)(0.75,-0.05)(1,-0.2)}
\psframe*(0.5,3.7)(5.5,3.8)
\psframe[fillstyle=solid,fillcolor=lightgray](5.5,1.2)(6,6)
\psframe[fillstyle=solid,fillcolor=lightgray](10.6,1.2)(11.1,6)
\psframe*(5.5,1.1)(15,1.2)
\psline[linewidth=1.25pt]{<->}(2.7,1.2)(2.7,5.4)
\psline[linewidth=1.25pt]{<->}(4.9,1.2)(4.9,2.3)\uput[l](4.9,1.7){$x$}
\psline(0.9,5.45)(0.5,6.1)(5.1,6.1)(4.7,5.4)%bateau
\psline(0.8,6.1)(0.8,6.5)(2.45,6.5)(2.6,6.1)%haut du bateau
\pscurve(0,5.4)(2,5.5)(4,5.4)(5.6,5.5)%ligne sous bateau
\pscurve(6,2.3)(8,2.2)(10,2.3)(12,2.2)(14,2.35)(15,2.3)% mer sous vantelles
\psframe[fillstyle=solid,fillcolor=lightgray](10.6,1.2)(11.1,6)
\psframe[fillstyle=solid,fillcolor=gray](5.3,4)(6.2,4.5)
\psframe[fillstyle=solid,fillcolor=gray](10.5,1.6)(11.3,2)
\psline(5.6,7.2)(5.6,8)
\psline(11.1,7.2)(11.1,8)
\psline{<->}(5.6,7.7)(11.1,7.7)
\psline{->}(2.7,1.2)(2.7,5.4)\uput[r](2.7,3.3){$h$}
\rput{30}(1.3,4.5){\vague}\rput{30}(3.3,4.5){\vague}\rput{30}(1.3,4){\vague}
\rput{30}(4.3,4){\vague}
\rput(2.6,8.5){Amont} 
\rput(8.2,8){Écluse} 
\rput(13.2,8){Aval} 
\rput(14.3,6.8){Portes}\psline{->}(14.3,6.6)(5.8,5.3)\psline{->}(14.3,6.6)(10.8,5.3) 
\rput(9.6,0.1){radier}\psline{->}(9,0.3)(8,1.1)  
\rput(13.8,3.5){vantelles}\psline{->}(12.9,3.3)(5.8,4.3)\psline{->}(12.9,3.3)(11,1.8)
\multido{\r=1.0+0.6}{6}{\rput(\r,5.8){\pscircle(0,0){3pt}}}
\rput{30}(6.3,1.8){\vague} \rput{30}(9,1.8){\vague}\rput{30}(13,1.8){\vague}
\rput{30}(8,1.2){\vague}\rput{30}(12.3,1.2){\vague}
\psline[linestyle=dotted](2,1.2)(5.6,1.2)
\psline[linestyle=dotted](4.7,2.3)(5.5,2.3)
\end{pspicture}
\end{center}

Toutes les mesures de longueur sont exprimées en mètres. 

On notera $h$ la hauteur du niveau de l'eau en amont et $x$ la hauteur du niveau de l'eau dans l'écluse. 

Ces hauteurs sont mesurées à partir du radier (fond) de l'écluse. (voir schéma ci-dessus). Lorsque la péniche se présente à l'écluse, on a : $h = 4,3$~m et $x = 1,8$~m.
 
La vitesse de l'eau s'écoulant par la vantelle (vanne) est donnée par la formule suivante : 

\[v = \sqrt{2g(h - x)}\]
 
où $g = 9,81$ (accélération en mètre par seconde au carré noté m.s$^{-2}$) et $v$ est la vitesse (en mètre par seconde noté m.s$^{-1}$) 

\medskip
 
\begin{enumerate}
\item \textit{Calculons l'arrondi à l'unité de la vitesse de l'eau s'écoulant par la vantelle à l'instant de son ouverture :}

 $v = \sqrt{2\times 9,81~\text{m.s}^{-2}\times(4,3~\text{m} - 1,8~\text{m})}=\sqrt{49,05~\text{m}^ 2~\text{s}^{-2}}$\fbox{$\approx 7~\text{m.s}^{-1}$} arrondi à l'unité

%(On considère l'ouverture comme étant instantanée). 
\item \textit{La vitesse d'écoulement de l'eau sera nulle lorsque $x=h$.} 

\textit{Dans ce cas, le niveau de l'eau dans l'écluse est le m\^eme qu'en amont.}
\item Le graphique donné en annexe 2 représente la vitesse d'écoulement de l'eau par la vantelle en fonction du niveau $x$ de l'eau dans l'écluse. 

\textit{Par lecture graphique, la vitesse d'écoulement est de \fbox{$4$ m/s} lorsque la hauteur de l'eau dans l'écluse est de $3,4$~m} (voir les pointillés*)\flushright{\scriptsize *d'où la celèbre maxime : <<Point de pointillés, point de point y est !>>[NDLR]}
\end{enumerate}
 
\vspace{0,5cm}

