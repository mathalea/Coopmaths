
\medskip
 
Voici deux programmes de calcul : 

\medskip

%\begin{tabularx}{\linewidth}{X| X}
%\multicolumn{1}{c|}{Programme de calcul \ding{192}} &\multicolumn{1}{c}{Programme de calcul \ding{193}}\\
%$\bullet~~$Soustraire 5 &$\bullet~~$ Multiplier par 6\\
%$\bullet~~$ Multiplier par 4 &$\bullet~~$ Soustraire 20\\
%&$\bullet~~$ Soustraire le double du nombre de départ\\
%\end{tabularx}
%
%\medskip

\begin{enumerate}
\item 
	\begin{enumerate}
		\item %Quel résultat obtient-on quand on applique le programme de calcul \ding{192} au nombre 3 ?
		On obtient $3 \to - 2 \to - 8$.
		\item %Quel résultat obtient-on quand on applique le programme de calcul \ding{193} au nombre 3 ?
On obtient $3 \to 18 /to ) 2 \to - 8$
	\end{enumerate}
\item %Démontrer qu'en choisissant le nombre $- 2$, les deux programmes donnent le même résultat.
Avec le programme de calcul \ding{192} on obtient $- 2 \to - 7 \to - 28$ ;

Avec le programme de calcul \ding{193} on obtient $- 2 \to - 12 \to - 32 \to - 28$ 
\item %On décide de réaliser davantage d'essais. Pour cela, on utilise un tableur et on obtient la copie d'écran suivante :

%\begin{center}
%\begin{tabularx}{\linewidth}{|c|*{3}{>{\centering\arraybackslash}X|}p{1.2cm}|}\hline
%\multicolumn{2}{|c|}{A6}&	&4	&\\ \hline
%&A&B&C&D\\ \hline
%1			&Nombre choisi	&Résultat avec le programme \ding{192}&Résultat avec le programme \ding{193}&\\ \hline
%2			&0				&$- 20$		&$- 20$						&\\ \hline
%3			&1				&$- 16$		&$-16$						&\\ \hline
%4			&2				&$- 12$		&$- 12$						&\\ \hline
%5			&3				&$- 8$		&$- 8$						&\\ \hline
%6			&4				&			&							&\\ \hline
%7			&5				&			&							&\\ \hline
%8			&6				&			&							&\\ \hline
%\end{tabularx}
%\end{center}

%Quelle formule a-t-on pu saisir dans la cellule B2 avant de la recopier vers le bas, jusqu'à la cellule B5 ?
Dans la case B2 : =4*(A2 - 5)
\item %Les résultats affichés dans les colonnes B et C sont égaux. Lucie pense alors que, pour n'importe quel nombre choisi au départ, les deux programmes donnent toujours le même résultat.

%Démontrer que Lucie a raison.
À partir du nombre $x$ le programme \ding{192} donne $4(x - 5)$.

À partir du nombre $x$ le programme \ding{193} donne $6x - 20 - 2x$.

Or $4(x - 5) = 4x - 20$ et $6x - 20 - 2x = 4x - 20$.

Les deux programmes conduisent donc à chaque fois au même résultat.
\end{enumerate}

\medskip

