\documentclass[10pt]{article}
\usepackage[T1]{fontenc}
\usepackage[utf8]{inputenc}%ATTENTION codage UTF8
\usepackage{fourier}
\usepackage[scaled=0.875]{helvet}
\renewcommand{\ttdefault}{lmtt}
\usepackage{amsmath,amssymb,makeidx}
\usepackage[normalem]{ulem}
\usepackage{diagbox}
\usepackage{fancybox}
\usepackage{tabularx,booktabs}
\usepackage{colortbl}
\usepackage{pifont}
\usepackage{multirow}
\usepackage{dcolumn}
\usepackage{enumitem}
\usepackage{textcomp}
\usepackage{lscape}
\newcommand{\euro}{\eurologo{}}
\usepackage{graphics,graphicx}
\usepackage{pstricks,pst-plot,pst-tree,pstricks-add}
\usepackage[left=3.5cm, right=3.5cm, top=3cm, bottom=3cm]{geometry}
\newcommand{\R}{\mathbb{R}}
\newcommand{\N}{\mathbb{N}}
\newcommand{\D}{\mathbb{D}}
\newcommand{\Z}{\mathbb{Z}}
\newcommand{\Q}{\mathbb{Q}}
\newcommand{\C}{\mathbb{C}}
\usepackage{scratch}
\renewcommand{\theenumi}{\textbf{\arabic{enumi}}}
\renewcommand{\labelenumi}{\textbf{\theenumi.}}
\renewcommand{\theenumii}{\textbf{\alph{enumii}}}
\renewcommand{\labelenumii}{\textbf{\theenumii.}}
\newcommand{\vect}[1]{\overrightarrow{\,\mathstrut#1\,}}
\def\Oij{$\left(\text{O}~;~\vect{\imath},~\vect{\jmath}\right)$}
\def\Oijk{$\left(\text{O}~;~\vect{\imath},~\vect{\jmath},~\vect{k}\right)$}
\def\Ouv{$\left(\text{O}~;~\vect{u},~\vect{v}\right)$}
\usepackage{fancyhdr}
\usepackage[french]{babel}
\usepackage[dvips]{hyperref}
\usepackage[np]{numprint}
%Tapuscrit : Denis Vergès
%\frenchbsetup{StandardLists=true}

\begin{document}
\setlength\parindent{0mm}
% \rhead{\textbf{A. P{}. M. E. P{}.}}
% \lhead{\small Brevet des collèges}
% \lfoot{\small{Polynésie}}
% \rfoot{\small{7 septembre 2020}}
\pagestyle{fancy}
\thispagestyle{empty}
% \begin{center}
    
% {\Large \textbf{\decofourleft~Brevet des collèges Polynésie 7 septembre 2020~\decofourright}}
    
% \bigskip
    
% \textbf{Durée : 2 heures} \end{center}

% \bigskip

% \textbf{\begin{tabularx}{\linewidth}{|X|}\hline
%  L'évaluation prend en compte la clarté et la précision des raisonnements ainsi que, plus largement, la qualité de la rédaction. Elle prend en compte les essais et les démarches engagées même non abouties. Toutes les réponses doivent être justifiées, sauf mention contraire.\\ \hline
% \end{tabularx}}

% \vspace{0.5cm}\textbf{\textsc{Exercice 6 \hfill 22 points}}

\medskip

%Dans le village de Jean, une brocante est organisée chaque année lors du premier week-end de juillet. Jean s'est engagé à s'occuper du stand de vente de frites. Pour cela, il fabrique des cônes en papier qui lui serviront de barquette pour les vendre. 
%
%Dans le fond de chaque cône, Jean versera de la sauce: soit de la mayonnaise, soit de la sauce tomate.
%
%\medskip
% 
%Il décide de fabriquer $400$ cônes en papier et il doit estimer le nombre de bouteilles de mayonnaise et de sauce tomate à acheter pour ne pas en manquer. 
%
%Voici les informations dont Jean dispose pour faire ses calculs : 
%
%\parbox{0.4\linewidth}{\textbf{Le cône de frites }: 
%
%\begin{center}
%\psset{unit=1cm}
%\begin{pspicture}(-2,0)(2,6)
%\psellipse(0,5.4)(1.8,0.35)
%\scalebox{.99}[0.3]{\psarc[linewidth=1.5pt](0,6.9){0.7}{180}{0}}%
%\scalebox{.99}[0.3]{\psarc[linewidth=1.5pt,linestyle=dashed](0,6.9){0.7}{0}{180}}%
%\psline(-1.8,5.4)(0,0)(1.8,5.4)
%\psline[linestyle=dashed](0,0)(0,5.4)
%\psline(-1.8,5.4)(1.8,5.4)\psframe(0,5.4)(-0.2,5.2)
%\uput[u](-1.8,5.4){A}\uput[u](0,5.4){B}\uput[u](1.8,5.4){C}
%\psline[linestyle=dashed](-0.7,2.05)(0.7,2.05)
%\uput[l](-0.7,2){E}\uput[r](0.7,2){G}\uput[ur](0,2){F}\uput[r](0,0){S}
%\end{pspicture}
%\end{center}
%
%La sauce sera versée dans le fond du cône jusqu'au cercle de diamètre [EG].}
%\hfill
%\parbox{0.57\linewidth}{
%
%\begin{center}\psset{unit=1cm}
%\begin{pspicture}(-3,0)(4,8.5)
%\rput(0.5,8){\textbf{Le schéma et les mesures de Jean }:}
%\pspolygon(-2.5,6.5)(2.5,6.5)(0,0)%CS
%\psline(-0.9,2.3)(0.9,2.3)%EG
%\psline(0,0)(0,6.5)%SB
%\psframe(0,6.5)(-0.2,6.3)%B
%\psframe(0,2.3)(-0.2,2.1)%F
%\uput[u](-2.5,6.5){A} \uput[ur](0,6.5){B} \uput[u](2.5,6.5){C} 
%\uput[ur](-0.9,2.3){E} \uput[ur](0,2.3){F} \uput[ul](0.9,2.3){G} 
%\uput[r](0,0){S} 
%\psline(-0.45,2.4)(-0.45,2.2)\psline(0.45,2.4)(0.45,2.2)
%\psline(-1.25,6.6)(-1.25,6.4)\psline(1.25,6.6)(1.25,6.4)
%\psline(-1.35,6.6)(-1.35,6.4)\psline(1.35,6.6)(1.35,6.4)
%\psline[linestyle=dashed]{<->}(1.2,0)(1.2,2.3)\uput[r](1.2,1.15){5 cm}
%\psline[linestyle=dashed]{<->}(3,0)(3,6.5)\uput[r](3,3.25){20 cm}
%\psline[linestyle=dashed]{<->}(-2.5,7.)(2.5,7.)\uput[u](0,7){12 cm}
%\end{pspicture}
%\end{center}
%
%B est le milieu de [AC] 
%
%F est le milieu de [EG] 
%
%BS = 20 cm ; FS = 5 cm ;  AC= 12cm 
%}
%
%\medskip
%
%\textbf{Les acheteurs }: 
%
%80\,\% des acheteurs prennent de la sauce tomate et tous les autres prennent de la mayonnaise. 
% 
%\medskip
%
%\textbf{Les sauces} : 
%
%La bouteille de mayonnaise est assimilée à un cylindre de révolution dont le diamètre de base est 5~cm et la hauteur est 15 cm. 
%
%La bouteille de sauce tomate a une capacité de $500$~mL. 
%
%\medskip

\begin{enumerate}
\item %Montrer que le rayon [EF] du cône de sauce a pour mesure $1,5$ cm. 
Les droite (FG) et (BC) car perpendiculaires à la même droite (SB). On peut donc d'après la propiété de Thalès avec les triangles SFG et SBC, écrire :

$\dfrac{\text{SF}}{\text{SB}} = \dfrac{\text{FG}}{\text{BC}} $, soit $\dfrac{5}{20} = \dfrac{\text{FG}}{6}$ d'où FG $= \dfrac{5}{20} \times 6 = \dfrac{6}{4} = \dfrac{3}{2} = 1,5$~cm.
\item %Montrer que le volume de sauce pour un cône de frites est d'environ $11,78$~cm$^3$.
Le volume d'un cône est égal à $\dfrac{\pi \times \text{EF}^2 \times \text{SF}}{3} =  \dfrac{\pi \times 1,5^2 \times 5\pi}{3} = \dfrac{15\pi}{4} \approx 11,781$, soit 11,78~cm$^3$ au centième près 
\item %Déterminer le nombre de bouteilles de chaque sauce que Jean devra acheter. 
Jean doit remplir 80\,\% des 400 cônes, de sauce tomate soit $400 \times \dfrac{80}{100} = 4 \times 80 = 320$ cônes.

320 cônes de 11,78~cm$^3$ de sauce tomate représentent $320 \times 11,78 = \np{3769,6}$~cm$^3$. 

Il a donc besoin de $\dfrac{\np{3769,6}}{500} \approx 7,5$ bouteilles soit 8 bouteilles de sauce tomate.

Pour la mayonnaise il lui faut remplir 80 cônes soit $80 \times 11,78 = 942,4$~cm$^3$.

Or chaque bouteille de mayonnaise a un volume de : $\pi \times 2,5^2 \times 15 = 93,75\pi \approx \np{294,524}$~cm$^3$. Il lui donc acheter 

$\dfrac{942,4}{294,524} \approx 3,2$ soit 4 bouteilles de mayonnaise.

Il lui faut donc 8 bouteilles de sauce tomate et 4 bouteilles de mayonnaise.
%\emph{Toute trace de recherche même non aboutie devra apparaitre sur la copie.} 
\end{enumerate}

\medskip

%\textbf{Rappels: } Volume d'un cône de révolution : $\dfrac{\pi \times \text{rayon}^2 \times \text{hauteur}}{3}$
%
%\phantom{Rappels: } Volume d'un cylindre de révolution  : $\pi \times  \text{rayon}^2 \times \text{hauteur}$ 
%
%\phantom{Rappels: } \np{1000} cm$^3 = 1$ Litre 

\newpage

\begin{center}\textbf{\large ANNEXE}

\vspace{1cm} 

\textbf{À DÉTACHER DU SUJET ET À JOINDRE AVEC VOTRE COPIE }
\end{center}

\vspace{1cm}


\textbf{EXERCICE 4} 

\bigskip

\textbf{Question 1 }

\medskip

Compléter le programme ci-dessous en remplaçant les pointillés par les bonnes valeurs pour que le losange soit dessiné tel qu'il est défini. 

\begin{center}
\parbox{0.4\linewidth}{
\begin{scratch}
\blockinit{Quand \greenflag est cliqu\'e}
\blockpen{effacer tout}
\blocklook{montrer}
\blockmove{s'orienter à \ovalnum{90} }
\blockmove{aller \`a x: \ovalnum0 y: \ovalnum0}
\blockvariable{mettre \ovalvariable{C\^ot\'e} à \ovaloperator{\red 50}}
\blockmove{Losange}
\blockmove{Cacher}
\end{scratch}
}\hfill
\parbox{0.4\linewidth}{
\begin{scratch}
\initmoreblocks{d\'efinir \namemoreblocks{Losange}}
\blockpen{stylo en position d'\'ecriture}
\blockrepeat{r\'ep\'eter \ovalnum{2} fois}
{
\blockmove{avancer de \ovalnum{C\^ot\'e}}
\blockmove{tourner \turnright{} de \ovalnum{\red 30} degr\'es}
\blockmove{avancer de \ovalnum{C\^ot\'e}}
\blockmove{tourner \turnright{} de \ovalnum{\red 150} degr\'es}
}
\end{scratch}
}
\end{center}
\end{document}\end{document}