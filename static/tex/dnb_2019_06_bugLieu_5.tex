\documentclass[10pt]{article}
\usepackage[T1]{fontenc}
\usepackage[utf8]{inputenc}%ATTENTION codage UTF8
\usepackage{fourier}
\usepackage[scaled=0.875]{helvet}
\renewcommand{\ttdefault}{lmtt}
\usepackage{amsmath,amssymb,makeidx}
\usepackage[normalem]{ulem}
\usepackage{diagbox}
\usepackage{fancybox}
\usepackage{tabularx,booktabs}
\usepackage{colortbl}
\usepackage{pifont}
\usepackage{multirow}
\usepackage{dcolumn}
\usepackage{enumitem}
\usepackage{textcomp}
\usepackage{lscape}
\newcommand{\euro}{\eurologo{}}
\usepackage{graphics,graphicx}
\usepackage{pstricks,pst-plot,pst-tree,pstricks-add}
\usepackage[left=3.5cm, right=3.5cm, top=3cm, bottom=3cm]{geometry}
\newcommand{\R}{\mathbb{R}}
\newcommand{\N}{\mathbb{N}}
\newcommand{\D}{\mathbb{D}}
\newcommand{\Z}{\mathbb{Z}}
\newcommand{\Q}{\mathbb{Q}}
\newcommand{\C}{\mathbb{C}}
\usepackage{scratch}
\renewcommand{\theenumi}{\textbf{\arabic{enumi}}}
\renewcommand{\labelenumi}{\textbf{\theenumi.}}
\renewcommand{\theenumii}{\textbf{\alph{enumii}}}
\renewcommand{\labelenumii}{\textbf{\theenumii.}}
\newcommand{\vect}[1]{\overrightarrow{\,\mathstrut#1\,}}
\def\Oij{$\left(\text{O}~;~\vect{\imath},~\vect{\jmath}\right)$}
\def\Oijk{$\left(\text{O}~;~\vect{\imath},~\vect{\jmath},~\vect{k}\right)$}
\def\Ouv{$\left(\text{O}~;~\vect{u},~\vect{v}\right)$}
\usepackage{fancyhdr}
\usepackage[french]{babel}
\usepackage[dvips]{hyperref}
\usepackage[np]{numprint}
%Tapuscrit : Denis Vergès
%\frenchbsetup{StandardLists=true}

\begin{document}
\setlength\parindent{0mm}
% \rhead{\textbf{A. P{}. M. E. P{}.}}
% \lhead{\small Brevet des collèges}
% \lfoot{\small{Polynésie}}
% \rfoot{\small{7 septembre 2020}}
\pagestyle{fancy}
\thispagestyle{empty}
% \begin{center}
    
% {\Large \textbf{\decofourleft~Brevet des collèges Polynésie 7 septembre 2020~\decofourright}}
    
% \bigskip
    
% \textbf{Durée : 2 heures} \end{center}

% \bigskip

% \textbf{\begin{tabularx}{\linewidth}{|X|}\hline
%  L'évaluation prend en compte la clarté et la précision des raisonnements ainsi que, plus largement, la qualité de la rédaction. Elle prend en compte les essais et les démarches engagées même non abouties. Toutes les réponses doivent être justifiées, sauf mention contraire.\\ \hline
% \end{tabularx}}

% \vspace{0.5cm}\textbf{Exercice 5 \hfill 22 points}

\medskip

Voici deux programmes de calcul:

\medskip

\parbox{0.48\linewidth}{\begin{center}\textbf{PROGRAMME A}

\psset{unit=1cm}
\begin{pspicture}(0,0.5)(6,6)
%\psgrid
\rput(3,5){Choisir un nombre}\psframe(1.4,4.7)(4.6,5.2)\psline{->}(2.5,4.7)(2,3.3)
\rput(2,3){Multiplier par 4}\psframe(0.8,2.8)(3.3,3.3)\psline{->}(3,4.7)(4.,4.3)
\rput(4,4){Soustraire 2}\psframe(3,3.8)(5,4.3)\psline{->}(2,2.8)(2.5,1.2)
\rput(4,2){Élever au carré}\psframe(2.8,1.7)(5.2,2.2)\psline{->}(4,3.8)(4,2.2)
\rput(3,1){ Ajouter les deux nombres}\psframe(1,0.7)(5,1.2)\psline{->}(4,1.7)(3.5,1.2)
\end{pspicture}
\end{center}}\hfill
\parbox{0.48\linewidth}{\begin{center}\textbf{PROGRAMME B}
\vspace{1.5cm}
\psset{unit=1cm}
\begin{pspicture}(6,4)
%\psgrid
\uput[r](1,3.5){$\bullet~~$Choisir un nombre}
\uput[r](1,2.5){$\bullet~~$Calculer son carré}
\uput[r](1,1.5){$\bullet~~$Ajouter 6 au résultat.}
\psframe(1,3.8)(4.6,1.2)
\end{pspicture}
\end{center}}

\medskip

\begin{enumerate}
\item 
	\begin{enumerate}
		\item Montrer que, si l'on choisit le nombre $5$, le résultat du programme A est $29$.
		\item Quel est le résultat du programme B si on choisit le nombre 5 ?
	\end{enumerate}
\item Si on nomme $x$ le nombre choisi, expliquer pourquoi le résultat du programme A peut s'écrire $x^2 + 4$.
\item Quel est le résultat du programme B si l'on nomme $x$ le nombre choisi ?
\item Les affirmations suivantes sont-elles vraies ou fausses? Justifier les réponses et écrire les étapes des éventuels calculs :
	\begin{enumerate}
		\item \og Si l'on choisit le nombre $\dfrac{2}{3}$, le résultat du programme B est $\dfrac{58}{9}$. \fg
		\item \og Si l'on choisit un nombre entier, le résultat du programme B est un nombre entier impair. \fg
		\item \og Le résultat du programme B est toujours un nombre positif. \fg
		\item \og Pour un même nombre entier choisi, les résultats des programmes A et B sont ou bien tous les deux des entiers pairs, ou bien tous les deux des entiers impairs. \fg
	\end{enumerate}
\end{enumerate}

\bigskip

\end{document}