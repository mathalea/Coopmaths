\textbf{\textsc{Exercice 4} \hfill 6 points}

\medskip

Un charpentier doit réaliser pour un de ses clients la charpente dont il a fait un
schéma ci-dessous:

\begin{center}
\psset{unit=1cm}
\begin{pspicture}(12,5)
%\psgrid
\pspolygon(0.3,1)(11.7,1)(6,4.7)
\pspolygon(2.2,1)(9.8,1)(6,3.3)
\psline(6,1)(6,4.7)
\psline(6,1)(5,2.7)
\psline(6,1)(7,2.7)
\psframe(6,1)(6.2,1.2)
\rput{210}(5,2.7){\psframe(0,0)(0.2,0.2)}
\rput{-120}(7,2.7){\psframe(0,0)(0.2,0.2)}
\uput[d](0.3,1){A} \uput[d](11.7,1){B} \uput[u](6,4.7){C} 
\uput[d](6,1){D} \uput[d](4.1,1){E} \uput[d](7.9,1){F} 
\uput[d](9.8,1){G} \uput[d](2.2,1){H} \uput[ur](6,3.3){I} 
\uput[ul](5,2.7){J} \uput[ur](7,2.7){K}
\psarc(0.3,1){6mm}{0}{32}\rput(1.2,1.25){25\degres} 
\psdots[dotstyle=+,dotangle=45](1.25,1)(3.15,1)(5.05,1)(6.95,1)(8.85,1)(10.75,1)
\psline{<->}(0.3,0.5)(11.7,0.5)
\uput[d](6,0.6){9~m}%
\psdots[dotstyle=+](4.1,1)(7.9,1)
\end{pspicture}
\end{center}

Il ne possède pas pour le moment toutes les dimensions nécessaires pour la réaliser mais il sait que :

\setlength\parindent{8mm}
\begin{itemize}
\item  la charpente est symétrique par rapport à la poutre [CD],
\item  les poutres [AC] et [HI] sont parallèles.
\end{itemize}
\setlength\parindent{0mm}

Vérifier les dimensions suivantes, calculées par le charpentier au centimètre près.

Toutes les réponses doivent être justifiées.

\medskip

\begin{enumerate}
\item Démontrer que hauteur CD de la charpente est égale à 2,10~m.
\item Démontrer, en utilisant la propriété de Pythagore, que la longueur AC est
égale à 4,97~m.
\item Démontrer, en utilisant la propriété de Thalès, que la longueur DI est égale à
1,40~m.
\item Proposer deux méthodes différentes pour montrer que la longueur JD est
égale à 1,27~m. On ne demande pas de les rédiger mais d'expliquer la
démarche.
\end{enumerate}
%%%%%%%%%%%%%
\vspace{0.25cm}

