\textbf{\textsc{Exercice 7} \hfill 5 points}

\medskip 

\parbox{0.6\linewidth}{Le Pentagone est un bâtiment hébergeant le ministère de la défense des Etats-Unis.
 
Il a la forme d'un pentagone régulier inscrit dans un cercle de rayon OA= 238 m.
 
Il est représenté par le schéma ci-contre.}\hfill 
\parbox{0.38\linewidth}{\psset{unit=1.25cm}\begin{pspicture}(-1.5,-1.5)(1.5,1.5)
\pspolygon(1.2;20)(1.2;92)(1.2;164)(1.2;236)(1.2;308)
\psline(1.2;92)(0;0)(1.2;164)
\psline(0;0)(0.97;128)
\uput[u](1.2;92){\footnotesize A} \uput[ul](1.2;164){\footnotesize B} \uput[dl](1.2;236){\footnotesize C} 
\uput[dr](1.2;308){\footnotesize D} \uput[ur](1.2;20){\footnotesize E} \uput[dr](0,0){\footnotesize O} 
\uput[ul](0.97;128){\footnotesize M}
\rput{-52}(0.97;128){\psframe(0.2,0.2)}  
\end{pspicture}}
\medskip

\begin{enumerate}
\item Calculer la mesure de l'angle $\widehat{\text{AOB}}$. 
\item La hauteur issue de O dans le triangle AOB coupe le côté [AB] au point M. 
	\begin{enumerate}
		\item Justifier que (OM) est aussi la bissectrice de $\widehat{\text{AOB}}$ et la médiatrice de [AB]. 
		\item Prouver que [AM] mesure environ $140$~m. 
		\item En déduire une valeur approchée du périmètre du Pentagone.
	\end{enumerate}
\end{enumerate}
 
\bigskip

