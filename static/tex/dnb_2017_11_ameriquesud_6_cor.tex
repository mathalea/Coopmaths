
\medskip

\begin{tabularx}{\linewidth}{|X|X|}\hline
Le bloc d'instruction \og carré\fg{} ci-dessous a été programmé puis utilisé dans les deux programmes ci-contre : 

\begin{scratch}
\initmoreblocks{définir {carré}}
\blockpen{stylo en position écriture}
\blockrepeat{répéter \ovalnum{4} fois}
	{
%	\blockmoreblocks{carréé}
	\blockmove{avancer de \ovalnum{longueur}}
	\blockmove{tourner \turnleft{} de \ovalnum{90} degrés}
	}
	\blockpen{relever le stylo}
\end{scratch}

\medskip

\textbf{Rappel : }

L'instruction \textbf{avancer de 10} fait avancer le lutin de 10 pixels.&

\textbf{Programme \no 1}

\begin{scratch}
\blockinit{quand \greenflag est pressé}
\blockvariable{mettre \selectmenu{longueur} à \ovalnum{10}}
%%%%%%
\blockrepeat{répéter \ovalnum{4} fois}
{
	\blockmoreblocks{carré}
	\blockvariable{mettre  \selectmenu{longueur} \ovaloperator{\ovalvariable{longueur}+ \ovalnum {20}}} }
	\blocklook{cacher}	
\end{scratch}

\textbf{Programme \no 2}

\begin{scratch}
\blockinit{quand \greenflag est pressé}
\blockvariable{mettre \selectmenu{longueur} à  \ovalnum{10}}
\blockrepeat{répéter \ovalnum{4} fois}
{
	\blockmoreblocks{carré}
	\blockvariable{mettre \selectmenu{longueur} à \ovaloperator{\ovalvariable{longueur}* \ovalnum {2}}} }
	\blocklook{cacher}	
\end{scratch}\\ \hline
\end{tabularx}

\medskip

\begin{enumerate}
\item Voici trois dessins : 

\begin{tabularx}{\linewidth}{|*{3}{>{\centering \arraybackslash}X|}}\hline
Dessin \no 1     &Dessin \no 2       &Dessin \no 3 \\ \hline 
\psset{unit=0.5cm}
\begin{pspicture}(4,4)
\multido{\n=1+1}{4}{\psframe(\n,\n)}
\end{pspicture}&
\psset{unit=0.9cm}
\begin{pspicture}(4,4)
\psframe(.5,.5)\psframe(1.4,1.4)\psframe(2.4,2.4)\psframe(3.4,3.4)
\end{pspicture}&
\psset{unit=1cm}
\begin{pspicture}(4,4.1)
\psframe(.5,.5)\psframe(1,1)\psframe(2,2)\psframe(4,4)
\end{pspicture}\\ \hline
\end{tabularx} 

	\begin{enumerate}
		\item Le dessin \no~2 est obtenu avec le programme \no~1. 
		\item Le dessin \no~3 est obtenu avec le programme \no~2. 
		\item Pour le programme \no~1 : 
		\begin{itemize}
			\item Le premier carré a une longueur de côté de 10 ;
			\item le deuxième carré a une longueur de côté de 30, $(10+20)$ ;
			\item le troisième carré a une longueur de côté de 50, $(30+20)$ ;
			\item le quatrième carré a une longueur de côté de 70, $(50+20)$ ;
		\end{itemize}
	L'instruction \textbf{avancer de 10} fait avancer le lutin de 10 pixels, donc la longueur, en pixel, du côté du plus grand carré dessiné est égale à 70~pixels.
	
	Pour le programme \no~2 : les dimensions du carré sont à chaque fois doublées ; la longueur, en pixel, du côté du plus grand carré dessiné est égale à 80~pixels.
	\end{enumerate}
%\end{enumerate}

\hspace{0.5cm}\begin{minipage}{5cm}
%\begin{enumerate}[\textbf{2.}]
\item On souhaite modifier le programme \no 2 pour obtenir le dessin ci-contre.

La modification 1  permet d'obtenir le dessin souhaité.
%\end{enumerate}
\end{minipage}
\end{enumerate}
\begin{minipage}{7cm}
\psset{unit=1cm}
\begin{pspicture}(-3,0)(6.5,4)
\psframe(.4,.4)\psframe(0.7,0)(1.4,0.7)\psframe(1.8,0)(3.2,1.4)\psframe(3.6,0)(6.5,2.9)
\end{pspicture}
\end{minipage}

\bigskip

\begin{tabularx}{\linewidth}{|*{3}{>{\centering \arraybackslash}X|}}\hline
Modification 1 &Modification 2 &Modification 3\\ \hline
\begin{footnotesize}
\begin{scratch}
\blockinit{quand \greenflag est pressé}
\blockvariable{mettre \selectmenu{longueur} {10}}
\blockrepeat{répéter \ovalnum{4} fois}
{
	\blockmoreblocks{carré}
	\blockmove{avancer de \ovaloperator{\ovalvariable{longueur + \ovalnum{10}}}}
	\blockvariable{mettre  \selectmenu{longueur} \ovaloperator{\ovalvariable{longueur}* \ovalnum {2}}} }
	\blocklook{cacher}	
\end{scratch}
\end{footnotesize}
& 
\begin{footnotesize}
\begin{scratch}
\blockinit{quand \greenflag est pressé}
\blockvariable{mettre \selectmenu{longueur} \ovalnum{10}}
\blockrepeat{répéter \ovalnum{4} fois}
{
	\blockmoreblocks{carré}
	\blockvariable{mettre  \selectmenu{longueur} \ovaloperator{\ovalvariable{longueur}* \ovalnum {2}}} 
	\blockmove{avancer de \ovaloperator{\ovalvariable{longueur + \ovalnum{10}}}}}
	\blocklook{cacher}	
\end{scratch}
\end{footnotesize}
& 
\begin{footnotesize}
\begin{scratch}
\blockinit{quand \greenflag est pressé}
\blockvariable{mettre \selectmenu{longueur} \ovalnum{10}}
\blockrepeat{répéter \ovalnum{4} fois}
{
	\blockmoreblocks{carré}
	\blockvariable{mettre  \selectmenu{longueur} \ovaloperator{\ovalvariable{longueur}* \ovalnum {2}}} }
	\blockmove{avancer de \ovaloperator{\ovalvariable{longueur + \ovalnum{10}}}}	
	\blocklook{cacher}	
\end{scratch}
\end{footnotesize}
\\ \hline
\end{tabularx} 

%\vspace{0.5cm}


