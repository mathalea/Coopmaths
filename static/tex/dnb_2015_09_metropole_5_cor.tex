\textbf{Exercice 5 \hfill 4 points}

\medskip 

%On considère le programme de calcul ci-dessous: 
%
%\medskip
%\begin{center}
%\begin{tabularx}{0.71\linewidth}{|lX|}\hline 
%$\bullet~~$&Choisir un nombre.\\ \hline 
%$\bullet~~$&Soustraire 6. \\ \hline 
%$\bullet~~$&Multiplier le résultat obtenu par le nombre choisi.\\ \hline 
%$\bullet~~$&Ajouter 9. \\ \hline 
%\end{tabularx}
%\end{center}

\begin{enumerate}
\item %Vérifier que lorsque le nombre choisi est 11, le résultat du programme est 64.
$11 - 3 = 5 \to 5 \times 11 = 55 \to 55 + 9 = 64$. 
\item %Lorsque le nombre choisi est $- 4$, quel est le résultat du programme ?
$- 4 - 6 = - 10\to - 10 \times (- 4) = 40 \to 40 + 9 = 49$. 
\item %Théo affirme que, quel que soit le nombre choisi au départ, le résultat du programme est toujours un nombre positif. A-t-il raison ?
Soit $x$ le nombre choisi ; on obtient successivement :

$x - 6 \to  x(x - 6)  \to  x(x - 6) + 9$.

On obtient donc finalement :

$x(x - 6) + 9 = x^2 - 6x + 9 = (x - 3)^2 \geqslant 0$.

Théo a raison. 
\end{enumerate}

\vspace{0.5cm}

