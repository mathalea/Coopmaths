
\medskip

%Sam préfère les bonbons bleus.
%
%Dans son paquet de 500 bonbons, 150 sont bleus, les  autres sont rouges, jaunes ou verts.
%
%\medskip

\begin{enumerate}
\item %Quelle est la probabilité qu'il pioche au hasard un bonbon bleu dans son paquet ?Il y a 150 bonbons bleus sur 500 bonbons ; la probabilité de tirer un bonbon bleu est donc égale à :

$\dfrac{150}{500} =  \dfrac{15}{50} = \dfrac{30}{100} = 0,30 = 30\,\%$.
\item %20\,\% dos bonbons de ce paquet sont rouges. Combien y a-t-il de bonbons rouges ?
20\,\ de 500 représentent $500 \times \dfrac{20}{100} = 5 \times 20 = 100$~bonbons rouges.
\item %Sachant qu'il y a 130 bonbons verts dons ce paquet, Sam a-t-il plus de chance de piocher au hasard un bonbon vert ou un bonbon jaune ?
Il y a sur 500 bonbons, 150 bleus, 100 rouges et 130 verts : il reste donc :

$500 - (150 + 100 + 130) = 500 - 380 = 120$~bonbons jaunes : il a donc plus de chance de tirer un bonbon vert qu'un bonbon jaune.
\item  %Aïcha avait acheté le même paquet il y a quinze jours, il ne lui reste que 140 bonbons bleus, 100 jaunes, 60 rouges et 100 verts.

%Elle dit à Sam : \og Tu devrais piocher dans mon paquet plutôt que dans le tien, tu aurais plus de chance d'obtenir un bleu \fg.
La probabilité de tirer un bonbon bleu dans le sachet d'Aïcha est égale à :

$\dfrac{140}{140 + 100 + 60 + 100} = \dfrac{140}{400} = \dfrac{4 \times 35}{4 \times 100} = \dfrac{35}{100} = 0,35$



%A-t-elle raison ?
Or on a vu à la question \textbf{1.} que la probabilité de tirer un bonbon bleu dans le sachet de Sam  est égale à 0,30.

$0,35 > 0,30$, Aïcha a raison.


\end{enumerate}

\bigskip

