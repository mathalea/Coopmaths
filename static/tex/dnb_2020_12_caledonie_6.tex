\textbf{EXERCICE 6 : Les étiquettes \hfill 14 points}

\medskip
 
\begin{enumerate}
\item Justifier que le nombre 102 est divisible par 3.
\item On donne la décomposition en produits de facteurs premiers de 85 : $85 = 5 \times 17$.

Décomposer 102 en produits de facteurs premiers.
\item Donner 3 diviseurs non premiers du nombre 102.
\end{enumerate}

Un libraire dispose d'une feuille cartonnée de 85 cm sur 102 cm.

Il souhaite découper dans celle-ci, en utilisant toute la feuille, des étiquettes carrées. 

Les côtés de ces étiquettes ont tous la même mesure.
\begin{enumerate}[resume]
\item Les étiquettes peuvent-elles avoir $34$ cm de côté ? Justifier. 
\item Le libraire découpe des étiquettes de $17$ cm de côté.

Combien d'étiquettes pourra-t-il découper dans ce cas ?
\end{enumerate}

\vspace{0,5cm}

