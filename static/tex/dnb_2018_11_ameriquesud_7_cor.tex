
\medskip

%En 2016 Marie-Amélie Le Fur a remporté la médaille d'or du 400~m aux
%Jeux Paralympiques (*) de Rio. Lors de la finale, elle a parcouru cette
%distance à la vitesse moyenne de $24,3$ km/h en battant ainsi son propre
%record du monde.
%
%Noémie met 20 minutes à vélo pour parcourir les $7$ km séparant le collège
%de sa maison. 
%
%Pour chacune des deux affirmations suivantes, dire en justifiant si elle est vraie ou fausse :
%
%\medskip
%
%\textbf{Affirmation 1 :} \og La vitesse moyenne de Noémie sur ces $7$ km est supérieure à la vitesse
%moyenne de Marie-Amélie Le Fur lors de cette finale. \fg
%
%\medskip
%
%\textbf{Affirmation 2 :} \og Marie-Amélie Le Fur a couru le 400~m en moins d'une minute lors de cette
%finale. \fg
%
%\medskip
%
%(*) Les Jeux Paralympiques sont les Jeux Olympiques pour athlètes en situation de handicap.
7~km en 20~minutes représente une vitesse de $7 \times 3$~km en $3 \times 20$~minutes soit 21~km/h.

Or $21 < 24,3$ : l'affirmation 1 est fausse.

On a $ v = \dfrac{d}{t}$, $d$ étant la distance parcourue et $t$ le temps mis pour parcourir cette distance.

Donc $v \times t = d$ et $t = \dfrac{d}{v} = \dfrac{0,400}{24,3} \approx \np{0,0164}$~h, soit environ $\np{0,0164} \times 60 = 0,99$~min soit moins d'une minute.

L'affirmation 2 est vraie.
