\textbf{Exercice 2 \hfill 6 points}

\medskip

Voici un programme de calcul : 
\begin{center}
\begin{tabularx}{0.45\linewidth}{|l X|}\hline
$\bullet~~$& Choisir un nombre entier positif\\
$\bullet~~$& Ajouter 1\\
$\bullet~~$& Calculer le carré du résultat obtenu\\
$\bullet~~$& Enlever le carré du nombre de départ.\\ \hline
\end{tabularx}
\end{center}

\begin{enumerate}
\item On applique ce programme de calcul au nombre 3. Montrer qu'on obtient 7.
\item Voici deux affirmations :

Affirmation \no 1 : \og Le chiffre des unités du résultat obtenu est 7 \fg.

Affirmation \no 2 : \og Chaque résultat peut s'obtenir en ajoutant le nombre entier de départ et le nombre entier qui le suit \fg.
	\begin{enumerate}
		\item Vérifier que ces deux affirmations sont vraies pour les nombres 8 et 13.
		\item Pour chacune de ces deux affirmations, expliquer si elle est vraie ou fausse quel que soit le nombre choisi au départ.
	\end{enumerate}
\end{enumerate}

\bigskip


\bigskip

