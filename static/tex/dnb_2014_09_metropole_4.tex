\textbf{Exercice 4 \hfill 5 points}

\medskip
 
\begin{center}
\psset{unit=1cm}
\begin{pspicture}(12,7)
%\psgrid
\psline(0,0.5)(12,0.5)
\psline(7.3,0.5)(7.3,7)\rput(9.5,4){Immeuble}
\psline(0.5,0.5)(3.1,3.85)(5.8,0.5)
\psline(4.5,0.5)(7.3,5.2)
\pspolygon*(2.8,3.5)(3.1,3.85)(3.4,3.5)
\psline(3.1,0.6)(3.1,0.4)\uput[d](3.1,0.4){P}
\psline(4.5,0.6)(4.5,0.4)\uput[d](4.5,0.4){M}
\psline(5.85,0.6)(5.85,0.4)\uput[d](5.85,0.4){L}
\psline(7.3,0.6)(7.3,0.4)\uput[d](7.3,0.4){C}
\uput[ul](7.3,5.2){F}
\pspolygon*(7.3,5.2)(7.3,4.8)(7.04,4.8)
\psarc[linewidth=1.5pt](2.95,3.8){0.175}{0}{180}
\psline[linewidth=1.5pt](2.77,3.8)(2.77,3.4)
\psline[linewidth=3pt](2.7,3.7)(2.7,0.5)
\pspolygon*(2.5,0.5)(2.7,1)(2.9,0.5)
\pspolygon[fillstyle=hlines](4.5,0.5)(5.04,1.42)(5.85,0.5)
\end{pspicture}
\end{center} 

On s'intéresse à la zone au sol qui est éclairée la nuit par deux sources de lumière : le lampadaire de la rue et le spot fixé en F sur la façade de l'immeuble.

\parbox{0.45\linewidth}{On réalise le croquis ci-contre qui n'est pas à l'échelle, pour modéliser la situation: 

On dispose des données suivantes : 
 
PC = 5,5~m ; CF = 5~m ; HP = 4~m ;

$\widehat{\text{MFC}} = 33\degres$ ; $\widehat{\text{PHL}} = 40\degres$}
\hfill 
\parbox{0.5\linewidth}{\psset{unit=1cm}
\begin{pspicture}(6.2,6.4)
\pspolygon(0.2,4.6)(0.2,0.4)(6.1,0.4)(6.1,5.9)(2.5,0.4)(3.8,0.4)%HPCFMLH
\uput[u](0.2,4.6){H} \uput[d](0.2,0.4){P} \uput[d](2.5,0.4){M} 
\uput[d](3.8,0.4){L} \uput[d](6.1,0.4){C} \uput[u](6.1,5.9){F}
\psframe(0.2,0.4)(0.5,0.7)\psframe(6.1,0.4)(5.8,0.7)
\psarc(0.2,4.6){0.7}{-90}{-50}
\psarc(6.1,5.9){0.7}{-123}{-90} 
\end{pspicture}}

\medskip

\begin{enumerate}
\item Justifier que l'arrondi au décimètre de la longueur PL est égal à 3,4~m. 
\item Calculer la longueur LM correspondant à la zone éclairée par les deux sources de lumière. On arrondira la réponse au décimètre. 
\item On effectue des réglages du spot situé en F afin que M et L soient confondus. 

Déterminer la mesure de l'angle $\widehat{\text{CFM}}$. On arrondira la réponse au degré. 
\end{enumerate} 

\vspace{0,5cm}

