\textbf{\textsc{Exercice 2 \hfill 5 points}}

\medskip

\begin{enumerate}
\item $\sqrt{(- 5)^2} = \sqrt{25} = 5$ ; c’est la réponse C.
\item  C'est la réponse C.
\item  $f(x)= 3x - (2x + 7) + (3x + 5) = 3x - 2x- 7 + 3x + 5 = 4x - 2$. C’est la réponse A.
\item  L'enquête ne peut pas l’aider car c’est un tirage aléatoire : chaque numéro a la même chance d'être sorti. C’est la réponse C.
\item $(x - 1)^2 −16 = (x - 1)^2 - 4^2 =[(x - 1) + 4[(x - 1) − 4]  = [x - 1+4] \times [x - 1 - 4] = (x + 3)(x - 5)$.

D’où $(x - 1)^2 - 16 = (x + 3)(x - 5)$ ; c’est la réponse A.
\end{enumerate}
 
\vspace{0,5cm}

