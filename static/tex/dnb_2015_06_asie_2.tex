\textbf{Exercice 2 \hfill 5 points}

\medskip

Julien est en retard pour aller rejoindre ses amis au terrain de basket.

Il décide alors de traverser imprudemment la route du point J au point F sans utiliser les
passages piétons.

Le passage piéton est supposé perpendiculaire au trottoir.

\begin{center}
\psset{unit=1cm}
\begin{pspicture}(6.5,4.3)
\psframe(6.5,4.3)
\psframe[fillstyle=solid,fillcolor=gray](6.5,0.8)
\psframe[fillstyle=solid,fillcolor=gray](0,3.5)(6.5,4.3)
\psframe[fillstyle=solid,fillcolor=lightgray](0,0.8)(6.5,3.5)
\multido{\n=1.000+0.625}{4}{\rput(0,\n){\psframe[fillstyle=solid,fillcolor=white](0,0)(1.7,0.425)}}
\uput[ul](0,3.5){F}\uput[dl](0,0.8){K}\uput[dr](6.5,0.8){J}
\psline{<->}(-0.3,0.8)(-0.3,3.5)\rput(-0.7,2.15){8 m}
\psline{<->}(0,0.5)(6.5,0.5)\rput(3.25,0.2){15 m}
\psline{<->}(0.1,3.5)(6.4,0.8)
\end{pspicture}
\end{center}

En moyenne, un piéton met $9$ secondes pour parcourir $10$ mètres. 

Combien de temps Julien a-t-il gagné en traversant sans utiliser le passage piéton ?

\newpage

