\textbf{\textsc{Exercice} 6 \hfill 5,5 points}

\medskip 

Dans les marais salants, le sel récolté est stocké sur une surface plane. On admet qu'un tas de sel a toujours la forme d'un cône de révolution. 

\medskip

\begin{enumerate}
\item 
	\begin{enumerate}
		\item Pascal souhaite déterminer la hauteur d'un cône de sel de diamètre 5~mètres. Il possède un bâton de longueur $1$~mètre. Il effectue des mesures et réalise les deux schémas ci-dessous :
		
\begin{center}
\psset{unit=0.7cm} 
\begin{pspicture}(0,-0.3)(15,12)
\rput(11.5,9.5){Cône de sel} 
\pspolygon[linestyle=dashed](11.5,8)(0.3,8)(11.5,11.7)
\psline[linewidth=1.8pt](4.5,8)(4.5,9.4)
\uput[r](4.5,8.7){Bâton}
\scalebox{.99}[0.3]{\psarc(11.5,26.8){2.5}{180}{0}}%
\scalebox{.99}[0.3]{\psarc[linestyle=dashed](11.5,26.8){2.5}{0}{180}}%
\psline(8.9,8)(11.5,11.7)(13.9,8)
\pspolygon[linestyle=dashed](11.5,0.5)(0.3,0.5)(11.5,4.2)
\pspolygon(8.9,0.5)(11.5,4.2)(13.9,0.5)
\psline[linewidth=1.8pt](4.5,0.5)(4.5,1.9)
\psframe(11.5,0.5)(11.8,0.8)
\psframe(4.5,0.5)(4.2,0.8)
\psline[linewidth=0.6pt,arrowsize=3pt 3]{<->}(0.2,0.2)(4.5,0.2)
\psline[linewidth=0.6pt,arrowsize=3pt 3]{<->}(4.5,0.2)(8.9,0.2)
\psline[linewidth=0.6pt,arrowsize=3pt 3]{<->}(8.9,0.2)(13.9,0.2)
\uput[d](2.35,0.2){3,20 m} \uput[d](6.7,0.2){2,30 m} \uput[d](11.4,0.2){5 m}
\uput[r](4.5,1.2){1 m}\uput[ul](0.2,0.5){A}\uput[u](4.5,1.9){C}
\uput[u](11.5,4.2){S}\uput[ul](11.5,0.5){O}
\uput[ul](8.9,0.5){E}\uput[ur](13.9,0.5){L} \uput[ur](4.5,0.5){B}
\end{pspicture}
\end{center}

Démontrer que la hauteur de ce cône de sel est égale à $2,50$~mètres.

\medskip
 
Dans cette question, on n'attend pas de démonstration rédigée. Il suffit d'expliquer brièvement le raisonnement suivi et de présenter clairement les calculs.
\item À l'aide de la formule  V$_{\text{c\^one}}= \dfrac{\pi \times \text{rayon}^2 \times \text{hauteur}}{3}$, déterminer en m$^3$ le volume de sel contenu dans ce cône. Arrondir le résultat au m$^3$ près. 
	\end{enumerate} 
\item Le sel est ensuite stocké dans un entrepôt sous la forme de cônes de volume \np{1000}~m$^3 $. Par mesure de sécurité, la hauteur d'un tel cône de sel ne doit pas dépasser $6$~mètres. Quel rayon faut-il prévoir au minimum pour la base ? Arrondir le résultat au décimètre près.
\end{enumerate}
 
\bigskip

