\documentclass[10pt]{article}
\usepackage[T1]{fontenc}
\usepackage[utf8]{inputenc}%ATTENTION codage UTF8
\usepackage{fourier}
\usepackage[scaled=0.875]{helvet}
\renewcommand{\ttdefault}{lmtt}
\usepackage{amsmath,amssymb,makeidx}
\usepackage[normalem]{ulem}
\usepackage{diagbox}
\usepackage{fancybox}
\usepackage{tabularx,booktabs}
\usepackage{colortbl}
\usepackage{pifont}
\usepackage{multirow}
\usepackage{dcolumn}
\usepackage{enumitem}
\usepackage{textcomp}
\usepackage{lscape}
\newcommand{\euro}{\eurologo{}}
\usepackage{graphics,graphicx}
\usepackage{pstricks,pst-plot,pst-tree,pstricks-add}
\usepackage[left=3.5cm, right=3.5cm, top=3cm, bottom=3cm]{geometry}
\newcommand{\R}{\mathbb{R}}
\newcommand{\N}{\mathbb{N}}
\newcommand{\D}{\mathbb{D}}
\newcommand{\Z}{\mathbb{Z}}
\newcommand{\Q}{\mathbb{Q}}
\newcommand{\C}{\mathbb{C}}
\usepackage{scratch}
\renewcommand{\theenumi}{\textbf{\arabic{enumi}}}
\renewcommand{\labelenumi}{\textbf{\theenumi.}}
\renewcommand{\theenumii}{\textbf{\alph{enumii}}}
\renewcommand{\labelenumii}{\textbf{\theenumii.}}
\newcommand{\vect}[1]{\overrightarrow{\,\mathstrut#1\,}}
\def\Oij{$\left(\text{O}~;~\vect{\imath},~\vect{\jmath}\right)$}
\def\Oijk{$\left(\text{O}~;~\vect{\imath},~\vect{\jmath},~\vect{k}\right)$}
\def\Ouv{$\left(\text{O}~;~\vect{u},~\vect{v}\right)$}
\usepackage{fancyhdr}
\usepackage[french]{babel}
\usepackage[dvips]{hyperref}
\usepackage[np]{numprint}
%Tapuscrit : Denis Vergès
%\frenchbsetup{StandardLists=true}

\begin{document}
\setlength\parindent{0mm}
% \rhead{\textbf{A. P{}. M. E. P{}.}}
% \lhead{\small Brevet des collèges}
% \lfoot{\small{Polynésie}}
% \rfoot{\small{7 septembre 2020}}
\pagestyle{fancy}
\thispagestyle{empty}
% \begin{center}
    
% {\Large \textbf{\decofourleft~Brevet des collèges Polynésie 7 septembre 2020~\decofourright}}
    
% \bigskip
    
% \textbf{Durée : 2 heures} \end{center}

% \bigskip

% \textbf{\begin{tabularx}{\linewidth}{|X|}\hline
%  L'évaluation prend en compte la clarté et la précision des raisonnements ainsi que, plus largement, la qualité de la rédaction. Elle prend en compte les essais et les démarches engagées même non abouties. Toutes les réponses doivent être justifiées, sauf mention contraire.\\ \hline
% \end{tabularx}}

% \vspace{0.5cm}\textbf{\textsc{Exercice 4} \hfill 7 points}

\medskip 
 
%Le nombre d'abonnés à une revue dépend du prix de la revue.
% 
%Pour un prix $x$ compris entre 0 et 20~\euro, le nombre d'abonnés est donné par la fonction $A$ telle que : $A(x) = - 50 x + \np{1250}$.
% 
%La recette, c'est-à-dire le montant perçu par l'éditeur de cette revue, est donnée par la fonction $R$ telle que : $R(x) = - 50 x^2 + \np{1250} x$. 
%
%
%\begin{center}\textbf{Représentation graphique de la fonction} \boldmath $A$ \unboldmath 
%
%\medskip
%
%\psset{xunit=0.4cm,yunit=0.004cm}
%\begin{pspicture}(-1,-100)(25,1400)
%\multido{\n=0+1}{25}{\psline[linestyle=dashed,linecolor=orange,linewidth=0.2pt](\n,0)(\n,1400)}
%\multido{\n=0+100}{15}{\psline[linestyle=dashed,linecolor=orange,linewidth=0.2pt](0,\n)(24,\n)}
%\psaxes[linewidth=1.5pt,Dx=2,Dy=200]{->}(0,0)(0,0)(24,1400)
%\psplot[plotpoints=100,linewidth=1.25pt]{0}{24}{1250 50 x mul sub}
%\uput[u](19,0){prix de la revue en euros}
%\uput[r](0,1350){nombre d'abonnés}
%\end{pspicture}

%\vspace{1cm}
%
%\textbf{Représentation graphique de la fonction } \boldmath $R$ \unboldmath
%
%\vspace{0,5cm}
%
%\psset{xunit=0.5cm,yunit=0.001cm}
%\begin{pspicture}(-1,-400)(20,8400)
%
%\multido{\n=0+1}{21}{\psline[linestyle=dashed,linecolor=orange,linewidth=0.2pt](\n,0)(\n,8400)}
%\multido{\n=0+400}{21}{\psline[linestyle=dashed,linecolor=orange,linewidth=0.2pt](0,\n)(20,\n)}
%\psaxes[linewidth=1.5pt,Dx=2,Dy=20000]{->}(0,0)(0,0)(20,8400)
%\multido{\n=0+2000}{5}{\uput[l](0,\n){\np{\n}}}
%\psplot[plotpoints=100,linewidth=1.25pt,linecolor=blue]{0}{20}{25  x  sub 50 mul x mul}
%\uput[u](16.5,0){prix de la revue en euros}
%\uput[r](0,8350){recette en euros}
%\end{pspicture} 
%\end{center}
%
%\medskip
 
\begin{enumerate}
\item %Le nombre d'abonnés est-il proportionnel au prix de la revue ? Justifier. 
C’est l’inverse : le nombre d’abonnés est inversement proportionnel au prix de la revue.
\item %Vérifier, par le calcul, que $A(10) = 750$ et interpréter concrètement ce résultat.
$A(10) = \np{1250} - 50 \times 10 = \np{1250} - 500 = 750$. 
\item %La fonction $R$ est-elle affine ? Justifier. 
La fonction $R$ n’est pas de la forme $R(x) = ax + b$ : ce n’est pas une fonction affine.
\item %Déterminer graphiquement pour quel prix la recette de l'éditeur est maximale. 
La recette semble maximale pour $x = 12,50$~\euro.
\item %Déterminer graphiquement les antécédents de \np{6800} par $R$.
La droite d’équation $y = \np{6800}$ coupe la représentation de $R$ aux points d’abscisse $x = 8$ et $x = 17$. 
\item %Lorsque la revue coûte $5$~euros, déterminer le nombre d'abonnés et la recette. 
On a $A(5) = \np{1250} - 50 \times 5 = \np{1250} - 250 = \np{1000}$.

$R(5) = \np{1250}\times 5 - 50 \times 5^2 = \np{6250} - \np{1250} = \np{5000}$ ou simplement $\np{1000} \times 5 = \np{5000}$~\euro.
\end{enumerate}

\bigskip

\end{document}