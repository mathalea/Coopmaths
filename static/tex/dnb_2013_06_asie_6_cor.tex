\textbf{Exercice 6 \hfill 3 points}

\medskip

%Voici un article trouvé sur internet.
% 
%\emph{D'après l'Observatoire des Usages Internet de Médiamétrie, au dernier trimestre $2011$, $28$ millions d'internautes\footnote{Un internaute est un utilisateur d'internet}  ont acheté en ligne. Au premier trimestre de $2012$, on constate une augmentation de $11$\,\% du nombre d'achats en ligne.}
%
%\medskip 

\begin{enumerate}
\item %En utilisant les données de cet article, calculer le nombre de cyberacheteurs au premier trimestre 2012. Arrondir le résultat à $0,1$~million près.
L'augmentation du nombre d'achats ne permet pas de dire quelle est l'augmentation du nombre d'internautes. 
\item %Si la progression sur le deuxième trimestre 2012 est, elle aussi, de 11\,\%, quelle serait la progression en pourcentage sur les deux trimestres ? Justifier la réponse.
Augmenter de 11\,/ù c'est multiplier par $1,11$ ; donc deux augmentations succesives de 11\;\% reviennent à multiplier par $1,11 \times 1,11 = 1,11^2 = \np{1,2321}$ soit une augmentation de 23,21\,\%.
\end{enumerate}
 
\bigskip

