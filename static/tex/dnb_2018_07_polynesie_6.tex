
\medskip

Voici un script saisi par Alice dans un logiciel d'algorithmique.

\medskip

\begin{center}
\begin{tabularx}{0.8\linewidth}{X}
\begin{scratch}
\blockinit{quand \greenflag est cliqué}
\blocksensing{demander \txtbox{Choisissez un nombre ?} et attendre}
\blockevent{envoyer à tous \ovalvariable{le nombre a été saisi}}
\blockvariable{mettre \ovalvariable{Nombre} à \ovaloperator{\ovalvariable{réponse} }}
\blockvariable{mettre \ovalvariable{Résultat 1} à \ovaloperator{2*\ovalvariable{Nombre} +3}}
\blockvariable{mettre \ovalvariable{Résultat 1} à \ovaloperator{Résultat 1}*\ovaloperator{Résultat 1}}
\blocklook{dire \txtbox{regroupe} \ovalvariable{le résultat 1 est \ovaloperator{Résultat 1}}pendant \ovalnum{2} secondes}
\end{scratch}\\[5pt]
\begin{scratch}
\blockinit{quand je reçois \txtbox{le nombre a été saisi}}
\blockvariable{mettre \ovalvariable{Résultat 2} à \ovaloperator{Nombre}*\ovaloperator{Nombre}}
\blockvariable{mettre \ovalvariable{Résultat 2} à \ovaloperator{Résultat 2}*\ovaloperator{4}}
\blockvariable{mettre \ovalvariable{Résultat 2} à \ovaloperator{Résultat 2}+\ovaloperator{12}*\ovaloperator{Nombre}}
\blockvariable{mettre \ovalvariable{Résultat 2} à \ovaloperator{Résultat 2}+\ovaloperator{9}}
\blockcontrol{attendre \ovalnum{3} seconde}
\blocklook{dire \txtbox{regroupe} \ovalvariable{le résultat 2 est \ovaloperator{Résultat 2}}}
\end{scratch}\\
\end{tabularx}
\end{center}

\begin{enumerate}
\item Alice a choisi 3 comme nombre, calculer les valeurs de Résultat 1 et de Résultat 2.

\emph{Justifier en faisant apparaître les calculs réalisés}.
\item Généralisation
	\begin{enumerate}
		\item En appelant $x$ le nombre choisi dans l'algorithme, donner une expression littérale
traduisant la première partie de l'algorithme correspondant à Résultat 1.
		\item  En appelant $x$ le nombre choisi dans l'algorithme, donner une expression littérale
traduisant la deuxième partie de l'algorithme correspondant à Résultat 2.
\item  Trouver le ou les nombres choisis par Alice qui correspondent au résultat affiché ci-dessous.

\begin{center}
\ovallook{Résultat 2 \ovalnum{9}}
\end{center}
 	\end{enumerate}
\end{enumerate}



