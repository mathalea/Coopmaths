\documentclass[10pt]{article}
\usepackage[T1]{fontenc}
\usepackage[utf8]{inputenc}%ATTENTION codage UTF8
\usepackage{fourier}
\usepackage[scaled=0.875]{helvet}
\renewcommand{\ttdefault}{lmtt}
\usepackage{amsmath,amssymb,makeidx}
\usepackage[normalem]{ulem}
\usepackage{diagbox}
\usepackage{fancybox}
\usepackage{tabularx,booktabs}
\usepackage{colortbl}
\usepackage{pifont}
\usepackage{multirow}
\usepackage{dcolumn}
\usepackage{enumitem}
\usepackage{textcomp}
\usepackage{lscape}
\newcommand{\euro}{\eurologo{}}
\usepackage{graphics,graphicx}
\usepackage{pstricks,pst-plot,pst-tree,pstricks-add}
\usepackage[left=3.5cm, right=3.5cm, top=3cm, bottom=3cm]{geometry}
\newcommand{\R}{\mathbb{R}}
\newcommand{\N}{\mathbb{N}}
\newcommand{\D}{\mathbb{D}}
\newcommand{\Z}{\mathbb{Z}}
\newcommand{\Q}{\mathbb{Q}}
\newcommand{\C}{\mathbb{C}}
\usepackage{scratch}
\renewcommand{\theenumi}{\textbf{\arabic{enumi}}}
\renewcommand{\labelenumi}{\textbf{\theenumi.}}
\renewcommand{\theenumii}{\textbf{\alph{enumii}}}
\renewcommand{\labelenumii}{\textbf{\theenumii.}}
\newcommand{\vect}[1]{\overrightarrow{\,\mathstrut#1\,}}
\def\Oij{$\left(\text{O}~;~\vect{\imath},~\vect{\jmath}\right)$}
\def\Oijk{$\left(\text{O}~;~\vect{\imath},~\vect{\jmath},~\vect{k}\right)$}
\def\Ouv{$\left(\text{O}~;~\vect{u},~\vect{v}\right)$}
\usepackage{fancyhdr}
\usepackage[french]{babel}
\usepackage[dvips]{hyperref}
\usepackage[np]{numprint}
%Tapuscrit : Denis Vergès
%\frenchbsetup{StandardLists=true}

\begin{document}
\setlength\parindent{0mm}
% \rhead{\textbf{A. P{}. M. E. P{}.}}
% \lhead{\small Brevet des collèges}
% \lfoot{\small{Polynésie}}
% \rfoot{\small{7 septembre 2020}}
\pagestyle{fancy}
\thispagestyle{empty}
% \begin{center}
    
% {\Large \textbf{\decofourleft~Brevet des collèges Polynésie 7 septembre 2020~\decofourright}}
    
% \bigskip
    
% \textbf{Durée : 2 heures} \end{center}

% \bigskip

% \textbf{\begin{tabularx}{\linewidth}{|X|}\hline
%  L'évaluation prend en compte la clarté et la précision des raisonnements ainsi que, plus largement, la qualité de la rédaction. Elle prend en compte les essais et les démarches engagées même non abouties. Toutes les réponses doivent être justifiées, sauf mention contraire.\\ \hline
% \end{tabularx}}

% \vspace{0.5cm}\textbf{\textsc{Exercice 6} \hfill 3 points}

\medskip

\begin{enumerate}
\item Si on choisit au hasard un macaron dans la boîte \no 1, la probabilité que ce soit un
macaron au café est $\dfrac{3}{12} = \dfrac{1}{4}$.
\item~

\begin{center}
\pstree[treemode=R,nodesep=8pt,levelsep=3.5cm]{\TR{}}
{\pstree{\TR{\footnotesize Chocolat}\taput{$\frac{3}{5}$}}
	{\TR{\footnotesize Chocolat}\taput{$\frac{2}{3}$}
	\TR{\footnotesize Fraise}\tbput{$\frac{1}{3}$}
	}
\pstree{\TR{\footnotesize Café}\tbput{$\frac{2}{5}$}}
	{\TR{\footnotesize Chocolat}\taput{$\frac{2}{3}$}
	\TR{\footnotesize Fraise}\tbput{$\frac{1}{3}$}
	}
}
\end{center} 

Pour obtenir deux macarons qui lui plaisent, Carole doit choisir un macaron au café dans
la boîte \no 1 et un macaron à la fraise dans la boîte \no 2.

Je calcule : $\dfrac{2}{5} \times \dfrac{1}{3} = \dfrac{2}{15}$.

La probabilité que Carole obtienne deux macarons qui lui plaisent est donc de $\dfrac{2}{15}$.
\end{enumerate}
 
\vspace{0,5cm}

\end{document}