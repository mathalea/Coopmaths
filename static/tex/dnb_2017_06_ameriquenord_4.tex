
\medskip

Les données et les questions de cet exercice concernent la France métropolitaine.

\begin{center}
\begin{tabularx}{\linewidth}{|m{5cm}|X|}\hline
\multicolumn{1}{|c|}{Document 1}&\multicolumn{1}{|c|}{Document 2}\\
\vspace*{-3cm}En 2015, environ 4,7\,\% de la population française souffrait d'allergies alimentaires.

En 2010, les personnes concernées par des allergies alimentaires étaient deux fois moins nombreuses qu'en 2015.

En 1970, seulement 1\,\% de la
population était concernée.

{\footnotesize \emph{Source : Agence nationale de la sécurité sanitaire de l'alimentation, de l'environnement et du
travail.}}&
\psset{xunit=0.15cm,yunit=0.25cm}
\begin{pspicture}(-7,-2)(45,16)
\multido{\n=0+5}{10}{\psline[linewidth=0.3pt](\n,0)(\n,16)}
\multido{\n=0+2}{9}{\psline[linewidth=0.3pt](0,\n)(45,\n)}
\psaxes[Ox=1970,Dx=5,Oy=50,Dy=2,labelFontSize=\scriptstyle](0,0)(0,0)(45,16)
\uput[d](22.5,-2){Années}\rput{90}(-7,8){Populations (en millions)}
\pscurve[linewidth=1.25pt](0,0.35)(5,2.2)(10,3.9)(15,5)(20,6.5)(25,7.95)(30,9)(35,11)(40,12.8)(45,14)
\end{pspicture}\\ \hline
\end{tabularx}
\end{center}

\medskip

\textbf{Partie 1 :}

\medskip

\begin{enumerate}
\item Déterminer une estimation du nombre de personnes, à \np{100000} près, qui souffraient
d'allergies alimentaires en France en 2010.
\item Est-il vrai qu'en 2015, il y avait environ 6 fois plus de personnes concernées qu'en
1970 ?
\end{enumerate}

\medskip

\textbf{Partie 2 :}

\medskip

En 2015, dans un collège de $681$ élèves, $32$ élèves souffraient d'allergies alimentaires.

Le tableau suivant indique les types d'aliments auxquels ils réagissaient.

\begin{center}
\begin{tabularx}{\linewidth}{|m{3cm}|*{5}{>{\centering \arraybackslash}X|}}\hline
Aliments &Lait &Fruits &Arachides &Poisson &Œuf\\ \hline
Nombre d'élèves concernés &6 &8 &11 &5 &9\\ \hline
\end{tabularx}
\end{center}

\begin{enumerate}
\item La proportion des élèves de ce collège souffrant d'allergies alimentaires est-elle
supérieure à celle de la population française ?
\item Jawad est étonné : \og J'ai additionné tous les nombres indiqués dans le tableau et j'ai
obtenu 39 au lieu de 32 \fg.

Expliquer cette différence.
\item Lucas et Margot ont chacun commencé un diagramme pour représenter les allergies
des 32 élèves de leur collège :

\begin{center}
\begin{tabularx}{\linewidth}{|*{2}{>{\centering \arraybackslash}X|}}\hline
\textbf{Diagramme de Lucas} &\textbf{Diagramme de Margot}\\ \hline
\psset{xunit=0.75cm,yunit=0.4cm}
\begin{pspicture}(6,10)
\multido{\n=0+1}{10}{\psline[linewidth=0.3pt](0,\n)(6,\n)}
\uput[r](0,9.5){\tiny Nombre d'élèves concernés}
\psaxes[Dx=10]{->}(0,0)(0,0)(6,10)
\rput{30}(1,-1.2){\footnotesize Lait}\rput{30}(2,-1.2){\footnotesize Fruits}\rput{30}(3,-1.2){\footnotesize Arachides}\rput{30}(4,-1.2){\footnotesize Poisson}\rput{30}(5,-1.2){\footnotesize Oeuf}
\psframe[fillstyle=solid,fillcolor=lightgray](0.75,0)(1.25,6)
\psframe[fillstyle=solid,fillcolor=lightgray](1.75,0)(2.25,8)
\end{pspicture}&
\psset{xunit=0.75cm,yunit=0.4cm} \begin{pspicture}(6,10)
\multido{\n=0+1}{10}{\psline[linewidth=0.3pt](0,\n)(6,\n)}
\psaxes[Dx=10]{->}(0,0)(0,0)(6,10)
\uput[r](0,9.5){\tiny Nombre d'élèves concernés}
\rput{30}(1,-1.2){\footnotesize Lait}\rput{30}(2,-1.2){\footnotesize Fruits}\rput{30}(3,-1.2){\footnotesize Arachides}\rput{30}(4,-1.2){\footnotesize Poisson}\rput{30}(5,-1.2){\footnotesize Oeuf}
\psline(0,0)(1,6)(2,8)
\end{pspicture}\\ \hline
\end{tabularx}
\end{center}
\vspace{0.6cm}
	\begin{enumerate}
		\item Qui de Lucas ou de Margot a fait le choix le mieux adapté à la situation ? Justifier
la réponse.
		\item Reproduire et terminer le diagramme choisi à la question \textbf{a.}
	\end{enumerate}
\end{enumerate}

\vspace{0,5cm}

