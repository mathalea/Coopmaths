
\medskip

On demande à quinze élèves d'une classe A et à dix élèves d'une classe B de compter le
nombre de SMS qu'ils envoient pendant un week-end.

Le lundi on récupère les résultats dans un tableur.

\begin{center}
\begin{tabularx}{\linewidth}{|c|c|*{15}{>{\footnotesize \centering \arraybackslash}X|}c|c|}\hline
	&A		&B	&C	&D	&E	&F	&G	&H	&I	&J	&K	&L	&M	&N	&O	&P	&Q	&R\\ \hline
1	&Classe&\multicolumn{15}{|c|}{Nombre de SMS envoyés par élève dans le week-end}&Moy.&Méd.\\ \hline
2	&A		&0	&0	&0	&0	&0	&5	&7	&12	&15	&15	&16	&18	&21	&34	&67	&	&\\ \hline
3	&B		&0	&1	&1	&2	&11	&17	&18	&18	&20	&32	&	&	&	&	&	&\footnotesize 12	&\footnotesize 14\\ \hline
\end{tabularx}
\end{center}

\medskip

\begin{enumerate}
\item Calculer le nombre moyen et le nombre médian de SMS envoyés pendant le week-end
par ces élèves de la classe A.
\item Quelles formules ont pu être écrites dans les cellules Q3 et R3 du tableur ?
\item Calculer le nombre moyen de SMS envoyés pendant le week-end par ces $2$5 élèves des
classes A et B.
\item Calculer le nombre médian de SMS envoyés pendant le week-end par ces $25$ élèves
des classes A et B.
\end{enumerate}

\bigskip

