
\medskip

Dans le village de Jean, une brocante est organisée chaque année lors du premier week-end de juillet. Jean s'est engagé à s'occuper du stand de vente de frites. Pour cela, il fabrique des cônes en papier qui lui serviront de barquette pour les vendre. 

Dans le fond de chaque cône, Jean versera de la sauce: soit de la mayonnaise, soit de la sauce tomate.

\medskip
 
Il décide de fabriquer $400$ cônes en papier et il doit estimer le nombre de bouteilles de mayonnaise et de sauce tomate à acheter pour ne pas en manquer. 

Voici les informations dont Jean dispose pour faire ses calculs : 

\parbox{0.4\linewidth}{\textbf{Le cône de frites }: 

\begin{center}
\psset{unit=1cm}
\begin{pspicture}(-2,0)(2,6)
\psellipse(0,5.4)(1.8,0.35)
\scalebox{.99}[0.3]{\psarc[linewidth=1.5pt](0,6.9){0.7}{180}{0}}%
\scalebox{.99}[0.3]{\psarc[linewidth=1.5pt,linestyle=dashed](0,6.9){0.7}{0}{180}}%
\psline(-1.8,5.4)(0,0)(1.8,5.4)
\psline[linestyle=dashed](0,0)(0,5.4)
\psline(-1.8,5.4)(1.8,5.4)\psframe(0,5.4)(-0.2,5.2)
\uput[u](-1.8,5.4){A}\uput[u](0,5.4){B}\uput[u](1.8,5.4){C}
\psline[linestyle=dashed](-0.7,2.05)(0.7,2.05)
\uput[l](-0.7,2){E}\uput[r](0.7,2){G}\uput[ur](0,2){F}\uput[r](0,0){S}
\end{pspicture}
\end{center}

La sauce sera versée dans le fond du cône jusqu'au cercle de diamètre [EG].}
\hfill
\parbox{0.57\linewidth}{

\begin{center}\psset{unit=1cm}
\begin{pspicture}(-3,0)(4,8.5)
\rput(0.5,8){\textbf{Le schéma et les mesures de Jean }:}
\pspolygon(-2.5,6.5)(2.5,6.5)(0,0)%CS
\psline(-0.9,2.3)(0.9,2.3)%EG
\psline(0,0)(0,6.5)%SB
\psframe(0,6.5)(-0.2,6.3)%B
\psframe(0,2.3)(-0.2,2.1)%F
\uput[u](-2.5,6.5){A} \uput[ur](0,6.5){B} \uput[u](2.5,6.5){C} 
\uput[ur](-0.9,2.3){E} \uput[ur](0,2.3){F} \uput[ul](0.9,2.3){G} 
\uput[r](0,0){S} 
\psline(-0.45,2.4)(-0.45,2.2)\psline(0.45,2.4)(0.45,2.2)
\psline(-1.25,6.6)(-1.25,6.4)\psline(1.25,6.6)(1.25,6.4)
\psline(-1.35,6.6)(-1.35,6.4)\psline(1.35,6.6)(1.35,6.4)
\psline[linestyle=dashed]{<->}(1.2,0)(1.2,2.3)\uput[r](1.2,1.15){5 cm}
\psline[linestyle=dashed]{<->}(3,0)(3,6.5)\uput[r](3,3.25){20 cm}
\psline[linestyle=dashed]{<->}(-2.5,7.)(2.5,7.)\uput[u](0,7){12 cm}
\end{pspicture}
\end{center}

B est le milieu de [AC] 

F est le milieu de [EG] 

BS = 20 cm ; FS = 5 cm ;  AC = 12 cm 
}

\medskip

\textbf{Les acheteurs }: 

80\,\% des acheteurs prennent de la sauce tomate et tous les autres prennent de la mayonnaise. 
 
\medskip

\textbf{Les sauces} : 

La bouteille de mayonnaise est assimilée à un cylindre de révolution dont le diamètre de base est 5~cm et la hauteur est 15 cm. 

La bouteille de sauce tomate a une capacité de $500$~mL. 

\medskip

\begin{enumerate}
\item Montrer que le rayon [EF] du cône de sauce a pour mesure $1,5$ cm. 
\item Montrer que le volume de sauce pour un cône de frites est d'environ $11,78$~cm$^3$ 
\item Déterminer le nombre de bouteilles de chaque sauce que Jean devra acheter. 

\emph{Toute trace de recherche même non aboutie devra apparaître sur la copie.} 
\end{enumerate}

\medskip

\textbf{Rappels: } Volume d'un cône de révolution : $\dfrac{\pi \times \text{rayon}^2 \times \text{hauteur}}{3}$

\phantom{Rappels: } Volume d'un cylindre de révolution  : $\pi \times  \text{rayon}^2 \times \text{hauteur}$ 

\phantom{Rappels: } \np{1000} cm$^3 = 1$ Litre 

