
\medskip 

Sur une facture de gaz, le montant à payer tient compte de l'abonnement annuel et du prix
correspondant au nombre de kilowattheures (kWh) consommés.

Deux fournisseurs de gaz proposent les tarifs suivants :

\begin{center}
\begin{tabularx}{0.75\linewidth}{|*{3}{>{\centering \arraybackslash}X|}}\hline
	&\textbf{Prix du kWh} &\textbf{Abonnement annuel}\\ \hline
\textbf{Tarif A (en \euro)}& \np{0,0609} &202,43\\ \hline
\textbf{Tarif B (en \euro)}& \np{0,0574} &258,39\\ \hline
\end{tabularx}
\end{center}

En 2016, la famille de Romane a consommé \np{17500} kWh. Le montant annuel de la facture de
gaz correspondant était de \np{1268,18}~\euro.

\medskip

\begin{enumerate}
\item Quel est le tarif souscrit par cette famille?
Depuis 2017, cette famille diminue sa consommation de gaz par des gestes simples (baisser le chauffage de quelques degrés, mettre un couvercle sur la casserole d'eau pour la porter à ébullition, réduire le temps sous l'eau dans la douche, etc.).
\item En 2017, cette famille a gardé le même fournisseur de gaz, mais sa consommation en kWh a diminué de 20\,\% par rapport à celle de 2016.
	\begin{enumerate}
		\item Déterminer le nombre de kWh consommés en 2017.
		\item Quel est le montant des économies réalisées par la famille de Romane entre 2016 et 2017 ?
	\end{enumerate}
\item On souhaite déterminer la consommation maximale assurant que le tarif A est le plus
avantageux. Pour cela:

\setlength\parindent{9mm}
\begin{itemize}
\item[$\bullet~~$] on note $x$ le nombre de kWh consommés sur l'année.
\item[$\bullet~~$] on modélise les tarifs A et B respectivement par les fonctions $f$ et $g$ :

\[f(x) = \np{0,0609}x + 202,43 \quad \text{et}\quad  g(x) = \np{0,0574}x + 258,39.\]
\end{itemize}
\setlength\parindent{0mm}

	\begin{enumerate}
		\item Quelles sont la nature et la représentation graphique de ces fonctions ?
		\item Résoudre l'inéquation: $f(x) < g(x)$.
		\item En déduire une valeur approchée au kWh près de la consommation maximale pour
laquelle le tarif A est le plus avantageux.
	\end{enumerate}
\end{enumerate}

\vspace{0.5cm}

