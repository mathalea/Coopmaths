
\medskip

Francis veut se lancer dans la production d'œufs biologiques. Son terrain est un rectangle
de $110$~m de long et $30$~m de large.

\medskip

\parbox{0.5\linewidth}{Il va séparer ce terrain en deux parties rectangulaires (voir schéma ci-contre qui n'est pas à l'échelle) :

\begin{itemize}
\item une partie couverte ;
\item une partie \og  plein air \fg.
\end{itemize}}\hfill
\parbox{0.45\linewidth}{\psset{unit=0.6cm}
\begin{pspicture}(11,6)
\psframe(1.5,0)(11,4.6)
\psline[linestyle=dotted,linewidth=1.5pt]{<->}(1.5,5)(11,5)
\uput[u](6.25,5){110~m}
\psline[linestyle=dotted,linewidth=1.5pt]{<->}(1.2,0)(1.2,4.6)\uput[l](1.2,2.3){30~m}
\psline[linestyle=dashed,linewidth=1.5pt](4.2,0)(4.2,4.6)
\rput(3,2.6){Partie}
\rput(3,1.8){couverte}
\rput(7.6,2.6){Partie}
\rput(7.6,1.8){\og plein air \fg}
\end{pspicture}}

\medskip

Pour avoir la qualification \og biologique \fg, Francis a l'obligation de respecter les deux règles ci-dessous.

\begin{center}
\begin{tabularx}{0.6\linewidth}{|*{2}{>{\centering \arraybackslash}X|}}\hline
\textbf{Partie couverte:}& \textbf{Partie \og Plein air \fg{} :}\\
utilisée pour toutes les& utilisée pour toutes les\\
poules quand il fait nuit &poules quand il fait jour\\ \hline
6 poules maximum par m$^2$& 4 m$^2$ minimum par poule\\ \hline
\multicolumn{2}{r}{\footnotesize \emph{(Source: Institut Technologique de l'agriculture Biologique)}}\\
\end{tabularx}
\end{center}

Il a prévu que la partie couverte ait une surface de 150 m$^2$.

\smallskip

\emph{Toute trace de recherche, même incomplète, pourra être prise en compte dans la notation.}

\medskip

\begin{enumerate}
\item Montrer que l'aire de la partie \og plein air\fg{} est de \np{3150} m$^2$.
\item Peut-il élever $800$ poules dans son installation?
\item Combien de poules au maximum pourrait-il élever dans son installation ?
\end{enumerate}

\bigskip

