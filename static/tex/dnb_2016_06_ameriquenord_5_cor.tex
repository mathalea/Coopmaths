\textbf{\textsc{Exercice 5 \hfill 5 points}}

\medskip

\begin{enumerate}
\item 
	\begin{enumerate}
		\item Tarif 1 : $2 \times 40,50 = 81$
		
Tarif 2 : $31 + 2 \times 32 = 31 + 64 = 95$.
		
Pour deux journées de ski, le tarif le plus intéressant est le tarif 1 avec 81~\euro contre 95~\euro
pour le tarif 2.
		\item Je cherche $x$ tel que : Tarif $2 <$ Tarif 1

$32x + 31 < 40,5x$

$32x - 32x + 31 < 40,5x - 32x$

$31 < 8,5x$

$\dfrac{31}{8,5} < \dfrac{8,5x}{8,5}$

$\dfrac{31}{8,5} < x$. Or $\dfrac{31}{8,5} \approx 3,6$.

Le tarif 2 est plus intéressant que le tarif 1 à partir de 4 journées de ski.
	\end{enumerate}
\item 
	\begin{enumerate}
		\item Le prix payé est proportionnel au nombre de jours skiés avec le tarif 1 puisque le
graphique est une droite qui passe par l'origine du repère.
		\item Pour 6 jours de ski, la différence entre les deux tarifs est d'environ 20~\euro.
		
$245 - 225 = 20$.
		\item Avec 275~\euro, Elliot peut skier 6 jours maximum avec le tarif 1 et 7 maximum avec le
tarif 2.
	\end{enumerate}
\end{enumerate}


\vspace{0,5cm}

