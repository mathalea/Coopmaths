

%\medskip
%\begin{minipage}{0.61\linewidth}
%Pour tracer une \og rue \fg, on a défini le tracé d'une \og maison \fg. \\
%\begin{minipage}{0.45\linewidth}
%\begin{center}
%\begin{scratch}
%\initmoreblocks{définir \namemoreblocks{maison}}
%\blockmove{tourner \turnleft{} de \ovalnum{90} degrés}
%\blockmove{avancer de \ovalnum{50}}
%\blockmove{tourner \turnright{} de \ovalnum{45} degrés}
%\blockmove{avancer de \ovalnum{50}}
%\blockmove{tourner \turnright{} de \ovalnum{90} degrés}
%\blockmove{avancer de \ovalnum{50}}
%\blockmove{tourner \turnright{} de \ovalnum{45} degrés}
%\blockmove{avancer de \ovalnum{50}}
%\blockmove{tourner \turnleft{} de \ovalnum{90} degrés}
%\end{scratch}
%%\includegraphics[width=65mm]{brevet-CE-17-fig1.eps}
%\end{center}
%\end{minipage}\hfill
%\begin{minipage}{0.345\linewidth}
%\begin{center}
%\psset{unit=1cm,arrowsize=2pt 4}
%\begin{pspicture}(-.1,-.1)(2,2.7)
%\psline(0,0)(0,1.5)(1,2.5)(2,1.5)(2,0)
%\psline{<->}(0,.9)(2,.9)\rput(1,1.1){$d$}
%\end{pspicture}\\
%
%tracé de la \og maison \fg. 
%\end{center}
%\end{minipage}
%\end{minipage}\hfill
%\begin{minipage}{0.35\linewidth}
%\begin{center}
%\begin{scratch}
%\blockinit{Quand \greenflag est cliqué}
%\blocklook{cacher}
%\blockpen{mettre la taille du stylo à \ovalnum{1}}
%\blockmove{aller à x:\ovalnum{-240} y: \ovalnum{0}}
%\blockpen{effacer tout}
%\blockpen{stylo en position écriture}
%\blockmove{s'orienter à  \ovalnum{90\selectarrownum}}
%\blockrepeat{répéter \ovalnum{$n$} fois}
%{
%\blockmoreblocks{maison}
%\blockmove{avancer de \ovalnum{20} }
%}
%\end{scratch}
%%\includegraphics[width=65mm]{brevet-CE-17-fig2.eps}
%programme principal
%\end{center}
%\end{minipage}\\

\begin{enumerate}
\item  %Vérifier que $d$ est environ égal à 71 à l'unité près.
Le sommet de la maison est un triangle rectangle d'hypoténuse $d$ et dont les autres côtés mesurent 50 unités. D'après le théorème de Pythagore on a donc :

$d^2 = 50^2 + 50^2 = \np{2500} + \np{2500} = \np{5000}$, donc $d = \sqrt{\np{5000}} \approx 70,7$ soit 71 unités à l'unité près. 

\item  %Un point dans une fenêtre d'exécution de votre programme a son abscisse qui peut varier de $-240$ à $240$ et son ordonnée qui peut varier de $-180$ à $180$. \\

%\begin{minipage}{0.61\linewidth}
%Quel est le plus grand, nombre entier $n$ que l'on peut utiliser dans le programme principal pour que le tracé de la \og rue \fg{} tienne dans la fenêtre de votre ordinateur où s'exécute le programme ? \\
Chaque motif (maison plus avancée de 20 unités) prend horizontalement environ 91 unités.

Or $5 \times 91 = 459$ et $6 \times 91 = 546$.

On peut donc démarrer à $- 240$ et dessiner 5 motifs soit 5 maisons.
%\end{minipage}\hfill
%\begin{minipage}{0.35\linewidth}
%\psset{unit=.6cm}
%\begin{pspicture}(-.1,-.1)(9,1.7)
%\psline(0,0)(0,1.5)(1,2.5)(2,1.5)(2,0)(2.3,0)(2.3,1.5)(3.3,2.5)(4.3,1.5)(4.3,0)(4.6,0)(4.6,1.5)(5.6,2.5)(6.6,1.5)(6.6,0)(6.9,0)
%\psline(7.2,0)(7.5,0)\psline(7.8,0)(8.1,0)\psline(8.4,0)(8.7,0)
%
%\end{pspicture}
%\end{minipage}\\
%\textit{Vous pourrez tracer sur votre copie tous les schémas (à main levée ou non) qui auront permis de répondre à la question précédente et ajouter toutes les informations utiles (valeurs, codages, traits supplémentaires, noms de points .. .) }


\item %\textit{ Attention, cette question est indépendante des questions précédentes et la « maison » est légèrement différente. }\\

%Si on désire rajouter une sortie de cheminée au tracé de la maison pour la rendre plus réaliste, il faut faire un minimum de calculs pour ne pas avoir de surprises.

\end{enumerate}

\medskip
%Exemples : 
%
%\psset{unit=0.75cm}
%\hspace{25mm}\begin{pspicture}(-1.1,-.1)(15,2.7)
%\psline(0,0)(0,1.5)(0.3,1.8)(.3,2.35)(.7,2.35)(.7,2.2)(1,2.5)(1.8,1.5)(1.8,0.3)
%\psline(2.5,0.2)(2.5,1.8)(2.8,2.1)(2.8,2.7)(3.2,2.7)(3.2,2.2)(3.5,2.5)(4.5,1.5)(4.5,0)
%\psline(5.5,0)(5.5,1.5)(5.9,1.9)(5.55,2.25)(5.75,2.45)(6.1,2.1)(6.5,2.5)(7.5,1.5)(7.5,0)
%\end{pspicture}
%
%\begin{minipage}{0.45\linewidth}
%On suppose que : 
%
%\begin{itemize}
%\item les points $H$, $E$ et $A$ sont alignés; 
%
%\item les points $C$, $M$ et $A$ sont alignés; 
%
%\item  $[CH]$ et $[EM]$ sont perpendiculaires à $[HA]$ ; 
%
%\item $AM = 16$ ; 
%
%\item $MC = 10$ ; 
%
%\item $\widehat{HAC}=30^\circ$. 
%\end{itemize}
%\medskip

%Calculer $EM$, $HC$ et $HE$ afin de pouvoir obtenir une belle sortie de cheminée. 

%\end{minipage}\hfill
%\begin{minipage}{0.45\linewidth}
%\psset{unit=.9cm}
%\begin{pspicture}(-.5,-.1)(9,5.2)
%\pstGeonode[PointName=none,PointSymbol=none](0,0){O}
%\pstGeonode[PointName=none,PointSymbol=none](0,1.5){T}
%\pstGeonode[PosAngle=45,PointSymbol=none](6.2,5){A}
%\pstHomO[PosAngle=-45,HomCoef=0.5417,PointSymbol=none]{A}{T}[C]
%\pstHomO[PosAngle=-45,HomCoef=0.2639,PointSymbol=none]{A}{T}[M]
%\pstGeonode[PointName=none,PointSymbol=none](0,5){Z}
%\pstProjection[PosAngle=90,PointSymbol=none]{Z}{A}{C}[H]
%\pstProjection[PosAngle=90,PointSymbol=none]{Z}{A}{M}[E]
%\pstLineAB{A}{M}
%\pstLineAB{E}{M}
%\pstLineAB{E}{H}
%\pstLineAB{H}{C}
%\pstLineAB{C}{T}
%\pstLineAB[linestyle=dashed]{C}{M}
%\pstLineAB{O}{T}
%\pstLineAB[linestyle=dashed]{E}{A}
%\pstMarkAngle[LabelSep=0.7]{E}{A}{M}{{\tiny $30^\circ$}}
%\pstGeonode[PointName=none,PointSymbol=none](8.2,3.87){R}
%\pstLineAB{A}{R}
%\rput(4.1,1.8){\textit{Ce schéma n'est pas}} 
%\rput(4.1,1.3){\textit{en vraie grandeur.}} 
%\end{pspicture}
%
%\end{minipage}
Dans le triangle AEM rectangle en A, on a $\sin \widehat{\text{EAM}} = \dfrac{\text{EM}}{\text{AM}}$, soit $\dfrac{1}{2} = \dfrac{\text{EM}}{16}$ soit $\text{EM} = 16 \times \dfrac{1}{2} = 8$.

De la même façon dans le triangle AHC rectangle en H, $\dfrac{1}{2} = \dfrac{\text{HC}}{16+10}$ soit 

$\text{HC} = 26 \times \dfrac{1}{2} = 13$.

D'autre part $\text{AE} = \text{AM} \times\cos 30 \approx 13,86$ et 

$\text{AH} = \text{AC} \times\cos 30 \approx 22,52$, donc $\text{HE} = \text{AH} - \text{AE} \approx 22,52 - 13,86$, donc  $\text{HE} \approx 8,66$.

\bigskip

