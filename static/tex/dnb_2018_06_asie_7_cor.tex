
\medskip

\begin{enumerate}
\item Le terrain a une aire de : $110 \times 30 = \np{3300}$~m$^2$.

Si la partie couverte a une aire de 150~m$^2$, il reste pour  la partie \og plein air \fg{} : $\np{3300} - 150 = \np{3150}$~m$^2$.
\item Il peut mettre au maximum dans la partie couverte : $6 \times 150 = 900$~poules ; il peut donc mettre dans la partie couverte 800~poules.

Ces  800~poules auront besoin dans la journée de $4 \times 800 = \np{3200}$~m$^2$ : or la partie \og plein air \fg{} ne fait que \np{3150}~m$^2$ : la règle 2 n'est pas respectée. Il ne peut pas élever 800~poules.
\item La partie \og plein air a une  d'aire de \np{3150}~m$^2$ et puisqu'il faut 4~m$^2$ minimum par poule, on pourra mettre au maximum $\dfrac{\np{3150}}{4} = 787,5$~poules.

On peut donc mettre au maximum 787~poules. 

\end{enumerate}

\bigskip

