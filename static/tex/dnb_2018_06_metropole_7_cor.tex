
\medskip

\begin{enumerate}
\item La représentation graphique n'est pas une droite passant par l'origine, donc le temps et la vitesse de rotation ne sont pas proportionnelles.
\item 
	\begin{enumerate}
		\item 20 tours par seconde.
		\item 1 min 20 s égale 80 s.
		
La vitesse de rotation est à 3 tours par seconde.
		\item Le hand-spinner s'arrêtera au bout de $93$ secondes.
 	\end{enumerate}
\item  
	\begin{enumerate}
		\item $V(t) = - 0,214 \times t + 20$ où $t = 30$ (s) ;
		
$V(t) = -0,214 \times 30 + 20$ ;
		
$V(t) = 13,58$ tours/s
		\item Lorsque le hand-spinner s'arrête, sa vitesse est égale à $0$.
		
$0 = -0,214 \times t + 20$ ; $0,214t = 20$ ;

$t = \dfrac{20}{0,214} \approx 93,46$~(s).
		\item On calcule le temps nécessaire pour que le hand-spinner s'arrête lorsque la vitesse initiale est de 40 (tours/s).

$0 = - 0,214t + 40$  soit $0,214t = 40$ et $t = \dfrac{40}{0,214} \approx 186,92$.

Or : $2 \times 93,46 = 186,92$.
	\end{enumerate}
\end{enumerate}
