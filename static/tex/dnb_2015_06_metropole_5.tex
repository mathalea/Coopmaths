\textbf{Exercice 5 \hfill 6 points}

\medskip

Agnês envisage de peindre la façade de son hangar.

\medskip

\begin{tabular}{|m{5.5cm}|m{6.5cm}|}\hline
\textbf{Information  1 : Caractéristiques de }

\textbf{la peinture utilisée.}

\begin{center}
Renseignements concernant un pot de peinture

\begin{tabular}{|c|}\hline
Volume : 6 l\\
Temps de séchage : 8 h\\
Surface couverte : 24 m$^2$\\
Monocouche*\\
Prix : 103,45~\euro\\ \hline
\end{tabular}
\end{center}
* Une seule couche de peinture suffit.&
\textbf{Information 2 : schéma de la façade}

(le schéma n'est pas à l'échelle)

La zone grisée est la zone à peindre.

\begin{center}
\psset{unit=1cm}
\begin{pspicture}(6,7)
%\psgrid
\pspolygon[fillstyle=solid,fillcolor=lightgray](0.5,0.5)(5.5,0.5)(5.5,4.5)(3,6.5)(0.5,4.5)%AEDCB
\psline[linestyle=dashed](0.5,4.5)(5.5,4.5)%BD
\psline{->}(3.4,0.3)(5.5,0.3)\psline{->}(2.6,0.3)(0.5,0.3)\rput(3,0.3){7,5 m}
\psline{->}(0.3,2.2)(0.3,0.5)  \psline{->}(0.3,2.8)(0.3,4.5)\rput(0.2,2.5){6 m}
\psline{->}(5.7,3.8)(5.7,6.5)  \psline{->}(5.7,3.2)(5.7,0.5)\rput(5.8,3.5){9 m}
\psline[linestyle=dashed](3,6.5)(5.7,6.5)
\uput[dl](0.5,0.5){A} \uput[dr](5.5,0.5){E}\uput[ur](5.5,4.5){D}\uput[u](3,6.5){C}
\uput[ul](0.5,4.5){B}
\psframe(0.5,4.5)(0.8,4.2)\psframe(0.5,0.5)(0.8,0.8)\psframe(5.5,0.5)(5.2,0.8)
\end{pspicture}
\end{center}
\\ \hline
\end{tabular}

\medskip

\begin{enumerate}
\item Quel est le montant minimum à prévoir pour l' achat des pots de peinture ?
\item Agnès achète la peinture et tout le matériel dont elle a besoin pour ses travaux. Le montant total de la facture est de 343,50~\euro.

Le magasin lui propose de régler $\frac{2}{5}$ de la facture aujourd'hui et le reste en trois mensualités identiques.

Quel sera le montant de chaque mensualité ?
\end{enumerate}
 
\vspace{0,5cm}

