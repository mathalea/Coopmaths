\textbf{Exercice 7 \hfill 5 points}

\medskip

%\textbf{Document 1 :} Extrait de la liste alphabétique des élèves de la 3\up{e} 4 et d'informations relevées en E. P. S. pour préparer des épreuves d'athlétisme.
%
%\medskip
%\begin{tabularx}{\linewidth}{|*{5}{>{\centering \arraybackslash}X|}}\hline 
%\textbf{Prénoms} &\textbf{Date de naissance} &\textbf{Année} &\textbf{Taille en m}&\textbf{Nombre de pas réalisés sur 100 m}\\ \hline
%Lahaina &26-oct. &1997 &1,81 &110\\ \hline 
%Manuarii &20-mai &1997 &1,62 &123\\ \hline 
%Maro-Tea &5-nov. &1998 &1,56 &128\\ \hline 
%Mehiti &5-juin &1997 &1,60 &125\\ \hline 
%Moana &10-déc. &1997 &1,80 &111\\ \hline 
%Rahina &14-mai &1997 &1,53 &130\\ \hline
%\end{tabularx}
%\bigskip
% 
%\textbf{Document 2 :} Dans le croquis ci-dessous, le tiki représente Moana, élève de 3\up{e} 4.
%
%\begin{center}
%\psset{unit=1cm}
%\begin{pspicture}(11,7.3)
%%\psgrid
%\pscurve[fillstyle=solid,fillcolor=lightgray]
%(2.1,0)(2.05,0.2)(2.1,3)(2.2,3.8)(2.25,3.75)(2,3.8)(1.6,3.54)(1.8,3.15)(1.9,3.1)
%(1.6,3.2)(1.5,3.5)(1.7,4.)(1.5,4.1)(1.1,4)(1,3.9)(0.9,3.6)(0.85,3.3)(0.75,3.6)
%(0.8,4)(1,4.3)(1.5,4.5)(1.7,4.4)(1.4,4.75)(1,4.7)(0.8,4.3)(1,4.9)
%(1.4,5.2)(1.85,5.1)(1.5,5.4)(1.3,5.5)(1.1,5.45)(1.5,5.6)(2,5.4)(2.1,5.6)(2.2,5.3)
%(2.6,5.25)(2.3,5.2)(2.6,5.2)(3,5.05)(2.6,5)(2.4,4.8)(3,4.95)
%(3.5,4.8)(3.7,4.7)(3.85,4.3)(3.8,4.1)(3.7,4.3)(3,4.7)(2.6,4.4)(2.8,4.45)(3,4.4)(3.75,4)(4,3.4)(3.85,3)(3.7,2.8)(3.5,3.6)(3.2,4)(3,4.1)(2.4,4)
%(2.3,4)(2.3,3)(2.4,0.3)(2.35,0)(2.1,0)
%\multido{\n=3.2+0.8}{10}{\pscircle*(\n,0.1){0.1}}
%\psline[linestyle=dashed](10.45,0)(1,6.65)
%\psline(7.8,0)(7.9,0.4)(8.1,0.4)(8.2,0)(8.3,0)(8.2,1)(8.22,1)(8.4,0.3)(8.5,0.32)(8.22,1.2)(8.12,1.2)
%\pscurve(8.12,1.2)(8.22,1.4)(8.13,1.58)(8,1.6)(7.87,1.58)(7.78,1.4)(7.88,1.2)
%\psline(7.88,1.2)(7.78,1.2)(7.5,0.32)(7.6,0.3)(7.78,1)(7.8,1)(7.7,0)(7.8,0)
%\rput{-35}(0.5,7.1){Soleil}\rput(8,2){Tiki}
%\psline(0,0)(11,0)
%\end{pspicture}
%\end{center}
% 
%Moana a d'abord posé sur le sol, \textbf{à partir du cocotier}, des noix de coco régulièrement espacées à chacun de ses pas, puis il s'est ensuite placé exactement comme indiqué sur le croquis, au niveau de la 7\up{e} noix de coco.
% 
%\textbf{À l'aide d'informations qui proviennent des documents précédents}, calcule la hauteur du cocotier en expliquant clairement ta démarche.
% 
%\emph{Dans cet exercice, tout essai, toute idée exposée et toute démarche, même non aboutis ou mal formulés seront pris en compte pour l'évaluation.} 
Moana a un pas qui fait en moyenne : $\dfrac{100}{111}$. D’après sa fiche Moana a une traille de 1,80~m.

Moana et l’arbre étant verticaux sont parallèles ; on a clairement une situation de Thalès  ; le théorème permet d’écrire, $h$ étant la hauteur du cocotier :

$\dfrac{1,8}{h} = \dfrac{3}{10}$, d’où $3h = 18$ et donc $h = 6$~(m).
\bigskip

