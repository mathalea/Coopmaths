\textbf{Exercice 5 \hfill 4 points}

\medskip

%\parbox{0.4\linewidth}{On découpe la pyramide FIJK dans le cube ABCDEFGH
%comme le montre le dessin ci-contre.
%
%Le segment [AB] mesure 6 cm.
%
%Les points I, J, et K sont les milieux respectifs des arêtes
%[FE], [FB] et [FG].}\hfill
%\parbox{0.58\linewidth}{\psset{unit=1cm}
%\begin{pspicture}(6.7,5.5)
%%\psgrid
%\psline(6.2,1)(2.8,0.6)(1.4,1.9)(1.4,4.8)(2.8,3.5)(2.8,0.6)%BADHEA
%\psline(2.8,3.5)(6.2,3.9)(4.8,5.2)(1.4,4.8)%EFGH
%\psline(6.2,3.9)(6.2,1)%FB
%\uput[d](2.8,0.6){A}\uput[dr](6.2,1){B}\uput[r](4.8,2.3){C}\uput[l](1.4,1.9){D}
%\uput[l](2.8,3.5){E}\uput[r](6.2,3.9){F}\uput[ur](4.8,5.2){G}\uput[ul](1.4,4.8){H}
%\uput[u](4.1,3.65){I}\uput[u](5.5,4.55){K}\uput[r](6.2,3){J}
%\psline(6.2,3)(4.1,3.65)(5.5,4.55)\psline[linestyle=dashed](6.2,3)(5.5,4.55)
%\pspolygon(2.8,3.5)(1.4,4.8)(1.4,1.9)(2.8,0.6)%EHDA
%\end{pspicture}
%}
%
%\medskip

\begin{enumerate}
\item %Tracer le triangle IFK en vraie grandeur.
IFK est un triangle rectangle en F, de côtés FI = FK = $\dfrac{8}{2} = 4$~cm.

D'après la propriété de Pythagore IK vérifie :

$\text{IK}^2 = \text{FI}^2 + \text{FK}^2 = 3^2 + 3^2 = 2 \times 9$, donc $\text{IK} = \sqrt{9 \times 2} = \sqrt{9}\sqrt{2} = 3\sqrt{2}$. (Il n'est pas nécessaire de calculer cette longueur pour construire le triangle).
\item %Un des quatre schémas ci-dessous correspond au patron de la pyramide FIIK.

%Indiquer son numéro sur la copie. Aucune justification n'est attendue.
%
%\begin{center}
%\begin{tabularx}{\linewidth}{|*{2}{>{\centering \arraybackslash}X|}}\hline
%\psset{unit=0.9cm}
%\begin{pspicture}(-2,-2)(2,2)
%\pspolygon(-2,0)(0,-2)(2,0)(0,2)
%\psline(0,-2)(0,2)\psline(-2,0)(2,0)
%\psdots(1,1)(1,-1)(-1,1)(-1,-1)
%\psframe(0.3,0.3) \psframe(-0.3,-0.3)\psframe(0.3,-0.3)
%\end{pspicture}&\psset{unit=0.9cm}\begin{pspicture}(-2,-2)(2,4)
%\pspolygon(-2,0)(0,-2)(2,0)(0,2)(0,-2)
%\psline(-2,0)(2,0)
%\psdots(1,1)(1,-1)(1,3)(-1,-1)
%\psframe(0.3,0.3) \psframe(-0.3,-0.3)\psframe(0.3,-0.3)
%\psline(2,0)(2,4)(0,2)\rput{-45}(0,2){\psframe(0,0)(0.3,0.3)}
%\end{pspicture}\\ 
%Schéma 1 &Schéma 2\\ \hline
%\psset{unit=0.9cm}
%\begin{pspicture}(-2,-3)(3,2.25)
%%\psgrid
%\pspolygon(-2,0)(0,-2)(2,0)(0,2)(0,-2)
%\psline(-2,0)(2,0)
%\psdots(1,1)(1,-1)(-1,-1)(2.3,-1.3)(1.3,-2.3)
%\psframe(0.3,0.3) \psframe(-0.3,-0.3)\psframe(0.3,-0.3)
%%\psline(2,0)(2,4)(0,2)%\rput{-45}(0,2){\psframe(0,0)(0.3,0.3)}
%\psline(2,0)(2.6,-2.6)(0,-2)
%\end{pspicture}&\psset{unit=0.8cm}\begin{pspicture}(0,-1.5)(6.5,2.6)
%
%%\psgrid
%\psline(2.5,0)(5.1,0.1)(6.2,2.5)(3.7,2.4)(5.1,0.1)
%\psline(3.7,2.4)(2.5,0)
%\psline(3.7,2.4)(0.2,1.2)(2.5,0)(1.95,1.8)
%\psdots(1.3,0.6)(3.1,1.2)(3.8,0.05)(4.35,1.3)(5,2.42)(5.75,1.5)
%\psdots[dotstyle=+,dotangle=45](2.25,0.8)(2.7,2.08)\rput{-70}(1.95,1.8){\psframe(0.3,0.3)}
%\end{pspicture}\\ 
%Schéma 3 &Schéma 4\\ \hline
%\end{tabularx}
%\end{center}
Les trois triangles rectangles IFK, IFJ et KFJ sont des triangles superposables, d'hypoténuses IK, IJ et KJ de longueur $3\sqrt{2}$.

Le patron se compose donc de trois triangles rectangles de même sommet F et d'un triangle équilatéral. Les seul patron possible est celui du schéma 3. 
\item %Calculer le volume de la pyramide FIJK.
En prenant par exemple comme base le triangle rectangle IFJ et donc [FK] comme hauteur, on a :

$V = \dfrac{3 \times 3}{2} \times 3\times \dfrac{1}{3} = \dfrac{9}{2} = 4,5$~cm$^3$.
%\emph{Rappel : Volume d'une pyramide} $= \dfrac{\text{\emph{Aire d'une base}} \times \text{\emph{hauteur}}}{3}$
\end{enumerate}

\bigskip

