\textbf{Exercice 8 : Jeu vidéo \hfill 7 points}

\bigskip
 
%Dans un jeu vidéo on a le choix entre trois personnages : un guerrier, un mage et un chasseur. 
%
%La force d'un personnage se mesure en points.
% 
%Tous les personnages commencent au niveau $0$ et le jeu s'arrête au niveau $25$. 
%
%Cependant ils n'évoluent pas de la même façon :
%
%\setlength\parindent{8mm} 
%\begin{itemize}
%\item[\decoone~~] Le guerrier commence avec 50 points et ne gagne pas d'autre point au cours du jeu. 
%\item[\decoone~~] Le mage n'a aucun point au début mais gagne 3 points par niveau. 
%\item[\decoone~~] Le chasseur commence à 40 points et gagne 1 point par niveau. 
%\end{itemize}
%\setlength\parindent{0mm}
%
%\medskip

\begin{enumerate}
\item %Au début du jeu, quel est le personnage le plus fort ? Et quel est le moins fort ? 
Le plus fort est le guerrier, le moins fort le mage.
\item %Compléter le tableau de l'annexe 2. 
\item %À quel niveau le chasseur aura-t-il autant de points que le guerrier ?
Il faut résoudre l'équation :

$50 = 40 + x$  soit $x = 10$.
\item %Dans cette question, x désigne le niveau de jeu d'un personnage. 

%Associer chacune des expressions suivantes à l'un des trois personnages: chasseur, mage ou guerrier :
% 
%\setlength\parindent{8mm} 
%\begin{itemize}
%\item[$\bullet~~$] $f(x) = 3x$ ; 
%\item[$\bullet~~$] $g(x) = 50$ ; 
%\item[$\bullet~~$] $h(x) = x + 40$.
%\end{itemize}
%\setlength\parindent{0mm}
 
\item %Dans le repère de l'annexe 2, la fonction $g$ est représentée. 
$f(x)$ désigne le nombre de points du mage ;

$g(x)$ désigne le nombre de points du guerrier ;

$h(x)$ désigne le nombre de points du chasseur.

%Tracer les deux droites représentant les fonctions $f$ et $h$.
Voir le dessin à la fin. 
\item %Déterminer à l'aide du graphique, le niveau à partir duquel le mage devient le plus fort.
Le mage devient le plus fort à partir du niveau 21. 
\end{enumerate} 

\newpage

\begin{center}
{\large \textbf{ANNEXE 1 - Exercice 7}}

\bigskip

\begin{tabularx}{\linewidth}{|c|m{2cm}|*{6}{>{\centering \arraybackslash}X|}}\hline
&A&B&C&D&E&F&G\\ \hline
1&Catégorie&\multicolumn{2}{|c|}{Junior}&\multicolumn{2}{|c|}{Intermédiaire}&\multicolumn{2}{|c|}{Sénior}\\ \hline 
2&Effectif par catégorie&\multicolumn{1}{>{\columncolor{lightgray}}c|}{\quad}&\np{1958}&\multicolumn{1}{>{\columncolor{lightgray}}c|}{\quad}&876&\multicolumn{1}{>{\columncolor{lightgray}}c|}{\quad}&308\\ \hline 
3&Niveau&5\up{e}&4\up{e}&3\up{e}&2\up{nde}&1\up{re}&Term\\ \hline 
4&Effectif par niveau& 989&969& 638&238& 172&136\\ \hline 
5&\multicolumn{6}{|c|}{Effectif total}&\np{3142}  \\ \hline
\end{tabularx}

\vspace{1cm}

{\large \textbf{ANNEXE 2 - Exercice 8}}

\bigskip

\begin{tabularx}{\linewidth}{|m{1.75cm}|*{6}{>{\centering \arraybackslash}X|}}\hline
Niveau						&0	&1	&5	&10	&15	&25\\ \hline
\small Points du guerrier	&50	&50	&50	&50	&50	&50\\ \hline
\small Points du mage		&0	&3	&6	&9	&12	&15\\ \hline
\small Points du chasseur	&40	&41	&42	&43	&44	&45\\ \hline
\end{tabularx}

\vspace{1cm}

\psset{xunit=0.4cm,yunit=0.08cm}
\begin{pspicture}(-1,-5)(27,75)
\multido{\n=0+1}{28}{\psline[linestyle=dashed,linewidth=0.3pt](\n,0)(\n,75)}
\multido{\n=0+5}{15}{\psline[linestyle=dashed,linewidth=0.3pt](0,\n)(27,\n)}
\psline[linewidth=1.5pt](0,50)(27,50)
\psaxes[linewidth=1.5pt,Dy=5]{->}(0,0)(-0.9,-4)(27,75)
\uput[u](25.5,0){Niveau}
\uput[r](0,72.5){Points}
\uput[dl](0,0){O}
\psline(24,72)
\psline(0,40)(27,67)
\rput{30}(2,7){mage}\uput[u](2,50){guerrier} \rput{12}(2,40){chasseur}
\psline[ArrowInside=->,linecolor=blue](0,60)(20,60)(20,0)
\end{pspicture}
\end{center}
