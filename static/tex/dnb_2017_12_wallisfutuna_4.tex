
\medskip

\begin{enumerate}
\item On souhaite tracer le motif ci-dessous en forme de losange.

Compléter sur l'annexe 1, le script du bloc Losange afin d'obtenir ce motif.

\medskip

\begin{tabularx}{\linewidth}{|X|X|}\hline
Le motif \textbf{Losange}&Le bloc \textbf{Losange}\\
\psset{unit=0.6cm}
\begin{pspicture}(0,-5)(10,4)
%\psgrid
\pspolygon(0.5,1)(5.6,1)(10,3.5)(4.9,3.5)
\psarc(0.5,1){4mm}{0}{30} \psarc(4.9,3.5){4mm}{210}{360}
\rput(1,2.5){\scriptsize Point de départ}
\psline{->}(0.5,2.2)(0.5,1.1)
\psline{<->}(0.5,0.8)(5.6,0.8)\uput[d](3.05,0.8){\footnotesize 60}
\rput(1.6,1.3){\footnotesize 30\degres}\rput(5.1,2.5){\footnotesize 150\degres}
\end{pspicture}&\footnotesize{\begin{scratch}
\initmoreblocks{définir \namemoreblocks{Losange}}
\blockpen{stylo en position d'écriture}
\blockmove{avancer de \ovalnum{}}
\blockmove{tourner \turnleft{} de \ovalnum{30} degrés}
\blockmove{avancer de \ovalnum{}}
\blockmove{tourner \turnleft{} de \ovalnum{150} degrés}
\blockmove{avancer de \ovalnum{}}
\blockmove{tourner \turnleft{} de \ovalnum{} degrés}
\blockmove{avancer de \ovalnum{}}
\blockmove{tourner \turnleft{} de \ovalnum{} degrés}
\blockpen{relever le stylo}
\end{scratch}}\\\hline
\end{tabularx}

\item On souhaite réaliser la figure ci-dessous construite à partir du bloc \textbf{Losange} complété à la question 1.

\begin{center}
\psset{unit=0.75cm}
\begin{pspicture}(-2.5,-2.5)(2.5,2.5)
\def\losange1{\pspolygon(0,0)(1.275,0)(2.375,0.625)(1.1,0.625)}
\multido{\n=0+30}{12}{\rput{\n}(0;0){\losange1}}
\end{pspicture}
\end{center}

\parbox{0.6\linewidth}{On rappelle que l'instruction \raisebox{-2.3ex}{\begin{scratch}\blockmove{s’orienter à \ovalnum{90\selectarrownum} degrés}\end{scratch}} signifie que l'on se dirige vers la droite.

\bigskip

Parmi les instructions ci-dessous, indiquer sur votre copie, dans
l'ordre, les deux instructions à placer dans la boucle ci-contre pour
finir le script.}\hfill
\parbox{0.37\linewidth}{\begin{scratch}
\blockinit{Quand \greenflag est cliqué}
\blockpen{effacer tout}
\blockmove{aller à x: \ovalnum0 y: \ovalnum0}
\blockmove{s’orienter à \ovalnum{90\selectarrownum} degrés}
\blockrepeat{répéter \ovalnum{12} fois}
{\blockspace[0.5]}
\end{scratch}}

\medskip

\begin{tabularx}{\linewidth}{|c|X|m{0.5cm}|c|X|}\cline{1-2}\cline{4-5}
\ding{'300}	&\raisebox{-2.3ex}{\begin{scratch}\blockmove{tourner \turnleft{} de \ovalnum{30} degrés}\end{scratch}}	&	&\ding{'301}		&\raisebox{-2.3ex}{\begin{scratch}\blockmove{tourner \turnleft{} de \ovalnum{150} degrés}\end{scratch}}\\ \cline{1-2}\cline{4-5}
\ding{'302}		&\raisebox{-2.3ex}{\begin{scratch} \blockmoreblocks{Losange}\end{scratch}}				&	&\ding{'303}		&\raisebox{-2.3ex}{\begin{scratch}\blockmove{avancer de \ovalnum{600}}\end{scratch}}\\ \cline{1-2}\cline{4-5}
\end{tabularx}
\end{enumerate}
\begin{center}
{\Large \textbf{Annexe 1 }}

\vspace{3cm}

\begin{scratch}
\initmoreblocks{définir \namemoreblocks{Losange}}
\blockpen{stylo en position d'écriture}
\blockmove{avancer de \ovalnum{}}
\blockmove{tourner \turnleft{} de \ovalnum{30} degrés}
\blockmove{avancer de \ovalnum{}}
\blockmove{tourner \turnleft{} de \ovalnum{150} degrés}
\blockmove{avancer de \ovalnum{}}
\blockmove{tourner \turnleft{} de \ovalnum{} degrés}
\blockmove{avancer de \ovalnum{}}
\blockmove{tourner \turnleft{} de \ovalnum{} degrés}
\blockpen{relever le stylo}
\end{scratch}
\end{center}

\vspace{0,5cm}

