
\medskip

\begin{enumerate}
\item On a $p(13) = \dfrac{1}{20}$.
\item Sur 20 boules, 10 portent un numéro pair, donc $p(\text{pair}) = \dfrac{10}{20} = \dfrac{1}{2}$.
\item Entre 1 et 20 ces deux nombres compris, les multiples de 4 sont : 4, 8, 12, 16 et 20 : il y a en a donc 5.

$p(\text{multiple de } 4) = \dfrac{5}{20} = \dfrac{5\times 1}{5\times 4} = \dfrac{1}{4}$.

Les diviseurs de 4 sont : 1, 2, et 4. Donc 

$p(\text{diviseur de } 4) = \dfrac{3}{20}$.

Comme $\dfrac{3}{20} < \dfrac{5}{20}$, la probabilité d'obtenir un multiple de 4 est plus grande que celle d'obtenir un diviseur de 4.
\item Les naturels premiers entre 1 et 20, sont :

2, 3, 5, 7, 11 , 13 17, 19, soit 8 naturels. Donc 

$p(\text{premier}) = \dfrac{8}{20} = \dfrac{4\times}{4 \times 5} = \dfrac{2}{5}$.
\end{enumerate}

\vspace{0,5cm}

