
\medskip

L'annexe 1 (à la fin) donne un tableau concernant les états et territoires de la Mélanésie.
\medskip

\begin{enumerate}
\item %Compléter les colonnes B et C du tableau dans l'annexe 1. Arrondir les fréquences au dixième.
Voir à la fin.
\item %Le tableau a été construit avec un tableur.

%Quelle formule peut-on saisir pour compléter la cellule B7 du tableau ?
On écrit en case B7  : =SOMME(B2 :~B6)

%L'annexe 2 (page 7/7) donne la répartition des superficies des différents territoires et états de la Mélanésie.

\item %Compléter la colonne des angles dans le tableau de l'annexe 2.
Voir l'annexe 2.
\item %Compléter le diagramme semi-circulaire dans l'annexe 2 en utilisant les données du tableau.
Voir l'annexe à la fin.
\end{enumerate}

\begin{center}
\textbf{ANNEXES À RENDRE AVEC LA COPIE}

\bigskip

\textbf{Annexe 1 - question 1}

\bigskip

\begin{tabularx}{\linewidth}{|l|*{3}{>{\centering \arraybackslash \small}X|}}\hline
	&A										&B								&C\\ \hline
1	&États ou territoires de la Mélanésie	&Superficie terrestre (en km$^2$)	& Fréquence (en \%)\\ \hline
2 	&îles Salomon							&\np{28530} 						&5,2\\ \hline
3 	&îles Fidji								&\np{18333}							&3,3\\ \hline
4 	&Nouvelle-Calédonie						&\np{18576}							& 3,4\\ \hline
5 	& Papouasie-Nouvelle-Guinée				&\np{472840}						&85,9\\ \hline
6	&Vanuatu 								&\np{12281}							&2,2\\ \hline
7	& TOTAL									&\np{558560}					&100\\ \hline
\end{tabularx}

\bigskip

\textbf{Annexe 2 - questions 3. et 4.}

\medskip

\begin{tabularx}{\linewidth}{|*{3}{>{\centering \arraybackslash \small}X|}}\hline
États ou territoires de la Mélanésie&Superficie terrestre (en km$^2$)&Angle (arrondi au degré près)\\ \hline
îles Salomon						&\np{28530}						&9\\ \hline
îles Fidji							&\np{18333}						&6\\ \hline
Nouvelle-Calédonie					&\np{18576} 					&6\\ \hline
Papouasie-Nouvelle-Guinée 			&\np{472840}					&155\\ \hline
Vanuatu 							&\np{12281}						&4\\ \hline
TOTAL 								&\np{550560}					&180\\ \hline
\end{tabularx}

\vspace{0,5cm}

\psset{unit=1cm}
\begin{pspicture}(-5,0)(5,5)
\psarc(0,0){5}{0}{180}\psline(-5,0)(5,0)
\multido{\n=0+1}{181}{\psline(4.8;\n)(5;\n)}
\multido{\n=0+5}{37}{\psline(4.6;\n)(5;\n)}
\multido{\n=0+10}{19}{\psline(4.4;\n)(5;\n)}
\multido{\n=0+10,\na=90+-10}{10}{\rput{\na}(4.2;\n){\n}}
\multido{\n=90+10,\na=0+10}{9}{\rput{\na}(4.2;\n){\n}}
\psline(5;9)\psline(5;15)\psline(5;21)\psline(5;176)
\rput{4.5}(4;4.5){îles Salomon}\rput{12}(4;12){îles Fidji}
\rput{18}(3;18){Nouvelle Calédonie} \rput{95}(3;95){Papouasie}\rput{178}(4;178){Vanuatu}
\end{pspicture}

\end{center}
\vspace{0,5cm}

