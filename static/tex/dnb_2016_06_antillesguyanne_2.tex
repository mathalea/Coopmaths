\textbf{\textsc{Exercice 2} \hfill 4,5 points}

\medskip

On considère les deux programmes de calcul ci-dessous. 

\begin{center}
\begin{tabularx}{0.75\linewidth}{|X|m{0.4cm}|X|}\cline{1-1}\cline{3-3}
\multicolumn{1}{|c|}{\textbf{Programme A}}&&\multicolumn{1}{|c|}{\textbf{Programme B}}\\\cline{1-1}\cline{3-3}  
1. Choisir un nombre.&&1. Choisir un nombre. \\ 
2. Multiplier par $-2$.&&2. Soustraire 7.  \\
3. Ajouter 13.&&3.  Multiplier par 3. \\\cline{1-1}\cline{3-3}
\end{tabularx}
\end{center} 

\begin{enumerate}
\item Vérifier qu'en choisissant 2 au départ avec le programme A, on obtient 9. 
\item Quel nombre faut-il choisir au départ avec le programme B pour obtenir 9 ? 
\item Peut-on trouver un nombre pour lequel les deux programmes de calcul donnent le même résultat ? 
\end{enumerate} 

\bigskip

