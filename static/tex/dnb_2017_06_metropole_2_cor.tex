\textbf{\textsc{Exercice 2} \hfill 6 points}

\medskip

\begin{enumerate}
\item Les coordonnées du point de départ du tracé sont $(-200~;~-100)$.
\item Le script permet de dessiner 5 triangles.
\item  
	\begin{enumerate}
		\item La longueur du côté du deuxième triangle tracé est de $80$ pixels.
		\item La figure obtenue :
\begin{center}
\psset{unit=1cm}
\begin{pspicture}(8,1.8)
\def\tria{\pspolygon(0,0)(2,0)(1,1.732)}
\def\trib{\pspolygon(0,0)(1.6,0)(0.8,1.386)}
\def\tric{\pspolygon(0,0)(1.2,0)(0.6,1.039)}
\def\trid{\pspolygon(0,0)(0.8,0)(0.4,0.693)}
\def\trie{\pspolygon(0,0)(0.4,0)(0.2,0.346)}
\rput(0,0){\tria}\rput(2,0){\trib}\rput(3.6,0){\tric}\rput(4.8,0){\trid}\rput(5.6,0){\trie}
\end{pspicture}
\end{center}
 	\end{enumerate}
\item  Il faut placer le bloc \og tournez le bloc de 60\degres \fg{} après l'instruction \no 9 du script initial pour obtenir cette nouvelle figure. 
\end{enumerate}

\bigskip

