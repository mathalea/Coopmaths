\textbf{\textsc{Exercice 3} \hfill 5,5 points}

\medskip

\parbox{0.47\textwidth}{La feuille de calcul ci-contre donne la production mondiale
de vanille en 2013.
\medskip
\begin{enumerate}
\item Quelle formule de tableur a été saisie dans la cellule B15 ?
\item À eux deux, l'Indonésie et Madagascar produisent-ils plus
des trois quarts de la production mondiale de vanille ?
\item On s'intéresse aux cinq  pays qui ont produit le moins de
vanille en 2013.

Quel pourcentage de la production mondiale représente la production de vanille de ces cinq pays ? Arrondir le résultat à l'unité.
\end{enumerate}}
\hfill
\parbox{0.47\textwidth}{
\begin{tabularx}{\linewidth}{c| X | X|}\hline
			&A	&B\\ \hline
1&Pays		&\footnotesize Production de vanille en 2013 (en milliers de tonnes)\\ \hline
2&Chine		&335\\ \hline
3&Comores	&35\\ \hline
4&France	&79\\ \hline
5&Indonésie	&\np{3200}\\ \hline
6&Kenya		&15\\ \hline
7&Madagascar&\np{3100}\\ \hline
8&Malawi	&22\\ \hline
9&Mexique	&463\\ \hline
10&Ouganda	&161\\ \hline
11&\footnotesize  Papouasie-Nouvelle-Guinée	& 433\\ \hline
12&Tonga	&198\\ \hline
13&Turquie	&290\\ \hline
14&Zinbabwe	&11\\ \hline
15&Total	&\np{8342}\\ \hline
\end{tabularx}}

\vspace{0,5cm}

