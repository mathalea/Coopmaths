
\begin{enumerate}
\item 
	\begin{enumerate}
		\item Léo choisit au départ le nombre $- 3$.
		Léo le multiplie par 6 : \quad $- 3 \times 6 = - 18$ ;
		puis Léo ajoute 5 : \quad  $- 18 + 5 = - 13$.
		Léo obtient $-13$.
		\item Julie choisit au départ le nombre $- 3$. Julie lui ajoute 8 : \quad  $- 3 + 8 = 5$ ;
Julie  multiplie le résultat $5$ par le nombre de départ $- 3$ : \quad  $5\times (-3)= - 15$ ;
puis Julie soustrait le carré du nombre de départ,   $-15 -(- 3)\times(- 3)$		$= - 15 - 9 =- 24$.
 Julie obtient $- 24$. 
\end{enumerate}
\pagebreak
\item ~

\vspace{-0.34cm}
\begin{minipage}{7cm}
Soit $x$ le nombre positif choisi au départ par Léo.

Léo le multiplie par 6, il obtient $x\times 6$, soit $6x$ ;

Puis Léo ajoute 5, il obtient $6x+5$.

\vspace{4cm}
\end{minipage}
\vline \hspace{0.5cm}\begin{minipage}{7cm}
Julie choisit le même nombre $x$ que Léo.

Julie lui ajoute 8, elle obtient $x+8$ ;

Julie  multiplie le résultat $x + 8$ par le nombre de départ $x$,

elle obtient $\textcolor{red}{\textbf{(}} x+8 \textcolor{red}{\textbf{)}} \times x$ ;

Puis Julie soustrait le carré du nombre de départ, soit $x^2$,

elle obtient $\underbrace{(x+8) \times x}_{\text{prioritaire}} - x^2$.

$\underbrace{(x+8) \times x}_{\text{On distribue}~x} - x^2=x\times x + x\times 8- x^2$

$= x^2 + 8x- x^2$

$= 8x$
\end{minipage}

Pour obtenir le même résultat, Léo et Julie doivent trouver $x$ tel que  $6x + 5 = 8x$

$\underbrace{6x+5}_{\text{1\ier\ membre}}=\underbrace{8x}_{\text{2$\up{nd}$ membre}}$

On met les termes en $x$ dans le premier membre (\textcolor{blue}{\textbf{on élimine les termes en $x$ dans le second membre)}}, les termes constants dans le second membre (\textcolor{orange}{\textbf{on élimine les termes constants dans le premier membre}}).

$6x\underbrace{+5}_{\textbf{\textcolor{orange}{éliminer}}} \textcolor{blue}{\textbf{-8x}} \textcolor{orange}{\textbf{~-5}}=\underbrace{8x}_{\textcolor{blue}{\textbf{éliminer}}} \textcolor{blue}{\textbf{-8x}} \textcolor{orange}{\textbf{~-5}}$



$-2x=-5$

donc $x=2,5$. \quad Léo et Julie doivent choisir le nombre 2,5 pour obtenir le même résultat.
\end{enumerate}

\vspace{0.5cm}

