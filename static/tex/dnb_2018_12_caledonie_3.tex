
\medskip

\begin{enumerate}
\item Décomposer les nombres $162$ et $108$ en produits de facteurs premiers.
\item Déterminer deux diviseurs communs aux nombres $162$ et $108$ plus grands que $10$.
\item Un snack vend des barquettes composées de nems et de samossas.

Le cuisinier a préparé $162$ nems et $108$ samossas.

Dans chaque barquette :

\setlength\parindent{9mm}
\begin{itemize}
\item le nombre de nems doit être le même.
\item le nombre de samossas doit être le même,
\end{itemize}
\setlength\parindent{0mm}

Tous les nems et tous les samossas doivent être utilisés.
	\begin{enumerate}
		\item Le cuisiner peut-il réaliser $36$ barquettes ?
		\item Quel nombre maximal de barquettes pourra-t-il réaliser ?
		\item Dans ce cas, combien y aura-t-il de nems et de samossas dans chaque barquette ?
	\end{enumerate}
\end{enumerate}

\vspace{0,5cm}

