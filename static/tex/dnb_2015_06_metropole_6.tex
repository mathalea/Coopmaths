\textbf{Exercice 6 \hfill 6 points}

\medskip

La distance parcourue par un véhicule entre le moment où 1e conducteur voit un obstacle et l'arrêt
complet du véhicule est schématisée ci-dessous.

\begin{center}
\psset{unit=1cm}
\begin{pspicture}(0,-3)(12,3)
%\psgrid
\rput(0.6,1.4){véhicule}\rput(4.4,1.4){véhicule}\rput(10,1.4){véhicule}
\psframe(10.7,1)(12,1.5)
\rput(11.4,1.25){obstacle} 
\psline[linewidth=5pt]{->}(1.2,1)(5,1)
\psset{arrowsize=1pt 1.5} 
\psline[linewidth=3pt,doubleline=true]{->}(5,1)(10.7,1)
\rput(2.5,0){Distance de réaction}
\rput(8.5,0){Distance de freinage}
\rput(2.5,-0.5){distance parcourue entre l'instant où}
\rput(2.5,-1){le conducteur voit l'obstacle et celui}
\rput(2.5,-1.5){où il commence à freiner}
\rput(8.5,-0.5){distance parcourue depuis le début du}
\rput(8.5,-1){freinage jusqu'à l'arrêt du véhicule}
\psline[linewidth=2.5pt,linestyle=dashed,doubleline=true]{->}(1.2,-2)(10.7,-2)
\rput(6,-2.6){Distance d'arrêt = distance de réaction + distance de freinage}
\end{pspicture}
\end{center}

\medskip

\begin{enumerate}
\item Un scooter roulant à 45 km/h freine en urgence pour éviter un obstacle. À cette vitesse, la distance de réaction est égale à 12,5 m et la distance de freinage à 10 m. Quelle est la distance d'arrêt ?
\item \textbf{Les deux graphiques, donnés en annexe} (dernière page du sujet) représentent, dans des conditions normales et sur route sèche, la distance de réaction et la distance de freinage en fonction de la vitesse du véhicule.

En utilisant ces graphiques, répondre aux questions suivantes :
	\begin{enumerate}
		\item La distance de réaction est de 15~m. À quelle vitesse roule-t-on ? (\emph{Aucune justification n'est attendue}).
		\item La distance de freinage du conducteur est-elle proportionnelle à la vitesse de son véhicule ?
		\item Déterminer la distance d'arrêt pour une voiture roulant à 90~km/h.
	\end{enumerate}
\item La distance de freinage en mètres, d'un véhicule sur route mouillée, peut se calculer à l'aide de la formule sui vante, où $v$ est la. vitesse en km/h du véhicule :

\[\text{distance de freinage sur route mouillée }\:= \dfrac{v^2}{152,4}\]

Calculer au mètre près la distance de freinage sur route mouillée à 110~km/h.
\end{enumerate}

\clearpage

