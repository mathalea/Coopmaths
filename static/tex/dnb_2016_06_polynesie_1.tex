\textbf{Exercice 1 \hfill 6 points}

\medskip

\parbox{0.5\linewidth}{Le Solitaire est un jeu de hasard de la \emph{Française des Jeux}.

Le joueur achète un ticket au prix de 2~\euro, gratte la case
argentée et découvre le \og montant du gain \fg.

Un ticket est gagnant si le \og montant du gain\fg{} est
supérieur ou égal à  2~\euro.

Les tickets de Solitaire sont fabriqués par lots de \np{750000}
tickets.

Le tableau ci-contre donne la composition d'un lot.}\hfill
\parbox{0.48\linewidth}{\begin{tabularx}{\linewidth}{c|*{2}{>{\centering \arraybackslash}X|}c|}\cline{2-3}
		&\scriptsize Nombre de tickets	&\scriptsize \og Montant du gain \fg{} par ticket	&\multicolumn{1}{|c}{}\\ \cline{2-3}
		&\np{532173} 		&0~\euro						&\multicolumn{1}{|c}{}\\\cline{2-4}
		&\np{100000} 		&2~\euro						&\multirow{8}{0.25cm}{}\\\cline{2-3}
		&\np{83000} 		&4~\euro						&\\ \cline{2-3}
		&\np{20860} 		&6~\euro						&\\\cline{2-3}
		&\np{5400} 			&12~\euro						&\\\cline{2-3}
		&\np{8150} 			&20~\euro						&\\\cline{2-3}
		&400 				&150~\euro 						&\\\cline{2-3}
		&15 				&\np{1000}~\euro				&\\\cline{2-3}
		&2 					&\np{15000}~\euro				&\\\hline
\multicolumn{1}{|c|}{Total}	& \np{750000}		&\multicolumn{1}{c}{}			&\multicolumn{1}{c}{}\\\cline{1-2}
\end{tabularx}}
\rput{-90}(-0.4,-0.35){Tickets gagnants}
\medskip

\begin{enumerate}
\item Si on prélève un ticket au hasard dans un lot,
	\begin{enumerate}
		\item quelle est la probabilité d'obtenir un ticket gagnant dont le \og montant du gain\fg{} est 4~\euro ?
		\item quelle est la probabilité d'obtenir un ticket gagnant ?
		\item expliquer pourquoi on a moins de 2\,\% de chances d'obtenir un ticket dont le \og montant du gain \fg{} est supérieur ou égal à  10~\euro.
	\end{enumerate}
\item Tom dit : \og Si j'avais assez d'argent, je pourrais acheter un lot complet de tickets Solitaire. Je deviendrais encore plus riche. \fg
	
Expliquer si Tom a raison.
\end{enumerate}

\bigskip

