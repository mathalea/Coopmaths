\textbf{Exercice 2 \hfill 6 points}

\medskip

\emph{On considère la figure ci-contre qui n'est pas à l'échelle.}

\medskip

%\parbox{0.55\linewidth}{\setlength\parindent{6mm} 
%\begin{itemize}
%\item[$\bullet~~$] Le triangle JAB est rectangle en A.   
%\item[$\bullet~~$] Les droites (MU) et (AB) sont parallèles.         
%\item[$\bullet~~$] Les points A, M et J sont alignés.         
%\item[$\bullet~~$] Les points C, U et J sont alignés.         
%\item[$\bullet~~$] Les points A, C et B sont alignés.         
%\item[$\bullet~~$] AB = 7,5 m.         
%\item[$\bullet~~$] MU = 3 m.         
%\item[$\bullet~~$] JM = 10 m.         
%\item[$\bullet~~$] JA = 18 m. 
%\end{itemize}       
%} \hfill
%\parbox{0.45\linewidth}{\psset{unit=0.9cm}
%\begin{pspicture}(6,7)
%%\psgrid
%\pspolygon(0.5,0.5)(0.5,6)(5.5,6)
%\psline(0.5,0.5)(3,6)
%\psline(0.5,2.6)(1.42,2.6)
%\uput[u](0.5,6){A} \uput[u](5.5,6){B} \uput[u](3,6){C} 
%\uput[l](0.5,2.6){M} \uput[r](1.42,2.6){U} \uput[d](0.5,0.5){J} 
%\psframe(0.5,6)(0.7,5.8)
%\end{pspicture}}

\begin{enumerate}
\item %Calculer la longueur JB.
Le triangle JAB est rectangle en A ; d'après le théorème de Pythagore :

$\text{JA}^2 + \text{AB}ô2 = \text{JB}^2$ soit $18^2 + 7,5^2 =  \text{JB}^2$ ou encore 

$\text{JB}^2 = 324 + 56,25 = 380,25$.

Donc JB $ = \sqrt{380,25} = 19,5$~(cm).
\item %Montrer que la longueur AC est égale à 5,4 m.
Dans le triangle JAC, les droites (MU) et (AC) sont parallèles, J, M et A sont alignés dans cet ordre, J, U et C  sont alignés dans cet ordre : on peut donc appliquer le théorème de Thalès :

$\dfrac{\text{JM}}{\text{JA}} = \dfrac{\text{JU}}{\text{JC}} = \dfrac{\text{MU}}{\text{AC}}$.

En particulier $\dfrac{\text{JM}}{\text{JA}} =  \dfrac{\text{MU}}{\text{AC}}$ donne $\dfrac{10}{18} = \dfrac{3}{\text{AC}}$ soit $10\text{AC} = 3 \times 18$ ou AC $ = 5,4$~(cm).
\item %Calculer l'aire du triangle JCB.
L'aire  du triangle JCB est égale à $\dfrac{1}{2} \text{JA} \times \text{CB} = \dfrac{1}{2} \times 18 \times (7,5 - 5,4)  = \dfrac{1}{2} \times 18 \times 2,1  = 9 \times 2,1 = 18,9~\text{cm}^2$.
\end{enumerate}

\vspace{0.5cm}

