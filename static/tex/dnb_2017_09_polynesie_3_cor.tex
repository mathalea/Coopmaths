
\medskip

%Le jardinier d'un club de football décide de semer à nouveau du gazon sur l'aire de
%jeu. Pour que celui-ci pousse correctement, il installe un système d'arrosage
%automatique qui se déclenche le matin et le soir, à chaque fois, pendant 15 minutes.
%
%\setlength\parindent{9mm}
%\begin{itemize}
%\item[$\bullet~~$] Le système d'arrosage est constitué de 12 circuits indépendants.
%\item[$\bullet~~$] Chaque circuit est composé de 4 arroseurs.
%\item[$\bullet~~$] Chaque arroseur a un débit de 0,4 m$^3$ d'eau par heure.
%\end{itemize}
%\setlength\parindent{0mm}
%
%Combien de litres d'eau auront été consommés si on arrose le gazon pendant tout le
%mois de juillet ?

%On rappelle que 1 m$^3 = \np{1000}$~litres et que le mois de juillet compte $31$~jours.
Chaque jour l'arrosage fonctionne pendant $2 \times 15 = 30$~min soit 0,5~h. Un arroseur débite donc pendant cette demi-heure 0,2~m$^3$.

Pendant le mois de juillet on aura donc déversé :

$31 \times 12 \times 4 \times 0,2 = 297,6$~m$^3$, soit \np{297600}~litres d'eau.
\vspace{0,5cm}

