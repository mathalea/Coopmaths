\documentclass[10pt]{article}
\usepackage[T1]{fontenc}
\usepackage[utf8]{inputenc}%ATTENTION codage UTF8
\usepackage{fourier}
\usepackage[scaled=0.875]{helvet}
\renewcommand{\ttdefault}{lmtt}
\usepackage{amsmath,amssymb,makeidx}
\usepackage[normalem]{ulem}
\usepackage{diagbox}
\usepackage{fancybox}
\usepackage{tabularx,booktabs}
\usepackage{colortbl}
\usepackage{pifont}
\usepackage{multirow}
\usepackage{dcolumn}
\usepackage{enumitem}
\usepackage{textcomp}
\usepackage{lscape}
\newcommand{\euro}{\eurologo{}}
\usepackage{graphics,graphicx}
\usepackage{pstricks,pst-plot,pst-tree,pstricks-add}
\usepackage[left=3.5cm, right=3.5cm, top=3cm, bottom=3cm]{geometry}
\newcommand{\R}{\mathbb{R}}
\newcommand{\N}{\mathbb{N}}
\newcommand{\D}{\mathbb{D}}
\newcommand{\Z}{\mathbb{Z}}
\newcommand{\Q}{\mathbb{Q}}
\newcommand{\C}{\mathbb{C}}
\usepackage{scratch}
\renewcommand{\theenumi}{\textbf{\arabic{enumi}}}
\renewcommand{\labelenumi}{\textbf{\theenumi.}}
\renewcommand{\theenumii}{\textbf{\alph{enumii}}}
\renewcommand{\labelenumii}{\textbf{\theenumii.}}
\newcommand{\vect}[1]{\overrightarrow{\,\mathstrut#1\,}}
\def\Oij{$\left(\text{O}~;~\vect{\imath},~\vect{\jmath}\right)$}
\def\Oijk{$\left(\text{O}~;~\vect{\imath},~\vect{\jmath},~\vect{k}\right)$}
\def\Ouv{$\left(\text{O}~;~\vect{u},~\vect{v}\right)$}
\usepackage{fancyhdr}
\usepackage[french]{babel}
\usepackage[dvips]{hyperref}
\usepackage[np]{numprint}
%Tapuscrit : Denis Vergès
%\frenchbsetup{StandardLists=true}

\begin{document}
\setlength\parindent{0mm}
% \rhead{\textbf{A. P{}. M. E. P{}.}}
% \lhead{\small Brevet des collèges}
% \lfoot{\small{Polynésie}}
% \rfoot{\small{7 septembre 2020}}
\pagestyle{fancy}
\thispagestyle{empty}
% \begin{center}
    
% {\Large \textbf{\decofourleft~Brevet des collèges Polynésie 7 septembre 2020~\decofourright}}
    
% \bigskip
    
% \textbf{Durée : 2 heures} \end{center}

% \bigskip

% \textbf{\begin{tabularx}{\linewidth}{|X|}\hline
%  L'évaluation prend en compte la clarté et la précision des raisonnements ainsi que, plus largement, la qualité de la rédaction. Elle prend en compte les essais et les démarches engagées même non abouties. Toutes les réponses doivent être justifiées, sauf mention contraire.\\ \hline
% \end{tabularx}}

% \vspace{0.5cm}\textbf{Exercice 4 \hfill 10 points}

\medskip

Leila est en visite à Paris. Aujourd'hui, elle est au Champ de Mars où l'on peut voir la tour Eiffel dont la hauteur totale BH est $324$~m.

Elle pose son appareil photo au sol à une distance AB = 600 m du monument et le programme pour prendre une photo (voir le dessin ci-dessous).

\medskip

\begin{enumerate}
\item Quelle est la mesure, au degré près, de l'angle $\widehat{\text{HAB}}$ ?
\item Sachant que Leila mesure $1,70$ m, à quelle distance AL de son appareil doit-elle se placer pour paraître aussi grande que la tour Eiffel sur sa photo ?

Donner une valeur approchée du résultat au centimètre près.
\end{enumerate}

\begin{center}
\psset{unit=0.9cm,arrowsize=2pt 3}
\begin{pspicture}(10,5)
\pspolygon(1,1)(9,1)(9,4.5)%ABL
\uput[dl](1,1){A}\uput[d](9,1){B}\uput[ur](9,4.5){H}
\uput[u](1,1){appareil photo}\uput[u](3.,1){Leila}\rput{90}(9.5,2){Tour Eiffel}
\uput[d](3,1){L}
\psline(3,1)(3,1.88)
\rput(5,-0.2){Le dessin n'est pas à l'échelle}
\uput[u](4.5,0.4){AB  = 600 m}\rput{90}(9.5,3.75){324~m}
\psline[linewidth=0.3pt]{<->}(1,0.4)(9,0.4)
\end{pspicture}
\end{center}

\bigskip

\end{document}