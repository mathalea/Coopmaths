\documentclass[10pt]{article}
\usepackage[T1]{fontenc}
\usepackage[utf8]{inputenc}%ATTENTION codage UTF8
\usepackage{fourier}
\usepackage[scaled=0.875]{helvet}
\renewcommand{\ttdefault}{lmtt}
\usepackage{amsmath,amssymb,makeidx}
\usepackage[normalem]{ulem}
\usepackage{diagbox}
\usepackage{fancybox}
\usepackage{tabularx,booktabs}
\usepackage{colortbl}
\usepackage{pifont}
\usepackage{multirow}
\usepackage{dcolumn}
\usepackage{enumitem}
\usepackage{textcomp}
\usepackage{lscape}
\newcommand{\euro}{\eurologo{}}
\usepackage{graphics,graphicx}
\usepackage{pstricks,pst-plot,pst-tree,pstricks-add}
\usepackage[left=3.5cm, right=3.5cm, top=3cm, bottom=3cm]{geometry}
\newcommand{\R}{\mathbb{R}}
\newcommand{\N}{\mathbb{N}}
\newcommand{\D}{\mathbb{D}}
\newcommand{\Z}{\mathbb{Z}}
\newcommand{\Q}{\mathbb{Q}}
\newcommand{\C}{\mathbb{C}}
\usepackage{scratch}
\renewcommand{\theenumi}{\textbf{\arabic{enumi}}}
\renewcommand{\labelenumi}{\textbf{\theenumi.}}
\renewcommand{\theenumii}{\textbf{\alph{enumii}}}
\renewcommand{\labelenumii}{\textbf{\theenumii.}}
\newcommand{\vect}[1]{\overrightarrow{\,\mathstrut#1\,}}
\def\Oij{$\left(\text{O}~;~\vect{\imath},~\vect{\jmath}\right)$}
\def\Oijk{$\left(\text{O}~;~\vect{\imath},~\vect{\jmath},~\vect{k}\right)$}
\def\Ouv{$\left(\text{O}~;~\vect{u},~\vect{v}\right)$}
\usepackage{fancyhdr}
\usepackage[french]{babel}
\usepackage[dvips]{hyperref}
\usepackage[np]{numprint}
%Tapuscrit : Denis Vergès
%\frenchbsetup{StandardLists=true}

\begin{document}
\setlength\parindent{0mm}
% \rhead{\textbf{A. P{}. M. E. P{}.}}
% \lhead{\small Brevet des collèges}
% \lfoot{\small{Polynésie}}
% \rfoot{\small{7 septembre 2020}}
\pagestyle{fancy}
\thispagestyle{empty}
% \begin{center}
    
% {\Large \textbf{\decofourleft~Brevet des collèges Polynésie 7 septembre 2020~\decofourright}}
    
% \bigskip
    
% \textbf{Durée : 2 heures} \end{center}

% \bigskip

% \textbf{\begin{tabularx}{\linewidth}{|X|}\hline
%  L'évaluation prend en compte la clarté et la précision des raisonnements ainsi que, plus largement, la qualité de la rédaction. Elle prend en compte les essais et les démarches engagées même non abouties. Toutes les réponses doivent être justifiées, sauf mention contraire.\\ \hline
% \end{tabularx}}

% \vspace{0.5cm}\textbf{Exercice 8 \hfill 7 points}

\medskip

\parbox{0.55\linewidth}{Soit l'expérience aléatoire suivante :
 
- tirer au hasard une boule noire, noter son numéro ; 

- tirer au hasard une boule blanche, noter son numéro;
 
- puis calculer la somme des 2 numéros tirés.} \hfill
\parbox{0.42\linewidth}{\psset{unit=0.8cm}
\begin{pspicture}(6,2)
\psline(0,1.5)(0,0)(2.5,0)(2.5,1.5)
\psline(3,1.5)(3,0)(5.5,0)(5.5,1.5)
\pscircle*(0.4,0.25){0.25} \rput(0.4,0.25){\white 1}
\pscircle*(0.9,0.25){0.25} \rput(0.9,0.25){\white 2}
\pscircle*(1.4,0.25){0.25} \rput(1.4,0.25){\white 3}
\pscircle*(1.9,0.25){0.25} \rput(1.9,0.25){\white 4}

\pscircle(3.4,0.25){0.25} \rput(3.4,0.25){ 2}
\pscircle(4.15,0.25){0.25}\rput(4.15,0.25){ 3}
\pscircle(4.9,0.25){0.25} \rput(4.9,0.25){ 5}
\end{pspicture}
} 

\bigskip
 
\begin{enumerate}
\item On a simulé l'expérience avec un tableur, en utilisant la fonction ALEA() pour obtenir les numéros des boules tirées au hasard.
 
Voici les résultats des premières expériences : 

\medskip

\parbox{0.52\linewidth}{\hspace{-0.5cm}\begin{tabularx}{1.1\linewidth}{|c|*{4}{>{\footnotesize \centering \arraybackslash}X|}}\hline 
&A &B& C&D\\ \hline 
1&Expé\-rience &Numéro de la boule noire &Numéro de la boule blanche& 
Somme\\ \hline 
2 &\no 1& 4 &2 &6\\ \hline 
3& \no 2& 1 &2 &3\\ \hline 
4&\no 3 &2& 3 &5\\ \hline 
5 &\no 4 &3&3&6\\ \hline 
6& \no 5& 3 &5& 8\\ \hline 
7& \no 6& 4 &3 &7\\ \hline
\end{tabularx}
}\hfill \parbox{0.46\linewidth}{\begin{enumerate}
\item Décris l'expérience \no 3. 
\item Parmi les 4 formules suivantes, recopie sur ta feuille celle qui est écrite dans la case D5 :
 
\fbox{$2\star\text{A}4$}\: \fbox{=\text{B}4+\text{C}4}\: \fbox{$=\text{B}5+\text{C}5$}\: \fbox{$=\text{SOMME}(\text{D}5)$} 
\item Peut-on obtenir la somme 2 ? Justifie. 
\item Quels sont les tirages possibles qui permettent d'obtenir la somme 4 ? 
Quelle est la plus grande somme possible ?
 
Justifie.
\end{enumerate}}

\medskip
 
\item Sur une seconde feuille de calcul, on a copié les résultats obtenus avec $50$~expériences, avec \np{1000}~expériences, avec \np{5000}~expériences et on a calculé les fréquences des différentes sommes.

\medskip
\begin{tabularx}{\linewidth}{|c|c|*{7}{>{\centering \arraybackslash \footnotesize }X|}c|}\hline 
	& A 		& B & C & D & E & F& G& H & I\\ \hline 
1 	&Somme 		&3 	&4 	&5 	&6 	&7 &8 &9&\footnotesize effectif total\\ \hline
2 	&effectif 	&5 	&10 &9 &8 &8 &8 &2 &50\\ \hline 
3	&\footnotesize fréquence &0,1 &0,2 &0,18 &0,16 &0,16 &0,16&&\\ \hline 
4	&\multicolumn{9}{c}{~}\\ \hline
5 	&Somme 		&3 	&4 &5 &6 &7 &8 &9&\footnotesize effectif total\\ \hline 
6	&effectif 	&79 &161 &167 &261 &166&72 &94 &\np{1000}\\ \hline 
7	&\footnotesize fréquence &0,079 &0,161 &0,167 &0,261 &0,166 &0,072&0,094&\\ \hline 
8	&\multicolumn{9}{c}{~}\\  \hline
9 	&Somme 		&3 	&4 &5 &6 &7 &8 &9 &\footnotesize effectif total\\ \hline 
10	& effectif 	&405&844 &851 &\np{1221} &871 &410 &398 &\np{5000}\\ \hline 
11	&\footnotesize fréquence& 0,081 &0,1688 &0,1702 &0,2442 &0,1742 &0,082 &0,0796&\\ \hline 
\end{tabularx} 
\medskip

	\begin{enumerate}
		\item Quelle est la fréquence de la somme $9$ au cours des $50$~premières expériences ? Justifie. 
		\item Quelle formule a-t-on écrite dans la case B7 pour obtenir la fréquence de la somme $3$ ? 
		\item Donne une estimation de la probabilité d'obtenir la somme $3$. 
	\end{enumerate}
\end{enumerate}
\end{document}\end{document}