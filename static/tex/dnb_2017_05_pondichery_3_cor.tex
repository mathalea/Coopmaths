
\medskip

\begin{enumerate}
\item 
	\begin{enumerate}
		\item $x = 5$
		
étape 1 $= 6 \times 5 = 30$
		
étape 2 $= 30 + 10 = 40$
		
résultat $= 40 : 2 = 20$
		
dire \og J'obtiens finalement 20 \fg.
		\item $x = 7$
		
étape 1 $= 6  \times 7 = 42$
		
étape 2 $= 42 + 10 = 52$
		
résultat = $52 : 2 = 26$
		
dire \og J'obtiens finalement 26 \fg.
	\end{enumerate}
\item Pour retrouver le nombre du départ il faut \og remonter \fg{} l'algorithme, d'où

résultat $= 8$ entraine que  étape 2 $= 8  \times 2 = 16$

étape 1 $= 16 - 10 = 6$

$x = 1$

Julie a choisi le nombre 1.
\item étape 1 $= 6  \times x = 6x$

étape 2 $= 6x + 10$

résultat $= (6x + 10) : 2 = \dfrac{6x + 10}{2} = \dfrac{2(3x + 5)}{2} = 3x + 5$, ou encore 

$= (6x + 10) : 2 = 6x : 2 + 10 : 2 = 3x + 5$.
\item Soit $x$ le nombre choisi.

Le programme de Maxime donne : $(x + 2)  \times 5 = 5(x + 2)  = 5x + 10$.

On veut que $5x + 10 = 3x + 5$, d'où 

$5x {\red - 3x} + 10 = 3x {\red - 3x} + 5$

$2x + 10 = 5$, puis

$2x + 10  {\red - 10}= 5 {\red - 10}$

$2x = - 5$, d'où $\dfrac{1}{2}\times 2x = - 5 \times \dfrac{1}{2}$ et enfin 

$x = \dfrac{- 5}{2} = \dfrac{- 25}{10} = - 2,5$.

Si on choisit $\dfrac{- 5}{2} = - 2,5$, les deux programmes donnent le même résultat.
\end{enumerate}

\vspace{0,5cm}


