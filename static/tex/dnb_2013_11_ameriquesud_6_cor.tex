\documentclass[10pt]{article}
\usepackage[T1]{fontenc}
\usepackage[utf8]{inputenc}%ATTENTION codage UTF8
\usepackage{fourier}
\usepackage[scaled=0.875]{helvet}
\renewcommand{\ttdefault}{lmtt}
\usepackage{amsmath,amssymb,makeidx}
\usepackage[normalem]{ulem}
\usepackage{diagbox}
\usepackage{fancybox}
\usepackage{tabularx,booktabs}
\usepackage{colortbl}
\usepackage{pifont}
\usepackage{multirow}
\usepackage{dcolumn}
\usepackage{enumitem}
\usepackage{textcomp}
\usepackage{lscape}
\newcommand{\euro}{\eurologo{}}
\usepackage{graphics,graphicx}
\usepackage{pstricks,pst-plot,pst-tree,pstricks-add}
\usepackage[left=3.5cm, right=3.5cm, top=3cm, bottom=3cm]{geometry}
\newcommand{\R}{\mathbb{R}}
\newcommand{\N}{\mathbb{N}}
\newcommand{\D}{\mathbb{D}}
\newcommand{\Z}{\mathbb{Z}}
\newcommand{\Q}{\mathbb{Q}}
\newcommand{\C}{\mathbb{C}}
\usepackage{scratch}
\renewcommand{\theenumi}{\textbf{\arabic{enumi}}}
\renewcommand{\labelenumi}{\textbf{\theenumi.}}
\renewcommand{\theenumii}{\textbf{\alph{enumii}}}
\renewcommand{\labelenumii}{\textbf{\theenumii.}}
\newcommand{\vect}[1]{\overrightarrow{\,\mathstrut#1\,}}
\def\Oij{$\left(\text{O}~;~\vect{\imath},~\vect{\jmath}\right)$}
\def\Oijk{$\left(\text{O}~;~\vect{\imath},~\vect{\jmath},~\vect{k}\right)$}
\def\Ouv{$\left(\text{O}~;~\vect{u},~\vect{v}\right)$}
\usepackage{fancyhdr}
\usepackage[french]{babel}
\usepackage[dvips]{hyperref}
\usepackage[np]{numprint}
%Tapuscrit : Denis Vergès
%\frenchbsetup{StandardLists=true}

\begin{document}
\setlength\parindent{0mm}
% \rhead{\textbf{A. P{}. M. E. P{}.}}
% \lhead{\small Brevet des collèges}
% \lfoot{\small{Polynésie}}
% \rfoot{\small{7 septembre 2020}}
\pagestyle{fancy}
\thispagestyle{empty}
% \begin{center}
    
% {\Large \textbf{\decofourleft~Brevet des collèges Polynésie 7 septembre 2020~\decofourright}}
    
% \bigskip
    
% \textbf{Durée : 2 heures} \end{center}

% \bigskip

% \textbf{\begin{tabularx}{\linewidth}{|X|}\hline
%  L'évaluation prend en compte la clarté et la précision des raisonnements ainsi que, plus largement, la qualité de la rédaction. Elle prend en compte les essais et les démarches engagées même non abouties. Toutes les réponses doivent être justifiées, sauf mention contraire.\\ \hline
% \end{tabularx}}

% \vspace{0.5cm}\textbf{Exercice 6 \hfill 4 points}

\medskip  

%\textbf{Dans cet exercice, si le travail n'est pas terminé, laisser tout de même une trace de la recherche. Elle sera prise en compte dans l'évaluation.}
%
%\medskip 
%
%Le même jour, à la caisse d'un cinéma, un adulte et deux enfants payent $21$~\euro, deux adultes et trois enfants payent $36$~\euro. 
%
%Trois adultes et trois enfants vont au cinéma ce jour-là. Le caissier leur réclame $43$~\euro.
%
%\og Vous vous trompez! \fg{} s'exclame un des enfants. A-t-il raison ? Pourquoi ?
Soit $a$ le prix du billet adulte et $e$ le prix du billet enfant.

On a $a + 2e = 21$, soit $a = 21 - 2e$.

On a aussi $2a + 3e = 36$ ou $2(21 - 2e) + 3e = 36$ ou $42 - 4e + 3e = 36$ soit $42 - 36 = e$. Donc $e = 6$.

Un adulte paie donc $a = 21 - 2\times 6 = 21 - 12 = 9$.

Un adulte et un enfant payent $9 + 6 = 15$, donc trois adultes et trois enfants payent trois fois plus soit $3 \times 15 = 45$~\euro. L'enfant a raison.

\emph{Remarque} : on peut également résoudre le système :

$\left\{\begin{array}{l c l}
a + 2e&=&21\\
2a + 3e&=&36
\end{array}\right.$ ou encore $\left\{\begin{array}{l c l}
2a + 4e&=&42\\
2a + 3e&=&36
\end{array}\right.$ et par différence $e = 6$, puis 

$a = 21 - 2\times 6 = 21 - 12 = 9$.

Donc $3a + 3e = 27 + 18 = 45$
\end{document}\end{document}