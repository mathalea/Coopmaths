\documentclass[10pt]{article}
\usepackage[T1]{fontenc}
\usepackage[utf8]{inputenc}%ATTENTION codage UTF8
\usepackage{fourier}
\usepackage[scaled=0.875]{helvet}
\renewcommand{\ttdefault}{lmtt}
\usepackage{amsmath,amssymb,makeidx}
\usepackage[normalem]{ulem}
\usepackage{diagbox}
\usepackage{fancybox}
\usepackage{tabularx,booktabs}
\usepackage{colortbl}
\usepackage{pifont}
\usepackage{multirow}
\usepackage{dcolumn}
\usepackage{enumitem}
\usepackage{textcomp}
\usepackage{lscape}
\newcommand{\euro}{\eurologo{}}
\usepackage{graphics,graphicx}
\usepackage{pstricks,pst-plot,pst-tree,pstricks-add}
\usepackage[left=3.5cm, right=3.5cm, top=3cm, bottom=3cm]{geometry}
\newcommand{\R}{\mathbb{R}}
\newcommand{\N}{\mathbb{N}}
\newcommand{\D}{\mathbb{D}}
\newcommand{\Z}{\mathbb{Z}}
\newcommand{\Q}{\mathbb{Q}}
\newcommand{\C}{\mathbb{C}}
\usepackage{scratch}
\renewcommand{\theenumi}{\textbf{\arabic{enumi}}}
\renewcommand{\labelenumi}{\textbf{\theenumi.}}
\renewcommand{\theenumii}{\textbf{\alph{enumii}}}
\renewcommand{\labelenumii}{\textbf{\theenumii.}}
\newcommand{\vect}[1]{\overrightarrow{\,\mathstrut#1\,}}
\def\Oij{$\left(\text{O}~;~\vect{\imath},~\vect{\jmath}\right)$}
\def\Oijk{$\left(\text{O}~;~\vect{\imath},~\vect{\jmath},~\vect{k}\right)$}
\def\Ouv{$\left(\text{O}~;~\vect{u},~\vect{v}\right)$}
\usepackage{fancyhdr}
\usepackage[french]{babel}
\usepackage[dvips]{hyperref}
\usepackage[np]{numprint}
%Tapuscrit : Denis Vergès
%\frenchbsetup{StandardLists=true}

\begin{document}
\setlength\parindent{0mm}
% \rhead{\textbf{A. P{}. M. E. P{}.}}
% \lhead{\small Brevet des collèges}
% \lfoot{\small{Polynésie}}
% \rfoot{\small{7 septembre 2020}}
\pagestyle{fancy}
\thispagestyle{empty}
% \begin{center}
    
% {\Large \textbf{\decofourleft~Brevet des collèges Polynésie 7 septembre 2020~\decofourright}}
    
% \bigskip
    
% \textbf{Durée : 2 heures} \end{center}

% \bigskip

% \textbf{\begin{tabularx}{\linewidth}{|X|}\hline
%  L'évaluation prend en compte la clarté et la précision des raisonnements ainsi que, plus largement, la qualité de la rédaction. Elle prend en compte les essais et les démarches engagées même non abouties. Toutes les réponses doivent être justifiées, sauf mention contraire.\\ \hline
% \end{tabularx}}

% \vspace{0.5cm}\textbf{Exercice 1 :  \hfill 5 points}

\medskip

\textbf{Question 1} : Dans un club sportif, $\dfrac{1}{8}$ des adhérents ont plus de 42~ans et $\dfrac{1}{4}$, soit $\dfrac{2}{8}$ ont moins de 25~ans.

$\dfrac{1}{8}+\dfrac{2}{8}=\dfrac{3}{8}$. Il reste une proportion de $1 - \dfrac{3}{8} = \dfrac{8 - 3}{8} = \dfrac{5}{8}$ d'adhérents ayant un âge de 25 à  42~ans. \textbf{Réponse~C}.

\textbf{Question 2} : Pour augmenter le prix de 20~\%\, on multiplie le prix de départ par 1,20. $46~000\times 1,20=\np{55200}$. \textbf{Réponse~B}.

\textbf{Question 3} : Si toutes les longueurs sont multipliées par $k$, alors les aires sont multipliées par $k^2$ et les volumes sont multipliés par $k^3$. Ici, toutes les longueurs du cube sont multipliées par 3, donc le volume du cube est multiplié par $3^3$, soit par 27.  \textbf{Réponse~D}. \\[2mm]
\textbf{Question 4} : Les nombres 23 et 37 sont impairs, donc on élimine la réponse~D.

Les nombres 23 et 37 ne sont pas divisibles par 3 (on ne les trouve pas dans la table de multiplication du 3 ; ou la somme de leurs chiffres n’est pas divisible par 3 ($2+3=5$ et 5 n’est pas divisible par 3 ; $3+7 = 10$ et 10 n’est pas divisible par 3)), donc on élimine la réponse~B.

Tous les nombres entiers sont divisibles par 1, donc les nombres 23 et 37 ont 1 comme diviseur commun, donc on élimine la réponse~C.

Il ne reste que la bonne réponse. Les nombres 23 et 37 ont exactement deux diviseurs (1 et le nombre lui-même), donc ils sont premiers. \textbf{Réponse~A}.

\textbf{Question 5} : On calcule $f(3)$ (en remplaçant $x$ par 3). 

$3^2 - 2\times 3+7 = 9 - 6 + 7 =  10$. \quad  \textbf{Réponse~A}. 

\vspace{0.5cm}

\end{document}