\textbf{Exercice 3 \hfill 6 points}

\medskip

%\parbox{0.48\linewidth}{Dans la figure ci-contre:
%
%\begin{itemize}
%\item[$\bullet~~$] ABE est un triangle;
%\item[$\bullet~~$] AB = 6~cm, AE = 8~cm et BE = 10~cm ;
%\item[$\bullet~~$] I et J sont les milieux respectifs des côtés [AB] et [AE] ;
%\item[$\bullet~~$] le cercle $(C)$ passe par les points I, J et A.
%\end{itemize}} \hfill
%\parbox{0.48\linewidth}{\psset{unit=0.6cm}
%\begin{pspicture}(10,5.5)
%%\psgrid
%\pspolygon(0.3,0.6)(9.5,0.5)(3.3,4.8)%BEA
%\uput[u](3.3,4.8){A} \uput[l](0.3,0.6){B} \uput[r](9.5,0.5){E} \uput[l](1.8,2.7){I} \uput[r](6.3,2.7){J}\uput[ur](5.4,4.5){$(C)$} 
%\pscircle(4.05,2.7){2.25}
%\psline(1.8,2.7)(6.3,2.7)%IJ
%\rput(5,0){La figure n'est pas à  l'échelle}
%\end{pspicture}}
%\bigskip

\begin{enumerate}
\item %Peut-on affirmer que les droites (IJ) et (BE) sont parallèles ?
La droite (IJ) contient les milieux de deux côtés du triangle ABE : elle est donc parallèle au troisième côté, donc (IJ) et (BE) sont parallèles.

\item %Montrer que le triangle ABE est rectangle.
On a d'une part $\text{AB}^2 + \text{AE}^2 = 6^2 + 8^2 = 36 + 64 = 100$, et d'autre part :

$\text{BE}^2 = 10^2 = 100$.

Donc $\text{AB}^2 + \text{AE}^2  = \text{BE}^2$, soit d'après la réciproque de Pythagore : ABE est un triangle rectangle en A.
\item %Quelle est la mesure de l'angle $\widehat{\text{AEB}}$ ? On donnera une valeur approchée au degré près.
On a dans le triangle rectangle en A, ABE :

$\cos \widehat{\text{AEB}} = \dfrac{\text{AE}}{\text{BE}} = \dfrac{8}{10} = 0,8$. La calculatrice donne $\widehat{\text{AEB}} \approx 36,8 \approx 37\,\degres$ au degré près.
\item  
	\begin{enumerate}
		\item %Justifier que le centre du cercle $(C)$ est le milieu du segment [IJ].
$\widehat{\text{IAJ}} = 90\,\degres$ ; l'angle droit intercepte un diamètre (l'angle inscrit a une mesure moitié de celle de l'angle 	au centre $\widehat{\text{IOJ}}$ si O est le centre du cercle ; donc $\widehat{\text{IOJ}} = 180\,\degres$, donc [IJ] est un diamètre. Le centre du cercle $(C)$ est le milieu du segment [IJ].
		\item %Quelle est la mesure du rayon du cercle $(C)$ ?
D'après la première question on sait que les droites (IJ) et (AB) sont parallèles ; de plus $\text{IJ} = \dfrac{\text{AB}}{2} = \dfrac{10}{2} = 5$. Or IJ $ = 2R = 5$, (avec $R$ rayon du cercle),  d'où $R = 2,5$.
	\end{enumerate}
\end{enumerate}

\bigskip

