\textbf{\textsc{Exercice} 2 \hfill 4 points}

\medskip

On a utilisé un tableur pour calculer les images de différentes valeurs de $x$ par une fonction affine $f$ et par une autre fonction $g$. Une copie de l'écran obtenu est donnée ci-dessous. 

\medskip

\begin{tabularx}{\linewidth}{|c|*{8}{>{\centering \arraybackslash}X|}}\hline
\multicolumn{3}{|c|}{C2}&$fx$&\multicolumn{5}{|l|}{$=-5\star\text{C}1+7$}\\ \hline
&A&B&C&D&R&F&G&H\\ \hline
1&$x$&$- 3$&$- 2$&$- 1$&$0$&1&2&3\\ \hline 
2&$f(x)$&22&\psframe(-0.57,-0.15)(0.91,0.3)17&12&7&2&$- 3$&$- 8$\\ \hline 
3&$g(x)$&13&8&5&4&5&8&13\\ \hline
4&&&&&&&&\\ \hline
\end{tabularx}

\medskip

\begin{enumerate}
\item Quelle est l'image de $- 3$ par  $f$ ? 
\item Calculer $f(7)$. 
\item Donner l'expression de $f(x)$. 
\item On sait que $g(x) = x^2 + 4$. Une formule a été saisie dans la cellule B3 et recopiée ensuite vers la droite pour compléter la plage de cellules C3:H3.  Quelle est cette formule ?
\end{enumerate}
 
\bigskip

