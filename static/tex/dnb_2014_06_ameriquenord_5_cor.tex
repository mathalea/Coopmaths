\textbf{\textsc{Exercice 5} \hfill 3 points}

\medskip
 
Pour une bonne partie de pêche au bord du canal, il faut un siège pliant adapté !

Nicolas est de taille moyenne et pour être bien assis, il est nécessaire que la hauteur de l'assise du siège soit comprise entre $44$~cm et $46$~cm.

\medskip

\parbox{0.58\linewidth}{Voici les dimensions d'un siège pliable qu'il a trouvé en vente sur internet : 
 
longueur des pieds : 56 cm

largeur de l'assise : 34 cm

profondeur de l'assise : 31 cm 

$\widehat{\text{ACE}}$ est droit

ABDC est un rectangle}
\hfill
\parbox{0.4\linewidth}{\psset{unit=.8cm}
\begin{pspicture}(-1,0)(4.2,5.8)
\pspolygon(0.2,1.2)(1.7,2)(4,4.1)(2.5,3.3)
\pspolygon(2.5,0.3)(3.8,1.15)(1.85,4.95)(0.4,4.2)
\psline(4,4.1)(1.85,4.95)
\psline(2.5,3.3)(0.4,4.2)
\psline[linewidth=0.5pt]{<->}(0.2,4)(2.2,0.3)
\psline[linewidth=0.5pt]{<->}(2.2,5.05)(4.15,4.3)
\psline[linewidth=0.5pt]{<->}(2.5,3.3)(2.5,0.3)
\uput[ul](0.2,1.2){F}\uput[l](0.4,4.2){A}
\uput[u](1.85,4.95){B}\uput[dr](2.4,3.3){C}
\uput[dr](4,4.1){D}\uput[dr](2.2,0.3){E}
\uput[ur](3.8,1.15){G}\uput[dr](1.8,2){H}
\rput(1,4.8){31 cm}\rput(3.4,4.9){34 cm}
\rput(0.4,2.7){56 cm}\rput(2.68,1.9){? ?}
\end{pspicture}} 

\textit{Calculons la longueur $CE$}

\textit{L'angle $\widehat{\text{ACE}}$ est droit donc le triangle $ACE$ est rectangle en $C$.}

\textit{D'après le théorème de Pythagore, $CE^2=AE^ 2-AC^2=56^2-34^2=1980$} 

\textit{(ABDC est un rectangle donc $AC=BD=34$~cm)}

\textit{Donc \fbox{$CE=\sqrt{1980}~\text{cm}\approx 44,5$~cm} arrondi au mm, donc la hauteur de ce siège est adaptée.} %La hauteur de ce siège lui est-elle adaptée ? 

\newpage

