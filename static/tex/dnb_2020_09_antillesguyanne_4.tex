
\medskip

Voici la série des temps exprimés en secondes, et réalisés par des nageuses lors de la finale du 100 mètres féminin nage libre lors des championnats d'Europe de natation de 2018 :

\begin{center}
\begin{tabularx}{\linewidth}{|*{8}{>{\centering \arraybackslash}X|}}\hline
53,23&54,04&53,61&54,52&53,35&52,93&54,56&54,07\\ \hline
\end{tabularx}
\end{center}

\begin{enumerate}
\item La nageuse française, Charlotte BONNET, est arrivée troisième à cette finale. Quel est le temps, exprimé en secondes, de cette nageuse ?
\item Quelle est la vitesse moyenne, exprimée en m/s, de la nageuse ayant parcouru les $100$ mètres en $52,93$ secondes ? Arrondir au dixième près.
\item Comparer moyenne et médiane des temps de cette série.

\medskip

Sur une feuille de calcul, on a reporté le classement des dix premiers pays selon le nombre de médailles d'or lors de ces championnats d'Europe de natation, toutes disciplines confondues :

\begin{center}
\begin{tabularx}{\linewidth}{|c|c|c|*{4}{>{\centering \arraybackslash}X|}}\hline
&A&B &C& D &E &F\\ \hline
1& Rang &Nation &Or 	&Argent &Bronze &Total\\ \hline
2&1		&Russie 		&23	&15 &9 	&47 \\ \hline
3&2		&Grande-Bretagne&13	&12 &9 	&34 \\ \hline
4&3		&Italie 		&8	&12 &19 &39\\ \hline
5&4		&Hongrie 		&6	&4	&2	&12\\ \hline
6&5		&Ukraine 		&5	&6 	&2 	&13\\ \hline
7&6		&Pays-Bas 		&5	&5 	&2 	&12\\ \hline
8&7		&France 		&4	&2 	&6 	&12\\ \hline
9&8		&Suède 			&4	&0 	&0 	&4\\ \hline
10&9	&Allemagne 		&3	&6 	&10 &19\\ \hline
11&10	&Suisse 		&1	&0 	&1 	&2\\ \hline
\end{tabularx}
\end{center}

\item Est-il vrai qu'à elles deux, la Grande-Bretagne et l'Italie ont obtenu autant de médailles d'or
que la Russie ?
\item Est-il vrai que plus de 35\,\% des médailles remportées par la France sont des médailles d'or ?
\item Quelle formule a-t-on pu saisir dans la cellule F2 de cette feuille de calcul, avant qu'elle soit
étirée vers le bas jusqu'à la cellule F11 ?

\end{enumerate}

\bigskip

