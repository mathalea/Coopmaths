\textbf{\textsc{Exercice 3} \hfill 6 points}

\medskip

\textbf{Attention les figures tracées ne respectent ni les mesures de longueur, ni les mesures d'angle} 

\medskip
 
Répondre par \og vrai \fg{} ou \og faux \fg{} ou \og on ne peut pas savoir \fg{} à chacune des affirmations suivantes et expliquer votre choix.

\medskip
\begin{enumerate}
\item Tout triangle inscrit dans un cercle est rectangle. 
\item Si un point M appartient à la médiatrice d'un segment [AB] alors le triangle 
AMB est isocèle. 
\item~

\parbox{0.45\linewidth}{Dans le triangle ABC suivant, 

AB = 4 cm.}\hfill
\parbox{0.45\linewidth}{\psset{unit=1cm}\begin{pspicture}(4,3.5)
\pspolygon (0.5,0.5)(3.5,0.5)(0.5,2.8)%ABC
\uput[dl](0.5,0.5){A}\uput[dr](3.5,0.5){B}\uput[u](0.5,2.8){C}
\uput[ur](1.75,1.8){8 cm}\psarc(3.5,0.5){1cm}{141}{180}
\rput(2.85,0.7){60\degres}
\end{pspicture}}
\item~

\parbox{0.45\linewidth}{Le quadrilatère ABCD ci-contre 
est un carré.}\hfill
\parbox{0.45\linewidth}{\psset{unit=1cm}\begin{pspicture}(4,3.5)
%\psgrid
\pspolygon(0.5,0.5)(3.5,0.3)(3.3,3.2)(0.4,3)%DCBA
\uput[ul](0.4,3){A}\uput[ur](3.3,3.2){B}\uput[dr](3.5,0.3){C}
\uput[dl](0.5,0.5){D}
\psline(1.9,0.5)(2.1,0.3)\psline(3.3,1.8)(3.5,1.6)
\psline(1.9,3.2)(2.1,3)\psline(0.3,1.8)(0.6,1.6)
\psline(0.4,2.6)(0.8,2.6)(0.8,3.03)
\end{pspicture}}
\end{enumerate}
 
\vspace{0,5cm}

