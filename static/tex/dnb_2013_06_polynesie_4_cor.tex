\textbf{Exercice 4 \hfill 7 points}

\medskip

%\og  Fenua Aquaculture \fg{} commercialise le paraha peue au prix de \np{2500} F/kg.
%
%La balance électronique a imprimé des tickets pour l'achat de trois poissons :
%
%\begin{center}
%\begin{tabularx}{\linewidth}{*{3}{>{\centering \arraybackslash}X}}
%\psset{unit=1cm}
%\begin{pspicture}(3,4)
%\psline(0.5,3.8)(0,3.8)(0,0)(3,0)(3,3.8)(2.5,3.8)
%\rput(1.5,3.8){Ticket \No 1}
%\uput[r](0,3){FENUA}
%\uput[r](0,2.5){AQUACULTURE}
%\uput[r](0,2.){SERVI LE  \quad 0,8 kg}
%\uput[r](0,1.5){17.06.13  \np{2500} F/kg}
%\uput[r](0,1){TOTAL}
%\uput[r](1.5,1){\fbox{\np{2000} F}}
%\end{pspicture}&\psset{unit=1cm}
%\begin{pspicture}(3,4)
%\psline(0.5,3.8)(0,3.8)(0,0)(3,0)(3,3.8)(2.5,3.8)
%\rput(1.5,3.8){Ticket \No 2}
%\uput[r](0,3){FENUA}
%\uput[r](0,2.5){AQUACULTURE}
%\uput[r](0,2.){SERVI LE  \quad 0,8 kg}
%\uput[r](0,1.5){17.06.13  \np{2500} F/kg}
%\uput[r](0,1){TOTAL}
%\uput[r](1.5,1){\fbox{1500 F}}
%\end{pspicture}&\psset{unit=1cm}
%\begin{pspicture}(3,4)
%\psline(0.5,3.8)(0,3.8)(0,0)(3,0)(3,3.8)(2.5,3.8)
%\rput(1.5,3.8){Ticket \No 3}
%\uput[r](0,3){FENUA}
%\uput[r](0,2.5){AQUACULTURE}
%\uput[r](0,2.){SERVI LE  \quad 0,8 kg}
%\uput[r](0,1.5){17.06.13  \np{2500} F/kg}
%\uput[r](0,1){TOTAL}
%\uput[r](1.5,1){\fbox{\np{3000} F}}
%\end{pspicture}
%\end{tabularx}
%\end{center}

\begin{enumerate}
\item Compléter le tableau ci-dessous à l'aide des 3 tickets :

\medskip

\begin{tabularx}{0.9\linewidth}{|m{3cm}|*{3}{>{\centering \arraybackslash}X|}c} \cline{1-4}
Ticket						&\No 1		&\No 2		&\No 3	&\\ \cline{1-4}
Masse du paraha peue (kg)	&0,8			&0,6	&1,2	&\\ \cline{1-4}
Prix (F)					&\np{2000}	&\np{1500}	&\np{3000}		&\psline(0,1)(0.25,0.75)\psframe(0.25,0.75)(1.35,0.25)\rput(0.8,0.45){$\times$ \np{2500}}\psline(0.25,0.25)(0,0)\\ \cline{1-4}
\end{tabularx}

\medskip

\item %Le prix est proportionnel à la masse du paraha peue acheté.

%Indiquer la valeur du coefficient de proportionnalité correspondant au tableau de la question 1 : \dotfill
On a $\dfrac{\np{2000}}{0,8} = \dfrac{\np{1500}}{0,6} = \dfrac{\np{3000}}{1,2} = \np{2500}$.
\item %Des données ont été effacées sur les deux tickets ci -dessous :

\parbox{0.4\linewidth}{\psset{unit=1cm} \begin{pspicture}(3,4)
\psline(0.5,3.8)(0,3.8)(0,0)(3,0)(3,3.8)(2.5,3.8)
\rput(1.5,3.8){Ticket \No 4}
\uput[r](0,3){FENUA}
\uput[r](0,2.5){AQUACULTURE}
\uput[r](0,2.){SERVI LE  \quad 0,7 kg}
\uput[r](0,1.5){17.06.13  \np{2500} F/kg}
\uput[r](0,1){TOTAL}
\uput[r](1.5,1){\fbox{\np{1750} F}}
\end{pspicture}}\hfill \parbox{0.55\linewidth}{%Calculer et compléter les valeurs effacées sur les deux tickets.

On a $\dfrac{\np{1750}}{\np{2500}} = 0,7$~kg

}

\parbox{0.4\linewidth}{\psset{unit=1cm}
\begin{pspicture}(3,4)
\psline(0.5,3.8)(0,3.8)(0,0)(3,0)(3,3.8)(2.5,3.8)
\rput(1.5,3.8){Ticket \No 5}
\uput[r](0,3){FENUA}
\uput[r](0,2.5){AQUACULTURE}
\uput[r](0,2.){SERVI LE  \: 0,900 kg}
\uput[r](0,1.5){17.06.13  \np{2500} F/kg}
\uput[r](0,1){TOTAL}
\uput[r](1.5,1){\fbox{\np{2250} F}}
\end{pspicture}}
\hfill \parbox{0.55\linewidth}{1 kg coûte \np{2500}~F, donc 0,9~kg coûte :

 $0,9 \times \np{2500} = \np{2250}$~F.

}

\item %On veut vérifier graphiquement les résultats des questions précédentes.
%Les points suivants ont pour abscisse la masse du paraha peue et pour ordonnée le prix payé :

\[\text{A}(1~;~\np{2500}), \text{B}(0,8~;~\np{2000}),  \text{C}(0,6~;~\np{1500})\quad  \text{et}\: \text{D}(1,2~;~\np{3000}).\]

	\begin{enumerate}
		\item %Placer les points C et D dans le repère ci-après.
		Voir ci-dessous
		\item Tracer la droite passant par les quatre points.
\begin{center}
\psset{xunit=7cm,yunit=0.00225cm,comma=true}
\begin{pspicture}(-0.1,-200)(1.4,3500)
\multido{\n=0.0+0.1}{15}{\psline[linestyle=dotted](\n,0)(\n,3500)}
\multido{\n=0+500}{8}{\psline[linestyle=dotted](0,\n)(1.4,\n)}
\psaxes[linewidth=1.25pt,Dx=0.1,Dy=500](0,0)(1.4,3500)
\psdots[dotstyle=+,dotangle=45,dotscale=1.6](0.8,2000)(1,2500)(0.6,1500)(1.2,3000)
\uput[u](1.3,0){Masse (kg)}\uput[r](0,3400){Prix (F)}
\uput[u](0.8,2000){B}\uput[u](1,2500){A}
\uput[u](0.6,1500){C}\uput[u](1.2,3000){D}
\psplot{0}{1.3}{2500 x mul}
\end{pspicture}
\end{center}
	\end{enumerate}
\end{enumerate}

\bigskip

