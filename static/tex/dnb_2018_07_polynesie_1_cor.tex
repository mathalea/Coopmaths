
\medskip

\begin{enumerate}
\item La probabilité d'arriver en A est égale à $\dfrac{1}{2} \times \dfrac{1}{2} = \dfrac{1}{4}$.

La probabilité d'arriver en A est égale à $\dfrac{1}{2} \times \dfrac{1}{2} + \dfrac{1}{2} \times \dfrac{1}{2} = \dfrac{1}{4} + \dfrac{1}{4} = \dfrac{1}{2}$ : l'affirmation est fausse.

\emph{Remarque : la probabilité d'arriver en {\rm C} est égale à }$\dfrac{1}{2} \times \dfrac{1}{2} = \dfrac{1}{4}$.

On a bien  $\dfrac{1}{4} +  \dfrac{1}{2} +  \dfrac{1}{4} =  1$
\item \np{1000} personnes ont besoin de $\np{1000} \times \np{7000} = \np{7000000}$~kWh par an.

L'éolienne produit 5~GWh = \np{5000000}~KWh soit moins : l'affirmation est vraie.
\item $45\,\% = \dfrac{45}{100} = 0,45$ ; $0,498 < \dfrac{305}{612} < 0,499$ ;

0,5 = 0,500 ; $730 \times 10^{-3} = 0,730$.

On a bien :

$0,45 < 0,498 < \dfrac{305}{612}  < 0,499 < 0,5 < 0,73$ : l'affirmation est vraie.
\item Il y a $20 \times (2 + 5 + 4 + 3 + 4) = 20 \times 18 = 360$ salariés.
Il y a $ 20 \times (3 + 4) = 140$ salariés gagnant plus de \np{1700}~euros.

Or : $\dfrac{140}{360}= \dfrac{14}{36} \approx 38,9\,\%$ : l'affirmation est fausse.
\end{enumerate}

\bigskip

