\textbf{\textsc{Exercice} 4 \hfill 5 points}

\medskip

Trois figures codées sont données ci-dessous. Elles ne sont pas dessinées en vraie grandeur. Pour chacune d'elles, déterminer la mesure de l'angle 
$\widehat{\text{ABC}}$.

\bigskip

\begin{tabularx}{\linewidth}{|*{2}{>{\centering \arraybackslash}X|}}\hline
\multicolumn{2}{|c|}{\psset{unit=0.6cm}
\begin{pspicture}(10,6)
\psframe(1,4.9)(1.3,5.2)\psarc(9,5.2){8mm}{-180}{-150}
\rput(-4,5.5){Figure 1}
\pspolygon(1,0.5)(1,5.2)(9,5.2)
\uput[ul](1,5.2){A} \uput[ur](9,5.2){B} \uput[dl](1,0.5){C} 
\rput{90}(0.1,2.85){AC = 3cm} 
\rput{32}(5,2.5){BC = 6cm}\rput(7.4,4.85){?}
\end{pspicture}}\\ \hline 
\psset{unit=0.6cm}
\begin{pspicture}(-5,-5)(5,5.5)
\pscircle(0,0){4.5}
\pspolygon(4.5;20)(4.5;90)(4.5;200)
\uput[ur](4.5;20){A} \uput[dl](4.5;200){B} \uput[u](4.5;90){C}
\uput[dr](0,0){O}
\psline(-1.8,-0.4)(-1.6,-0.8)\psline(-1.9,-0.4)(-1.7,-0.8)
\psline(1.8,0.4)(1.6,0.8)\psline(1.9,0.4)(1.7,0.8)
\psline(-0.2,1.9)(0.2,1.9)\psline(-0.2,2)(0.2,2)
\psline(0;0)(4.5;90)
\rput(2.1,1.75){59\degres}
\rput(-4.25,4.5){Figure 2 }
\psarc(4.5;20){8mm}{-215}{-160}
\psarc(4.5;200){8mm}{20}{54} \rput(-2.8,-0.4){?}
\rput(0,-4.8){[AB] est un diamètre du cercle de centre O.} 
\end{pspicture}&\psset{unit=0.6cm}
\begin{pspicture}(-5,-5)(5,5)
\pscircle(0,0){4}
\pspolygon(4;20)(4;92)(4;164)(4;236)(4;308)
\uput[u](4;92){A}\uput[l](4;164){B}\uput[dl](4;236){C}\uput[dr](4;308){D}
\uput[ur](4;20){E}
\psarc(4;164){8mm}{-70}{40}\rput(-2.2,1){?}
\rput(-4.25,4.5){Figure 3}
\rput{20}(2;20){\psline(0,0.2)(0,-0.2)\psline(0.1,0.2)(0.1,-0.2)}
\rput{92}(2;92){\psline(0,0.2)(0,-0.2)\psline(0.1,0.2)(0.1,-0.2)}
\rput{164}(2;164){\psline(0,0.2)(0,-0.2)\psline(0.1,0.2)(0.1,-0.2)}
\rput{236}(2;236){\psline(0,0.2)(0,-0.2)\psline(0.1,0.2)(0.1,-0.2)}
\rput{308}(2;308){\psline(0,0.2)(0,-0.2)\psline(0.1,0.2)(0.1,-0.2)}
\multido{\n=20+72,\na=19+72,\nb=21+72}{5}{\psline(0;0)(4;\n)\psline(2;\na)(2;\nb)}
\uput[r](0,-0.08){O}
\psline(3,-0.8)(3.2,-1.)
\psline(1.7,2.5)(2,2.8)
\psline(-1.9,2.45)(-2,2.8)
\psline(-3.1,-1.3)(-2.9,-1.2)
\psline(0.1,-3.4)(0.05,-3.05)
\end{pspicture}\\ \hline
\end{tabularx}

\bigskip

