\textbf{\textsc{Exercice 4} \hfill 4,5 points}

\medskip

Cet exercice est un questionnaire à choix multiple. Aucune justification n'est attendue. Pour chacune des questions, une seule réponse est exacte.

Recopier sur la copie le numéro de la question et la réponse exacte.

Toute réponse exacte vaut 1.5 point. Toute réponse inexacte ou toute absence de réponse n'enlève pas de point.

\medskip

\textbf{Question 1}

Le nombre 2 est solution de l'inéquation :

\medskip
\begin{tabularx}{\linewidth}{*{4}{X}}
\textbf{a.~~}$x < 2$&\textbf{b.~~} $- 4x - 3 > - 10$
&\textbf{c.~~} $5x - 4 \leqslant 7$ &\textbf{d.~~} $8 - 3x \geqslant 3$
\end{tabularx}

\medskip

\textbf{Question 2}

la fonction $f$ qui à tout nombre $x$ associe le nombre $2x - 8$ est représentée par le

\medskip

\begin{tabularx}{\textwidth}{X c X}
\textbf{graphique a.}&~~&\textbf{graphique b.}\\
\psset{xunit=0.8cm,yunit=0.2cm}
\begin{pspicture*}(-1,-10)(5,4)
\psaxes[linewidth=1.25pt,Dy=4]{->}(0,0)(-1,-10)(5,4)
\psplot[plotpoints=1000,linecolor=blue]{-1}{4}{4 x mul 8 sub}
\uput[u](4.8,0){$x$}\uput[l](0,3.5){$y$}
\end{pspicture*}
&&
\psset{xunit=0.8cm,yunit=0.2cm}
\begin{pspicture}(-2,-10)(6,12)
\psaxes[linewidth=1.25pt,Dy=4]{->}(0,0)(-2,-10)(6,12)
\psplot[plotpoints=1000,linecolor=blue]{-2}{6}{8 2 x mul sub}
\uput[u](5.8,0){$x$}\uput[l](0,11.5){$y$}
\end{pspicture}
\\
\textbf{graphique c.}&&\textbf{graphique d.}\\
\psset{xunit=0.8cm,yunit=0.2cm}
\begin{pspicture*}(-1,-10)(5,4)
\psaxes[linewidth=1.25pt,Dy=4]{->}(0,0)(-1,-10)(5,4)
\psplot[plotpoints=1000,linecolor=blue]{-1}{6}{2 x mul 8 sub }
\uput[u](4.8,0){$x$}\uput[l](0,3.5){$y$}
\end{pspicture*}
&&
\psset{xunit=0.8cm,yunit=0.2cm}
\begin{pspicture*}(-2,-10)(6,12)
\psaxes[linewidth=1.25pt,Dy=4]{->}(0,0)(-2,-10)(6,12)
\psplot[plotpoints=1000,linecolor=blue]{-2}{6}{2 x mul 4 sub}
\uput[u](5.8,0){$x$}\uput[l](0,11.5){$y$}
\end{pspicture*}
\\
\end{tabularx}

\medskip

\textbf{Question 3}

Un coureur qui parcourt 100 mètres en 10 secondes a une vitesse égale :

\medskip
\begin{tabularx}{\linewidth}{*{4}{X}}
\textbf{a.~~} 6 km/min &\textbf{b.~~} 36 km/h &\textbf{c.~~} \np{3600} m/h &\textbf{d.~~} 10 km/h
\end{tabularx}
\vspace{0,5cm}

