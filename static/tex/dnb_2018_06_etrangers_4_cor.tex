
\medskip

\textbf{Partie A. Le gros sel}

\medskip

On commence par ranger la série dans l'ordre croissant:

30 -- 31 -- 31 -- 32 -- 32 -- 33 -- 34 -- 34 -- 36 -- 37 -- 38 -- 38 -- 39 -- 39 -- 40 -- 40 -- 42 -- 42 -- 43 -- 43 -- 45 -- 45 -- 46 -- 47 -- 48

\medskip

\begin{enumerate}
\item $e = 48 - 30 = 18$.
\item La série comporte 25 données.

$25 \div 2 = 12,5$. La médiane est donc la 13\up{e} donnée. $m = 39$.

La moitié des carreaux produit au moins 39 kg de gros sel.
\item  moyenne $= \dfrac{\text{somme totale}}{25} = \dfrac{965}{25} = 38,6$ kg de sel par carreau en moyenne.
\end{enumerate}

\bigskip

\textbf{Partie B. La fleur de sel}

\medskip

\begin{enumerate}
\item $V = \dfrac{(40 + 70) \times 35}{2} \times  40 = \np{77000}$~cm$^3$ = $77$~dm$^3$ = $77$ litres.
\item $77 \times  900 = \np{69300}$~g = $69,3$ kg.
\end{enumerate}

\vspace{0,5cm}

