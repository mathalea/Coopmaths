\textbf{Exercice 5 \hfill 4,5 points}

\medskip

Les \og 24 heures du Mans\fg est le nom d'une course automobile.

\begin{center}
\begin{tabularx}{\linewidth}{|X|X|}\hline
\textbf{Document 1 : principe de la course}

Les voitures tournent sur un circuit pendant 24 heures. La voiture gagnante est celle qui a
parcouru la plus grande distance.&\textbf{Document 2 : schéma du circuit}

\psset{unit=0.7cm}
\begin{pspicture}(8,5)
\pscurve[linewidth=2pt](0.4,3)(0.5,3.5)(1,3.7)(2,3.75)(3,3.8)(6,3.85)(7,3.65)(7.9,3.5)(7.5,2.4)(7,1.9)(6.6,1.7)(6.8,1.2)(6,1)(5,0.4)(4,0.6)(3,0.75)(2,1)(1.5,1.05)(1,1.6)(0.7,2.7)(0.4,3)
\psline{->}(2.4,0.7)(2,0.8)\rput{75}(2.65,1){\scriptsize Départ}\rput(7.3,1.2){\scriptsize Arnage}\rput(7,3.2){\scriptsize Pulsante}\rput(4.2,3.3){\scriptsize Hunaudières}\rput(1.5,3){\scriptsize Tertre rouge}
\rput(4,2.6){La longueur d'un tour}
\rput(4,2.1){est de 13,629 km}
\end{pspicture}\\ \hline
\textbf{Document 3 : article extrait d'un journal}

\begin{center} {\large \np{5405,470}}

C'est le nombre de kilomètres parcourus
par l'Audi R15+ à l'issue de la course.
\end{center}&\textbf{Document 4 : unités anglo-saxonnes}

L'unité de mesure utilisée par les anglo-saxons est le mile par heure (mile per hour) noté mph.

1 mile $\approx$ \np{1609}~mètres\\ \hline
\end{tabularx}
\end{center}

À l'aide des documents fournis:

\medskip

\begin{enumerate}
\item Déterminer le nombre de tours complets que la voiture Audi R15+ a effectués lors de cette course.
\item Calculer la vitesse moyenne en km/h de cette voiture. Arrondir à l'unité.
\item On relève la vitesse de deux voitures au même moment :

\setlength\parindent{8mm}
\begin{itemize}
\item[$\bullet~~$] Vitesse de la voiture \no 37 : 205 mph.
\item[$\bullet~~$] Vitesse de la voiture \no 38 : 310 km/h.
\end{itemize}
\setlength\parindent{0mm}

Quelle est la voiture la plus rapide ?
\end{enumerate}

\vspace{0,5cm}

