\textbf{\textsc{Exercice 6} \hfill 6 points}

\medskip

\begin{enumerate}
\item Si la reproduction se fait à l'échelle 1/300 (coefficient de réduction), il suffit alors de diviser toutes les longueurs par 300 pour connaître les dimensions du plan :

Hauteur : $\dfrac{32}{300}\approx  0,107$~m, soit environ 11~cm ;

Longueur : $\dfrac{317}{300}\approx 1,057$~m, soit environ 106~cm

Largeur : $\dfrac{317}{300}\approx 0,93$~m, soit 93~cm.
\item  
	\begin{enumerate}
		\item Pour réduire une superficie (exprimée ici en m$^2$), il faut la diviser par le coefficient de réduction au carré.
		
Aire de la reproduction  : $\dfrac{\np{69500}}{300^2} \approx  0,77$~m$^2$.
		\item On sait que la longueur du stade est d'environ 1,057 m et que la largeur est d'environ 0,93 m. L'ire de la reproduction du stade ne pourra donc pas dépasser l'aire du rectangle, soit : $1,057 \times  0,93 = \np{0,98301}$~m$^2$ soit moins 
que l'espace de 1~m$^2$ dont il dispose.
	\end{enumerate}
\end{enumerate}

\vspace{0,25cm}

