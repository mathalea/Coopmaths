\textbf{Exercice 5 : Sécurité routière \hfill 4 points} 

\bigskip

%En se retournant lors d'une marche arrière, le conducteur d'une camionnette voit le sol à 6 mètres derrière son camion. 
%
%Sur le schéma, la zone grisée correspond à ce que le conducteur ne voit pas lorsqu'il regarde en arrière. 
%
%\begin{center}
%\psset{unit=0.65cm}
%\begin{pspicture}(17,4.75)
%\pspolygon[fillstyle=solid,fillcolor=lightgray](6.35,0)(13.9,0)(6.35,2.7)%ECA
%\pscurve(0,1)(0.1,1.3)(0.3,2)(0.4,2.5)(0.75,2.7)(1,3)(1.6,4)(1.8,4.2)(2,4.3)(2.4,4.4)(5,4.3)(6,4.2)(6.2,4)(6.4,3)(6.45,2.35)(6.6,2.3)(6.5,1.6)(6.35,1.6)(6.3,1.3)(6.1,1.3)(6,1)(6,0.8)(5.85,0.8)(5.8,1)(5.7,1.3)(5.2,1.6)(5,1.6)(4.5,1.4)(4.3,1)(4.2,0.7)(3,0.65)(2.3,0.7)(2.2,1)(2,1.4)(1.5,1.6)(1,1.4)(0.8,1)(0.7,0.7)(0.3,0.7)(0,1)
%\pscurve(0.7,2.7)(1.2,2.7)(1.4,3)(1.8,4.2)
%\pscurve(1.5,2.7)(2.8,2.7)(3,2.8)(3,3.2)(3.,4.1)(2,4.1)(1.8,3.5)(1.5,2.7)
%\pscurve(3.2,2.7)(4.6,2.8)(4.6,3)(4.65,4.05)(3.3,4)(3.25,3.5)(3.2,2.7)
%\pscurve(4.8,2.7)(6,2.65)(6.2,3)(6,4)(5.1,4.1)(4.9,3.7)(4.8,2.7)
%\pscurve(0.4,2.5)(3,2.4)(6.3,2.4)
%\pscurve(0.3,2.3)(2,2.2)(6.3,2.1)
%\pscurve(1.5,0)(1,0.2)(0.85,0.5)(1,1.15)(1.5,1.45)(2.1,1)(2,0.3)(1.5,0)
%\pscurve(5.1,0)(4.6,0.2)(4.45,0.5)(4.6,1.15)(5.1,1.45)(5.7,1)(5.6,0.3)(5.1,0)
%\pscurve(1.1,0.6)(1.5,1)(1.9,0.5)(1.5,0.3)(1.1,0.6)
%\pscurve(1.3,0.7)(1.5,0.9)(1.7,0.7)(1.5,0.4)(1.3,0.7)
%\pscurve(4.8,0.7)(5,1.1)(5.4,0.6)(5,0.3)(4.8,0.7)
%\pscurve(4.9,0.7)(5.05,1)(5.25,0.7)(5,0.5)(4.9,0.7)
%\pscurve(6,0.8)(6.4,0.85)(6.75,1)(6.75,1.2)(6.3,1.3)
%\pscurve(6.4,2.4)(6.3,2)(6.3,1.6)
%\pscurve(0,1)(0.2,1.2)(0.8,1)
%\pscircle(2.5,3.6){0.4}
%\psline(2.4,3.2)(2.2,3)(1.9,2.6)
%\psline(2.6,3.2)(2.8,2.7)
%\psline(2.3,3.7)(2.4,3.6)
%\psline(2.5,3.7)(2.6,3.6)
%\psline(8.6,1.9)(8.6,0)%BD
%\psdots(8.6,1.9)(8.6,0)(6.35,0)(13.9,0)(6.35,2.7)
%\psline(0,0)(6.4,0)
%\uput[u](8.6,1.9){B}\uput[d](8.6,0){D}
%\uput[ur](6.35,2.7){A}\uput[d](6.35,0){E}
%\uput[d](13.9,0){C}
%\rput(14.5,3.5){Données :}
%\rput(14.5,1.9){\begin{tabular}{l}
%(AE) // (BD)\\ AE = 1,50 m\\ BD = 1,10 m\\ EC = 6 m
%\end{tabular}}
% \end{pspicture}
%\end{center} 

\begin{enumerate}
\item %Calculer DC.
Les droites (AE) et (BD) sont parallèles ; les points E, D, C d'une part, A, B, C de l'autre sont alignés dans cet ordre ; le théorème de Thalès permet d'écrire :

$\dfrac{\text{DC}}{\text{EC}} =  \dfrac{\text{BD}}{\text{AE}}$ soit $\dfrac{\text{DC}}{6} = \dfrac{1,1}{1,5}$ soit DC $ = 6\times \dfrac{1,1}{1,5} = 4,4$~m.
\item %En déduire que ED = 1,60~m.
On a ED = $\text{EC} - \text{DC} = 6 - 4,4 = 1,6$~m. 
\item %Une fillette mesure 1,10~m. Elle passe à 1,40~m derrière la camionnette. 

%Le conducteur peut-il la voir ? Expliquer.
Comme $1,4 < 1,6$ et que la jeune fille a pour taille BD, elle sera entièrement dans la zone grisée  : le conducteur ne la verra pas.
\end{enumerate}
 
\bigskip 

