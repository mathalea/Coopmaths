
\medskip

\begin{enumerate}
\item 
	\begin{enumerate}
		\item Avec la formule $f(x) = 220 - x$, on remplace $x$ par 5.
		
$220-5 = 215$. La fréquence cardiaque maximale recommandée pour un enfant de $5$ ans est de 215~pulsations/minute.
		\item Avec la formule $g(x) = 208 - 0,7x$, on remplace $x$ par 5. 
		
$208- 0,7\times 5 = 208 - 3,5 = 204,5$. La fréquence cardiaque maximale recommandée pour un enfant de $5$ ans est de 204~pulsations/minute (on ne compte pas de demi-pulsation !).
	\end{enumerate}
\item  
	\begin{enumerate}
		\item Sur l'annexe 2, on complète le tableau de valeurs comme ci-dessous : \\[2mm]
		\begin{tabularx}{\linewidth}{|*{12}{>{\centering \arraybackslash}X|}}\hline
$x$		&5		&10		&20		&30		&40	&50		&60		&70		&80		&90		&100\\ \hline
$f(x)$	&215 	&210 	&200 	&190 	&180&170 	&160 	&150 	&140	&130 	&120 \\ \hline
$g(x)$	&204,5 	&201 	&194 	&187 	&180&173 	&166 	&159 	&152 	&145 	&138\\ \hline
\end{tabularx}
		\item Sur l'annexe 2, on a tracé en rouge la droite $d$ représentant la fonction $f$ dans le repère tracé.
		\item Sur le même repère, on a tracé en violet la droite $d'$ représentant la fonction $g$.
	\end{enumerate}
\item  Selon la nouvelle formule, à  partir de 40~ans la fréquence cardiaque maximale
recommandée est supérieure ou égale à  celle calculée avec l'ancienne formule. Ceci se voit dans le tableau : avant la colonne correspondant à 40~ans, $f(x)$ est supérieur à $g(x)$ et après cette colonne,  $f(x)$ est inférieur à $g(x)$.

Ceci se voit aussi sur la représentation graphique : avant le point d’intersection de $d$ et $d’$ correspondant à 40~ans, $d$ est au-dessus de $d’$ et après ce point,  $d$ est en-dessous de $d’$.
\item  L’exercice physique, pour une personne de $30$~ans, est le plus efficace lorsque la
fréquence cardiaque atteint 80\,\% de  187~pulsations/minute.

$\dfrac{80}{100}\times187=149,6$

Pour que l'exercice physique soit le plus efficace pour une personne de $30$~ans, la fréquence cardiaque doit être de 149 pulsations/minute (on ne compte pas 6 dixièmes de pulsation !).
\end{enumerate}
\begin{center}
{\Large \textbf{Annexe 2}}

\bigskip

\bigskip

\psset{xunit =0.08cm,yunit=0.08cm}
\begin{pspicture}(-5,-5)(100,220)
\multido{\n=0+5}{21}{\psline[linecolor=cyan,linewidth=0.3pt](\n,0)(\n,220)}
\multido{\n=0+5}{45}{\psline[linecolor=cyan,linewidth=0.3pt](0,\n)(100,\n)}
\psaxes[linewidth=1.25pt,Dx=10,Dy=10,labelFontSize=\scriptstyle]{->}(0,0)(0,0)(100,220)
\psaxes[linewidth=1.25pt,Dx=10,Dy=10,labelFontSize=\scriptstyle](0,0)(0,0)(100,220)
\psline[linewidth=1.25pt,linecolor=red](5,215)(100,120)
\psline[linewidth=1.25pt,linecolor=violet](5,204,5)(100,138)
\psline[linestyle=dashed](40,0)(40,180)
\psline[linestyle=dashed](0,180)(40,180)
\uput[r](90,120){\textcolor{red}{$d$}}
\uput[r](90,148){\textcolor{violet}{$d’$}}
\uput[r](0,217.5){fréquence cardiaque}
\uput[u](93,0){$x$ (âge)}
\end{pspicture}
\end{center}

\vspace{0,5cm}

