\documentclass[10pt]{article}
\usepackage[T1]{fontenc}
\usepackage[utf8]{inputenc}%ATTENTION codage UTF8
\usepackage{fourier}
\usepackage[scaled=0.875]{helvet}
\renewcommand{\ttdefault}{lmtt}
\usepackage{amsmath,amssymb,makeidx}
\usepackage[normalem]{ulem}
\usepackage{diagbox}
\usepackage{fancybox}
\usepackage{tabularx,booktabs}
\usepackage{colortbl}
\usepackage{pifont}
\usepackage{multirow}
\usepackage{dcolumn}
\usepackage{enumitem}
\usepackage{textcomp}
\usepackage{lscape}
\newcommand{\euro}{\eurologo{}}
\usepackage{graphics,graphicx}
\usepackage{pstricks,pst-plot,pst-tree,pstricks-add}
\usepackage[left=3.5cm, right=3.5cm, top=3cm, bottom=3cm]{geometry}
\newcommand{\R}{\mathbb{R}}
\newcommand{\N}{\mathbb{N}}
\newcommand{\D}{\mathbb{D}}
\newcommand{\Z}{\mathbb{Z}}
\newcommand{\Q}{\mathbb{Q}}
\newcommand{\C}{\mathbb{C}}
\usepackage{scratch}
\renewcommand{\theenumi}{\textbf{\arabic{enumi}}}
\renewcommand{\labelenumi}{\textbf{\theenumi.}}
\renewcommand{\theenumii}{\textbf{\alph{enumii}}}
\renewcommand{\labelenumii}{\textbf{\theenumii.}}
\newcommand{\vect}[1]{\overrightarrow{\,\mathstrut#1\,}}
\def\Oij{$\left(\text{O}~;~\vect{\imath},~\vect{\jmath}\right)$}
\def\Oijk{$\left(\text{O}~;~\vect{\imath},~\vect{\jmath},~\vect{k}\right)$}
\def\Ouv{$\left(\text{O}~;~\vect{u},~\vect{v}\right)$}
\usepackage{fancyhdr}
\usepackage[french]{babel}
\usepackage[dvips]{hyperref}
\usepackage[np]{numprint}
%Tapuscrit : Denis Vergès
%\frenchbsetup{StandardLists=true}

\begin{document}
\setlength\parindent{0mm}
% \rhead{\textbf{A. P{}. M. E. P{}.}}
% \lhead{\small Brevet des collèges}
% \lfoot{\small{Polynésie}}
% \rfoot{\small{7 septembre 2020}}
\pagestyle{fancy}
\thispagestyle{empty}
% \begin{center}
    
% {\Large \textbf{\decofourleft~Brevet des collèges Polynésie 7 septembre 2020~\decofourright}}
    
% \bigskip
    
% \textbf{Durée : 2 heures} \end{center}

% \bigskip

% \textbf{\begin{tabularx}{\linewidth}{|X|}\hline
%  L'évaluation prend en compte la clarté et la précision des raisonnements ainsi que, plus largement, la qualité de la rédaction. Elle prend en compte les essais et les démarches engagées même non abouties. Toutes les réponses doivent être justifiées, sauf mention contraire.\\ \hline
% \end{tabularx}}

% \vspace{0.5cm}\textbf{\textsc{Exercice 8} \hfill 4 points}

\medskip  

%\parbox{0.5\linewidth}{Les longueurs sont données en centimètres. 
%
%ABCD est un trapèze.}\hfill
%\parbox{0.48\linewidth}{\psset{unit=0.8cm}
%\begin{pspicture}(7,3)
%\pspolygon(0,0)(1,3)(4,3)(7,0)
%\psline[linestyle=dashed](0,0)(0,3)(1,3)
%\psline[linestyle=dashed](4,3)(7,3)(7,0)
%\psframe(0,3)(0.2,2.8)
%\psframe(7,3)(6.8,2.8)
%\rput(2.5,3){o}\rput(5.5,3){o}\rput(7,1.5){o}
%\uput[r](7,1.5){3}\uput[u](0.5,3){1}\uput[d](3.5,0){7}
%\uput[ur](1,3){A} \uput[ur](4,3){B} \uput[dr](7,0){C} \uput[dl](0,0){D} 
%\end{pspicture}} 
%
%\medskip

\begin{enumerate}
\item 
	\begin{enumerate}
		\item %Donner une méthode permettant de calculer l'aire du trapèze ABCD.
\emph{Méthode $1$} : on part de l'aire du rectangle à laquelle on retire l'aire des deux triangles rectangles :

$7 \times 3 - \left(\dfrac{1}{2} \times 3 \times 1 \right)  - \left(\dfrac{1}{2} \times 3 \times  3\right) = 21 -  \dfrac{3}{2} - \dfrac{9}{2} = 21 - 6 = 15$~cm$^2$.

\emph{Méthode $1$} : on utilise la formule de l'aire du trapèze : 

$\dfrac{(B + b) \times h}{2} = \dfrac{(7 + 3)\times 3}{2} = 15$~cm$^2$.
		\item %Calculer l'aire de ABCD.
		Voir ci-dessus.
	\end{enumerate} 
\item %\textbf{Dans cette question, si le travail n'est pas terminé, laisser tout de même une trace de la recherche. Elle sera prise en compte dans l'évaluation.}
 
%L'aire d'un trapèze $A$ est donnée par l'une des formules suivantes. Retrouver la formule juste en expliquant votre choix.
C'est la deuxième expression qui est correcte.

Il suffit de racer une diagonale du trapèze pour retrouver cette formule  : l'aire du trapèze est la somme des aires de deux triangles : $\dfrac{Bh}{2} + \dfrac{bh}{2} = \dfrac{(b + B)h}{2}$.

\begin{center}
\psset{unit=0.8cm}
\begin{pspicture}(0,-0.2)(7,3.2)
\pspolygon(0,0)(1,3)(4,3)(7,0)
\psline[linestyle=dotted,linecolor=blue](0,0)(4,3)
\psline[linestyle=dashed](4,3)(7,3)(7,0)
\psframe(7,3)(6.8,2.8)
\uput[u](2.5,3){$b$}
\uput[d](3.5,0){$B$}
\uput[r](7,1.5){$h$} 
\end{pspicture}
\end{center}
\medskip

%\begin{tabularx}{\linewidth}{*{3}{X}}
%$A = \dfrac{(b . B)h}{2}$& 
%$A = \dfrac{(b + B)h}{2}$& 
%$A = 2(b + B)h$
%\end{tabularx} 
\end{enumerate}
\end{document}\end{document}