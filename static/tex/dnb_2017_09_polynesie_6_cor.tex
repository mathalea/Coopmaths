
\medskip

%La figure ci-après est la copie d'écran d'un programme réalisé avec le logiciel
%\og Scratch \fg.
%
%\medskip

%\parbox{0.4\linewidth}{
\begin{enumerate}
\item %Montrer que si on choisit 2 comme nombre de départ, alors le programme renvoie $- 5$
Avec $x = 2, \: y = x^2 - 9 = 4 - 9 = - 5$.
\item %Que renvoie le programme si on choisit au départ :
	\begin{enumerate}
		\item %le nombre $5$ ?
si $x =  5, \: y = 5^2 - 9 = 25 - 9 = 16$ ;
		\item %le nombre $- 4$ ?
si $x = - 4, \: y = (- 4)^2 - 9 = 16 - 9 = 7$.
	\end{enumerate}
\item  %Déterminer les nombres qu'il faut choisir au départ pour que le programme renvoie $0$.
Il faut que $y = x^2 - 9 = 0$, soit $(x + 3)(x - 3) = 0$ ou $\left\{\begin{array}{l c l}
x + 3	&=&0\\
		&\text{ou}&\\
x - 3	&=&0
\end{array}\right.$ et finalement $\left\{\begin{array}{l c r}
x	&=&- 3\\
	&\text{ou}&\\
x	&=&3
\end{array}\right.$

Pour obtenir 0 à la fin du programme on peut choisir au départ $- 3$ ou 3.
\end{enumerate}
%}\hfill
%\parbox{0.54\linewidth}{\begin{\small}{\begin{scratch}
%  \blockinit{quand \greenflag est cliqué}
%  \blockvariable{cacher la variable \txtbox{\ovalvariable{x}\selectarrownum}}
%  \blockvariable{cacher la variable \txtbox{\ovalvariable{y}\selectarrownum}}
%  \blocksensing{demander \txtbox{Choisis un nombre} et attendre}
%  \blockvariable{mettre \txtbox{\ovalvariable{x}\selectarrownum} à \ovalsensing{réponse}}
%  \blockvariable{mettre \txtbox{\ovalvariable{y}\selectarrownum} à \ovaloperator{\ovalvariable{x} * \ovalvariable{x} - \ovalnum{9}}}
%   \blocklook{dire \txtbox{En choississant} pendant \ovalnum{1} seconde} 
%   \blocklook{dire \ovalsensing{réponse} pendant \ovalnum{1} seconde}
%   \blocklook{dire \txtbox{On obtient} pendant \ovalnum{1} seconde}
%   \blocklook{dire \ovalvariable{y}}                 
%\end{scratch}\end{small}}
%}	


