
\begin{enumerate}
\item Calculons la moyenne pour la ville de Grenoble:

$m_{\text{Grenoble}} = \dfrac{634}{10} = 63,4~\mu$g/m$^3$.

Or $63,4 \mu$g/m$^3 < 72,5~\mu$g/m$^3$, 
donc la moyenne $m_{\text{Lyonnaise}}$ est supérieure.
\item  $E_{\text{Grenoble}} = 89 - 32 = 57~\mu$g/m$^3$.

$E_{\text{Lyon}} = 107 - 22 = 85~\mu$g/m$^3$.

L'étendue la plus importante est celle de la ville de Lyon.
\item  La médiane est de $83,5$~g/m$^3$.

La série possède 10 valeurs. La médiane nous indique qu'au moins 50\,\% des
valeurs sont égales à $83,5~\mu$~g/m$^3$.

L'affirmation est juste.
\end{enumerate}

\bigskip

