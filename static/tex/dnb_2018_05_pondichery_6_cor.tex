
\medskip

%Chris fait une course à vélo tout terrain (VTT). Le graphique ci-dessous représente sa fréquence cardiaque (en battements par minute) en fonction du temps lors de la course.
%
%\begin{center}
%
%\psset{xunit=0.225cm,yunit=0.05cm}
%\begin{pspicture}(-2,-15)(53,180)
%\uput[u](27,170){Fréquence cardiaque de Chris}
%\multido{\n=0+5}{11}{\psline[linewidth=0.15pt](\n,0)(\n,170)}
%\multido{\n=0+10}{18}{\psline[linewidth=0.15pt](0,\n)(53,\n)}
%\psaxes[linewidth=1.25pt,Dx=5,Dy=10](0,0)(0,0)(53,170)
%\uput[d](27,-9){Durée (en min)}
%\rput{90}(-5,85){Fréquence cardiaque (bat./min)}
%\psline[linewidth=1.5pt](0,52)(2.5,57)(3.3,53)(5,60)(7,70)(8.5,142)(10,149.5)(12,151)(15,146)(16.5,148)(17.5,140)(18,151)(20,140)(21.5,143)(22,142)(23,150)(25,139.5)(26,144)(28,142)(28.5,160)(30,147)(32,139)(33,141)(34,136)(35,141)(36.5,140)(38,150)(39,139)(40,145)(44,75)(45,70)(46.5,69.5)(49,70)(50,71)(52,69)(53,69.5)
%\end{pspicture}
%\end{center}
%
%\medskip

\begin{enumerate}
\item %Quelle est la fréquence cardiaque de Chris au départ de sa course ?
On lit à peu près 52 battements par minute au départ de la course.
\item %Quel est le maximum de la fréquence cardiaque atteinte par Chris au cours de sa course ?
La fréquence la plus haute est voisine de 160 battements par minute.
\item %Chris est parti à 9~h~33 de chez lui et termine sa course à 10~h~26.

%Quelle a été la durée, en minutes de sa course ?
La durée de la course est :

9~h 86 - 9~h~33 = 53~min.
\item  %Chris a parcouru 11~km lors de cette course.

%Montrer que sa vitesse moyenne est d'environ $12,5$ km/h.
On a $v = \dfrac{d}{t} = \dfrac{11}{53}$~km/min soit $\dfrac{11 \times 60}{53} \approx 12,45$ soit environ 12,5~km/h au dixième près.
\item %On appelle FCM (Fréquence Cardiaque Maximale) la fréquence maximale que peut supporter l'organisme. Celle de Chris est FCM $= 190$~battements par minute. 

%En effectuant des recherches sur des sites internet spécialisés, il a trouvé le tableau suivant :
%
%\begin{center}
%\begin{tabularx}{\linewidth}{|p{2.5cm}|*{4}{>{\centering \arraybackslash}X|}}\hline
%Effort &léger &soutenu &tempo &seuil anaérobie\\ \hline
%Fréquence cardiaque mesurée&Inférieur à 70\,\% de la FCM&70 à 85\,\% de la FCM&85 à 92\,\% de la FCM&92 à 97\,\% de la FCM\\ \hline
%\end{tabularx}
%\end{center}
%
%Estimer la durée de la période pendant laquelle Chris a fourni un effort soutenu au cours de sa course.
On a $190 \times \dfrac{70}{100} = 133$ et $190\times \dfrac{85}{100} = 161,5$.

Il faut donc estimer le temps pendant lequel la fréquence a été comprise entre 133 et 161,5 battements par minute, soit en fait supérieure à 133.

On lit approximativement que cette fréquence a dépassé 133 de la 8\up{e} à la 42\up{e} minute, soit pendant 34 minutes.
\end{enumerate}

\vspace{0,5cm}

