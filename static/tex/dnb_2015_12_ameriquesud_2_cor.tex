\textbf{\textsc{Exercice 2} \hfill 4 points}

\medskip

%On considère deux fonctions 
%
%\[f :\: x \longmapsto - 8x \quad \text{et} \quad  g :\: x \longmapsto - 6x + 4.\]
%
%On utilise un tableur pour calculer des images par $f$ et $g$.
%
%\begin{center}
%\begin{tabularx}{\linewidth}{|c|c|*{4}{>{\centering \arraybackslash}X|}}\hline
%&A & B & C&D & E \\ \hline
%1&$x$ &$-3$& 0& 2&\\ \hline
%2&$f(x) = - 8x$& 24& 0& $- 16$& $- 24$\\ \hline
%3&$g(x) = - 6x + 4$& 22& 4& $- 8$& $- 14$\\ \hline
%\end{tabularx}
%\end{center}

\begin{enumerate}
\item %Quelle formule peut-on saisir dans la cellule B2 avant de la recopier vers la droite ?
=B2*$(- 3)$.
\item %Le contenu de la cellule E1 a été effacé. Peux-tu le retrouver ?
On a à trouver l'antécédent de $- 24$ qui est $\dfrac{-24}{- 8} = 3$.
\item %On fabrique une nouvelle fonction $h :\: x \longmapsto f(x)\times g(x)$. 

%La fonction $h$ est-elle une fonction affine ?
On a $h(x) = - 8x (- 6x + 4) = 48x^2 - 32x$ : ce n'est pas une fonction affine.
\end{enumerate}
%%%%%%%%%%%%%
\vspace{0.25cm}

