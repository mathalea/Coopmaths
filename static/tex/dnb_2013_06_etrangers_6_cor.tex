\textbf{\textsc{Exercice 6} \hfill 4 points}

\medskip

%\textbf{Dans cet exercice, toute trace de recherche, même incomplète, sera prise en compte dans l'évaluation.}
%
%\medskip
% 
%\emph{On considère la figure ci-dessous, qui n'est pas en vraie grandeur.}
%
%\bigskip
%
%\parbox{0.6\linewidth}{\psset{unit=0.6cm}
%\begin{pspicture}(-0.2,-0.2)(10,7.3)
%\psframe(3,7)(9,0)
%\psline(3,7)(0,7)(9,2.8)
%\uput[u](0,7){A} \uput[u](3,7){B} \uput[u](9,7){C} 
%\uput[ur](3,5.5){M} \uput[r](9,2.8){F} \uput[ur](3,0){E} 
%\uput[ur](9,0){D} \uput[u](1.5,7){3} \uput[u](6,7){6} 
%\end{pspicture}}\hfill
%\parbox{0.38\linewidth}{BCDE est un carré de 6 cm de côté.
% 
%Les points A, B et C sont alignés et AB = 3 cm.
% 
%F est un point du segment [CD].
% 
%La droite (AF) coupe le segment [BE] en M.} 
%
%\medskip
%
%Déterminer la longueur CF par calcul ou par construction pour que les longueurs BM et FD soient égales. 
Appelons $x$ les longueurs égales BM et FD.

Les droites (BM) et (CF) sont parallèles (côtés opposés du carré).

Les points A, B C d’une part, A, M, F d’autre part sont alignés dans cet ordre. Le théorème de Thalès permet d’écrire :

$\dfrac{\text{AB}}{\text{AC}} = \dfrac{\text{BM}}{\text{CF}}$.

Or CF $ = 6 - x$ ; donc $\dfrac{3}{9} = \dfrac{x}{6 - x}$ d’où $3x = 6 - x$ ou $4x  = 6$ et $x = \dfrac{3}{2} = 1,5$~cm.

Conclusion : CF $ = 6 - x = 6 - 1,5 = 4,5$~(cm).

\emph{Remarque} : méthode par construction

Si les conditions sont remplies les segments [BM] et [FD] sont parallèles et de même longueur. Le quadrilatère BMDF est donc un parallélogramme ; ses diagonales [BD] et [MF] ont donc le même milieu O centre du carré BCDE.

D'où la construction : on construit les diagonales [BD] et [CE] du carré qui se coupent en O ; la droite (AO) coupe [BE] en M et [CD] en F. On mesure CF $ = 4,5$~cm.

\begin{center}
\psset{unit=0.6cm}
\begin{pspicture}(-0.5,-0.2)(10,7.3)
%\psgrid
\psframe(3,7)(9,1)
\psline(3,7)(0,7)%(9,2.5)
\uput[u](0,7){A} \uput[u](3,7){B} \uput[u](9,7){C} 
\uput[ur](3,5.5){M} \uput[r](9,2.8){F} \uput[ur](3,0){E} 
\uput[ur](9,0){D} \uput[u](1.5,7){3} \uput[u](6,7){6}
\psline[linestyle=dotted](3,7)(9,1)
\psline[linestyle=dotted](3,1)(9,7)
\psline[linestyle=dotted](0,7)(9,2.5) 
\uput[d](6,4){O}
\end{pspicture}
\end{center}

\bigskip

