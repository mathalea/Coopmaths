
\medskip 

\parbox{0.35\linewidth}{Un TeraWattheure est noté: 1~TWh.

La géothermie permet la production d'énergie électrique grâce à la chaleur des nappes d'eau souterraines.

Le graphique ci-contre représente les productions d'électricité par différentes
sources d'énergie en France en 2014.}\hfill
\parbox{0.64\linewidth}{\psset{unit=0.7cm}
\begin{pspicture}(-7.5,-3.8)(3.1,4)
%\psgrid
\psarc(0,0){2.9}{-134.6}{137.6}
\psarc(0,0){2.25}{-134.6}{137.6}
\psline(2.9;137.6)(2.25;137.6)\psline(2.9;-134.6)(2.25;-134.6)
\psarc(0,0){2.9}{140.6}{155.8}
\psarc(0,0){2.25}{140.6}{155.8}
\psline(2.9;140.6)(2.25;140.6)\psline(2.9;155.8)(2.25;155.8)
\psarc(0,0){2.9}{157.8}{202.7}
\psarc(0,0){2.25}{157.8}{202.7}
\psline(2.9;157.8)(2.25;157.8)\psline(2.9;202.7)(2.25;202.7)
\psarc(0,0){2.9}{204.7}{222.7}
\psarc(0,0){2.25}{204.7}{222.7}
\psline(2.9;204.7)(2.25;204.7)\psline(2.9;222.7)(2.25;222.7)
\rput(-2,-3.5){\scriptsize Statistiques de l'électricité en France 2014 RTE - chiffres de production 2014 - EDF}
\rput(1,3.6){\small Nucléaire : 415,9~TWh}\psline{->}(1,3.4)(1.2,2.6)
\rput(-4.8,3.2){\small Thermique à flamme :}
\rput(-4.8,2.8){\small 25,8 TWh}\psline{->}(-4.8,2.6)(-2.4,1.6)
\rput(-5,0.4){\small Hydraulique : }\psline{->}(-3.8,0.2)(-2.95,-0.2)
\rput(-5,0){\small 67,5 TWh}
\rput(-5,-2.2){\small Autres énergies}\psline{->}(-3.2,-2.2)(-2.4,-1.6)
\rput(-4.7,-2.8){\small (dont la géothermie) : 31 TWh}
\end{pspicture}}

\medskip

\begin{enumerate}
\item 
	\begin{enumerate}
		\item Calculer la production totale d'électricité en France en 2014.
		\item Montrer que la proportion d'électricité produite par les \og Autres énergies (dont la géothermie) \fg{} est environ égale à 5,7\,\%.
	\end{enumerate}
\item Le tableau suivant présente les productions d'électricité par les différentes sources d'énergie, en France, en 2013 et en 2014.
	
\begin{center}
\begin{tabularx}{\linewidth}{|m{3cm}|*{4}{>{\centering \arraybackslash}X|}}\cline{2-5}
\multicolumn{1}{c|}{~}&	Thermique à flamme& Hydrauli\-que &\footnotesize Autres énergies (dont la géothermie) & Nucléaire\\ \hline
Production en 2013 (en TWh) &43,5 &75,1 &28,1 &403,8\\ \hline
Production en 2014 (en TWh) &25,8 &67,5 &31 &415,9\\ \hline
Variation de production entre 2013 et 2014&$- 40,7\,\%$& $-10,1\,\%$& $+ 10,3\,\%$&$+ 3\,\% $\\ \hline
\end{tabularx}	
\end{center}

Alice et Tom ont discuté pour savoir quelle est la source d'énergie qui a le plus augmenté sa production d'électricité. 

Tom pense qu'il s'agit des \og Autres énergies (dont la géothermie) \fg{} et Alice pense qu'il s'agit du \og Nucléaire \fg. 

Quel est le raisonnement tenu par chacun d'entre eux ?
\item La centrale géothermique de Rittershoffen (Bas Rhin) a été inaugurée le 7 juin 2016. On y a creusé un puits pour capter de l'eau chaude sous pression, à \np{2500}~m de profondeur, à une température de $170$~degrés Celsius.
	
\medskip
	
\parbox{0.5\linewidth}{Ce puits a la forme du tronc de cône représenté ci-contre.

Les proportions ne sont pas respectées.

On calcule le volume d'un tronc de cône grâce à la formule suivante:

\[V = \dfrac{\pi}{3} \times h \times \left(R^2 + R \times r  + r^2\right)\]

où $h$ désigne la hauteur du tronc de cône, $R$ le rayon de la grande base et $r$ le
rayon de la petite base.

\textbf{a.} Vérifier que le volume du puits est environ égal à $225$~m$^3$.

\textbf{b.} La terre est tassée quand elle est dans le sol. Quand on l'extrait, elle n'est
plus tassée et son volume augmente de 30\,\%.
		
Calculer le volume final de terre à stocker après le forage du puits.}\hfill
\parbox{0.48\linewidth}{\psset{unit=0.8cm,arrowsize=2pt 4}
\begin{pspicture}(-3,0)(3,9.5)
%\psgrid
\psellipse(0,7)(2.3,0.7)
\scalebox{1}[0.3]{\psarc[linewidth=1.25pt](0,2.7){1.1}{180}{0}}%
\scalebox{1}[0.3]{\psarc[linestyle=dashed,linewidth=1.5pt](0,2.7){1.1}{0}{180}}%
\psline{<->}(-2.3,8)(2.3,8)
\psline{<->}(-1.1,0.4)(1.1,0.4)
\psline{<->}(2.4,7)(2.4,0.7)
\rput(0,0){\small Petite base de 20 cm de diamètre}
\rput(0,8.4){\small Grande base de 46 cm de diamètre}
\rput{-90}(2.8,3.85){\small Hauteur : \np{2500}~m}
\psline[linestyle=dashed](0,0.7)(0,7)
\psline(1.1,0.8)(1.4,2.4)\psline(-1.1,0.8)(-1.4,2.4)\psline[linestyle=dashed](1.4,2.4)(1.8,4.4)
\psline(1.8,4.4)(2.3,7)\psline(-1.8,4.4)(-2.3,7)\psline[linestyle=dashed](-1.4,2.4)(-1.8,4.4) 
\end{pspicture}}

\end{enumerate}

\bigskip
 
