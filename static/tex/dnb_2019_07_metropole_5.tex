
\medskip

\parbox{0.6\linewidth}{Olivia s'est acheté un tableau pour décorer le mur de son salon.

Ce tableau. représenté ci-contre, est constitué de quatre rectangles
identiques nommés \textcircled{1}, \textcircled{2}, \textcircled{3} et \textcircled{4} dessinés à l'intérieur d'un grand rectangle ABCD d'aire égale à 1,215~m$^2$. Le ratio longueur : largeur est égal à $3 : 2$ pour chacun des cinq rectangles.}\hfill
\parbox{0.35\linewidth}{\psset{unit=0.5cm}
\begin{pspicture}(9,6.5)
\psframe(9,6)\psframe[fillstyle=solid,fillcolor=lightgray](6,0)(9,2)%4
\rput(7.5,1){\textcircled{4}}
\psframe[fillstyle=solid,fillcolor=lightgray](0,0)(3,2)%2
\rput(1.5,1){\textcircled{2}}
\psframe[fillstyle=solid,fillcolor=lightgray](6,4)(9,6)%3
\rput(7.5,5){\textcircled{3}}
\psframe[fillstyle=solid,fillcolor=lightgray](2,2)(4,5)%1
\rput(3,3.5){\textcircled{1}}
\psline[linestyle=dashed](3,2)(6,2)
\psdots[dotstyle=+,dotangle=45](1.5,0)(1.5,2)(4.5,2)(4.5,0)(7.5,0)
\psdots(9,1)(9,3)(9,5)
\uput[ul](0,6){A} \uput[ur](9,6){B} \uput[dr](9,0){C} 
\uput[dl](0,0){D} \uput[r](9,4){E} \uput[u](6,2){F} 
\end{pspicture}
}
\bigskip

\begin{enumerate}
\item Recopier, en les complétant, les phrases suivantes. Aucune justification n'est demandée.

	\begin{enumerate}
		\item Le rectangle \ldots est l'image du rectangle \ldots par la translation qui transforme C en E.
		\item Le rectangle \textcircled{3} est l'image du rectangle \ldots par la rotation de centre F et d'angle $90$\degres dans le sens des aiguilles d'une montre.
		\item Le rectangle ABCD est l'image du rectangle \ldots par l'homothétie de centre \ldots et de rapport~$3$.
		
(Il y a plusieurs réponses possibles, une seule est demandée.)
 	\end{enumerate}
\item Quelle est l'aire d'un petit rectangle ?
\item Quelles sont la longueur et la largeur du rectangle ABCD ?
\end{enumerate}

\bigskip

