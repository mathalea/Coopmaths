\textbf{Exercice 2 : Pierre, feuille, ciseaux}

\begin{enumerate}
\item Je joue une partie face à  un adversaire qui joue au hasard et je choisis de jouer \og \textbf{pierre} \fg{}. 
	\begin{enumerate}
		\item Je perds la partie si mon adversaire choisit \og \textbf{feuille} \fg{} parmi les trois \linebreak possibilités \og \textbf{pierre} \fg{}, \og \textbf{ciseaux} \fg{} et \og \textbf{feuille} \fg{}, donc la probabilité que je perde la partie est égale à $\dfrac13$.
		\item La probabilité que je ne perde pas la partie est égale à $\dfrac23$. \quad Car $\left( 1-\dfrac13=\dfrac23\right)$
	\end{enumerate}
\item  Arbre des possibles de l'adversaire pour les deux parties. 

\makeatletter
 
\begin{forest}
	% Paramètres généraux de l'arbre :
	% - s sep :  distance verticale entre les nœuds.
	% - l :      distance horizontale entre les nœuds.
	% - grow ' : orientation de l'arbre (ici, vers la droite).
	for tree={s sep=.5cm, l=2.5cm, grow=0, grow'=east, align=left},
	before typesetting nodes={for tree={content=#1}},
	decision edge label/.style n args=3{
		% On met la probabilité de l'événement au milieu de la branche. S'il n'y en a pas, on ne met pas de fond.
		edge label/.expanded={node[midway, fill={\if#3\@empty none\else white\fi}]{#3}}
		% Autre possibilité : probabilité au-dessus de la branche.
		%edge label/.expanded={node[midway, above]{#3}}
	},
	decision/.style={if n=1
		{decision edge label={below}{north}{#1}}
		{decision edge label={left}{east}{#1}}
	},
	delay={for descendants={
		% Gestion de l'événement.
		decision/.expanded/.wrap pgfmath arg={\getProba#1\endget}{content},
		% Gestion de la probabilité.
		content/.expanded/.wrap pgfmath arg={\getEvene#1\endget}{content}
	}},
	% On créé le nœud principale (l'univers oméga).
%	[$\Omega$
	[$$
		[\text{P}/\frac13, name=premierLance % On créé un sous-nœud que l'on nomme `premierLance`.
			[\text{P}/\frac13, name=secondLance % Un sous-sous-nœud. On marque `événement/probabilité` (voir structure plus haut).
				% La probabilité en fin de branche. On la nomme `proba`. On enlève la branche avec `no edge`. 
				% Pensez à entourer la probabilité de trois { et } si vous obtenez une erreur.
				[{{{P(\text{P}, \text{P}) = \frac13 \times \frac13 = \frac19}}}/, no edge, name=proba]
			]
			[\text{F}/\frac13
				[{{{P(\text{P}, \text{F}) = \frac13 \times \frac13 = \frac19}}}/, no edge]
			]
			[\text{C}/\frac13
				[{{{P(\text{P}, \text{C}) = \frac13 \times \frac13 = \frac19}}}/, no edge]
			]
		]
		[\text{F}/\frac13 % Second sous-nœud.
			[\text{P}/\frac13
				[{{{P(\text{F}, \text{P}) = \frac13 \times \frac13 = \frac19}}}/, no edge]
			]
			[\text{F}/\frac13
				[{{{P(\text{F}, \text{F}) = \frac13 \times \frac13 = \frac19}}}/, no edge]
			]
			[\text{C}/\frac13
				[{{{P(\text{F}, \text{C}) = \frac13 \times \frac13 = \frac19}}}/, no edge]
			]
		]
		[\text{C}/\frac13 % Troisième sous-nœud.
			[\text{P}/\frac13
				[{{{P(\text{C}, \text{P}) = \frac13 \times \frac13 = \frac19}}}/, no edge]
			]
			[\text{F}/\frac13
				[{{{P(\text{C}, \text{F}) = \frac13 \times \frac13 = \frac19}}}/, no edge]
			]
			[C/\frac13
				[{{{P(\text{C}, \text{C}) = \frac13 \times \frac13 = \frac19}}}/, no edge]
			]
		]
	]
	% Noms des colonnes
	% ¯¯¯¯¯¯¯¯¯¯¯¯¯¯¯¯¯
	% "Probabilités" au-dessus de `proba`. On nomme ce texte `nomProba`.
	\node [above=5pt of proba, name=nomProba] {Probabilités};
	% "2de lancé" à l'intersection de droite verticale passant par `nomProba` et de la droite horizontale passant par `secondLance`.
	\node [name=nomSecondLance] at (secondLance |- nomProba) {2\up{de} partie}; 
	% Idem mais pour `premierLance` et `nomSecondLance`.
	\node at (premierLance |- nomSecondLance) {1\up{re} partie};
\end{forest}
 
\makeatother

\item  	\begin{enumerate}
		\item Je gagne les deux parties si mon adversaire choisit \og \textbf{ciseaux} \fg{} aux deux \linebreak parties, donc la probabilité que je gagne les deux parties est égale à \linebreak $P$(C, C), soit $\dfrac19$. 
		\item Je ne perds aucune des deux parties si mon adversaire choisit \og \textbf{ciseaux} \fg{} ou \og \textbf{pierre} \fg{} dans les parties donc la probabilité que je ne perde aucune des deux parties est égale à $P(\text{P}, \text{P})+P(\text{P}, \text{C})+P(\text{C}, \text{P}) + P(\text{C}, \text{C})=4\times \dfrac13=\dfrac49$. 
	\end{enumerate}
\end{enumerate}

\vspace{0,5cm}


\begin{minipage}{4.2cm}
