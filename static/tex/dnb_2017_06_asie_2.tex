
\medskip

Dans une classe de 24 élèves, il y a 16 filles.

\medskip

\begin{enumerate}
\item L'un des deux diagrammes ci-dessous peut-il représenter correctement la
répartition des élèves de cette classe ?

\begin{center}
\psset{unit=1cm}
\begin{pspicture}(12,2.5)
\psframe(0,1.3)(0.4,1.7)
\psframe[fillstyle=solid,fillcolor=lightgray](0,0.7)(0.4,1.1)
\uput[r](0.7,1.5){Garçons}
\uput[r](0.7,0.9){Filles}
\psframe(4,0.5)(7,2.5)\psframe[fillstyle=solid,fillcolor=lightgray](5.5,0.5)(7,2.5)
\uput[u](5.5,0){Diagramme 1} 
\uput[u](9.5,0){Diagramme 2}
\pspolygon[fillstyle=solid,fillcolor=lightgray](9.5,1.5)(9.5,0.5)(11,0.5)(11,2.5)(8,2.5)
\psframe(8,0.5)(11,2.5)
\psline(8,1.5)(11,1.5)\psline(9.5,0.5)(9.5,2.5)\psline(8,0.5)(11,2.5)\psline(8,2.5)(11,0.5)
\end{pspicture}
\end{center}

\item On a représenté la répartition des élèves de cette classe par un diagramme
circulaire.

\begin{center}
\psset{unit=1cm}
\begin{pspicture}(6,2.5)
\psframe(0,1.3)(0.4,1.7)
\psframe[fillstyle=solid,fillcolor=lightgray](0,0.7)(0.4,1.1)
\uput[r](0.7,1.5){Garçons}
\uput[r](0.7,0.9){Filles}
\pscircle(4.5,1.25){1.25}
\pswedge[fillstyle=solid,fillcolor=lightgray](4.5,1.25){1.25}{-150}{90}
\end{pspicture}
\end{center}

Écrire le calcul permettant de déterminer la mesure de l'angle du secteur qui représente
les garçons.
\end{enumerate}

\vspace{0,5cm}

