\textbf{Exercice 4 :\hfill 7 points}

\medskip

Voici le parcours du cross du collège La Bounty schématisé par la figure ci-dessous :

\begin{center}
\psset{unit=1cm,arrowsize=3pt 4}
\begin{pspicture}(12,8)
\pspolygon(1,1.5)(5.7,1.5)(5.7,3.7)(11.6,7.3)(1,7.3)%BYTNO
\psline[linestyle=dashed](5.7,3.7)(5.7,7.3)
\psframe(1,1.5)(1.2,1.7) \psframe(1,7.3)(1.2,7.1)
\psframe(5.7,7.3)(5.9,7.1)
\uput[dl](1,1.5){B} \uput[dr](5.7,1.5){Y} \uput[l](5.7,3.7){T} 
\uput[r](11.6,7.3){N} \uput[dl](5.7,7.3){U} \uput[ul](1,7.3){O}
\psline[linewidth=0.5pt]{<->}(1,7.5)(11.6,7.5)\uput[u](6.3,7.5){234 m}
\psline[linewidth=0.5pt]{<->}(0.8,7.3)(0.8,1.5)\rput{90}(0.3,4.4){155 m}
\psline[linewidth=0.5pt]{<->}(5.9,1.5)(5.9,3.7)\rput{90}(6.1,2.6){25 m}
\psline[linewidth=0.5pt]{<->}(1,1.3)(5.7,1.3)\uput[d](3.35,1.3){90 m}  
\end{pspicture}
\end{center}
 
\begin{enumerate}
\item Montrer que la longueur NT est égale à $194$~m. 
\item Le départ et l'arrivée de chaque course du cross se trouvent au point B. 

Calculer la longueur d'un tour de parcours. 
\item Les élèves de 3\up{e} doivent effectuer 4 tours de parcours. Calculer la longueur totale de leur course. 
\item Terii, le vainqueur de la course des garçons de 3ème a effectué sa course en 10 minutes et 42 secondes. 

Calculer sa vitesse moyenne et l'exprimer en mis. Arrondir au centième près. 
\item Si Terii maintenait sa vitesse moyenne, penses-tu qu'il pourrait battre le champion Georges Richmond qui a gagné dernièrement la course sur $15$~km des Foulées du Front de mer en $55$~minutes et 11 secondes ?
 
\textbf{Pour cette question, toute trace de recherche, même incomplète, sera prise en compte dans l'évaluation.} 
\end{enumerate}
 
\bigskip

