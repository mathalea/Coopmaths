\documentclass[10pt]{article}
\usepackage[T1]{fontenc}
\usepackage[utf8]{inputenc}%ATTENTION codage UTF8
\usepackage{fourier}
\usepackage[scaled=0.875]{helvet}
\renewcommand{\ttdefault}{lmtt}
\usepackage{amsmath,amssymb,makeidx}
\usepackage[normalem]{ulem}
\usepackage{diagbox}
\usepackage{fancybox}
\usepackage{tabularx,booktabs}
\usepackage{colortbl}
\usepackage{pifont}
\usepackage{multirow}
\usepackage{dcolumn}
\usepackage{enumitem}
\usepackage{textcomp}
\usepackage{lscape}
\newcommand{\euro}{\eurologo{}}
\usepackage{graphics,graphicx}
\usepackage{pstricks,pst-plot,pst-tree,pstricks-add}
\usepackage[left=3.5cm, right=3.5cm, top=3cm, bottom=3cm]{geometry}
\newcommand{\R}{\mathbb{R}}
\newcommand{\N}{\mathbb{N}}
\newcommand{\D}{\mathbb{D}}
\newcommand{\Z}{\mathbb{Z}}
\newcommand{\Q}{\mathbb{Q}}
\newcommand{\C}{\mathbb{C}}
\usepackage{scratch}
\renewcommand{\theenumi}{\textbf{\arabic{enumi}}}
\renewcommand{\labelenumi}{\textbf{\theenumi.}}
\renewcommand{\theenumii}{\textbf{\alph{enumii}}}
\renewcommand{\labelenumii}{\textbf{\theenumii.}}
\newcommand{\vect}[1]{\overrightarrow{\,\mathstrut#1\,}}
\def\Oij{$\left(\text{O}~;~\vect{\imath},~\vect{\jmath}\right)$}
\def\Oijk{$\left(\text{O}~;~\vect{\imath},~\vect{\jmath},~\vect{k}\right)$}
\def\Ouv{$\left(\text{O}~;~\vect{u},~\vect{v}\right)$}
\usepackage{fancyhdr}
\usepackage[french]{babel}
\usepackage[dvips]{hyperref}
\usepackage[np]{numprint}
%Tapuscrit : Denis Vergès
%\frenchbsetup{StandardLists=true}

\begin{document}
\setlength\parindent{0mm}
% \rhead{\textbf{A. P{}. M. E. P{}.}}
% \lhead{\small Brevet des collèges}
% \lfoot{\small{Polynésie}}
% \rfoot{\small{7 septembre 2020}}
\pagestyle{fancy}
\thispagestyle{empty}
% \begin{center}
    
% {\Large \textbf{\decofourleft~Brevet des collèges Polynésie 7 septembre 2020~\decofourright}}
    
% \bigskip
    
% \textbf{Durée : 2 heures} \end{center}

% \bigskip

% \textbf{\begin{tabularx}{\linewidth}{|X|}\hline
%  L'évaluation prend en compte la clarté et la précision des raisonnements ainsi que, plus largement, la qualité de la rédaction. Elle prend en compte les essais et les démarches engagées même non abouties. Toutes les réponses doivent être justifiées, sauf mention contraire.\\ \hline
% \end{tabularx}}

% \vspace{0.5cm}\textbf{\textsc{Exercice 5 \hfill 14 points}}

\medskip

\begin{enumerate}
\item On compare les longueurs des côtés des triangles OAB et ODC :

On a $\dfrac{\text{OA}}{\text{OD}} = \dfrac{36}{64} = \dfrac{4 \times 9}{4 \times 16} = \dfrac{9}{16}$ ;

$\dfrac{\text{OB}}{\text{OC}} = \dfrac{27}{48} = \dfrac{3 \times 9}{3 \times 16} = \dfrac{9}{16}$, donc 

$\dfrac{\text{OA}}{\text{OD}} = \dfrac{\text{OB}}{\text{OC}}$ : d'après la réciproque de la propriété de Thalès cette égalité montre que les droites(AB) et (CD) sont parallèles.
\item On sait que l'on a également $\dfrac{\text{OA}}{\text{OD}} =\dfrac{\text{AB}}{\text{CD}}$ ou encore en remplaçant par les valeurs connues :

$\dfrac{9}{16} = \dfrac{\text{AB}}{80}$, d'où en multipliant chaque membre par 80 : 

AB $ = 80 \times \dfrac{9}{16} = 16 \times 5 \times \dfrac{9}{16} = 5 \times 9 = 45$~(cm).
\item On sait que le triangle ACD est rectangle en C ; donc le théorème de Pythagore permet d'écrire :

$\text{AC}^2 + \text{CD}^2 = \text{AD}^2$. \quad (1)

Or CD $ = 80$ et AD = AO + OD $= 36 + 64 = 100$.

L'égalité (1) devient :

$\text{AC}^2 + 80^2 = 100^2$, d'où $\text{AC}^2 = 100^2 - 80^2 = \np{10000} - \np{6400} = \np{3600}$; d'où AC $ = \sqrt{3600} = 60$.

Chaque étagère a une hauteur de 60~cm avec un plateau de 2~cm soit une hauteur de 62~cm ; il y a 4 étagères, donc la hauteur totale du meuble est égale à : $4 \times 62 = 248$~(cm) plus le dernier plateau donc une hauteur totale de 250~cm.
\end{enumerate}

\vspace{0,5cm}

\end{document}