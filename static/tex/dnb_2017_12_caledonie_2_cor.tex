
\medskip

%\parbox{0.3\linewidth}{Voici un programme de calcul :}\hfill
%\parbox{0.68\linewidth}{
%\begin{tabular}{|l|}\hline
%$\bullet~~$Choisir un nombre\\
%$\bullet~~$Ajouter 1 à ce nombre\\
%$\bullet~~$Calculer le carré du résultat\\
%$\bullet~~$Soustraire le carré du nombre de départ au résultat précédent.\\
%$\bullet~~$Écrire le résultat.\\ \hline
%\end{tabular}
%}

\medskip

\begin{enumerate}
\item %On choisit 4 comme nombre de départ. Prouver par le calcul que le résultat obtenu avec le programme est 9.
On obtient successivement : $4 \to 4 + 1 = 5 \to 5^2 = 25 \to 25 - 4^2 = 25 - 16 = 9$.
\item On note $x$ le nombre choisi.
	\begin{enumerate}
		\item %Exprimer le résultat du programme en fonction de $x$.
On obtient successivement : $x \to x + 1 \to (x + 1)^2 \to (x + 1)^2 - x^2$.
		\item %Prouver que ce résultat est égal à $2x + 1$.
$(x + 1)^2 - x^2 = x^2 + 2x + 1 - x^2 = 2x + 1$.
	\end{enumerate}
\item Soit $f$ la fonction définie par $f(x) = 2x + 1$.
	\begin{enumerate}
		\item %Calculer l'image de 0 par $f$.
L'image de 0 par $f$ est $f(0) = 2\times 0 + 1 = 1$.
		\item %Déterminer par le calcul l'antécédent de $5$ par $f$.
On a $f(x) = 2x + 1 = 5$ ou $2x = 4$ ou $x = 2$. L'antécédent de $5$ par $f$ est $2$.
		\item %En annexe 1, tracer la droite représentative de la fonction $f$.
Voir à la fin.
		\item %Par lecture graphique, déterminer le résultat obtenu en choisissant $- 3$ comme nombre de départ dans le programme de calcul. Sur l'annexe, laisser les traits de construction apparents.
La verticale passant par le point d'abscisse $- 3$ coupe la droite en un point d'ordonnée $- 5$. 
	\end{enumerate}	
\end{enumerate}
\begin{center}
\textbf{\large À RENDRE AVEC LA COPIE}

\bigskip

\textbf{\large ANNEXE 1}

\medskip

\psset{unit=0.5cm,arrowsize=2pt 4}
\begin{pspicture}(-10,-9)(10,9)
\psgrid[gridlabels=0pt,subgriddiv=1,griddots=10]
\psaxes[linewidth=1.25pt,Dx=2,Dy=2]{->}(0,0)(-10,-9)(10,9)
\psplot[plotpoints=3000,linewidth=1.25pt,linecolor=blue]{-5}{4}{x 2 mul 1 add}
\psline[linestyle=dashed,ArrowInside=->,linewidth=1.5pt](-3,0)(-3,-5)(0,-5)
\uput[u](9.5,0){$x$}\uput[r](0,9.5){$y$}
\end{pspicture}

\end{center}
\vspace{0,5cm}

