
\medskip
 
%Valentin souhaite acheter un écran de télévision ultra HD (haute définition).
%
%Pour un confort optimal, la taille de l'écran doit être adaptée aux dimensions de son salon.
%
%Voici les caractéristiques du téléviseur que Valentin pense acheter:
%
%\begin{center}
%\begin{tabularx}{0.4\linewidth}{|X m{1.5cm}|}\hline
%Hauteur de l'écran	&60 cm\\
%Format de l'écran	&16/9\\
%Ultra HD			&Oui\\ \hline
%\end{tabularx}
%\end{center}
%
%\textbf{Question :} Valentin a-t-il fait un choix adapté ?
%
%\medskip
%
%Utiliser les informations ci-dessous et les caractéristiques du téléviseur pour répondre.
%
%Toute trace de recherche, même incomplète, pourra être prise en compte dans la notation.
%
%\textbf{Information 1.} 
%
%Distance écran-téléspectateur du salon de Valentin : 3,20~m. 
%
%\medskip
%
%\textbf{Information 2.} Format 16/9
%
%Pour un écran au format 16/9, on a : Largeur $= \dfrac{16}{9} \times $ Hauteur
%
%\medskip
%
%\textbf{Information 3.} Graphique pour aider au choix de la taille de l'écran
La hauteur de l'écran envisagé est de $h = 60$~cm, donc sa largeur est : $l = \frac{16}{9} \times 60 = \dfrac{16}{3} \times 20 = \dfrac{320}{3}$~cm.

D'après le théorème de Pythagore la diagonale $d$ de son écran est telle que :

$d^2 = h^2 + l^2 = 60^2 + \left(\dfrac{320}{3} \right)^2 = \dfrac{\np{134800}}{9} \approx 122,4$~cm.

Sur le graphique ci-dessous on trace donc la droite verticale d'équation $x = 122,4$ et horizontalement la droite d'équation $y = 3,20$ ; ces deux droites sont sécantes en un point de coordonnées (122,4~;~3,2) et ce point est bien dans la région conseillée une distance à l'écran entre 200 et 415 cm).

Valentin peut acheter le téléviseur.
\begin{center}
\psset{xunit=0.06cm,yunit=0.015cm,arrowsize=2pt 4}
\begin{pspicture}(-10,-20)(200,550)
\multido{\n=0+10}{16}{\psline[linewidth=0.2pt](\n,0)(\n,500)}
\multido{\n=0+50}{4}{\psline[linewidth=1.2pt](\n,0)(\n,500)}
\multido{\n=0+20}{26}{\psline[linewidth=0.2pt](0,\n)(150,\n)}
\multido{\n=0+100}{6}{\psline[linewidth=1.2pt](0,\n)(150,\n)}
\psaxes[linewidth=1.25pt,Dx=50,Dy=100]{->}(0,0)(0,0)(150,500)
\psaxes[linewidth=1.25pt,Dx=50,Dy=100](0,0)(0,0)(150,500)
\uput[r](0,520){\small Distance écran-téléspectateur (en cm)}
\uput[r](150,460){\small  Distance maximale}\psline{->}(150,460)(132,460)
\uput[r](150,230){\small Distance minimale}\psline{->}(150,230)(135,230)
\uput[u](140,0){\small Longueur de la diagonale de l'écran (en cm)}
\psline[linewidth=1.25pt](30,100)(140,485)
\psline[linewidth=1.25pt](30,48)(140,240)
\psline[linecolor=green](122.4,0)(122.4,420)
\psline[linecolor=green](0,320)(140,320)
\end{pspicture}
\end{center}
\medskip

