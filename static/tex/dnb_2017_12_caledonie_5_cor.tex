
\medskip

%\emph{Dans cet exercice, toute trace de recherche, même incomplète ou non fructueuse, sera prise en compte dans l'évaluation.}
%
%\medskip
%
%--- AUREL : Belle pêche ! Combien de poissons et de coquillages vas-tu pouvoir vendre au marché ?
%
%--- ANTOINE : En tout, je vais pouvoir vendre au marché 30 poissons et 500 coquillages.
%
%\medskip
%
%Antoine est un pêcheur professionnel. Il veut vendre des paniers contenant des coquillages et des
%poissons. Il souhaite concevoir le plus grand nombre possible de paniers identiques. Enfin, il voudrait qu'il ne lui reste aucun coquillage et aucun poisson dans son congélateur.
%
%\medskip

\begin{enumerate}
\item %Combien de paniers au maximum Antoine pourra t-il concevoir ? Justifier.
Si chaque panier contient $c$ coquillages et $p$ poissons, le nombre de paniers doit être un diviseur de 30 et de 500.

Le plus grand nombre de paniers sera donc le plus grand diviseur commun à 50 et 300, donc le P. G. C. D de 30 et 500 qui est de façon évidente 10.
\item %Quelle sera la composition de chaque panier ? Justifier.
On aura donc $c = \dfrac{500}{10} = 50$~coquillages  et $p = \dfrac{30}{10} = 3$~poissons.
\end{enumerate}

\vspace{0,5cm}

