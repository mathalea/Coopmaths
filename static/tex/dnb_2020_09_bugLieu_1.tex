\documentclass[10pt]{article}
\usepackage[T1]{fontenc}
\usepackage[utf8]{inputenc}%ATTENTION codage UTF8
\usepackage{fourier}
\usepackage[scaled=0.875]{helvet}
\renewcommand{\ttdefault}{lmtt}
\usepackage{amsmath,amssymb,makeidx}
\usepackage[normalem]{ulem}
\usepackage{diagbox}
\usepackage{fancybox}
\usepackage{tabularx,booktabs}
\usepackage{colortbl}
\usepackage{pifont}
\usepackage{multirow}
\usepackage{dcolumn}
\usepackage{enumitem}
\usepackage{textcomp}
\usepackage{lscape}
\newcommand{\euro}{\eurologo{}}
\usepackage{graphics,graphicx}
\usepackage{pstricks,pst-plot,pst-tree,pstricks-add}
\usepackage[left=3.5cm, right=3.5cm, top=3cm, bottom=3cm]{geometry}
\newcommand{\R}{\mathbb{R}}
\newcommand{\N}{\mathbb{N}}
\newcommand{\D}{\mathbb{D}}
\newcommand{\Z}{\mathbb{Z}}
\newcommand{\Q}{\mathbb{Q}}
\newcommand{\C}{\mathbb{C}}
\usepackage{scratch}
\renewcommand{\theenumi}{\textbf{\arabic{enumi}}}
\renewcommand{\labelenumi}{\textbf{\theenumi.}}
\renewcommand{\theenumii}{\textbf{\alph{enumii}}}
\renewcommand{\labelenumii}{\textbf{\theenumii.}}
\newcommand{\vect}[1]{\overrightarrow{\,\mathstrut#1\,}}
\def\Oij{$\left(\text{O}~;~\vect{\imath},~\vect{\jmath}\right)$}
\def\Oijk{$\left(\text{O}~;~\vect{\imath},~\vect{\jmath},~\vect{k}\right)$}
\def\Ouv{$\left(\text{O}~;~\vect{u},~\vect{v}\right)$}
\usepackage{fancyhdr}
\usepackage[french]{babel}
\usepackage[dvips]{hyperref}
\usepackage[np]{numprint}
%Tapuscrit : Denis Vergès
%\frenchbsetup{StandardLists=true}

\begin{document}
\setlength\parindent{0mm}
% \rhead{\textbf{A. P{}. M. E. P{}.}}
% \lhead{\small Brevet des collèges}
% \lfoot{\small{Polynésie}}
% \rfoot{\small{7 septembre 2020}}
\pagestyle{fancy}
\thispagestyle{empty}
% \begin{center}
    
% {\Large \textbf{\decofourleft~Brevet des collèges Polynésie 7 septembre 2020~\decofourright}}
    
% \bigskip
    
% \textbf{Durée : 2 heures} \end{center}

% \bigskip

% \textbf{\begin{tabularx}{\linewidth}{|X|}\hline
%  L'évaluation prend en compte la clarté et la précision des raisonnements ainsi que, plus largement, la qualité de la rédaction. Elle prend en compte les essais et les démarches engagées même non abouties. Toutes les réponses doivent être justifiées, sauf mention contraire.\\ \hline
% \end{tabularx}}

% \vspace{0.5cm}\textbf{Exercice 1 \hfill 20 points}

\medskip

\parbox{0.49\linewidth}{La figure ci-contre est dessinée à main levée. On donne les informations suivantes :

\begin{itemize}[label=\textbullet]
\item ABC est un triangle tel que :
AC = 10,4 cm, AB = 4 cm et BC = 9,6 cm ;
\item les points A, L et C sont alignés ;
\item les points B, K et C sont alignés ;
\item la droite (KL) est parallèle à la droite (AB) ;
\item CK = 3~cm.
\end{itemize}}\hfill
\parbox{0.49\linewidth}{\psset{unit=1cm}
\begin{pspicture}(6,2.8)
%\psgrid
\pslineByHand(0.5,0.5)(5.5,1)(4.75,2.5)(0.5,0.5)%CBA
\uput[ur](4.75,2.5){A} \uput[d](5.5,1){B} \uput[l](0.5,0.5){C} \uput[ul](1.9,1.1){L} \uput[dl](2.2,0.65){K} 
\pslineByHand(2.5,0)(1.5,2)%LK

\end{pspicture}
}
\medskip

\begin{enumerate}
\item À l'aide d'instruments de géométrie, construire la figure en vraie grandeur sur la copie en laissant apparents les traits de construction.
\item Prouver que le triangle ABC est rectangle en B.
\item Calculer la longueur CL en cm.
\item À l'aide de la calculatrice, calculer une valeur approchée de la mesure de l'angle $\widehat{\text{CAB}}$, au degré près.
\end{enumerate}

\bigskip

\end{document}