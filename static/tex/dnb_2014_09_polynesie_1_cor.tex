\textbf{Exercice 1 \hfill 3 points}

%\bigskip
% 
%Voici trois calculs effectués à  la calculatrice. Détailler ces calculs afin de comprendre les résultats donnés par la calculatrice : 
%
%\smallskip
 
Calcul \no 1: %$\dfrac{5}{6} - \dfrac{3}{4} = \dfrac{1}{12}$
$\dfrac{5}{6} - \dfrac{3}{4} = \dfrac{10}{12} - \dfrac{9}{12} =  \dfrac{10 - 9}{12} = \dfrac{1}{12}$.

\smallskip

Calcul \no 2 : %$\sqrt{18} = 3\sqrt{2}$
$\sqrt{18} = \sqrt{9 \times 2} = \sqrt{9} \times \sqrt{2} = 3\sqrt{2}$.
\smallskip
 
Calcul \no 3 : %$8 \times 10^{15} + 2 \times 10^{15} = 1 \times 10^{16}$
$8 \times 10^{15} + 2 \times 10^{15} =  10^{15}(8 + 2) = 10^{15} \times 10^1  = 10^{16} = 1 \times 10^{16}$
\bigskip
 
