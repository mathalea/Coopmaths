\textbf{\textsc{Exercice 4 \hfill 6 points}}

\medskip

%L'inspecteur G. est en mission dans l'Himalaya. Un hélicoptère est chargé de le transporter en haut
%d'une montagne puis de l'amener vers son quartier général.
%
%Le pilote : \og Alors, je vous emmène, inspecteur ? \fg
%
%L'inspecteur : \og OK, allons-y ! Mais d'abord, puis-je voir le plan de vol ?\fg
%\medskip
%
%Le trajet ABCDEF modélise le plan de vol. Il est constitué de déplacements rectilignes. On a de plus
%les informations suivantes :
%
%\begin{itemize}
%\item AF = 12,5 km ; AC = 7,5 km ; CF = 10 km ; AB = 6 km ; DG = 7 km et EF = 750 m.
%\item (DE) est parallèle à (CF).
%\item ABCH et ABGF sont des rectangles
%\end{itemize}
%
%\begin{center}
%\psset{unit=1cm,arrowsize=3pt 4}
%\begin{pspicture}(10,5.2)
%%\psgrid
%\psline[ArrowInside=->,linewidth=1.5pt](0,0)(0,5)(5,5)(10,0.75)(10,0)
%\psline(10,0)
%\pscurve[fillstyle=solid,fillcolor=lightgray](0,0)(1,0.8)(2,1.9)(3,3)(3.5,4)(3.8,4.6)(4,5)(4.4,4.2)(5,3.8)(6,3.)(7,2.2)(8,1.6)(9,0.7)(9.9,0)
%\uput[l](0,0){A}\uput[l](0,5){B}\uput[u](4.2,5){C}\uput[u](5,5){D} \uput[u](10,5){G}
%\uput[r](10,0.75){E}
%\uput[r](10,0){F}\uput[d](4,0){H}
%\pspolygon(0,0)(4,5)(10,0)
%\psline(4,5)(4,0)
%\psline(5,5)(10,5)(10,0.75)
%\end{pspicture}
%\end{center}
%
%\bigskip
%
%Le pilote : \og Je dois faire le plein \ldots \fg
%
%L'inspecteur : \og Combien consomme votre hélico ? \fg
%
%Le pilote : \og 1,1~L par km  pour ce genre de trajet \fg
%
%L'inspecteur : \og Mais le plein nous surchargerait !  20 L  de carburant seront très largement suffisants.\fg
%
%\medskip

\begin{enumerate}
\item %Vérifier que la longueur du parcours est de 21 kilomètres.

%Dans cette question, toute trace de recherche sera valorisée.
Le triangle ABC est rectangle en B ; le théorème de Pythagore permet d'écrire :

$\text{AC}^2 = \text{AB}^2 + \text{BC}^2$ soit $\text{BC}^2 = \text{AC}^2 - \text{AB}^2 = 7,5^2 - 6^2 = 56,25 - 36 = 20,25$, 

d'où $\text{BC} = \sqrt{20,25} = 4,5$~(km).

Puis $\text{CD} = \text{BG} - \text{BC} - \text{DG} = 12,5 - 4,5 - 7 = 1$~(km).

Enfin GE $ = \text{GF} - \text{FE} = 6 - 0,750 = 5,25$~(km).

Le théorème de Pythagore appliqué au triangle DGE s'écrit :

$\text{DE}^2 = \text{DG}^2 + \text{GE}^2 = 7^2 + 5,25^2 = \np{76,5625}$ ; donc DE $ = \sqrt{\np{76,5625}} = 8,75$~(km).

Le trajet a donc une longueur de :

$6 + 4,5 + 1 + 8,75 + 0,75 = 21$~(km).

\item %Le pilote doit-il avoir confiance en l'inspecteur G ? Justifier votre réponse.
Pour faire ces 21 km il faut à l'hélicoptère : $21 \times 1,1 = 23,1$ litres de carburant. Donc le pilote ne doit pas faire confiance à l'inspecteur.
\end{enumerate}

\vspace{0,5cm}

