\documentclass[10pt]{article}
\usepackage[T1]{fontenc}
\usepackage[utf8]{inputenc}%ATTENTION codage UTF8
\usepackage{fourier}
\usepackage[scaled=0.875]{helvet}
\renewcommand{\ttdefault}{lmtt}
\usepackage{amsmath,amssymb,makeidx}
\usepackage[normalem]{ulem}
\usepackage{diagbox}
\usepackage{fancybox}
\usepackage{tabularx,booktabs}
\usepackage{colortbl}
\usepackage{pifont}
\usepackage{multirow}
\usepackage{dcolumn}
\usepackage{enumitem}
\usepackage{textcomp}
\usepackage{lscape}
\newcommand{\euro}{\eurologo{}}
\usepackage{graphics,graphicx}
\usepackage{pstricks,pst-plot,pst-tree,pstricks-add}
\usepackage[left=3.5cm, right=3.5cm, top=3cm, bottom=3cm]{geometry}
\newcommand{\R}{\mathbb{R}}
\newcommand{\N}{\mathbb{N}}
\newcommand{\D}{\mathbb{D}}
\newcommand{\Z}{\mathbb{Z}}
\newcommand{\Q}{\mathbb{Q}}
\newcommand{\C}{\mathbb{C}}
\usepackage{scratch}
\renewcommand{\theenumi}{\textbf{\arabic{enumi}}}
\renewcommand{\labelenumi}{\textbf{\theenumi.}}
\renewcommand{\theenumii}{\textbf{\alph{enumii}}}
\renewcommand{\labelenumii}{\textbf{\theenumii.}}
\newcommand{\vect}[1]{\overrightarrow{\,\mathstrut#1\,}}
\def\Oij{$\left(\text{O}~;~\vect{\imath},~\vect{\jmath}\right)$}
\def\Oijk{$\left(\text{O}~;~\vect{\imath},~\vect{\jmath},~\vect{k}\right)$}
\def\Ouv{$\left(\text{O}~;~\vect{u},~\vect{v}\right)$}
\usepackage{fancyhdr}
\usepackage[french]{babel}
\usepackage[dvips]{hyperref}
\usepackage[np]{numprint}
%Tapuscrit : Denis Vergès
%\frenchbsetup{StandardLists=true}

\begin{document}
\setlength\parindent{0mm}
% \rhead{\textbf{A. P{}. M. E. P{}.}}
% \lhead{\small Brevet des collèges}
% \lfoot{\small{Polynésie}}
% \rfoot{\small{7 septembre 2020}}
\pagestyle{fancy}
\thispagestyle{empty}
% \begin{center}
    
% {\Large \textbf{\decofourleft~Brevet des collèges Polynésie 7 septembre 2020~\decofourright}}
    
% \bigskip
    
% \textbf{Durée : 2 heures} \end{center}

% \bigskip

% \textbf{\begin{tabularx}{\linewidth}{|X|}\hline
%  L'évaluation prend en compte la clarté et la précision des raisonnements ainsi que, plus largement, la qualité de la rédaction. Elle prend en compte les essais et les démarches engagées même non abouties. Toutes les réponses doivent être justifiées, sauf mention contraire.\\ \hline
% \end{tabularx}}

% \vspace{0.5cm}\textbf{\textsc{Exercice 4 \hfill 13 points}}

\medskip

\begin{enumerate}
\item Tableau complété :

\begin{center}
\begin{tabularx}{\linewidth}{|*{4}{>{\centering \arraybackslash}X|}}\hline
Modèle	&Pour la ville	&Pour le sport	& Total\\ \hline
Noir	&\blue 15		&\black 5		&\black 20\\ \hline
Blanc	&\black 7		&\blue 10		&\blue 17\\ \hline
Marron	&\blue 5		&\black 3		&\blue 8\\ \hline
Total	&\black 27		&\blue 18		&\black 45\\ \hline
\end{tabularx}
\end{center}

\item 
	\begin{enumerate}
		\item La probabilité de choisir un modèle de couleur noire est égale à $\dfrac{20}{45} = \dfrac{5 \times 4}{5 \times 9} = \dfrac{4}{9}$.
		\item La probabilité de choisir un modèle pour le sport est égale à $\dfrac{18}{45} = \dfrac{9 \times 2}{9 \times 5}  = \dfrac{2}{5} = \dfrac{4}{10} = 0,4$.
		\item La probabilité de choisit un modèle pour la ville de couleur marron est égale à $\dfrac{5}{45} = \dfrac{5 \times 1}{5 \times 9} = \dfrac{1}{9}$.	
	\end{enumerate}
\item Dans le magasin B la probabilité de choisir un modèle de couleur noire est égale à $\dfrac{30}{54} = \dfrac{6 \times 5}{6 \times 9} = \dfrac{5}{9}$. 

Comme $\dfrac{5}{9} > \dfrac{4}{9}$ on a plus de chance d'obtenir un modèle de couleur noire dans le magasin B.
\end{enumerate}

\vspace{0,5cm}

\end{document}