\textbf{Exercice 5 \hfill 7 points}

\medskip

%Un bateau se trouve à une distance $d$ de la plage.
%
%\begin{center}
%\psset{unit=1cm}
%\begin{pspicture}(8,4.3)
%%\psgrid
%\pscustom[fillstyle=solid,fillcolor=lightgray]{\pscurve(0,1.7)(2,1.4)(4,1.5)(6,1.6)(8,1.45)
%\psline(8,0)(0,0)}
%\pspolygon(1,1)(7,1)(5,4)
%\psline(5,4)(5,1)\psframe(5,1)(5.3,1.3)
%\psarc(1,1){5mm}{0}{42}\psarc(7,1){5mm}{127}{180}
%\uput[u](5,4){Bateau}
%\uput[d](4,1){$L = 80$~m}\uput[l](5,2.5){$d$}
%\rput(4,0.25){Plage}\rput(1.7,1.3){$\alpha$}\rput(6.25,1.3){$\beta$}
%\end{pspicture}
%\end{center}
%
%Supposons dans tout le problème que $\alpha = 45\degres, \beta = 65\degres$ et que $L = 80$ m.
%
%\medskip

\begin{enumerate}
\item \textbf{Conjecturons la distance \boldmath$d$ \unboldmath à l'aide d'une construction}

\medskip

%Mise au point par Thalès (600 avant JC), la méthode dite de TRIANGULATION
%propose une solution pour estimer la distance $d$.
	\begin{enumerate}
		\item ~%Faire un schéma à l'échelle 1/\np{1000} (1 cm pour 10 m).
\begin{center}
\psset{unit=1cm}
\begin{pspicture}(9.5,6.8)
\pspolygon(1,1)(9,1)(5.46,6.46)
\psline(5.46,6.46)(5.46,1)\psframe(5.46,1)(5.76,1.3)
\psarc(1,1){5mm}{0}{47}\psarc(9,1){5mm}{117}{180}
\uput[u](5.46,6.46){C}
\uput[d](5,1){$80$~m}%\uput[l](5,2.5){$?$}
\rput(1.7,1.3){$45\degres$}\rput(8.25,1.3){$65\degres$}
\end{pspicture}
\end{center}
		\item %Conjecturer en mesurant sur le schéma la distance $d$ séparant le bateau de la côte.
En mesurant sur le schéma, on trouve environ 5,5~cm, on suppose donc que $d$ est égale
à $55$~m. 
	\end{enumerate}
\item \textbf{Déterminons la distance \boldmath$d$ \unboldmath par le calcul}
	\begin{enumerate}
		\item %Expliquer pourquoi la mesure de l'angle $\widehat{\text{ACB}}$ est de $70$\degres.
Dans un triangle, la somme des mesures des angles est égale à $180\degres$. On a donc :

$\widehat{\text{ACB}} = 180  - \left(\widehat{\text{CAB}} + \widehat{\text{CBA}}\right) = 180 - (45 + 65) = 180  - 110  = 70$~(\degres).
		\item %Dans tout triangle ABC, on a la relation suivante appelée \og loi des sinus \fg :
On utilise la \og  loi des sinus \fg{} : 
 
\[\dfrac{\text{BC}}{\sin \widehat{\text{A}}} = \dfrac{\text{AC}}{\sin \widehat{\text{B}}} = \dfrac{\text{AB}}{\sin \widehat{\text{C}}}.\]

%En utilisant cette formule, calculer la longueur BC. Arrondir au cm près.
Soit $\dfrac{\text{BC}}{\sin 45} = \dfrac{\text{AC}}{\sin 65} = \dfrac{80}{\sin 70}$.

En particulier $\dfrac{\text{BC}}{\sin 45} =  \dfrac{80}{\sin 70}$, d'où par produit en croix :

$\text{BC} = 80 \times \dfrac{\sin 45}{\sin 70} \approx 60,20$~(m) au centimètre près.
		\item %En déduire la longueur CH arrondie au cm près.
CBH est un triangle rectangle en H, on a :

$\sin \widehat{\text{CBH}} = \dfrac{\text{CH}}{\text{CB}}$, soit $\sin 65 \approx  \dfrac{\text{CH}}{60,2}$ ou encore $\text{CH} \approx 60,2 \times \sin 65$

$\text{CH} \approx 54,56$m (valeur arrondie au cm près)
	\end{enumerate}
\end{enumerate}

\vspace{0,5cm}

