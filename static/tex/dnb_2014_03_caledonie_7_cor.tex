\textbf{Exercice 7 : Le marché municipal \hfill 4 points) }

\medskip

%Au marché municipal de Nouméa, on trouve toutes sortes de légumes et de fines herbes. 
%
%\setlength\parindent{6mm}
%\begin{itemize}
%\item[$\bullet~~$] la botte de persil vaut 20 F de plus que la botte d'oignons verts ; 
%\item[$\bullet~~$] la botte de basilic coûte le même prix que la botte de menthe; 
%\item[$\bullet~~$] la botte de menthe coûte cinq fois moins cher que le kilogramme de salade verte; 
%\item[$\bullet~~$] le kilogramme de salade verte est à 900 F, soit six fois le prix d'une botte d'oignons verts.
%\end{itemize}
%\setlength\parindent{0mm} 
%
%Chacune des affirmations suivantes est-elle vraie ou fausse ? Les réponses doivent être justifiées.

%\medskip 

\textbf{Affirmation 1 :} %avec 700 F, on peut acheter 6 bottes d'oignons verts. 
La botte d'oignons vaut $\dfrac{900}{6} = \dfrac{300}{2} = 150$~F.

La botte de persil vaut donc $150 + 20 = 170$~F.

Le prix de la botte de menthe (ou de basilic) est $\dfrac{900}{5} = 180$~F.

On a vu que 6 bottes d'oignons verts revienne à acheter un kg de salade qui vaut 900~F, donc avec 700~F on ne peut pas effectuer cet achat. \textbf{Affirmation 1 : fausse}

\textbf{Affirmation 2 :} %avec 700 F, on peut acheter une botte de menthe, une botte d'oignons verts, une botte de basilic et une botte de persil. 
Une botte de menthe, une botte d'oignons verts, une botte de basilic et une botte de persil coutent $180 + 150 + 180 + 170 = 680$~F. \textbf{Affirmation 2 : vraie}

\textbf{Affirmation 3 :} %avec 1 500F, on peut acheter 2 bottes de chacune des fines herbes (la salade ne fait pas partie des fines herbes). 
2 bottes de chacune des fines herbes reviennent à : $2\times (170 + 180 + 180 + 150) = 2 \times 680 = \np{1360}$~F. \textbf{Affirmation 3 : vraie}
\vspace{0,5cm}

