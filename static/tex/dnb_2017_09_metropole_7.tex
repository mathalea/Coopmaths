
\medskip

Les légionelles sont des bactéries présentes dans l'eau potable. Lorsque la température de l'eau est comprise
entre 30~\degres C et 45~\degres C, ces bactéries prolifèrent et peuvent atteindre, en 2 ou 3 jours, des concentrations dangereuses pour l'homme.

On rappelle que \og $\mu$ m \fg{} est l'abréviation de micromètre. Un micromètre est égal à un millionième de mètre.

\medskip

\begin{enumerate}
\item La taille d'une bactérie légionelle est 0,8 $\mu$m.

Exprimer cette taille en m et donner le résultat sous la forme d'une écriture scientifique.
\item  Lorsque la température de l'eau est 37~\degres C, cette population de bactéries légionelles double tous les quarts d'heure.

Une population de $100$ bactéries légionelles est placée dans ces conditions.

On a créé la feuille de calcul suivante qui permet de donner le nombre de bactéries légionelles en fonction du nombre de quarts d'heure écoulés:

\begin{center}
\begin{tabularx}{0.7\linewidth}{|c|*{2}{>{\centering \arraybackslash}X|}}\hline
&A &B\\ \hline
1 &Nombre de quarts d'heure &Nombre de bactéries\\ \hline
2 &0 &100\\ \hline
3 &1&\\ \hline
4 &2&\\ \hline
5 &3&\\ \hline
6 &4&\\ \hline
7 &5&\\ \hline
8 &6&\\ \hline
9 &7&\\ \hline
10 &8&\\ \hline
\end{tabularx}
\end{center}
	\begin{enumerate}
		\item Dans la cellule B3, on veut saisir une formule que l'on pourra étirer vers le bas dans la colonne B pour calculer le nombre de bactéries légionelles correspondant au nombre de quarts d'heure écoulés. Quelle est cette formule ?
		\item Quel est le nombre de bactéries légionelles au bout d'une heure ?
		\item Le nombre de bactéries légionelles est-il proportionnel au temps écoulé ?
		\item Après combien de quarts d'heure cette population dépasse-t-elle dix mille bactéries légionelles ?
 	\end{enumerate}
\item On souhaite tester l'efficacité d'un antibiotique pour lutter contre la bactérie légionelle. On introduit l'antibiotique dans un récipient qui contient $10^4$ bactéries légionelles au temps $t = 0$. La représentation graphique, sur \textbf{l'annexe, à rendre avec la copie}, donne le nombre de bactéries dans le récipient en
fonction du temps.
	\begin{enumerate}
		\item Au bout de 3 heures, combien reste-t-il environ de bactéries légionelles dans le récipient ?
		\item Au bout de combien de temps environ reste-t-il \np{6000} bactéries légionelles dans le récipient?
		\item On estime qu'un antibiotique sera efficace sur l'être humain s'il parvient à réduire de $80$\,\% le nombre initial de bactéries dans le récipient en moins de $5$~heures.
		
En s'aidant du graphique, étudier l'efficacité de l'antibiotique testé sur l'être humain.
 	\end{enumerate}
\end{enumerate}
 
\begin{center}
\textbf{\Large Annexe à rendre avec la copie}

\bigskip

 
Faire apparaître les traits justifiant les réponses de la question 3.
 
\vspace{2cm}

\psset{xunit=1cm,yunit=0.0008cm}
\begin{pspicture}(-1,-700)(10.5,12000) 
\multido{\n=0+1}{11}{\psline[linewidth=0.2pt](\n,0)(\n,12000)}
\multido{\n=0+2000}{7}{\psline[linewidth=0.2pt](0,\n)(10,\n)}
\psaxes[linewidth=1.25pt,Dy=15000](0,0)(0,0)(10.5,12000)
\psplot[plotpoints=3000,linewidth=1.25pt,linecolor=blue]{0}{10}{10000 2.71828 0.223 x mul exp div}
\uput[d](9,-650){temps $t$ en $h$}
\uput[r](0,12500){Nombre de bactéries}
\multido{\n=0+2000}{7}{\uput[l](0,\n){\np{\n}}}
\end{pspicture}
\end{center}
