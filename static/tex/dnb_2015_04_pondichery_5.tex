\textbf{\textsc{Exercice 5\hfill 4 points}}

\medskip

Un jeu télévisé propose à des candidats deux épreuves :

\setlength\parindent{6mm}
\begin{itemize}
\item[$\bullet~~$]Pour la première épreuve, le candidat est face à 5 portes : une seule
porte donne accès à la salle du trésor alors que les 4 autres s'ouvrent
sur la salle de consolation.
\item[$\bullet~~$]Pour la deuxième épreuve, le candidat se retrouve dans une salle face
à 8 enveloppes.

\textbf{Dans la salle du trésor }: 1 enveloppe contient \np{1000}~\euro, 5 enveloppes
contiennent 200~\euro. Les autres contiennent 100~\euro.

\textbf{Dans la salle de consolation} : 5 enveloppes contiennent 100~\euro{} et les
autres sont vides.
\end{itemize}
\setlength\parindent{0mm}

Il doit choisir une seule enveloppe et découvre alors le montant qu'il a gagné.

\medskip

\begin{enumerate}
\item Quelle est la probabilité que le candidat accède à la salle du trésor ?
\item Un candidat se retrouve dans la salle du trésor.
	\begin{enumerate}
		\item Représenter par un schéma la situation.
		\item Quelle est la probabilité qu'il gagne au moins 200~\euro{} ?
	\end{enumerate}
\item Un autre candidat se retrouve dans la salle de consolation.

Quelle est la probabilité qu'il ne gagne rien ?
\end{enumerate}

\vspace{0.5cm}

