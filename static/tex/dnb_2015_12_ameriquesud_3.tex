\textbf{\textsc{Exercice 3} \hfill 4 points}

\medskip

Un « DJ »\footnote{DJ signifie « disk jokey » c'est à dire animateur musical} possède 96 titres de musique rap et 104 titres de musique électro. Lors
de ses concerts, il choisit les titres qu'il mixe au hasard.

\medskip

\begin{enumerate}
\item Calculer la probabilité que le premier titre soit un titre de musique rap.
\item Pour varier ses concerts, le DJ souhaite répartir tous ses titres en réalisant des « mix »\footnote{mix est une abréviation de mixage} identiques, c'est-à-dire comportant le même nombre de titres et la même
répartition de titres de musique « rap » et de musique « électro ».
	\begin{enumerate}
		\item Quel est le nombre maximum de concerts différents pourra-t-il réaliser ?
		\item Combien y aura-t-il dans ce cas de titres de musique rap et de musique
électro par concert ?
	\end{enumerate}
\end{enumerate}
%%%%%%%%%%%%%%
\vspace{0.25cm}

