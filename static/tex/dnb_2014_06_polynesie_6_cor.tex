\textbf{Exercice 6 \hfill 8 points}

\medskip
 
%Une famille de quatre personnes hésite entre deux modèles de piscine. Elle regroupe des informations afin de prendre sa décision.
%
%\medskip
%
%\begin{tabularx}{\linewidth}{|*{2}{>{\centering \arraybackslash}X|}}\hline
%\textbf{Information 1} : &les deux modèles de piscine:\\
%La piscine \og ronde \fg& 	La piscine \og octogonale \fg \\
%\psset{unit=0.8cm}
%\begin{pspicture}(-3,-1)(3,2.5)
%\psellipse[linewidth=1.5pt](0,2)(2.5,0.45)
%\psellipse[linecolor=red](0,0)(2.5,0.45)
%\pscustom[fillstyle=solid,fillcolor=lightgray]{
%\psline(-2.5,0)(-2.5,2)
%\pscurve(-2.5,2)(-2,1.78)(-1.5,1.67)(-1,1.6)(0,1.57)(1,1.6)(2,1.78)(2.5,2)
%\psline(2.5,2)(2.5,0)
%\pscurve(2.5,0)(2,-.24)(1.5,-0.33)(1,-0.4)(0,-0.43)(-1,-0.4)(-2,-0.29)(-2.5,0)
%}
%\end{pspicture}&\psset{unit=0.8cm}
%\begin{pspicture}(-3,-0.5)(3,2.5)
%\pspolygon(2,1.6)(0,1.5)(-2,1.7)(-1,2)(1,2.1)(2.7,1.8)
%\psline(2.7,0.6)(2,0.2)(0,0)(-2,0.3)
%\pscustom[fillstyle=solid,fillcolor=lightgray]{
%\psline(-2,0.3)(-2,1.7)(0,1.5)(2,1.6)(2.7,1.8)(2.7,0.6)(2,0.2)(0,0)(-2,0.3)
%}
%\end{pspicture}\\
%Hauteur intérieure : 1,20 m& 	Hauteur intérieure : 1,20 m\\ 
%Vue du dessus : un cercle de rayon 1,70 m& 	Vue du dessus : un octogone régulier de diamètre extérieur 4,40 m\\ 
%\psset{unit=1cm}
%\begin{pspicture}(6,3.5)
%\pscircle(3,1.75){1.75}
%\psline(3,1.75)(4.75,1.75)
%\psline{<->}(3,1.85)(4.75,1.85)
%\uput[u](3.875,1.85){1,70~m}
%\end{pspicture}&\psset{unit=1cm}
%\begin{pspicture}(-3,-1.75)(3,1.75)
%\pscircle[linestyle=dotted](0,0){1.75}
%\pspolygon(1.75;0)(1.75;45)(1.75;90)(1.75;135)(1.75;180)(1.75;225)(1.75;270)(1.75;315)
%\psline{<->}(1.75;0)(1.75;180)
%\uput[u](0,0){4,40 m}
%\end{pspicture}\\ \hline
%\end{tabularx}
%
%\begin{tabular}{|m{11.68cm}|}
%\textbf{Information 2} :\\
% La construction d'une piscine de surface au sol de moins de 10m$^2$ ne nécessite aucune démarche administrative.\\ \hline
%\textbf{Information 3} :\\
%Surface minimale conseillée par baigneur: 3,40 m$^2$\\ \hline
%\textbf{Information 4} :\\
% Aire d'un octogone régulier : $A_{\text{octogone}} = 2\sqrt{2} \times  R^2$. 
%
%où $R$ est le rayon du disque extérieur à l'octogone.\\ \hline
%\textbf{Information 5} :\\
%Débit du robinet de remplissage : 12 litres d'eau par minute.\\ \hline
%\end{tabular}
%
%\medskip
 
\begin{enumerate}
\item %Chacun des modèles proposés impose-t-il des démarches administratives ? 
La piscine \og ronde \fg{} a une emprise au sol de : $\pi R^2 = \pi \times 1,7^2 \approx 9,08$~m$^2$ soit moins de 10~m$^2$ : pas de formalité.

La piscine \og octogonale \fg{} a une emprise au sol de : $2\sqrt{2}\times R^2 = 2\sqrt{2} \times 2,2^2 \approx 13,69$~m$^2$ soit plus de 10~m$^2$ : il faudra une démarche administrative. 
\item %Les quatre membres de la famille veulent se baigner en même temps. Expliquer pourquoi la famille doit dans ce cas choisir la piscine octogonale.
Pour quatre baigneurs il est conseillé une surface minimale de $4 \times 3,4 = 13,6~$m$^2$, donc la piscine \og ronde est trop petite et la piscine \og octogonale \fg{} est juste suffisante car $13,69 > 13,6$.

Il faut donc choisir la piscine \og octogonale \fg.  
\item %On commence le remplissage de cette piscine octogonale le vendredi à 14~h~00 et on laisse couler l'eau pendant la nuit, jusqu'au samedi matin à 10~h~00. La piscine va-t-elle déborder ?
La piscine \og octogonale \fg a un volume de $2\sqrt{2} \times 2,2^2 \times 1,2 \approx 16,43~$m$^3$.

L'eau coule pendant $10 + 10 = 20$~h soit $20 \times 60 = \np{1200}$~min  ; avec un débit de 12~l par minute la piscine s'est remplie de $12 \times \np{1200} = \np{14400}$~litres soit 14,4~m$^3$ : elle ne sera pas donc pleine. Pas de débordement ! 
\end{enumerate} 

\bigskip

