\textbf{\textsc{Exercice 1} \hfill 4 points}

\medskip

Cet exercice est un questionnaire à choix multiples (QCM). Pour chaque question, une seule réponse est exacte. Aucune justification n'est demandée. 
 
\emph{Pour chacune des quatre questions, écrire sur votre copie le numéro de la question et la lettre A, B, ou C correspondant à la réponse choisie.} 

\renewcommand\arraystretch{2.7}
\begin{center}
\begin{tabularx}{\linewidth}{|p{6.5cm}|*{3}{>{\centering \arraybackslash}X|}}\hline 
&A& B& C \\ \hline
\textbf{1.~~} $\left(\dfrac{2}{7} +  \dfrac{3}{7}\right) ~:~\dfrac{1}{5}=\dfrac{5}{7}\times 5 =\dfrac{25}{7}$&$\dfrac{1}{7}$&\fbox{$\dfrac{25}{7}$}&$\dfrac{17}{7}$\\ \hline
\textbf{2.~~} Le PGCD des nombres $84$ et $133$ est \ldots \ldots&1&\fbox{7}&3\\ \hline
\textbf{3.~~} Les solutions de l'inéquation $- 3x + 5 \geqslant 9$ 
sont les nombres $x$ tels que \ldots &\fbox{$x \leqslant \dfrac{- 4}{3}$}&$x = \dfrac{- 4}{3}$&$x \geqslant \dfrac{- 4}{3}$\\ \hline 
\textbf{4.~~} $\left(1 + \sqrt{2}\right)^2=1^2+2\times1\times\sqrt{2}+(\sqrt{2})^2=3+2\sqrt{2}$&3&$3 - \sqrt{2}$&\fbox{$3 + 2\sqrt{2}$}\\ \hline
\end{tabularx}

\bigskip

\renewcommand\arraystretch{1.}

\begin{tabularx}{\linewidth}{|X|}\hline
\textbf{Les 8 exercices qui suivent traitent du même thème \og  le canal du Midi\footnote{Le canal du Midi est un canal qui rejoint l'Atlantique à la Méditerranée.} \fg{} mais sont indépendants. Le vocabulaire spécifique est donné sur le schéma de l'exercice 7 }\\ \hline
\end{tabularx}
\end{center} 

