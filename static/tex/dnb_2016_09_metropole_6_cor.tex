\textbf{\textsc{Exercice 6} \hfill 7 points}

\medskip

\begin{enumerate}
\item $19 \times 1,24 = 23,56$~(\euro).
\item $\text{DC} = \text{BD} - \text{BC} = 3,10 - 2,10 = 1$~(m).

Dans le triangle DEC rectangle en C, $\tan \widehat{\text{DEC}} = \dfrac{\text{DC}}{\text{EC}} = \dfrac{1}{2,85}$.  La calculatrice donne donc $\widehat{\text{DEC}} \approx 19~\degres$.

La pente du toit de la véranda permet donc la pose de chaque modèle.

\item $\bullet~~$ On sait que le triangle DEC est rectangle en C.

D'après la propriété de Pythagore dans le triangle rectangle EDC, on a : 

$\text{ED}^2 = \text{EC}^2 + \text{DC}^2 = = 2,85^2 + 1^2 = \np{9,1225}$, 
donc ED $= \sqrt{\np{9,1225}} \approx 3$~(m).

$\mathcal{A}_{\text{EGDF}} = \text{ED} \times \text{EF}\approx  3 \times  6,10 \approx  18,3~\left(\text{m}^2\right)$.

$\bullet~~$ On augmente la surface de 5\,\% : $18,3 \times  (1 + 0,05) \approx 19,215~\left(\text{m}^2\right)$.

$\bullet~~$ $19,215 \times  13 = 249,795$. Mélanie doit prévoir d'acheter 250 tuiles.
\end{enumerate}

\vspace{0,5cm}

