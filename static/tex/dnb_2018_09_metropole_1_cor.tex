
\medskip

\textbf{Partie 1}

\medskip

%On s'intéresse à une course réalisée au début de l'année 2018. Il y a $80$ participants, dont $32$ femmes et $48$ hommes.
%
%Les femmes portent des dossards rouges numérotés de 1 à 32. Les hommes portent des dossards verts numérotés de 1 à 48.
%
%Il existe donc un dossard \no 1 rouge pour une femme, et un dossard \no 1 vert pour un homme, et ainsi de suite ...
%
%\medskip

\begin{enumerate}
\item %Quel est le pourcentage de femmes participant à la course ?
Il y a 32 femmes sur un total de 80 participants ; le pourcentage de femmes est donc : $\dfrac{32}{80} \times 100 = \dfrac{8 \times 4}{8 \times 10} \times 100 = \dfrac{4}{10} \times 100 = \dfrac{2}{5} \times 100 = 40$. Il y a 40\,\% de femmes.
\item %Un animateur tire au hasard le dossard d'un participant pour remettre un prix de consolation.
	\begin{enumerate}
		\item %Soit l'évènement $V$ : \og Le dossard est vert \fg. Quelle est la probabilité de l'évènement $V$?
Vert correspond à un homme et il y a $80 - 32 = 48$ hommes, donc $p(V) = \dfrac{48}{80} = \dfrac{8 \times 6}{8 \times 10} = \dfrac{6}{10} = \dfrac{60}{100} = 60\,\%$.

\emph{Remarque} : on aurait pu faire directement le complément à 100\,\% des 40\,\% de femmes.
		\item %Soit l'évènement $M$ : \og Le numéro du dossard est un multiple de 10 \fg. Quelle est la probabilité de l'évènement $M$ ?
Il y a deux 10, deux 20, deux 30 et un 40, soit en tout 7 dossards dont le numéro est un multiple de 10.

La probabilité de cet évènement est donc $p(M) = \dfrac{7}{80}$.
		\item %L'animateur annonce que le numéro du dossard est un multiple de $10$. Quelle est alors la probabilité qu'il appartienne à une femme ?
Sur les 7 multiples de 10, 3 sont ceux d'une femme. La probabilité est donc égale à $\dfrac{3}{7}$.
	\end{enumerate}
\end{enumerate}

\bigskip

%\parbox{0.55\linewidth}{
\textbf{Partie 2}

%À l'issue de la course, le classement est affiché ci-contre.
%
%\textbf{On s'intéresse aux années de naissance des 20 premiers coureurs.}
%
%\medskip
%
\begin{enumerate}
\item %On a rangé les années de naissance des coureurs dans l'ordre croissant :
%
%\begin{tabularx}{\linewidth}{*{5}{X}}
%1959&1959&1960&1966&1969\\
%1970&1972&1972&1974&1979\\
%1981&1983&1986&1988&1989\\
%1993&1997&1998&2002&2003\\
%\end{tabularx}
%
%Donner la médiane de la série.
Il y a 10 coureurs nés avant 1980 et 10 coureurs nés après 1980 ; 1980 est donc la médiane de cette série.
\item %La moyenne de la série a été calculée dans la cellule B23.

%Quelle formule a été saisie dans la cellule B23 ?
On écrit dans la cellule B23 : =SOMME(B2:~B21)/20
\item %Astrid remarque que la moyenne et la médiane de cette série sont égales.

%Est-ce le cas pour n'importe quelle autre série statistique ?
En général la moyenne calcul de la somme divisé par le nombre d'éléments n'est pas égal à la médiane qui partage la série en deux séries de même effectif.
%Expliquer votre réponse.
\end{enumerate}
%} %\hfill
%\parbox{0.43\linewidth}{\begin{tabularx}{\linewidth}{|c|*{2}{X|}}\hline
% 	&A 			&B\\ \hline
%1 	&\footnotesize Classement &\footnotesize Année de naissance\\ \hline
%2 	&1 			&1983\\ \hline
%3 	&2 			&1972\\ \hline
%4 	&3 			&1966\\ \hline
%5 	&4 			&2003\\ \hline
%6 	&5 			&1986\\ \hline
%7 	&6 			&1972\\ \hline
%8 	&7 			&1979\\ \hline
%9 	&8 			&1997\\ \hline
%10 	&9 			&1959\\ \hline
%11 	&10 		&1981\\ \hline
%12 	&11 		&1970\\ \hline
%13 	&12 		&1989\\ \hline
%14 	&13 		&1988\\ \hline
%15 	&14 		&1959\\ \hline
%16 	&15 		&1993\\ \hline
%17 	&16 		&1974\\ \hline
%18 	&17 		&1960\\ \hline
%19 	&18 		&1998\\ \hline
%20 	&19 		&1969\\ \hline
%21 	&20 		&2002\\ \hline
%22	&			&\\ \hline
%23 &moyenne		& 1980\\ \hline
%\end{tabularx}}

\vspace{0,5cm}

