\textbf{Exercice 7 \hfill 5 points}

\bigskip
 
%On considère ces deux programmes de calcul :
%
%\begin{center}
%\begin{tabularx}{\linewidth}{X p{0.2cm}X} 
%\textbf{Programme A :}& &\textbf{Programme B :}\\
%\psframebox[framearc=0.25]{\parbox{4.9cm}{Choisir un nombre\\Soustraire 0,5\\Multiplier le résultat par le double\\du nombre choisi au départ}}
%&& 
%\psframebox[framearc=0.25]{\parbox{4.9cm}{
%Choisir un nombre\\
%Calculer son carré\\
%Multiplier le résultat par 2\\
%Soustraire \`a ce nouveau résultat\\
%le nombre choisi au départ}}\\
%\end{tabularx}
%\end{center} 
 
\begin{enumerate}
\item 
	\begin{enumerate}
		\item %Montrer que si on applique le programme A au nombre 10, le résultat est 190.
$10 \to 10 - 0,5 = 9,5 \to 9,5 \times 20 = 190$.
		\item %Appliquer le programme B au nombre 10.
$10 \to 10^2 = 100 \to 2 \times 100 = 200 \to 200 - 10 = 190$
	\end{enumerate} 
\item %On a utilisé un tableur pour calculer des résultats de ces deux programmes. Voici ce qu'on a obtenu : 

%\begin{center}
%\begin{tabularx}{0.8\linewidth}{|c|*{3}{>{\centering \arraybackslash}X|}}\hline
%&A& B& C\\ \hline 
%1& Nombre choisi& Programme A& Programme B\\ \hline 
%2 &1 &1 &1\\ \hline 
%3 &2 &6 &6 \\ \hline 
%4 &3 &15 &15 \\ \hline 
%5 &4 &28 &28 \\ \hline 
%6 &5 &45 &45 \\ \hline 
%7& 6 &66 &66\\ \hline 
%\end{tabularx}
%\end{center} 
	\begin{enumerate}
		\item %Quelle formule a-t-on saisie dans la cellule C2 puis recopiée vers le bas ?
=A$2^2 *2 - $A2 
		\item %Quelle conjecture peut-on faire à  la lecture de ce tableau ?
Il semble que les deux programmes conduisent au même résultat. 
		\item %Prouver cette conjecture.
Programme A : $x \to x - 0,5 \to (x - 0,5) \times 2x = 2x(x - 0,5) = 2x^2 - x$.

Programme B : $x \to x^2 \to 2 \times x^2 \to 2x^2 - x$.

Les deux programmes donnent le même résultat : le double du carré du nombre initial auquel on retranche le nombre initial. 
	\end{enumerate}
\item %Quels sont les deux nombres à  choisir au départ pour obtenir 0 à  l'issue de ces programmes ? 
Il faut trouver $x$ tel que $2x^2 - x = 0$ soit $x(2x - 1) = 0$ d'o๠deux possibilités :

$x = 0$ ou $2x - 1 = 0$ soit $2x = 1$ et enfin $x = 0,5$.

\bigskip
 
