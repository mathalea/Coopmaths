
\medskip

Dans un laboratoire A, pour tester le vaccin contre la grippe de la saison hivernale prochaine, on a injecté la même souche de virus à  5 groupes comportant 29 souris chacun.

3 de ces groupes avaient été préalablement vaccinés contre ce virus.

Quelques jours plus tard, on remarque que :

\setlength\parindent{10mm}
\begin{itemize}
\item[$\bullet~~$] dans les $3$ groupes de souris vaccinées, aucune souris n'est malade ;
\item[$\bullet~~$] dans chacun des groupes de souris non vaccinées, $23$ souris ont développé la maladie.
\end{itemize}
\setlength\parindent{0mm} 

\medskip
 
\begin{enumerate}
\item 
	\begin{enumerate}
		\item Il y a $5$~groupes de $29$~souris. $5\times29=145$.
		
Il y a $2$~groupes de souris non vaccinées contenant chacun $23$~souris ayant développé la maladie.$2\times23=46$.
		
		 La proportion de souris malades lors de ce test est $\dfrac{46}{145}$ car il y a $46$~souris ayant développé la maladie sur $145$~souris.
		\item Les décompositions en facteurs premiers de 46 et 145 sont : \quad $46=2\times23$ et $145=5\times29$. \\
		Ces deux décompositions permettent de dire que le seul diviseur commun à 46 et 145 est 1, on ne peut donc pas simplifier cette fraction.
	\end{enumerate}	
\end{enumerate}
		
Dans un laboratoire B on informe que $\dfrac{140}{870}$ des souris ont été malades.

\begin{enumerate}		
\item[\textbf{2.}] 
	\begin{enumerate}
		\item  ~	
	\vspace{-0.5cm}
	\hspace{-0.3cm}\getprime{20}%
\primedecomp{140}

\vspace{0.4cm}

La décomposition en facteurs premiers de 140 est : \quad $140=2\times2\times5\times7$.

\hspace{-0.3cm}\getprime{20}%
\primedecomp{870}

\vspace{0.2cm}

La décomposition en facteurs premiers de 870 est : \quad $870 = 2 \times 3\times 5 \times 29$.
		\item $\dfrac{140}{870}=\dfrac{\cancel{2}\times2\times\cancel{5}\times7}{\cancel{2}\times3\times\cancel{5}\times29} = \dfrac{14}{87}$.
		
La forme irréductible de la proportion de souris malades dans le laboratoire B est $\dfrac{14}{87}$.
	\end{enumerate}
\end{enumerate}

\vspace{0,5cm}

