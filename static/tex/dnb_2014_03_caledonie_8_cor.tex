\documentclass[10pt]{article}
\usepackage[T1]{fontenc}
\usepackage[utf8]{inputenc}%ATTENTION codage UTF8
\usepackage{fourier}
\usepackage[scaled=0.875]{helvet}
\renewcommand{\ttdefault}{lmtt}
\usepackage{amsmath,amssymb,makeidx}
\usepackage[normalem]{ulem}
\usepackage{diagbox}
\usepackage{fancybox}
\usepackage{tabularx,booktabs}
\usepackage{colortbl}
\usepackage{pifont}
\usepackage{multirow}
\usepackage{dcolumn}
\usepackage{enumitem}
\usepackage{textcomp}
\usepackage{lscape}
\newcommand{\euro}{\eurologo{}}
\usepackage{graphics,graphicx}
\usepackage{pstricks,pst-plot,pst-tree,pstricks-add}
\usepackage[left=3.5cm, right=3.5cm, top=3cm, bottom=3cm]{geometry}
\newcommand{\R}{\mathbb{R}}
\newcommand{\N}{\mathbb{N}}
\newcommand{\D}{\mathbb{D}}
\newcommand{\Z}{\mathbb{Z}}
\newcommand{\Q}{\mathbb{Q}}
\newcommand{\C}{\mathbb{C}}
\usepackage{scratch}
\renewcommand{\theenumi}{\textbf{\arabic{enumi}}}
\renewcommand{\labelenumi}{\textbf{\theenumi.}}
\renewcommand{\theenumii}{\textbf{\alph{enumii}}}
\renewcommand{\labelenumii}{\textbf{\theenumii.}}
\newcommand{\vect}[1]{\overrightarrow{\,\mathstrut#1\,}}
\def\Oij{$\left(\text{O}~;~\vect{\imath},~\vect{\jmath}\right)$}
\def\Oijk{$\left(\text{O}~;~\vect{\imath},~\vect{\jmath},~\vect{k}\right)$}
\def\Ouv{$\left(\text{O}~;~\vect{u},~\vect{v}\right)$}
\usepackage{fancyhdr}
\usepackage[french]{babel}
\usepackage[dvips]{hyperref}
\usepackage[np]{numprint}
%Tapuscrit : Denis Vergès
%\frenchbsetup{StandardLists=true}

\begin{document}
\setlength\parindent{0mm}
% \rhead{\textbf{A. P{}. M. E. P{}.}}
% \lhead{\small Brevet des collèges}
% \lfoot{\small{Polynésie}}
% \rfoot{\small{7 septembre 2020}}
\pagestyle{fancy}
\thispagestyle{empty}
% \begin{center}
    
% {\Large \textbf{\decofourleft~Brevet des collèges Polynésie 7 septembre 2020~\decofourright}}
    
% \bigskip
    
% \textbf{Durée : 2 heures} \end{center}

% \bigskip

% \textbf{\begin{tabularx}{\linewidth}{|X|}\hline
%  L'évaluation prend en compte la clarté et la précision des raisonnements ainsi que, plus largement, la qualité de la rédaction. Elle prend en compte les essais et les démarches engagées même non abouties. Toutes les réponses doivent être justifiées, sauf mention contraire.\\ \hline
% \end{tabularx}}

% \vspace{0.5cm}\textbf{Exercice 8 :  \hfill 3 points}

\medskip 

%Une feuille de calcul d'un tableur est représentée ci-dessous. Pour chaque question, une seule des trois réponses proposées est exacte. Sur la copie, indiquer le numéro de la question et recopier, sans justifier, la proposition choisie. Aucun point ne sera enlevé en cas de mauvaise réponse.
%
%\begin{center} 
%\begin{tabularx}{0.6\linewidth}{|c|*{4}{>{\centering \arraybackslash}X|}}\hline
%	&A									&B	&C	&D\\ \hline   
%1	&35									&21	&18	&\\ \hline        
%2	&\rput(-0.85,-0.2){\psframe[linewidth=2pt](1.7,0.5)}	&	&	&\\ \hline   
%3	&									&	&	&\\ \hline   
%4	&									&	&	&\\ \hline   
%5	&									&	&	&\\ \hline
%\end{tabularx}            
%\end{center} 
%
%\bigskip

%\begin{tabularx}{\linewidth}{|m{6cm}|*{3}{>{\centering \arraybackslash}X|}}\hline     
%Propositions								&\multicolumn{3}{|c|}{Réponses }\\ \cline{2-4}    
%											&A	&B 	&C\\ \hline   
%\textbf{1.} Dans la cellule A3, lorsqu'on écrit : &	&	& \\          
%\fbox{=A1*B1+C1} &\np{1365}					&74 &753 \\  
%on obtient alors :							&	&	& \\ \hline          
%\textbf{2.} Dans la cellule B3, lorsqu'on écrit : &	&	& \\         
%\fbox{=MAX(A1~;~C1)} 						&35	&21	&18 \\     
%on obtient alors :							&	&	&   \\ \hline        
%\textbf{3.} Dans la cellule C3, lorsqu'on écrit :&	&	&  \\          
%\fbox{=SOMME(A1~:~C1)} 						&35	&74 &56 \\    
%      
%on obtient alors : 							&	&	&\\ \hline
%\end{tabularx}
\begin{enumerate}
\item On a $35 \times 21 + 18 = 753$. Réponse \textbf{C}.
\item Le plus grand des deux nombres 35 et 21 est 35 : réponse \textbf{A}.
\item On a $35 + 21 + 18 = 74$. Réponse \textbf{B}.
\end{enumerate}
\end{document}\end{document}