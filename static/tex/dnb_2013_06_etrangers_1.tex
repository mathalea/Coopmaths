\textbf{\textsc{Exercice 1} \hfill 6 points}

\medskip

\emph{Cet exercice est un questionnaire à choix multiple (QCM). Pour chaque ligne du tableau, trois réponses sont proposées, mais une seule est exacte. Toute réponse exacte vaut $1$ point. Toute réponse inexacte ou toute absence de réponse n'enlève pas de point. Pour chacune des questions, on indiquera sur sa feuille le numéro de la question et la réponse choisie.}

\medskip

\begin{tabularx}{\linewidth}{|c|m{3cm}|*{3}{>{\centering \arraybackslash}X|}}\hline 
& & réponse A &réponse B &réponse C\\ \hline 
1& Les solutions de l'équation $(x+7)(2x - 7)= 0$ sont& 
$-7$ et $3,5$& 7 et $- 3,5$& $- 7$ et $5$\\ \hline 
2&La (ou les) solution(s) de l'inéquation $-2(x + 7) \leqslant  - 16$ est (sont)&tous les nombres inférieurs ou égaux à 1&tous les nombres supérieurs ou égaux à 1&1\\ \hline 
3&La forme développée de $(7x - 5)^2$ est&$49x^2 - 25$&$49x^2 - 70x + 25$& $49x^2 - 70x - 25$\\ \hline
4&La forme factorisée de $9 - 64x^2$ est& $- 55 x^2$&$(3 - 8x)^2$&$(3 - 8x)(3 + 8x)$\\ \hline
5&\psset{unit=0.8cm}\begin{pspicture}(-1,0)(1,5)
%\psgrid
\psellipse[fillstyle=solid,fillcolor=lightgray](0,0.2)(0.4,0.2)
\pspolygon[fillstyle=solid,fillcolor=lightgray](-0.3,3.1)(0.3,3.1)(0,1.8)
\psline[linewidth=1.4pt](0,0.2)(0,1.8)
\psellipse(0,4.3)(0.6,0.3)
\psline[linewidth=1.4pt](-0.6,4.3)(0,1.8)(0.6,4.3)
\psellipse[fillstyle=solid,fillcolor=lightgray](0,3.1)(0.3,0.15)
\psline{<->}(0.8,1.8)(0.8,4.3) \uput[r](0.8,3.1){$h$}
\psline{<->}(-0.8,1.8)(-0.8,3.1) \uput[r](-0.8,2.45){$\frac{h}{2}$}
\end{pspicture}

Le liquide remplit-il à moitié le verre ?&oui&non, c'est moins de la moitié&non, c'est plus de la moitié\\ \hline
6&La section KMEH du cube ABCDEFGH par un plan parallèle à une de ses arêtes est \ldots

\psset{unit=0.6cm}\begin{pspicture}(-0.2,-0.2)(4.6,4.8)
\pspolygon[fillstyle=solid,fillcolor=lightgray](1.1,1.1)(1.1,4.2)(2.2,3.1)(2.2,0)
\psframe(3.1,3.1)%ABCD
\psline(3.1,0)(4.2,1.1)(4.2,4.2)(3.1,3.1)%DFGC
\psline(4.2,4.2)(1.1,4.2)(0,3.1)%GHD
\psline[linestyle=dashed](0,0)(1.1,1.1)(4.2,1.1)%AEF
\uput[l](0.15,0.15){\footnotesize A} \uput[r](3.1,0){\footnotesize B} \uput[r](3.1,3.1){\footnotesize C} 
\uput[ul](0,3.1){\footnotesize D} \uput[l](1.1,1.1){\footnotesize E} \uput[ur](4.2,1.1){\footnotesize F} 
\uput[ur](4.2,4.2){\footnotesize G} \uput[ul](1.1,4.2){\footnotesize H} \uput[ur](2.2,3.1){\footnotesize K} 
\uput[ur](2.2,0){\footnotesize M}
 
\end{pspicture}
&un parallélogramme non rectangle&un carré &un rectangle\\ \hline
\end{tabularx} 

\bigskip

