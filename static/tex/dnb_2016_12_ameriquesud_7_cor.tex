\documentclass[10pt]{article}
\usepackage[T1]{fontenc}
\usepackage[utf8]{inputenc}%ATTENTION codage UTF8
\usepackage{fourier}
\usepackage[scaled=0.875]{helvet}
\renewcommand{\ttdefault}{lmtt}
\usepackage{amsmath,amssymb,makeidx}
\usepackage[normalem]{ulem}
\usepackage{diagbox}
\usepackage{fancybox}
\usepackage{tabularx,booktabs}
\usepackage{colortbl}
\usepackage{pifont}
\usepackage{multirow}
\usepackage{dcolumn}
\usepackage{enumitem}
\usepackage{textcomp}
\usepackage{lscape}
\newcommand{\euro}{\eurologo{}}
\usepackage{graphics,graphicx}
\usepackage{pstricks,pst-plot,pst-tree,pstricks-add}
\usepackage[left=3.5cm, right=3.5cm, top=3cm, bottom=3cm]{geometry}
\newcommand{\R}{\mathbb{R}}
\newcommand{\N}{\mathbb{N}}
\newcommand{\D}{\mathbb{D}}
\newcommand{\Z}{\mathbb{Z}}
\newcommand{\Q}{\mathbb{Q}}
\newcommand{\C}{\mathbb{C}}
\usepackage{scratch}
\renewcommand{\theenumi}{\textbf{\arabic{enumi}}}
\renewcommand{\labelenumi}{\textbf{\theenumi.}}
\renewcommand{\theenumii}{\textbf{\alph{enumii}}}
\renewcommand{\labelenumii}{\textbf{\theenumii.}}
\newcommand{\vect}[1]{\overrightarrow{\,\mathstrut#1\,}}
\def\Oij{$\left(\text{O}~;~\vect{\imath},~\vect{\jmath}\right)$}
\def\Oijk{$\left(\text{O}~;~\vect{\imath},~\vect{\jmath},~\vect{k}\right)$}
\def\Ouv{$\left(\text{O}~;~\vect{u},~\vect{v}\right)$}
\usepackage{fancyhdr}
\usepackage[french]{babel}
\usepackage[dvips]{hyperref}
\usepackage[np]{numprint}
%Tapuscrit : Denis Vergès
%\frenchbsetup{StandardLists=true}

\begin{document}
\setlength\parindent{0mm}
% \rhead{\textbf{A. P{}. M. E. P{}.}}
% \lhead{\small Brevet des collèges}
% \lfoot{\small{Polynésie}}
% \rfoot{\small{7 septembre 2020}}
\pagestyle{fancy}
\thispagestyle{empty}
% \begin{center}
    
% {\Large \textbf{\decofourleft~Brevet des collèges Polynésie 7 septembre 2020~\decofourright}}
    
% \bigskip
    
% \textbf{Durée : 2 heures} \end{center}

% \bigskip

% \textbf{\begin{tabularx}{\linewidth}{|X|}\hline
%  L'évaluation prend en compte la clarté et la précision des raisonnements ainsi que, plus largement, la qualité de la rédaction. Elle prend en compte les essais et les démarches engagées même non abouties. Toutes les réponses doivent être justifiées, sauf mention contraire.\\ \hline
% \end{tabularx}}

% \vspace{0.5cm}\textbf{\textsc{Exercice 7} \hfill 6 points}

\medskip

\begin{enumerate}
\item On sait que le triangle USO est rectangle en O. 

On a OS $ = 396 - 220 = 176$.

Pour calculer la valeur de l'angle $\widehat{\text{GUS}}$, on recourt à
la formule du sinus.

$\sin \widehat{\text{OUS}} = \dfrac{\text{côté opposé}}{\text{hypoténuse}} = \dfrac{\text{OS}}{\text{US}} = \dfrac{176}{762} \approx  0,231$.

Il ne reste plus qu'à calculer avec la calculatrice et l'inverse du sinus : on obtient $\widehat{\text{OUS}} \approx  13$\degres.

\emph{Remarque} : on ne peut pas utiliser la formule du cosinus car nous ne disposons pas de la longueur du côté adjacent à l'angle, donné par [UO]
\item On utilise la formule de la vitesse :

$\text{vitesse} = \dfrac{\text{distance}}{\text{temps}}$.

Le temps est de 6 min 30 s soit $360 + 30 = 390$~s.

$\text{vitesse} = \dfrac{762}{390} = \dfrac{254}{130} \approx 1,954$ soit 2~m/s à l'unité près. 
\item 
	\begin{enumerate}
		\item Le nombre calculé avec la formule est égal à la somme des affluences de de 8 h à 20 h par tranches de 2 h, 615 visiteurs sur la journée.
		\item La somme des nombres visibles est : $122 + 140 + 63 + 75 + 118 = 518$.
		
Le nombre de visiteurs entre 12~h et 14~h est donc égal à $615 - 518 = 97$
	\end{enumerate}
\item Pour qu'un tableur puisse appliquer un calcul, il faut toujours commencer par le symbole \og = \fg. Il suffit alors de taper \og  =MOYENNE(B2:G2)/2\fg.

\emph{Remarque} : on peut également pour avoir cette moyenne entrer la formule : \og =H2/12 \fg.
\end{enumerate}
\end{document}\end{document}