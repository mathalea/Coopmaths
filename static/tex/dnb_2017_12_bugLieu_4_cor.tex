\documentclass[10pt]{article}
\usepackage[T1]{fontenc}
\usepackage[utf8]{inputenc}%ATTENTION codage UTF8
\usepackage{fourier}
\usepackage[scaled=0.875]{helvet}
\renewcommand{\ttdefault}{lmtt}
\usepackage{amsmath,amssymb,makeidx}
\usepackage[normalem]{ulem}
\usepackage{diagbox}
\usepackage{fancybox}
\usepackage{tabularx,booktabs}
\usepackage{colortbl}
\usepackage{pifont}
\usepackage{multirow}
\usepackage{dcolumn}
\usepackage{enumitem}
\usepackage{textcomp}
\usepackage{lscape}
\newcommand{\euro}{\eurologo{}}
\usepackage{graphics,graphicx}
\usepackage{pstricks,pst-plot,pst-tree,pstricks-add}
\usepackage[left=3.5cm, right=3.5cm, top=3cm, bottom=3cm]{geometry}
\newcommand{\R}{\mathbb{R}}
\newcommand{\N}{\mathbb{N}}
\newcommand{\D}{\mathbb{D}}
\newcommand{\Z}{\mathbb{Z}}
\newcommand{\Q}{\mathbb{Q}}
\newcommand{\C}{\mathbb{C}}
\usepackage{scratch}
\renewcommand{\theenumi}{\textbf{\arabic{enumi}}}
\renewcommand{\labelenumi}{\textbf{\theenumi.}}
\renewcommand{\theenumii}{\textbf{\alph{enumii}}}
\renewcommand{\labelenumii}{\textbf{\theenumii.}}
\newcommand{\vect}[1]{\overrightarrow{\,\mathstrut#1\,}}
\def\Oij{$\left(\text{O}~;~\vect{\imath},~\vect{\jmath}\right)$}
\def\Oijk{$\left(\text{O}~;~\vect{\imath},~\vect{\jmath},~\vect{k}\right)$}
\def\Ouv{$\left(\text{O}~;~\vect{u},~\vect{v}\right)$}
\usepackage{fancyhdr}
\usepackage[french]{babel}
\usepackage[dvips]{hyperref}
\usepackage[np]{numprint}
%Tapuscrit : Denis Vergès
%\frenchbsetup{StandardLists=true}

\begin{document}
\setlength\parindent{0mm}
% \rhead{\textbf{A. P{}. M. E. P{}.}}
% \lhead{\small Brevet des collèges}
% \lfoot{\small{Polynésie}}
% \rfoot{\small{7 septembre 2020}}
\pagestyle{fancy}
\thispagestyle{empty}
% \begin{center}
    
% {\Large \textbf{\decofourleft~Brevet des collèges Polynésie 7 septembre 2020~\decofourright}}
    
% \bigskip
    
% \textbf{Durée : 2 heures} \end{center}

% \bigskip

% \textbf{\begin{tabularx}{\linewidth}{|X|}\hline
%  L'évaluation prend en compte la clarté et la précision des raisonnements ainsi que, plus largement, la qualité de la rédaction. Elle prend en compte les essais et les démarches engagées même non abouties. Toutes les réponses doivent être justifiées, sauf mention contraire.\\ \hline
% \end{tabularx}}

% \vspace{0.5cm}\textbf{Exercice 4 :  \hfill 5 points}

\medskip

\begin{enumerate}
\item On souhaite tracer le motif ci-dessous en forme de losange.

\vspace{0.2cm}

\begin{tabularx}{\linewidth}{|X|X|}\hline
Le motif \textbf{Losange} &Le bloc \textbf{Losange}\\
\psset{unit=0.6cm}
\begin{pspicture}(0,-5)(10,4)
\pspolygon(0.5,1)(5.6,1)(10,3.5)(4.9,3.5)
\psarc(0.5,1){4mm}{0}{30} \psarc(4.9,3.5){4mm}{210}{360}
\rput(1,2.5){\scriptsize Point de départ}
\psline{->}(0.5,2.2)(0.5,1.1)
\rput(1.6,1.3){\footnotesize 30\degres}\rput(5.1,2.5){\footnotesize 150\degres}
\psline{<->}(0.5,0.8)(5.6,0.8)
\uput[d](3.05,0.8){\footnotesize 60}
\end{pspicture}&\footnotesize{\begin{scratch}
\initmoreblocks{définir \namemoreblocks{Losange}}
\blockpen{stylo en position d'écriture}
\blockmove{avancer de \ovalnum{}}
\blockmove{tourner \turnleft{} de \ovalnum{30} degrés}
\blockmove{avancer de \ovalnum{}}
\blockmove{tourner \turnleft{} de \ovalnum{150} degrés}
\blockmove{avancer de \ovalnum{}}
\blockmove{tourner \turnleft{} de \ovalnum{} degrés}
\blockmove{avancer de \ovalnum{}}
\blockmove{tourner \turnleft{} de \ovalnum{} degrés}
\blockpen{relever le stylo}
\end{scratch}}\\ \hline
\end{tabularx}

\item On souhaite réaliser la figure ci-dessous construite à  partir du bloc \textbf{Losange} complété à  la question 1.

\parbox{0.6\linewidth}{
La figure :
\begin{center}
\psset{unit=0.75cm}
\begin{pspicture}(-2.5,-2.5)(2.5,2.5)
\def\losange1{\pspolygon(0,0)(1.275,0)(2.375,0.625)(1.1,0.625)}
\multido{\n=0+30}{12}{\rput{\n}(0;0){\losange1}}
\end{pspicture}
\end{center}

L’instruction \raisebox{-2.3ex}{\begin{scratch}\blockmove{s’orienter à  \ovalnum{90\selectarrownum}}\end{scratch}} signifie que l'on se dirige vers la droite. \\[2mm]
Les deux instructions à  placer dans la boucle pour
finir le script sont les \linebreak instruction \ding{'302}, puis \ding{'300}.}\hfill
\parbox{0.37\linewidth}{Le script : \\
\begin{scratch}
\blockinit{Quand \greenflag est cliqué}
\blockpen{effacer tout}
\blockmove{aller à  x: \ovalnum0 y: \ovalnum0}
\blockmove{s'orienter à  \ovalnum{90\selectarrownum}}
\blockrepeat{répéter \ovalnum{12} fois}
{\initmoreblocks{Losange}
\blockmove{tourner \turnleft{} de \ovalnum{30} degrés}
}
\end{scratch}}
\end{enumerate}

\vspace{0.5cm}

\end{document}