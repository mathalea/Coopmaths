\textbf{\textsc{Exercice 4} \hfill 5 points}

\medskip

\parbox{0.48\linewidth}{Pour présenter ses macarons, une boutique souhaite utiliser des présentoirs dont la forme est une pyramide régulière à base carrée de côté 30 cm et dont les
arêtes latérales mesurent 55~cm.

On a schématisé le présentoir par la figure suivante :}\hfill \parbox{0.48\linewidth}{
\psset{unit=1cm}
\begin{pspicture}(5,4.5)
\pspolygon(0.5,0.5)(3.2,0.5)(4.5,2)(2.5,4)%ABCS
\psline(3.2,0.5)(2.5,4)
\psline[linestyle=dotted](0.5,0.5)(4.5,2)(1.8,2)(3.2,0.5)%ACDB
\psline[linestyle=dotted](0.5,0.5)(1.8,2)(2.5,4)
\psline[linestyle=dashed](2.5,4)(2.5,1.3)
\psline[linewidth=0.3pt](0.5,0.5)(1,0.5)(1.2,0.8)(0.7,0.8)
\psline[linewidth=0.3pt](3.2,0.5)(2.7,0.5)(2.9,0.8)(3.4,0.8)
\psline[linewidth=0.3pt](4.5,2)(4,2)(3.75,1.7)(4.25,1.7)
\psline[linewidth=0.3pt](1.8,2)(2.3,2)(2.1,1.7)(1.6,1.7)
\uput[dl](0.5,0.5){A} \uput[dr](3.2,0.5){B} \uput[ur](4.5,2){C} \uput[ul](1.8,2){D} \uput[d](2.5,1.3){O}\uput[u](2.5,4){S} 
\end{pspicture}
}

Peut-on placer ce présentoir dans une vitrine réfrigérée parallélépipédique dont la hauteur est de
50 cm ?

\bigskip

