\textbf{Exercice 2 \hfill 6 points}

\medskip

%Voici un programme de calcul :
% 
%\begin{center}
%\begin{tabularx}{0.6\linewidth}{|l X|}\hline
%$\bullet~~$& Choisir un nombre entier positif\\
%$\bullet~~$& Ajouter 1\\
%$\bullet~~$& Calculer le carré du résultat obtenu\\
%$\bullet~~$& Enlever le carré du nombre de départ.\\ \hline
%\end{tabularx}
%\end{center}

\begin{enumerate}
\item %On applique ce programme de calcul au nombre 3. Montrer qu'on obtient 7.

On a successivement : $3 \to 3 + 1 = 4 \to 4^2 = 16 \to 16 - 3^2 = 16 - 9 = 7$.
\item %Voici deux affirmations :

%Affirmation \no 1 : \og Le chiffre des unités du résultat obtenu est 7 \fg.

%Affirmation \no 2 : \og Chaque résultat peut s'obtenir en ajoutant le nombre entier de départ et le nombre entier qui le suit \fg.
	\begin{enumerate}
		\item %Vérifier que ces deux affirmations sont vraies pour les nombres 8 et 13.
$\bullet~~$Avec 8 on obtient : $8 \to 9 \to 81 \to 81- 64 = 17$. Le chiffre des unités du résultat obtenu est 7.

D'autre part $8 + (8 + 1) = 8 + 9 = 17$. le résultat s'obtient en ajoutant le nombre entier de départ et le nombre entier qui le suit.

$\bullet~~$Avec 13 on obtient $13 \to 14 \to 196 \to 196 - 169 = 27$. Le chiffre des unités du résultat obtenu est 7.

D'autre part $13 + (13 + 1) = 13 + 14 = 27$. le résultat s'obtient en ajoutant le nombre entier de départ et le nombre entier qui le suit.
		\item %Pour chacune de ces deux affirmations, expliquer si elle est vraie ou fausse quel que soit le nombre choisi au départ.
		Pour l'affirmation 1, en partant de 4, on obtient :
		
		$4 \to 5 \to 25 \to 25 - 16 = 9$. Le chiffre des unités n'est pas 7. l'affirmation 1 n'est pas vraie quel que soit le nombre de départ.
		
		Pour l'affirmation 2. Soit $x$ le nombre de départ, on obtient :
		
$x \to (x + 1) \to (x + 1)^2 \to (x + 1)^2 - x^2 = x^2 + 2x + 1 - x ^2 = 2x + 1 = x + x + 1 = x + (x + 1)$ : le résultat s'obtient en ajoutant le nombre entier de départ et le nombre entier qui le suit.

L'affirmation 2 est vraie quel que soit le nombre choisi au départ.
	\end{enumerate}
\end{enumerate}

\bigskip


\bigskip

