\textbf{\textsc{Exercice 6} \hfill 5 points}

\medskip

%\textbf{Dans cet exercice, toute trace de recherche, même non aboutie, sera prise en compte dans l'évaluation.}
%
%\medskip
%
%Monsieur et Madame Jean vont faire construire une piscine et l'entourer de dalles en bois sur une largeur de 2~m.
%\medskip
%
%\textbf{Information 1 :} les modèles de piscine
%
%\begin{center}
%\begin{tabularx}{\linewidth}{*{3}{>{\centering \arraybackslash}X}}
%Modèle A& Modèle B& Modèle C\\
%\psset{unit=1cm}
%\begin{pspicture}(4,3.5)
%\psframe(0.5,1)(3.5,2.5)
%\psline{<->}(0.5,2.7)(3.5,2.7)
%\psline{<->}(0.3,1)(.3,2.5)
%\uput[u](2,2.7){\footnotesize 500 cm}
%\rput{90}(0.1,1.9){\footnotesize 300 cm}
%\end{pspicture}&\psset{unit=1cm}
%\begin{pspicture}(4,3.5)
%\psframe(0.5,1)(3.8,3)
%\psline{<->}(0.5,3.2)(3.8,3.2)
%\psline{<->}(0.3,1)(.3,3)
%\uput[u](2,3.1){\footnotesize 850 cm}
%\rput{90}(0.1,2){\footnotesize 350 cm}
%
%\end{pspicture}&\psset{unit=1cm}
%\begin{pspicture}(4,3.5)
%\psframe(0.2,1)(3.7,2.5)
%\psline{<->}(0.2,2.7)(3.7,2.7)
%\psline{<->}(0.3,1)(.3,2.5)
%\uput[u](2,2.6){\footnotesize 800 cm}
%\rput{90}(0.05,1.9){\footnotesize 400 cm}
%\end{pspicture}\\
%profondeur : 133 cm&profondeur: 138 cm&profondeur: 144 cm\\
%pompe: débit 8 m$^3$/h&pompe: débit 10 m$^3$/h&pompe: débit 12 m$^3$/h
%\end{tabularx}
%\end{center}
%
%\emph{Les figures ci-dessus ne sont pas représentées à l'échelle.}
%
%\medskip
%
%\textbf{Information 2 :}  les dalles en bois
%
%\medskip
%
%
%Dalle Jécoba en bois,
%L 100 cm $\times$ larg. 100 cm $\times$ ép. 28 mm
%
%Référence 628 051
%
%Quantité pour 1 m$^2$ : 1
%
%Epaisseur du produit (en mm) : 28
%
%Couleur: Naturel
%
%Prix indicatif: 13,90~\euro le mètre carré
%
%\medskip
%
%\textbf{Information 3 :} la promotion sur les dalles en bois
%
%\begin{center}
%\textbf{Vente flash :} 15\,\% de remise
%\end{center}
%
%Ils choisissent le modèle de piscine qui a la plus grande surface.
%
%Quel prix payent-ils pour leurs dalles s'ils profitent de la vente flash ?

Aire du modèle A : $5 \times 3 = 15$~m$^2$ ;

Aire du modèle B : $8,5 \times 3,5 = 29,75$~m$^2$ ;

Aire du modèle A : $8 \times 4 = 32$~m$^2$ : ils choisissent le modèle C.

Aire des dalles : $(8 + 2 + 2) \times 2 \times 2 + 4 \times 2 \times 2 = 64$~m$^2$ soit 64 dalles.

La promotion revient à payer 85\,\% du prix initial. Le coût des dalles est donc de :

$64 \times 13,9 \times 0,85 = 756,16$~\euro.
%%%%%%%%%%%%%
\vspace{0.25cm}

