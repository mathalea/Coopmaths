\documentclass[10pt]{article}
\usepackage[T1]{fontenc}
\usepackage[utf8]{inputenc}%ATTENTION codage UTF8
\usepackage{fourier}
\usepackage[scaled=0.875]{helvet}
\renewcommand{\ttdefault}{lmtt}
\usepackage{amsmath,amssymb,makeidx}
\usepackage[normalem]{ulem}
\usepackage{diagbox}
\usepackage{fancybox}
\usepackage{tabularx,booktabs}
\usepackage{colortbl}
\usepackage{pifont}
\usepackage{multirow}
\usepackage{dcolumn}
\usepackage{enumitem}
\usepackage{textcomp}
\usepackage{lscape}
\newcommand{\euro}{\eurologo{}}
\usepackage{graphics,graphicx}
\usepackage{pstricks,pst-plot,pst-tree,pstricks-add}
\usepackage[left=3.5cm, right=3.5cm, top=3cm, bottom=3cm]{geometry}
\newcommand{\R}{\mathbb{R}}
\newcommand{\N}{\mathbb{N}}
\newcommand{\D}{\mathbb{D}}
\newcommand{\Z}{\mathbb{Z}}
\newcommand{\Q}{\mathbb{Q}}
\newcommand{\C}{\mathbb{C}}
\usepackage{scratch}
\renewcommand{\theenumi}{\textbf{\arabic{enumi}}}
\renewcommand{\labelenumi}{\textbf{\theenumi.}}
\renewcommand{\theenumii}{\textbf{\alph{enumii}}}
\renewcommand{\labelenumii}{\textbf{\theenumii.}}
\newcommand{\vect}[1]{\overrightarrow{\,\mathstrut#1\,}}
\def\Oij{$\left(\text{O}~;~\vect{\imath},~\vect{\jmath}\right)$}
\def\Oijk{$\left(\text{O}~;~\vect{\imath},~\vect{\jmath},~\vect{k}\right)$}
\def\Ouv{$\left(\text{O}~;~\vect{u},~\vect{v}\right)$}
\usepackage{fancyhdr}
\usepackage[french]{babel}
\usepackage[dvips]{hyperref}
\usepackage[np]{numprint}
%Tapuscrit : Denis Vergès
%\frenchbsetup{StandardLists=true}

\begin{document}
\setlength\parindent{0mm}
% \rhead{\textbf{A. P{}. M. E. P{}.}}
% \lhead{\small Brevet des collèges}
% \lfoot{\small{Polynésie}}
% \rfoot{\small{7 septembre 2020}}
\pagestyle{fancy}
\thispagestyle{empty}
% \begin{center}
    
% {\Large \textbf{\decofourleft~Brevet des collèges Polynésie 7 septembre 2020~\decofourright}}
    
% \bigskip
    
% \textbf{Durée : 2 heures} \end{center}

% \bigskip

% \textbf{\begin{tabularx}{\linewidth}{|X|}\hline
%  L'évaluation prend en compte la clarté et la précision des raisonnements ainsi que, plus largement, la qualité de la rédaction. Elle prend en compte les essais et les démarches engagées même non abouties. Toutes les réponses doivent être justifiées, sauf mention contraire.\\ \hline
% \end{tabularx}}

% \vspace{0.5cm}\textbf{\textsc{Exercice 6} \hfill 7 points}

\medskip

Avec des ficelles de 20~cm, on construit des polygones comme ci-dessous :
\medskip

\begin{center} 

\textbf{Méthode de construction des polygones}

\begin{tabularx}{\linewidth}{|l|m{6cm}|X|}\hline 
Étape 1&\psset{unit=1cm}\begin{pspicture}(6,1)%\psgrid
\psline(0,0.5)(6,0.5)\psline(1.5,0.8)(1.7,0.4)\rput{-65}(1.78,0.25){\psellipse(0,0)(0.16,0.1)}
\psline(1.8,0.8)(1.55,0.4)\rput{-125}(1.45,0.25){\psellipse(0,0)(0.16,0.1)} \end{pspicture}&On coupe la ficelle de 20~cm en deux   morceaux.\\ \hline   
Étape 2&\psset{unit=1cm}\begin{pspicture}(6,1.5)%\psgrid
\psline(0,0.5)(1.6,0.5)\psline(1.8,0.5)(6,0.5)  \rput(0.8,1){morceau \no 1}
   \rput(4,1){morceau \no 2}\end{pspicture}   &On sépare les deux morceaux.\\ \hline   
Étape 3&\psset{unit=0.65cm}\begin{pspicture}(6,4.5)%\psgrid
\psframe(0.5,1)(1.5,2) \pspolygon(4,0.5)(7,0.5)(5.5,3.398)
\end{pspicture}
&$\bullet~~$Avec le \og morceau \no 1 \fg,  on construit un carré.
   
$\bullet~~$Avec le \og morceau \no 2 \fg,  on construit un triangle équilatéral.\\ \hline   
\end{tabularx}
\end{center}

\textbf{Partie 1 :}

\smallskip 

Dans cette partie, on découpe à l'étape 1 une ficelle pour que le \og morceau \no 1 \fg{} mesure 8~cm.

\medskip 

\begin{enumerate}
\item Dessiner en grandeur réelle les deux polygones obtenus. 
\item Calculer l'aire du carré obtenu. 
\item Estimer l'aire du triangle équilatéral obtenu en mesurant sur le dessin. 
\end{enumerate}

\medskip

\textbf{Partie 2 :}

\medskip 

Dans cette partie, on cherche maintenant à étudier l'aire des deux polygones obtenus à l'étape 3 en fonction de la longueur du \og morceau \no 1 \fg. 

\medskip

\begin{enumerate}
\item Proposer une formule qui permet de calculer l'aire du carré en fonction de la longueur du 
\og morceau \no 1 \fg. 
\item Sur le graphique ci-dessous: 

\setlength\parindent{8mm}
\begin{itemize}
\item[$\bullet~~$] la courbe A représente la fonction qui donne l'aire du carré en fonction de la longueur du \og morceau \no 1 \fg{} ; 
\item[$\bullet~~$]la courbe B représente la fonction qui donne l'aire du triangle équilatéral en fonction de la longueur du \og morceau \no 1 \fg. 
\end{itemize}
\setlength\parindent{0mm}

\textbf{Graphique représentant les aires des polygones en fonction de la longueur du \og morceau \no 1 \fg }

\begin{center}
\psset{unit=0.5cm}
\begin{pspicture}(-1.5,-2)(21,26)
\multido{\n=0+2}{11}{\psline[linewidth=0.3pt,linecolor=cyan](\n,0)(\n,26)}
\multido{\n=0+2}{14}{\psline[linewidth=0.3pt,linecolor=cyan](0,\n)(21,\n)}
\psaxes[linewidth=1.25pt,Dx=2,Dy=2](0,0)(0,0)(21,26)
\uput[d](15.6,-1){Longueur du \og morceau \no 1 \fg{} (en cm)}
\uput[r](0,25.5){Aire $\left(\text{en cm}^2\right)$}
\psplot[plotpoints=4000,linewidth=1.25pt,linecolor=blue]{0}{20}{x 4 div dup mul}
\psplot[plotpoints=4000,linewidth=1.25pt]{0}{20}{20 x sub 3 div dup mul 0.866025 mul 2 div}
\rput(18.2,16.5){\blue Courbe A}
\rput(3.7,16.5){Courbe B} 
\end{pspicture}
\end{center}
 
En utilisant ce graphique, répondre aux questions suivantes. Aucune justification n'est attendue. 
	\begin{enumerate}
		\item Quelle est la longueur du \og morceau \no 1 \fg{} qui permet d'obtenir un triangle équilatéral d'aire 14~cm$^2$ ? 
		\item Quelle est la longueur du \og morceau \no 1 \fg{} qui permet d'obtenir deux polygones d'aires égales ? 
	\end{enumerate}
\end{enumerate}

\bigskip

\end{document}