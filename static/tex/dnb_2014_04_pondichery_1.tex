\textbf{\textsc{Exercice 1 \hfill 6 points}}

\medskip
 
Emma et Arthur ont acheté pour leur mariage \np{3003} dragées au chocolat et \np{3731} dragées aux amandes.

\medskip
 
\begin{enumerate}
\item Arthur propose de répartir ces dragées de façon identique dans 20 corbeilles.
 
Chaque corbeille doit avoir la même composition.
 
Combien lui reste-t-il de dragées non utilisées ? 
\item Emma et Arthur changent d'avis et décident de proposer des petits ballotins* dont la composition est identique. Ils souhaitent qu'il ne leur reste pas de dragées. 
	\begin{enumerate}
		\item Emma propose d'en faire $90$. Ceci convient-il ? Justifier. 
		\item Ils se mettent d'accord pour faire un maximum de ballotins.
		
Combien en feront-ils et quelle sera leur composition ? 

 
	\end{enumerate}
\end{enumerate}

\emph{* Un ballotin est un emballage pour confiseries, une boîte par exemple.}

\vspace{0,5cm}

