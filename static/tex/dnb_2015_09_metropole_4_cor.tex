\textbf{Exercice 4 \hfill 4 points}

\medskip

%Trois amis se rendent chez un apiculteur pour réaliser quelques achats. 
%
%\medskip
%
%Le premier achète deux pots de miel et trois pains d'épices pour un montant de 24 euros.
%
%Le deuxième achète un pot de miel et deux pains d'épices pour un montant de 14,50 euros. 
%
%Le troisième achète trois pots de miel et un pain d'épices. 
%
%Combien va-t-il payer ? 
Soit $m$ le prix d'un pot de miel et $e$ le prix d'un pain d'épices.

Les deux achats se traduisent par :

\[\left\{\begin{array}{l c l}
2m + 3e&=&24\\
\phantom{2}m + 2e&=&14,50
\end{array}\right.\]

Par différence on obtient $m + e = 9,50$.

On a donc $\left\{\begin{array}{l c r}
m + 2e&=&14,50\\
m + \phantom{2}e &=& 9,50
\end{array}\right.$ qui donnent par différence $e = 5$ et par conséquent $m = 4,50$.

La troisième personne va donc payer $3 \times 4,50 + 5 = 18,50$~\euro.
\vspace{0.5cm}

