
\medskip

%Cet exercice est un questionnaire à choix multiples (QCM). Pour chaque question, une seule des trois réponses proposées est exacte. 
%
%Sur la copie, indiquer le numéro de la question et la réponse choisie. On
%ne demande pas de justifier. 
%
%Aucun point ne sera enlevé en cas de mauvaise réponse.
%
%\begin{center}
%\begin{tabularx}{\linewidth}{|c|m{5cm}|*{3}{>{\centering \arraybackslash}X|}}\hline
%\multicolumn{2}{c|}{Question}&\multicolumn{3}{c|}{Réponses proposées}\\\cline{3-5}
%1&\psset{unit=0.8cm}\begin{pspicture}(5,2)
%\psframe(0.5,0.3)(4.5,1.6)
%\psline(1.8,0.3)(1.8,1.6)
%\uput[ul](0.5,1.6){A}\uput[ur](4.5,1.6){B}
%\uput[dl](0.5,0.3){D}\uput[dr](4.5,0.3){C}
%\rput{30}(0.3,0.95){\psline(0,0)(0.4,0)}\rput{30}(1.6,0.95){\psline(0,0)(0.4,0)}
%\rput{30}(4.3,0.95){\psline(0,0)(0.4,0)}\rput{30}(0.95,1.5){\psline(0,0)(0.4,0)}
%\rput{30}(0.95,0.2){\psline(0,0)(0.4,0)}
%\rput(0.3,0.8){$x$}
%\psline{<-}(1.8,1.8)(3,1.8)\psline{->}(3.3,1.8)(4.5,1.8)\rput(3.15,1.8){2}
%\end{pspicture}
%
%{\footnotesize Quelle est l'aire du rectangle ABCD ?}&$x + 2$& $x^2 + 2x$& $ 4x + 4$\\ \hline
%2 &{\footnotesize Alexandra achète 2 cahiers et 3 crayons, elle paie 810 F
%
%Nathalie achète 1 cahier et 5 crayons elle paie 650 F
% 
%Combien coûte  un cahier et combien coûte un crayon ?}& un cahier  coûte 250 F
%  
%un crayon coûte 100 F &un cahier  coûte 250 F
%  
%un crayon coûte 110 F &un cahier  coûte 300 F
%  
%un crayon coûte 70 F \\ \hline 
%3&\psset{unit=1cm}
%\begin{pspicture}(5,2.8)
%%\psgrid
%\pspolygon(0.5,2.1)(1,1.9)(0.9,1.4)(1.3,1)(1.5,0.8)(1.8,1.5)(2,2.1)(3,2.5)(4.4,1.9)(4.5,0.8)(4,0.75)(3.8,1.5)(3.2,1.8)(2.4,1.7)(2.2,0.8)(1.6,0.2)(0.9,0.5)(0.4,1.2)
%%\psdots(0.5,2.1)(1,1.9)(0.9,1.4)(1.3,1)(1.5,0.8)(1.8,1.5)(2,2.1)(3,2.5)(4.4,1.9)(4.5,0.8)(4,0.75)(3.8,1.5)(3.2,1.8)(2.4,1.7)(2.2,0.8)(1.6,0.2)(0.9,0.5)(0.4,1.2)
%\psline(0.9,1.4)(0.4,1.2)\psline(1.3,1)(0.9,0.5)\psline(1.5,0.8)(1.6,0.2)
%\psline(1.8,1.5)(2.2,0.8)\psline(2,2.1)(2.4,1.7)\psline(3,2.5)(3.2,1.8)
%\psline(4.4,1.9)(3.8,1.5)\psline(4.5,0.8)(4,0.75)\psline(4,0.75)(4.5,0.8)
%\rput(0.7,1.8){\Large$\bullet$}\rput(0.68,1.6){\Large $\bullet$}
%\psline[linewidth=2.5pt]{->}(1.1,2.6)(0.8,2)\psline[linewidth=2.5pt]{->}(4.3,0.77)(4.32,0.2)
%\end{pspicture}
%
%{\footnotesize À l'entrée du chemin, sur la première case, sont placés deux cailloux noirs.
%
%Le but du jeu est de sortir du chemin en passant par toutes les cases.
%
%\textbf{Attention} : pour pouvoir se déplacer sur
%la case suivante il faut pouvoir déposer un nombre de cailloux égal au double du
%nombre de cailloux sur la case précédente.
%
%Combien de cailloux doit-on placer sur la dernière case ?}&64 cailloux&128 cailloux&256 cailloux
%\\ \hline
%4 &$\dfrac{5}{14} + \dfrac{3}{7} \times \dfrac{5}{2} = $ ?\rule[-3mm]{0mm}{9mm}&$\dfrac{40}{42}$&$\dfrac{20}{28}$&
%$\dfrac{20}{14}$ \\ \hline 
%5 &{\footnotesize Voici un schéma du garage qu'Eli veut construire sur son terrain (l'unité est le
%mètre) :} 
%
%\psset{unit=0.6cm}
%\begin{pspicture}(-0.2,0)(7,4.5)
%%\psgrid
%\psline(0,0.8)(6,0.8)(6,3.9)
%\psline[linewidth=2.pt](6,3.9)(2.3,2)(2.3,0.8)
%\psline[linestyle=dotted](2.3,2)(0,0.8)
%\psframe[fillstyle=vlines](6,0.8)(6.8,3.9)
%\rput(0.6,3){\footnotesize Poteau}\psline{->}(0.6,2.8)(2.3,1.4)
%\psline{<-}(0,0.5)(0.9,0.5)\psline{->}(1.4,0.5)(2.3,0.5)\rput(1.15,0.5){\footnotesize 3}
%\psline{<-}(2.3,0.5)(3.8,0.5)\psline{->}(4.5,0.5)(6,0.5)\rput(4.15,0.5){\footnotesize 4,5}
%\uput[dl](0,0.5){A} \uput[d](2.3,0.5){L}\uput[dr](6,0.5){C}
%\uput[u](6,3.9){B}\uput[ul](2.3,2){M}
%\psline{<-}(7.2,0.8)(7.2,2.)\psline{->}(7.2,2.4)(7.2,3.9)\rput(7.2,2.2){\footnotesize 3}
%\end{pspicture}
%
%{\footnotesize \emph{Données} : M $\in$ (AB); L $\in$ (AC) ;} 
%
%{\footnotesize (ML) // (BC)}
%
%{\footnotesize Quelle est la hauteur du poteau ?}&1,5 mètre &1,2 mètre &On ne peut pas savoir.\\ \hline
%\end{tabularx}
%\end{center}
\begin{enumerate}
\item La longueur est $x + 2$ et la largeur $x$ ; l'aire est donc égale à $x(2 + x) = 2x + x^2$.
\item Soit $x$ le prix d'un cahier et $y$ le pris d'un crayon. On a donc :

$\left\{\begin{array}{l c l}
2x + 3y&=&810\\
x + 5y &=& 650
\end{array}\right.$ Donc $x = 650 - 5y$ et en reportant dans la première équation :

$2(650 - 5y) + 3y = 810 $ soit $\np{1300}  - 10y + 3y = 810$ ou  $490 = 7y$ et enfin $70 = y$.

D'où $x = 650 - 5y = 650 - 5 \times 70 = 650 - 350 = 300$.

Un cahier coûte 300~F et un crayon 70~F.
\item Il faut déposer successivement : 2,\: 4,\: 8,\:16,\:32,\:64,\:128,\:256 cailloux.
\item $\dfrac{5}{14} + \dfrac{3}{7} \times \dfrac{5}{2} = \dfrac{5}{14} + \dfrac{15}{14} = \dfrac{5 + 15}{14} + \dfrac{20}{14}  = \dfrac{10 \times 2}{7 \times 2}  = \dfrac{10}{7}$ !
\item Dans le triangle rectangle ABC, les droites (ML) et (BC) sont parallèles. On peut donc écrire d'après Thalès : $\dfrac{\text{AL}}{\text{AC}} = \dfrac{\text{ML}}{\text{BC}}$ soit $\dfrac{3}{3 + 4,5} = \dfrac{\text{ML}}{3}$, d'où $\text{ML} = 3 \times \dfrac{3}{7,5} = \dfrac{9}{7,5} = \dfrac{3}{2,5} = \dfrac{6}{5} = \dfrac{12}{10} = 1,2$~(m). 
\end{enumerate}

\vspace{0,5cm}

