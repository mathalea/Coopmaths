
\medskip

\begin{enumerate}
\item 
	\begin{enumerate}
		\item Augmenter de 10\,\%, revient à multiplier par $1 + \dfrac{10}{100} = 1,1$.
		
Le nombre total d'adhérents le 31 décembre 2012 était donc : $\np{1000} \times 1,1 = \np{1100}$.		
		\item De même augmenter de 5\,\%, c'est multiplier par 1,05 ; le nombre total d'adhérents le 31 décembre 2015 était donc : $\np{1100} \times 1,05 = \np{1155}$.
		\item On ne peut ajouter des pourcentages.
		
		Du 1\up{er} janvier 2010 au 31 décembre 2015, l'augmentation a été de $\np{1155} - \np{1000} = 155$ adhérents, soit un pourcentage de $\dfrac{155}{\np{1000}}\times 100 = 15,5$.
	\end{enumerate}
\item
	\begin{enumerate}
		\item \np{1260} adhérents sont représentés pas 360\degres, donc 1 adhérent par $\dfrac{360}{1260} = \dfrac{40}{140} = \dfrac{4}{14} = \dfrac{2}{7}$.
		
Donc 392 adhérents sont représentés pas : $392 \times \dfrac{2}{7} = \dfrac{7 \times 56 \times 2}{7} = 112$(\degres) ; la fréquence est égale à : $\dfrac{392}{1260}\times 100 \approx 31,11$ ;

224 adhérents sont représentés pas : $224 \times \dfrac{2}{7} = \dfrac{7 \times 32 \times 2}{7} = 64$(\degres) ; la fréquence est égale à : $\dfrac{224}{1260}\times 100 \approx 17,78$

644 adhérents sont représentés pas : $644 \times \dfrac{2}{7} = \dfrac{7 \times 92 \times 2}{7} = 184$(\degres) ; la fréquence est égale à : $\dfrac{392}{1260}\times 100 \approx 51,11$.
		\item Diagramme circulaire :
\begin{center}
\psset{unit=1cm}
\begin{pspicture}(-2,-2)(2,2)
\pscircle(0,0){2}
\psline(2;0)\psline(2;112) \psline(2;176)
\rput(1.3;56){Planche à }
\rput(0.7;56){voile}
\rput(1.3;144){Beach }
\rput(0.75;150){volley}
\rput(1;268){Surf}
\end{pspicture}
\end{center}
		\item Voir ci-dessus les calculs.
	\end{enumerate} 
\end{enumerate}

\bigskip

