\documentclass[10pt]{article}
\usepackage[T1]{fontenc}
\usepackage[utf8]{inputenc}%ATTENTION codage UTF8
\usepackage{fourier}
\usepackage[scaled=0.875]{helvet}
\renewcommand{\ttdefault}{lmtt}
\usepackage{amsmath,amssymb,makeidx}
\usepackage[normalem]{ulem}
\usepackage{diagbox}
\usepackage{fancybox}
\usepackage{tabularx,booktabs}
\usepackage{colortbl}
\usepackage{pifont}
\usepackage{multirow}
\usepackage{dcolumn}
\usepackage{enumitem}
\usepackage{textcomp}
\usepackage{lscape}
\newcommand{\euro}{\eurologo{}}
\usepackage{graphics,graphicx}
\usepackage{pstricks,pst-plot,pst-tree,pstricks-add}
\usepackage[left=3.5cm, right=3.5cm, top=3cm, bottom=3cm]{geometry}
\newcommand{\R}{\mathbb{R}}
\newcommand{\N}{\mathbb{N}}
\newcommand{\D}{\mathbb{D}}
\newcommand{\Z}{\mathbb{Z}}
\newcommand{\Q}{\mathbb{Q}}
\newcommand{\C}{\mathbb{C}}
\usepackage{scratch}
\renewcommand{\theenumi}{\textbf{\arabic{enumi}}}
\renewcommand{\labelenumi}{\textbf{\theenumi.}}
\renewcommand{\theenumii}{\textbf{\alph{enumii}}}
\renewcommand{\labelenumii}{\textbf{\theenumii.}}
\newcommand{\vect}[1]{\overrightarrow{\,\mathstrut#1\,}}
\def\Oij{$\left(\text{O}~;~\vect{\imath},~\vect{\jmath}\right)$}
\def\Oijk{$\left(\text{O}~;~\vect{\imath},~\vect{\jmath},~\vect{k}\right)$}
\def\Ouv{$\left(\text{O}~;~\vect{u},~\vect{v}\right)$}
\usepackage{fancyhdr}
\usepackage[french]{babel}
\usepackage[dvips]{hyperref}
\usepackage[np]{numprint}
%Tapuscrit : Denis Vergès
%\frenchbsetup{StandardLists=true}

\begin{document}
\setlength\parindent{0mm}
% \rhead{\textbf{A. P{}. M. E. P{}.}}
% \lhead{\small Brevet des collèges}
% \lfoot{\small{Polynésie}}
% \rfoot{\small{7 septembre 2020}}
\pagestyle{fancy}
\thispagestyle{empty}
% \begin{center}
    
% {\Large \textbf{\decofourleft~Brevet des collèges Polynésie 7 septembre 2020~\decofourright}}
    
% \bigskip
    
% \textbf{Durée : 2 heures} \end{center}

% \bigskip

% \textbf{\begin{tabularx}{\linewidth}{|X|}\hline
%  L'évaluation prend en compte la clarté et la précision des raisonnements ainsi que, plus largement, la qualité de la rédaction. Elle prend en compte les essais et les démarches engagées même non abouties. Toutes les réponses doivent être justifiées, sauf mention contraire.\\ \hline
% \end{tabularx}}

% \vspace{0.5cm}\textbf{\textsc{Exercice 5} \hfill 4 points}

\medskip

%\begin{tabularx}{\linewidth}{|c|c|X|}\hline
%Année &SMIC&\multirow{12}{9cm}{On considère la série statistique donnant le SMIC\\
%\textbf{1.} Quelle est l'étendue de cette série ? Interpréter ce résultat.
%\textbf{2.} Quelle est la médiane ?\\
%\textbf{3.} Paul remarque qu'entre 2001 et 2002, l'augmentation du SMIC horaire brut est de 16 centimes alors qu'entre 2007 et 2008, elle est de 19 centimes.\\
%Il affirme que \og le pourcentage d'augmentation entre  2007 et 2008 est supérieur à celui pratiqué entre 2001 et 2002 \fg.\\
%A-t-il raison ?}\\ \cline{1-2} 	 
%2011& 	9,40& \\ \cline{1-2}	 
%2010& 	9,00& \\ \cline{1-2}	 
%2009& 	8,82& \\ \cline{1-2}	 
%2008& 	8,63& \\ \cline{1-2}	 
%2007& 	8,44& \\ \cline{1-2}	 
%2006& 	8,27& \\ \cline{1-2}	 
%2005& 	8,03& \\ \cline{1-2}	 
%2004& 	7,61& \\ \cline{1-2}	 
%2003& 	7,19& \\ \cline{1-2}
%2002& 	6,83& \\ \cline{1-2}	 
%2001& 	6,67&\\ \hline
%\multicolumn{3}{r}{SMIC : salaire minimum interprofessionnel de croissance horaire brut en euros }\\
%\multicolumn{3}{r}{de 2001 à 2011 (source : INSEE)}
%\end{tabularx}
\begin{enumerate}
\item Étendue : $9,40 - 6,67 = 2,73$.
\item La médiane est 8,27.
\item Augmentation de 2001 à 2002 : $\dfrac{6,83 - 6,67}{6,67} \times 100 = \dfrac{16}{6,67} \approx 2,4\,\%$.

Augmentation de 2007 à 2008 : $\dfrac{8,63 - 8,44}{8,44} \times 100 = \dfrac{19}{8,44} \approx 2,3\,\%$. Donc Paul a tort.
\end{enumerate}

\medskip

\end{document}