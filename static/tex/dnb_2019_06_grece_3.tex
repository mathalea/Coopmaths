
\medskip

Marc et Jim, deux amateurs de course à pied, s'entraînent sur une piste d'athlétisme dont la longueur du tour mesure $400$~m. 

\smallskip

Marc fait un temps moyen de $2$ minutes par tour. 

Marc commence son entrainement par un échauffement d'une longueur d'un kilomètre. 

\medskip

\begin{enumerate}
\item Combien de temps durera l'échauffement de Marc? 
\item Quelle est la vitesse moyenne de course de Marc en km/h ? 
\end{enumerate}

À la fin de l'échauffement, Marc et Jim décident de commencer leur course au même point de départ A et vont effectuer un certain nombre de tours. 

Jim a un temps moyen de 1 minute et 40 secondes par tour. 

Le schéma ci-dessous représente la piste d'athlétisme de Marc et Jim constituée de deux segments [AB] et [CD] et de deux demi-cercles de diamètre [AD] et [BC]. 

(\emph{Le schéma n'est pas à l'échelle et les longueurs indiquées sont arrondies à l'unité.}) 

\medskip

\begin{center}
\psset{unit=0.75cm}
\begin{pspicture}(18,6)
\psline(3,0.5)(9.7,0.5)\psline(3,5.7)(9.7,5.7)
\psline[linestyle=dotted](3,0.5)(3,5.7)
\psline[linestyle=dotted](9.7,0.5)(9.7,5.7)
\psarc(3,3.1){2.6}{90}{270}
\psarc(9.7,3.1){2.6}{-90}{90}
\uput[u](3,5.7){A} \uput[u](9.7,5.7){B} \uput[d](9.7,0.5){C} \uput[d](3,0.5){D} 
\psframe(3,5.7)(3.2,5.5)\psframe(9.7,5.7)(9.5,5.5)\psframe(9.7,0.5)(9.5,0.7)
\psframe(3,0.5)(3.2,0.7)
\psline[linestyle=dashed]{<->}(3,5.9)(9.7,5.9)\uput[u](6.35,5.9){90 m}
\psline[linestyle=dashed]{<->}(3.4,5.7)(3.4,0.5)\uput[r](3.4,3.1){70 m}
\rput(15.5,3){ABCD est un rectangle}
\rput(15.5,2.2){AB = 90 m et AD = 70 m}
\end{pspicture}
\end{center}

\begin{enumerate}[resume]
\item Calculer le temps qu'il faudra pour qu'ils se retrouvent ensemble, au même moment, et pour la première fois au point A. 

Puis déterminer combien de tours de piste cela représentera pour chacun d'entre eux. 

\smallskip

\emph{Toute trace de recherche, même non aboutie, devra apparaître sur la copie. Elle sera prise en compte dans l'évaluation.} 
\end{enumerate}

\vspace{0.5cm}

