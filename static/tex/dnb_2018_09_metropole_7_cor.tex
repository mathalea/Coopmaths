
\medskip

%\parbox{0.45\linewidth}{Pour la course à pied en montagne, certains sportifs mesurent leur
%performance par la \textbf{vitesse ascensionnelle}, notée $V_a$.
%
%$V_a$ est le quotient du dénivelé de la course, exprimé en mètres, par la durée, exprimée en heure.} \hfill
%\parbox{0.52\linewidth}{\psset{unit=0.75cm}
%\begin{pspicture}(10,5.5)
%\pspolygon(0.5,0.5)(6.7,0.5)(6.7,4.6)%ABC
%\psframe(6.7,0.5)(6.2,1)
%\psline{<->}(6.9,0.5)(6.9,4.6)
%\uput[d](0.5,0.5){A}\uput[d](6.7,0.5){B}\uput[ur](6.5,4.6){C}
%\rput(2.8,0.9){angle de pente}
%\uput[r](6.9,2.9){\footnotesize dénivelé}
%\uput[r](6.9,0.5){\footnotesize altitude de départ}
%\uput[r](6.9,4.6){\footnotesize altitude d'arrivée}
%\psarc(0.5,0.5){0.5}{0}{30}
%\rput(5,0){Figure 1}
%\end{pspicture}}
%
%\medskip
%
%Par exemple: pour un dénivelé de \np{4500} m et une durée de parcours de 3 h : $V_a = \np{1500}$ m/h.
%
%Rappel: le dénivelé de la course est la différence entre l'altitude à l'arrivée et l'altitude au départ.
%
%\medskip
%
%Un coureur de haut niveau souhaite atteindre une vitesse ascensionnelle d'au moins \np{1400} m/h lors de sa prochaine course.

\begin{center}
\emph{La figure ci-dessous n'est pas représentée en vraie grandeur.}

\psset{unit=0.6cm}
\begin{pspicture}(20.5,6)
%\psgrid
\psline(0,1.3)(20,1.3)
\psline(8.8,1.3)(8.8,2.25)(15.6,2.25)(15.6,5.2)%EFGH
\uput[dl](1.9,1.3){D}\uput[ur](15.6,5.2){H}
\uput[d](15.6,2.25){G}\uput[ul](8.9,2.45){F}\uput[dr](8.9,1.3){E}
\psline{<->}(1.9,1)(8.8,1)
\psline(1.9,1.3)(8.8,2.25)(15.6,5.2)%DFH
\psframe(8.8,1.3)(8.3,1.8)
\psframe(15.6,2.25)(15.1,2.75)
\uput[r](16.1,5.2){\small Sommet}
\psline{<->}(16.1,1.3)(16.1,2.25)
\psline{<->}(16.1,5.2)(16.1,2.25)
\psline{<->}(2,1.5)(8.7,2.4)
\psline{<->}(8.7,2.5)(15.6,5.5)
\uput[r](16.1,3.7){\small dénivelé seconde étape}
\uput[u](1.1,1.3){\small Départ}
\uput[r](16.1,1.8){\small dénivelé première étape}
\uput[d](5.35,1.2){\small \np{3790} m}
\uput[u](5,2.1){\small \np{3800} m}
\uput[u](11.4,4.2){\small \np{4100} m}
\rput(10.25,0.1){Figure 2}
\psarc(1.9,1.3){1.3}{0}{10}
\psarc(8.8,2.25){1.3}{0}{25}
\rput(10.5,2.6){\small 12\degres}
\end{pspicture}
\end{center}

%Le parcours se décompose en deux étapes (voir figure 2) :
%
%\medskip
%
%\setlength\parindent{12mm}
%\begin{itemize}
%\item[$\bullet~~$]Première étape de \np{3800} m pour un déplacement horizontal de \np{3790} m.
%\item[$\bullet~~$]Seconde étape de $4,1$ km avec un angle de pente d'environ $12$\degres.
%\end{itemize}
%\setlength\parindent{0mm}

\bigskip

\begin{enumerate}
\item %Vérifier que le dénivelé de la première étape est environ $275,5$ m.
Le triangle DEF étant rectangle en E, le théorème de Pythagore permet d'écrire :

$\text{DF}^2 = \text{DE}^2 + \text{EF}^2$ ou $\text{EF}^2 = \text{DF}^2 - \text{DE}^2 = \np{3800}^2 - \np{3790}^2 = \np{14440000} - \np{14364100} = \np{75900}$, d'où $\text{EF} = \sqrt{\np{75900}} \approx 275,499$ soit 275,5~(m) au dixième près.
\item %Quel est le dénivelé de la seconde étape ?
Dans le triangle DEF  rectangle en E, on a $\sin \widehat{\text{GFH}} = \dfrac{\text{GH}}{\text{FH}}$, d'où :

$\text{GH} = \text{FH} \times \sin \widehat{\text{GFH}} = \np{4100}\times \sin 12\degres \approx 852,4$environ.
\item %Depuis le départ, le coureur met $48$ minutes pour arriver au sommet.

Le dénivelé total est donc : $275,5 + \np{852,4} = \np{1127,9}$ pour un temps de $\dfrac{48}{60} =\dfrac{8}{10} = \dfrac{4}{5} = 0,8$.

La vitesse ascensionnelle est donc égale à : 

$\dfrac{\np{1127,9}}{0,8} \approx  \np{1409,9} > \np{1400}$~(m/h) : le coureur a atteint son objectif.

%Le coureur atteint-il son objectif ?
\end{enumerate}
