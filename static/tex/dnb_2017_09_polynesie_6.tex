
\medskip

La figure ci-après est la copie d'écran d'un programme réalisé avec le logiciel
\og Scratch \fg.

\medskip

\parbox{0.48\linewidth}{\begin{enumerate}
\item Montrer que si on choisit 2 comme
nombre de départ, alors le programme renvoie $- 5$.
\item Que renvoie le programme si on
choisit au départ :
	\begin{enumerate}
		\item le nombre $5$ ?
		\item le nombre $- 4$ ?
	\end{enumerate}
\item  Déterminer les nombres qu'il faut choisir au départ pour que le
programme renvoie $0$.
\end{enumerate}}\quad\hfill
\parbox{0.48\linewidth}{\small{%
\begin{scratch}
\blockinit{quand \greenflag est cliqué}
    \blockvariable{cacher la variable \selectmenu{x}}
    \blockvariable{cacher la variable \selectmenu{y}}
    \blocksensing{demander \txtbox{Choisis un nombre} et attendre}
    \blockvariable{mettre \selectmenu{x} à \ovalsensing{réponse}}
    \blockvariable{mettre \selectmenu{y} à \ovaloperator{\ovaloperator{\ovalvariable{x} * \ovalvariable{x}} - \ovalnum{9}}}
    \blocklook{dire \txtbox{En choisissant} pendant \ovalnum{1} seconde}
    \blocklook{dire \ovalsensing{réponse} pendant \ovalnum{1} seconde}
    \blocklook{dire \txtbox{On obtient} pendant \ovalnum{1} seconde}
    \blocklook{dire \ovalvariable{y}}
\end{scratch}
}
}	



