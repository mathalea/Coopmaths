\textbf{\textsc{Exercice 5 \hfill 5 points}}

\medskip

Lors d'une course en moto-cross, après avoir franchi une rampe, Gaëtan a effectué un saut record en moto.

\medskip

\parbox{0.45\linewidth}{Le saut commence dès que Gaëtan quitte
la rampe. 

On note $t$ la durée (en secondes) de ce saut.

La hauteur (en mètres) est déterminée en fonction de la durée $t$ par la fonction $h$ suivante : 

$h\: : t \longmapsto  (-5t - 1,35)(t - 3,7)$.}\hfill
\parbox{0.51\linewidth}{\psset{unit=1cm}
\begin{pspicture}(-0.5,-1.2)(4,4)
\psline(-0.5,0)(4,0)
\psplot[plotpoints=2000,linewidth=1.75pt,linecolor=blue]{0}{3}{x dup mul 0.125  mul}
\psplot[plotpoints=2000,linewidth=1.25pt]{3}{4}{x 1.2 mul 1.575  sub x dup mul 0.1 mul sub}
\rput(3,-1){distance horizontale $d$}
\rput(5,1){hauteur $h$}
\rput(4,3){durée $t = 0$ s}
\rput(1.8,1.2){\blue rampe}
\psset{arrowsize=3pt 2}
\psline{->}(3,3)(3,1.125)\psline[linestyle=dotted](3,1.12)(3,0)
\psline[linestyle=dotted](4,1.65)(4,0)
\psline{<->}(4.2,0)(4.2,1.65)\psline{<->}(3,-0.5)(4,-0.5)
\end{pspicture}}

Voici la courbe représentative de cette fonction $h$.

\begin{center}
\psset{xunit=2cm,yunit=0.3cm,comma=true}
\begin{pspicture}(-.5,-2)(4.5,22.5)
\multido{\n=0.0+0.5}{9}{\psline[linewidth=0.3pt](\n,0)(\n,22.5)}
\multido{\n=0+5}{5}{\psline[linewidth=0.3pt](0,\n)(4.5,\n)}
\psaxes[linewidth=1.5pt,Dx=0.5,Dy=5](0,0)(-.4,-2)(4.4,22.5)
\psplot[plotpoints=2000,linewidth=1.25pt,linecolor=blue]{0}{3.7}{x 3.7 sub 5 x mul 1.35 add mul neg}
\uput[dl](0,0){O}
\end{pspicture}
\end{center}

Les affirmations suivantes sont-elles vraies ou fausses ? Justifier en utilisant soit le graphique soit des calculs.

\medskip

\begin{enumerate}
\item En développant et en réduisant l'expression de $h$ on obtient 

$h(t) = - 5t^2 - 19,85t -  4,995$.
\item Lorsqu'il quitte la rampe, Gaëtan est à 3,8~m de hauteur.
\item Le saut de Gaëtan dure moins de 4 secondes.
\item Le nombre 3,5 est un antécédent du nombre 3,77 par la fonction $h$.
\item Gaetan a obtenu la hauteur maximale avant 1,5~seconde.
\end{enumerate}

\vspace{0,5cm}

