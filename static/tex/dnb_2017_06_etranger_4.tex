\documentclass[10pt]{article}
\usepackage[T1]{fontenc}
\usepackage[utf8]{inputenc}%ATTENTION codage UTF8
\usepackage{fourier}
\usepackage[scaled=0.875]{helvet}
\renewcommand{\ttdefault}{lmtt}
\usepackage{amsmath,amssymb,makeidx}
\usepackage[normalem]{ulem}
\usepackage{diagbox}
\usepackage{fancybox}
\usepackage{tabularx,booktabs}
\usepackage{colortbl}
\usepackage{pifont}
\usepackage{multirow}
\usepackage{dcolumn}
\usepackage{enumitem}
\usepackage{textcomp}
\usepackage{lscape}
\newcommand{\euro}{\eurologo{}}
\usepackage{graphics,graphicx}
\usepackage{pstricks,pst-plot,pst-tree,pstricks-add}
\usepackage[left=3.5cm, right=3.5cm, top=3cm, bottom=3cm]{geometry}
\newcommand{\R}{\mathbb{R}}
\newcommand{\N}{\mathbb{N}}
\newcommand{\D}{\mathbb{D}}
\newcommand{\Z}{\mathbb{Z}}
\newcommand{\Q}{\mathbb{Q}}
\newcommand{\C}{\mathbb{C}}
\usepackage{scratch}
\renewcommand{\theenumi}{\textbf{\arabic{enumi}}}
\renewcommand{\labelenumi}{\textbf{\theenumi.}}
\renewcommand{\theenumii}{\textbf{\alph{enumii}}}
\renewcommand{\labelenumii}{\textbf{\theenumii.}}
\newcommand{\vect}[1]{\overrightarrow{\,\mathstrut#1\,}}
\def\Oij{$\left(\text{O}~;~\vect{\imath},~\vect{\jmath}\right)$}
\def\Oijk{$\left(\text{O}~;~\vect{\imath},~\vect{\jmath},~\vect{k}\right)$}
\def\Ouv{$\left(\text{O}~;~\vect{u},~\vect{v}\right)$}
\usepackage{fancyhdr}
\usepackage[french]{babel}
\usepackage[dvips]{hyperref}
\usepackage[np]{numprint}
%Tapuscrit : Denis Vergès
%\frenchbsetup{StandardLists=true}

\begin{document}
\setlength\parindent{0mm}
% \rhead{\textbf{A. P{}. M. E. P{}.}}
% \lhead{\small Brevet des collèges}
% \lfoot{\small{Polynésie}}
% \rfoot{\small{7 septembre 2020}}
\pagestyle{fancy}
\thispagestyle{empty}
% \begin{center}
    
% {\Large \textbf{\decofourleft~Brevet des collèges Polynésie 7 septembre 2020~\decofourright}}
    
% \bigskip
    
% \textbf{Durée : 2 heures} \end{center}

% \bigskip

% \textbf{\begin{tabularx}{\linewidth}{|X|}\hline
%  L'évaluation prend en compte la clarté et la précision des raisonnements ainsi que, plus largement, la qualité de la rédaction. Elle prend en compte les essais et les démarches engagées même non abouties. Toutes les réponses doivent être justifiées, sauf mention contraire.\\ \hline
% \end{tabularx}}

% \vspace{0.5cm} \textbf{\textsc{Exercice 4} \hfill 4 points}

\medskip 



 Un fabricant de volets roulants électriques réalise une étude statistique pour connaître leur fiabilité. Il fait donc fonctionner un échantillon de 500 volets sans s'arrêter, jusqu'à une panne éventuelle. Il inscrit les résultats dans le tableur ci-dessous :


\begin{center}
{\tiny
\begin{tabularx}{\linewidth}{|c|c|*{6}{>{\centering \arraybackslash}X|}c|}\hline
\multicolumn{2}{|c|}{H2}&\cellcolor{lightgray}{\hspace{4mm}$f_x\quad \Sigma =$}&\multicolumn{6}{|c|}{$\quad$}\\\hline
\cellcolor{lightgray}{\quad}&\cellcolor{lightgray}{A}&\cellcolor{lightgray}{B}&\cellcolor{lightgray}{C}&\cellcolor{lightgray}{D}&\cellcolor{lightgray}{E}&\cellcolor{lightgray}{F}&\cellcolor{lightgray}{G}&\cellcolor{gray}{{\white H}}\\ \hline
\cellcolor{lightgray}{1}&\begin{tabular}{c}Nombre de \\montée-descente\end{tabular}&\begin{tabular}{c}Entre\\ 0 et 999\end{tabular}&\begin{tabular}{c}Entre\\ 1000 et 1999\end{tabular} &\begin{tabular}{c}Entre \\2000 et 2999\end{tabular}&\begin{tabular}{c}Entre\\ 3000 et 3999\end{tabular}&\begin{tabular}{c}Entre\\ 4000 et 4999\end{tabular}&Plus de 5000&TOTAL\\ \hline
\cellcolor{gray}{{\white 2}}&\begin{tabular}{c}Nombre de volets roulants\\ tombés en panne\end{tabular}& 20&54&137&186&84&19&\\ \hline
\cellcolor{lightgray}{3}&&&&&&&&
\end{tabularx}
}
\end{center}


\begin{enumerate}
\item  Quelle formule faut-il saisir dans la cellule H2 du tableur pour obtenir le nombre total de volets testés ? 

\item  Un employé prend au hasard un volet dans cet échantillon. Quelle est la probabilité que ce volet fonctionne plus de 3000 montées descentes? 

\item  Le fabricant juge ses volets fiables si plus de 95 \% des volets fonctionnent plus de 1000 montées descentes. Ce lot de volets roulants est-il fiable? Expliquer votre raisonnement. 
\end{enumerate}

\bigskip

\end{document}