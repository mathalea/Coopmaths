\documentclass[10pt]{article}
\usepackage[T1]{fontenc}
\usepackage[utf8]{inputenc}%ATTENTION codage UTF8
\usepackage{fourier}
\usepackage[scaled=0.875]{helvet}
\renewcommand{\ttdefault}{lmtt}
\usepackage{amsmath,amssymb,makeidx}
\usepackage[normalem]{ulem}
\usepackage{diagbox}
\usepackage{fancybox}
\usepackage{tabularx,booktabs}
\usepackage{colortbl}
\usepackage{pifont}
\usepackage{multirow}
\usepackage{dcolumn}
\usepackage{enumitem}
\usepackage{textcomp}
\usepackage{lscape}
\newcommand{\euro}{\eurologo{}}
\usepackage{graphics,graphicx}
\usepackage{pstricks,pst-plot,pst-tree,pstricks-add}
\usepackage[left=3.5cm, right=3.5cm, top=3cm, bottom=3cm]{geometry}
\newcommand{\R}{\mathbb{R}}
\newcommand{\N}{\mathbb{N}}
\newcommand{\D}{\mathbb{D}}
\newcommand{\Z}{\mathbb{Z}}
\newcommand{\Q}{\mathbb{Q}}
\newcommand{\C}{\mathbb{C}}
\usepackage{scratch}
\renewcommand{\theenumi}{\textbf{\arabic{enumi}}}
\renewcommand{\labelenumi}{\textbf{\theenumi.}}
\renewcommand{\theenumii}{\textbf{\alph{enumii}}}
\renewcommand{\labelenumii}{\textbf{\theenumii.}}
\newcommand{\vect}[1]{\overrightarrow{\,\mathstrut#1\,}}
\def\Oij{$\left(\text{O}~;~\vect{\imath},~\vect{\jmath}\right)$}
\def\Oijk{$\left(\text{O}~;~\vect{\imath},~\vect{\jmath},~\vect{k}\right)$}
\def\Ouv{$\left(\text{O}~;~\vect{u},~\vect{v}\right)$}
\usepackage{fancyhdr}
\usepackage[french]{babel}
\usepackage[dvips]{hyperref}
\usepackage[np]{numprint}
%Tapuscrit : Denis Vergès
%\frenchbsetup{StandardLists=true}

\begin{document}
\setlength\parindent{0mm}
% \rhead{\textbf{A. P{}. M. E. P{}.}}
% \lhead{\small Brevet des collèges}
% \lfoot{\small{Polynésie}}
% \rfoot{\small{7 septembre 2020}}
\pagestyle{fancy}
\thispagestyle{empty}
% \begin{center}
    
% {\Large \textbf{\decofourleft~Brevet des collèges Polynésie 7 septembre 2020~\decofourright}}
    
% \bigskip
    
% \textbf{Durée : 2 heures} \end{center}

% \bigskip

% \textbf{\begin{tabularx}{\linewidth}{|X|}\hline
%  L'évaluation prend en compte la clarté et la précision des raisonnements ainsi que, plus largement, la qualité de la rédaction. Elle prend en compte les essais et les démarches engagées même non abouties. Toutes les réponses doivent être justifiées, sauf mention contraire.\\ \hline
% \end{tabularx}}

% \vspace{0.5cm}\textbf{\textsc{Exercice 5} \hfill 8 points}

\medskip

Il existe différentes unités de mesure de la température. En France, on utilise le degré
Celsius (\degres C), aux États-Unis on utilise le degré Fahrenheit (\degres F). Voici deux représentations de cette correspondance :

\medskip

\parbox{0.3\linewidth}{\psset{unit=0.9cm}
\begin{pspicture}(3,9.5)
%\psgrid
\psframe[framearc=0.3](0.6,0.5)(2.6,9)
\multido{\n=1.3000+0.0823}{86}{\psline(1.3,\n)(1.5,\n)}
\multido{\n=1.3823+0.4118}{18}{\psline(1.2,\n)(1.5,\n)}
\multido{\n=1.3630+0.0908}{77}{\psline[linecolor=red](1.7,\n)(1.9,\n)}
\multido{\n=1.454+0.454}{16}{\psline[linecolor=red](1.7,\n)(2,\n)}
\multido{\n=1.3823+0.4118,\na=-35+5}{18}{\rput(0.8,\n){\footnotesize \np{\na}}}
\multido{\nb=1.491+0.454,\nc=-30+10}{16}{\rput(2.25,\nb){\footnotesize \np{\nc}}}
\pscircle*(1.6,0.9){3.5mm}
\psline(1.5,1.3)(1.5,8.5)(1.7,8.5)(1.7,1.3)
\rput(1,8.7){\degres C}\rput(2,8.7){\red \degres F}
\rput(1.6,0.2){Représentation 1}
\end{pspicture}} \hfill
\parbox{0.67\linewidth}{
\psset{unit=0.09cm}
\begin{pspicture}(-35,-45)(52.5,55)
\multido{\n=-35.0+2.5}{36}{\psline[linewidth=0.2pt](\n,-35)(\n,55)}
\multido{\n=-35.0+2.5}{37}{\psline[linewidth=0.2pt](-35,\n)(55,\n)}
\psaxes[linewidth=1.25pt,Dx=10,Dy=10,labelFontSize=\scriptstyle]{->}(0,0)(-35,-35)(52.5,52.5)
\psaxes[linewidth=1.25pt,Dx=10,Dy=10,labelFontSize=\scriptstyle](0,0)(-35,-35)(52.5,52.5)
\psplot[plotpoints=2000,linewidth=1.25pt,linecolor=blue]{-35}{11.5}{1.8 x mul 32 add}
\uput[u](40,0){\footnotesize Température en \degres C}
\uput[r](0,53){\footnotesize  Température en \degres F}
\rput(8.5,-42){Représentation 2}
\end{pspicture}}

\medskip

\begin{enumerate}
\item En vous appuyant sur les représentations précédentes, déterminer s'il y a
proportionnalité entre la température en degré Celsius et la température en degré
Fahrenheit. Justifier votre réponse.
\item  Soit $f$ la fonction qui à une température $x$ en degré Celsius associe la température
$f(x)$ en degré Fahrenheit correspondante. On propose trois expressions de $f(x)$ :

\begin{center}
\begin{tabularx}{0.8\linewidth}{|*{3}{>{\centering \arraybackslash}X|}}\hline
Proposition 1 &Proposition 2& Proposition 3\\ \hline
$f(x) = x+32$& $f(x) = 1,8x + 32$& $f(x) = 2x + 30$\\ \hline
\end{tabularx}
\end{center}

\og Les propositions 1 et 3 ne peuvent pas être correctes. C'est donc la proposition 2 qui
convient. \fg. Justifier cette affirmation.
\item  On considère la fonction $f$ définie par $f(x) = 1,8x + 32$.

Calculer $f(10)$ et $f(-40)$.
\item  Existe-t-il une valeur pour laquelle la température exprimée en degré Celsius est égale
à la température exprimée en degré Fahrenheit ? Justifier votre réponse.
\end{enumerate}

\vspace{0,5cm}

\end{document}