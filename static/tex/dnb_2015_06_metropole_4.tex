\textbf{Exercice 4 \hfill 7,5 points}

\medskip

\textbf{Toutes les questions sont indépendantes}

\medskip

\begin{enumerate}
\item On considère la fonction $f$ définie par $f(x) = - 6x + 7$.

Déterminer l'image de 3 par la fonction $f$.
\item  Arthur a le choix pour s'habiller aujourd'hui entre trois chemisettes (une verte, une bleue et une rouge) et deux shorts (un vert et un bleu). Il décide de s'habiller en choisissent au hasard une chemisette puis un short.

Quelle est la probabilité qu'Arthur soit habillé uniquement en vert ?
\item  Ariane affirme que $2^{40}$ est le double de $2^{39}$.  A-t-elle raison ?
\item  Loïc affirme que le PGCD d'un nombre pair et d'un nombre impair est toujours égal à 1.

A-t-il raison ?
\item  Résoudre l'équation :  $5x - 2 = 3x + 7$.
\end{enumerate}

\vspace{0,5cm}

