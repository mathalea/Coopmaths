\documentclass[10pt]{article}
\usepackage[T1]{fontenc}
\usepackage[utf8]{inputenc}%ATTENTION codage UTF8
\usepackage{fourier}
\usepackage[scaled=0.875]{helvet}
\renewcommand{\ttdefault}{lmtt}
\usepackage{amsmath,amssymb,makeidx}
\usepackage[normalem]{ulem}
\usepackage{diagbox}
\usepackage{fancybox}
\usepackage{tabularx,booktabs}
\usepackage{colortbl}
\usepackage{pifont}
\usepackage{multirow}
\usepackage{dcolumn}
\usepackage{enumitem}
\usepackage{textcomp}
\usepackage{lscape}
\newcommand{\euro}{\eurologo{}}
\usepackage{graphics,graphicx}
\usepackage{pstricks,pst-plot,pst-tree,pstricks-add}
\usepackage[left=3.5cm, right=3.5cm, top=3cm, bottom=3cm]{geometry}
\newcommand{\R}{\mathbb{R}}
\newcommand{\N}{\mathbb{N}}
\newcommand{\D}{\mathbb{D}}
\newcommand{\Z}{\mathbb{Z}}
\newcommand{\Q}{\mathbb{Q}}
\newcommand{\C}{\mathbb{C}}
\usepackage{scratch}
\renewcommand{\theenumi}{\textbf{\arabic{enumi}}}
\renewcommand{\labelenumi}{\textbf{\theenumi.}}
\renewcommand{\theenumii}{\textbf{\alph{enumii}}}
\renewcommand{\labelenumii}{\textbf{\theenumii.}}
\newcommand{\vect}[1]{\overrightarrow{\,\mathstrut#1\,}}
\def\Oij{$\left(\text{O}~;~\vect{\imath},~\vect{\jmath}\right)$}
\def\Oijk{$\left(\text{O}~;~\vect{\imath},~\vect{\jmath},~\vect{k}\right)$}
\def\Ouv{$\left(\text{O}~;~\vect{u},~\vect{v}\right)$}
\usepackage{fancyhdr}
\usepackage[french]{babel}
\usepackage[dvips]{hyperref}
\usepackage[np]{numprint}
%Tapuscrit : Denis Vergès
%\frenchbsetup{StandardLists=true}

\begin{document}
\setlength\parindent{0mm}
% \rhead{\textbf{A. P{}. M. E. P{}.}}
% \lhead{\small Brevet des collèges}
% \lfoot{\small{Polynésie}}
% \rfoot{\small{7 septembre 2020}}
\pagestyle{fancy}
\thispagestyle{empty}
% \begin{center}
    
% {\Large \textbf{\decofourleft~Brevet des collèges Polynésie 7 septembre 2020~\decofourright}}
    
% \bigskip
    
% \textbf{Durée : 2 heures} \end{center}

% \bigskip

% \textbf{\begin{tabularx}{\linewidth}{|X|}\hline
%  L'évaluation prend en compte la clarté et la précision des raisonnements ainsi que, plus largement, la qualité de la rédaction. Elle prend en compte les essais et les démarches engagées même non abouties. Toutes les réponses doivent être justifiées, sauf mention contraire.\\ \hline
% \end{tabularx}}

% \vspace{0.5cm}\textbf{Exercice 2 \hfill 14 points}

\medskip

Un amateur de football, après l'Euro 2016, décide de s'intéresser à l'historique des
treize dernières rencontres entre la France et le Portugal, regroupées dans le tableau
ci-dessous.

On rappelle la signification des résultats ci-dessous en commentant deux exemples :

\setlength\parindent{6mm}
\begin{itemize}
\item[$\bullet~~$]la rencontre du 3 mars 1973, qui s'est déroulée en France, a vu la victoire du Portugal par 2 buts à 1 ;
\item[$\bullet~~$]la rencontre du 8 mars 1978, qui s'est déroulée en France, a vu la victoire de la France par 2 buts à 0.
\end{itemize}
\setlength\parindent{0mm}

\begin{center}
\begin{tabularx}{\linewidth}{|*{3}{>{\centering \arraybackslash}X|}}
\hline
\multicolumn{3}{|c|}{\textbf{Rencontres de football opposant la France et le Portugal depuis 1973}}\\ \hline
3 mars 1973		& France - Portugal& 1-2\\ \hline
26 avril 1975	& France - Portugal& 0-2\\ \hline
8 mars 1978		& France - Portugal& 2-0\\ \hline
16 février 1983	& Portugal - France& 0-3\\ \hline
23 juin 1984	& France - Portugal& 3-2\\ \hline
24 janvier 1996	& France - Portugal& 3-2\\ \hline
22 janvier 1997	& Portugal - France& 0-2\\ \hline
28 juin 2000	& Portugal - France& 1-2\\ \hline
25 avril 2001	& France - Portugal& 4-0\\ \hline
5 juillet 2006	& Portugal - France& 0-1\\ \hline
11 octobre 2014	& France - Portugal& 2-1\\ \hline
4 septembre 2015& Portugal - France& 0-1\\ \hline
10 juillet 2016	& France - Portugal& 0-1\\ \hline
\end{tabularx}
\end{center}

\begin{enumerate}
\item Depuis 1973, combien de fois la France a-t-elle gagné contre le Portugal ?
\item Calculer le pourcentage du nombre de victoires de la France contre le Portugal
depuis 1973. Arrondir le résultat à l'unité de \%.
\item Le 3 mars 1973, 3 buts ont été marqués au cours du match. Calculer le nombre
moyen de buts par match sur l'ensemble des rencontres. Arrondir le résultat au
dixième.
\end{enumerate}

\bigskip

\end{document}