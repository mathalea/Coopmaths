\documentclass[10pt]{article}
\usepackage[T1]{fontenc}
\usepackage[utf8]{inputenc}%ATTENTION codage UTF8
\usepackage{fourier}
\usepackage[scaled=0.875]{helvet}
\renewcommand{\ttdefault}{lmtt}
\usepackage{amsmath,amssymb,makeidx}
\usepackage[normalem]{ulem}
\usepackage{diagbox}
\usepackage{fancybox}
\usepackage{tabularx,booktabs}
\usepackage{colortbl}
\usepackage{pifont}
\usepackage{multirow}
\usepackage{dcolumn}
\usepackage{enumitem}
\usepackage{textcomp}
\usepackage{lscape}
\newcommand{\euro}{\eurologo{}}
\usepackage{graphics,graphicx}
\usepackage{pstricks,pst-plot,pst-tree,pstricks-add}
\usepackage[left=3.5cm, right=3.5cm, top=3cm, bottom=3cm]{geometry}
\newcommand{\R}{\mathbb{R}}
\newcommand{\N}{\mathbb{N}}
\newcommand{\D}{\mathbb{D}}
\newcommand{\Z}{\mathbb{Z}}
\newcommand{\Q}{\mathbb{Q}}
\newcommand{\C}{\mathbb{C}}
\usepackage{scratch}
\renewcommand{\theenumi}{\textbf{\arabic{enumi}}}
\renewcommand{\labelenumi}{\textbf{\theenumi.}}
\renewcommand{\theenumii}{\textbf{\alph{enumii}}}
\renewcommand{\labelenumii}{\textbf{\theenumii.}}
\newcommand{\vect}[1]{\overrightarrow{\,\mathstrut#1\,}}
\def\Oij{$\left(\text{O}~;~\vect{\imath},~\vect{\jmath}\right)$}
\def\Oijk{$\left(\text{O}~;~\vect{\imath},~\vect{\jmath},~\vect{k}\right)$}
\def\Ouv{$\left(\text{O}~;~\vect{u},~\vect{v}\right)$}
\usepackage{fancyhdr}
\usepackage[french]{babel}
\usepackage[dvips]{hyperref}
\usepackage[np]{numprint}
%Tapuscrit : Denis Vergès
%\frenchbsetup{StandardLists=true}

\begin{document}
\setlength\parindent{0mm}
% \rhead{\textbf{A. P{}. M. E. P{}.}}
% \lhead{\small Brevet des collèges}
% \lfoot{\small{Polynésie}}
% \rfoot{\small{7 septembre 2020}}
\pagestyle{fancy}
\thispagestyle{empty}
% \begin{center}
    
% {\Large \textbf{\decofourleft~Brevet des collèges Polynésie 7 septembre 2020~\decofourright}}
    
% \bigskip
    
% \textbf{Durée : 2 heures} \end{center}

% \bigskip

% \textbf{\begin{tabularx}{\linewidth}{|X|}\hline
%  L'évaluation prend en compte la clarté et la précision des raisonnements ainsi que, plus largement, la qualité de la rédaction. Elle prend en compte les essais et les démarches engagées même non abouties. Toutes les réponses doivent être justifiées, sauf mention contraire.\\ \hline
% \end{tabularx}}

% \vspace{0.5cm}\textbf{\textsc{Exercice 6 \hfill 7 points}}

\medskip

Voici le classement des médailles d'or reçues par les pays participant aux jeux olympiques pour le cyclisme masculin (Source : Wikipédia). 
 
\begin{tabularx}{\linewidth}{|l X|l|l X|}
\multicolumn{5}{l}{\textbf{Bilan des médailles d'or de 1896 à 2008}}\\ \cline{1-2}\cline{4-5}
\textbf{Nation}&\textbf{Or}&&\textbf{Nation}&\textbf{Or}\\
France				&40&~~~~~~~~~~~	& Russie	& 	4\\ 
Italie				&32&	&Suisse 	&	3\\ 
Royaume-Uni			&18&	&Suède		& 	3 \\
Pays-Bas 			&15&	&Tchécoslovaquie& 	2\\
États-Unis 			&14&	&Norvège	& 	2\\
Australie 			&13&	&Canada		& 	1\\
Allemagne 			&13&	&Afrique du Sud &1\\
Union soviétique	&11&	&Grèce		&1\\
Belgique 			&6&	&Nouvelle-Zélande&1\\ 
Danemark 			&6&	&Autriche	&1\\
Allemagne de l'Ouest&6&	&Estonie	&1 \\
Espagne 			&5&	&Lettonie	&1\\ 
Allemagne de l'Est	&4&	&Argentine	&1\\ \cline{1-2}\cline{4-5}
\end{tabularx}

\bigskip
 
\begin{enumerate}
\item Voici un extrait du tableur :

\medskip

\begin{tabularx}{\linewidth}{|c|m{1.5cm}|*{14}{X|}}\hline
 &A									&B	&C	&D	&E	&F	&G	&H	&I	&J	&K	&L	&M	&N	&O\\ \hline 
1&\footnotesize Nombre de médailles d'or	&1	&2	&3	&4	&5	&6	&11	&13&14&15&18&32&40&\\ \hline 
2&\footnotesize Effectif					&8 	&2 	&2	&2	&1	&3	&1	&2&1&1&1&1&1&26\\ \hline
\end{tabularx}

\medskip

Quelle formule a-t-on saisie dans la cellule O2 pour obtenir le nombre total de pays ayant eu une médaille d'or ? 
\item 	
	\begin{enumerate}
		\item Calculer la moyenne de cette série (arrondir à l'unité). 
		\item Déterminer la médiane de cette série. 
		\item En observant les valeurs prises par la série, donner un argument qui explique pourquoi les	valeurs de	la moyenne et de la	médiane sont différentes.
	\end{enumerate}
\item Pour le cyclisme masculin, 70\,\% des pays médaillés ont obtenu au moins une médaille d'or. Quel est le nombre de	pays qui n'ont obtenu que des médailles d'argent ou de bronze (arrondir le résultat à l'unité) ? 
\end{enumerate}

\textbf{Si la travail n'est pas terminé, laisser tout de même une trace de recherche.\\  
Elle sera prise en compte dans l'évaluation.}
\end{document}\end{document}