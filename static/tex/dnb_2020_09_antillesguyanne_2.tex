
\medskip

Cet exercice est un questionnaire à choix multiple (QCM).

Pour chacune des cinq questions, quatre réponses sont proposées, une seule d'entre elles est exacte.

Pour chacune des cinq questions, indiquer sur la copie le numéro de la question et la réponse choisie .
 
\textbf{On rappelle que toute réponse doit être justifiée}.

Une réponse fausse ou l'absence de réponse ne retire pas de point.

\begin{center}
\begin{tabularx}{\linewidth}{|c m{4cm}|*{4}{>{\centering \arraybackslash}X|}}\hline
&Question&Réponse A &Réponse B &Réponse C &Réponse D\\ \hline
\textbf{1.}&Si on multiplie la longueur de chaque arête
 d'un cube par 3, alors le volume du cube sera multiplié par:&3 &9 &12 &27\\ \hline
\textbf{2.}&Lorsque $x$ est égal à $-4$,\: $x^2 +3x + 4$ est égal à :&8 &0 &$-24$ &$-13$\\ \hline
\textbf{3.}&$\dfrac{1}{3} + \dfrac{1}{4} = $&$\dfrac{2}{7}$&0,583&$\dfrac{7}{12}$&$\dfrac{1}{7}$\\ \hline
\textbf{4.}&La notation scientifique de \np{1500000000} est &$15 \times 10^{-8}$& $15 \times 10^8$&
$1,5 \times 10^{-9}$& $1,5 \times 10^9$\\ \hline
\textbf{5.}&$(x - 2)\times (x + 2)$	&$x^2 - 4$&	$x^2 +4$	&$2x - 4$ 	&$2x$\\ \hline
\end{tabularx}
\end{center}

\bigskip

