
\medskip

\begin{enumerate}
\item 
	\begin{enumerate}
		\item La production totale d'électricité en France en 2014 est égale à :  
		
$25,8 + 67,5 + 31 + 415,9 = 540,2$~TWh
		\item La proportion d'électricité produite par les \og Autres énergies (dont la géothermie) \fg{} est :

$\dfrac{31}{540,2} \approx  0,0574$ soit environ $0,057  = 5,7\,\%$.
 	\end{enumerate}
\item Tom considère les pourcentages : ce sont les autres énergies qui ont le plus augmenté leur production par rapport à la production de 2013.

Alice a calculé les variations de production en TWh : avec une augmentation de 12,1 TWh, c'est la nucléaire qui a le plus augmenté sa production (en quantité), alors que les autres énergies ont augmenté de $31 - 28,1 = 2,9$~TWh.
\item 
	\begin{enumerate}
		\item $R = 23$ cm $= 0,23$ m ;  $r = 10$ cm $= 0,1$ m
		
$V = \dfrac{\pi}{3}\times \np{2500} \times \left(0,23^2 + 0,23 \times 0,1 + 0,1^2\right) \approx  225$~m$^3$.
		\item Augmenter de 30\,\% c'est multiplier par $1 + \dfrac{30}{100}$, d'où
		
		$V_{\text{terre extraite}} = 225 \left(1 + \frac{30}{100}\right) = 225 \times  1,30 = 292,5$~ m$^3$.
 	\end{enumerate}
\end{enumerate}

\vspace{0,5cm}

