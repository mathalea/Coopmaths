\documentclass[10pt]{article}
\usepackage[T1]{fontenc}
\usepackage[utf8]{inputenc}%ATTENTION codage UTF8
\usepackage{fourier}
\usepackage[scaled=0.875]{helvet}
\renewcommand{\ttdefault}{lmtt}
\usepackage{amsmath,amssymb,makeidx}
\usepackage[normalem]{ulem}
\usepackage{diagbox}
\usepackage{fancybox}
\usepackage{tabularx,booktabs}
\usepackage{colortbl}
\usepackage{pifont}
\usepackage{multirow}
\usepackage{dcolumn}
\usepackage{enumitem}
\usepackage{textcomp}
\usepackage{lscape}
\newcommand{\euro}{\eurologo{}}
\usepackage{graphics,graphicx}
\usepackage{pstricks,pst-plot,pst-tree,pstricks-add}
\usepackage[left=3.5cm, right=3.5cm, top=3cm, bottom=3cm]{geometry}
\newcommand{\R}{\mathbb{R}}
\newcommand{\N}{\mathbb{N}}
\newcommand{\D}{\mathbb{D}}
\newcommand{\Z}{\mathbb{Z}}
\newcommand{\Q}{\mathbb{Q}}
\newcommand{\C}{\mathbb{C}}
\usepackage{scratch}
\renewcommand{\theenumi}{\textbf{\arabic{enumi}}}
\renewcommand{\labelenumi}{\textbf{\theenumi.}}
\renewcommand{\theenumii}{\textbf{\alph{enumii}}}
\renewcommand{\labelenumii}{\textbf{\theenumii.}}
\newcommand{\vect}[1]{\overrightarrow{\,\mathstrut#1\,}}
\def\Oij{$\left(\text{O}~;~\vect{\imath},~\vect{\jmath}\right)$}
\def\Oijk{$\left(\text{O}~;~\vect{\imath},~\vect{\jmath},~\vect{k}\right)$}
\def\Ouv{$\left(\text{O}~;~\vect{u},~\vect{v}\right)$}
\usepackage{fancyhdr}
\usepackage[french]{babel}
\usepackage[dvips]{hyperref}
\usepackage[np]{numprint}
%Tapuscrit : Denis Vergès
%\frenchbsetup{StandardLists=true}

\begin{document}
\setlength\parindent{0mm}
% \rhead{\textbf{A. P{}. M. E. P{}.}}
% \lhead{\small Brevet des collèges}
% \lfoot{\small{Polynésie}}
% \rfoot{\small{7 septembre 2020}}
\pagestyle{fancy}
\thispagestyle{empty}
% \begin{center}
    
% {\Large \textbf{\decofourleft~Brevet des collèges Polynésie 7 septembre 2020~\decofourright}}
    
% \bigskip
    
% \textbf{Durée : 2 heures} \end{center}

% \bigskip

% \textbf{\begin{tabularx}{\linewidth}{|X|}\hline
%  L'évaluation prend en compte la clarté et la précision des raisonnements ainsi que, plus largement, la qualité de la rédaction. Elle prend en compte les essais et les démarches engagées même non abouties. Toutes les réponses doivent être justifiées, sauf mention contraire.\\ \hline
% \end{tabularx}}

% \vspace{0.5cm}\textbf{\textsc{Exercice 9} \hfill 5 points}

\medskip

Pour son mariage, le samedi 20 août 2016, Norbert souhaite se faire livrer des macarons.

L'entreprise lui demande de payer 402~\euro{} avec les frais de livraison compris.

À l'aide des documents ci-dessous, déterminer dans quelle zone se trouve l'adresse de livraison.


\begin{center}
\parbox{0.31\linewidth}{\begin{tabularx}{\linewidth}{|X|}\hline
\textbf{Document 1 : Bon de commande de Norbert}\\
10 boîtes de 12 petits macarons chocolat\\
10 boîtes de 12 petits macarons vanille\\
5 boîtes de 12 petits macarons framboise\\
2 boîtes de 12 petits macarons café\\
1 boîte de 6 petits macarons caramel\\ \hline
\end{tabularx}}\hfill \parbox{0.66\linewidth}{\begin{tabularx}{\linewidth}{|*{2}{>{\centering \arraybackslash}X|}m{1.9cm}|}\hline
\multicolumn{3}{|c|}{\textbf{Document 2 : Tarifs de la boutique}}\\ \hline
Parfum au choix& Jusqu'à 5 boites achetées&\multirow{5}{1.9cm}{\emph{À partir de la sixième boîte identique achetée, profitez de $20$\,\% de réduction sur toutes vos boîtes de ce parfum}}\\ \cline{1-2}
Boîte de 6 petits macarons& 9~\euro{} la boîte&\\ \cline{1-2}
Boîte de 12 petits macarons&16~\euro{} la boîte&\\ \cline{1-2}
Boîte de 6 gros macarons &13,50~\euro{} la boîte&\\ \cline{1-2}
Boîte de 12 gros macarons&25~\euro{} la boîte&\\ \hline
\multicolumn{3}{|m{7cm}|}{Les frais de livraison, en supplément, sont détaillés
ci-dessous en fonction de la zone de livraison.}\\ \hline
\end{tabularx}}

\bigskip

\begin{tabularx}{0.5\linewidth}{|*{3}{>{\centering \arraybackslash}X|}}\hline
\multicolumn{3}{|c|}{\textbf{Document 3 : Tarifs de livraison}}\\ \hline
&En semaine&Samedi et dimanche\\ \hline
Zone A &12,50~\euro &17,50~\euro\\ \hline
Zone B &20~\euro &25~\euro\\ \hline
Zone C &25~\euro &30~\euro\\ \hline
\end{tabularx}

\vspace{1cm}
\psset{unit=0.6cm}
\begin{pspicture}(11,7)
%\psgrid

\pscircle[fillstyle=solid,fillcolor=lightgray](3.5,3.5){1.75}
\pscircle[fillstyle=solid,fillcolor=white](3.5,3.5){0.75}
\pscircle[fillstyle=vlines](3.5,3.5){0.75}
\pscircle(3.5,3.5){3.4}
\rput(9.5,6){\textbf{Zones de livraison}}
\rput(9.5,4.5){Zone A}\psline{->}(8.4,4.4)(3.5,3.5)
\rput(9.5,3){Zone B}\psline{->}(8.4,3)(4.6,3)
\rput(9.5,1.5){Zone C}\psline{->}(8.4,1.4)(5.8,2.2)
\end{pspicture}
\end{center}
\end{document}\end{document}