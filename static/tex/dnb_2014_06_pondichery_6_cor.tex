\documentclass[10pt]{article}
\usepackage[T1]{fontenc}
\usepackage[utf8]{inputenc}%ATTENTION codage UTF8
\usepackage{fourier}
\usepackage[scaled=0.875]{helvet}
\renewcommand{\ttdefault}{lmtt}
\usepackage{amsmath,amssymb,makeidx}
\usepackage[normalem]{ulem}
\usepackage{diagbox}
\usepackage{fancybox}
\usepackage{tabularx,booktabs}
\usepackage{colortbl}
\usepackage{pifont}
\usepackage{multirow}
\usepackage{dcolumn}
\usepackage{enumitem}
\usepackage{textcomp}
\usepackage{lscape}
\newcommand{\euro}{\eurologo{}}
\usepackage{graphics,graphicx}
\usepackage{pstricks,pst-plot,pst-tree,pstricks-add}
\usepackage[left=3.5cm, right=3.5cm, top=3cm, bottom=3cm]{geometry}
\newcommand{\R}{\mathbb{R}}
\newcommand{\N}{\mathbb{N}}
\newcommand{\D}{\mathbb{D}}
\newcommand{\Z}{\mathbb{Z}}
\newcommand{\Q}{\mathbb{Q}}
\newcommand{\C}{\mathbb{C}}
\usepackage{scratch}
\renewcommand{\theenumi}{\textbf{\arabic{enumi}}}
\renewcommand{\labelenumi}{\textbf{\theenumi.}}
\renewcommand{\theenumii}{\textbf{\alph{enumii}}}
\renewcommand{\labelenumii}{\textbf{\theenumii.}}
\newcommand{\vect}[1]{\overrightarrow{\,\mathstrut#1\,}}
\def\Oij{$\left(\text{O}~;~\vect{\imath},~\vect{\jmath}\right)$}
\def\Oijk{$\left(\text{O}~;~\vect{\imath},~\vect{\jmath},~\vect{k}\right)$}
\def\Ouv{$\left(\text{O}~;~\vect{u},~\vect{v}\right)$}
\usepackage{fancyhdr}
\usepackage[french]{babel}
\usepackage[dvips]{hyperref}
\usepackage[np]{numprint}
%Tapuscrit : Denis Vergès
%\frenchbsetup{StandardLists=true}

\begin{document}
\setlength\parindent{0mm}
% \rhead{\textbf{A. P{}. M. E. P{}.}}
% \lhead{\small Brevet des collèges}
% \lfoot{\small{Polynésie}}
% \rfoot{\small{7 septembre 2020}}
\pagestyle{fancy}
\thispagestyle{empty}
% \begin{center}
    
% {\Large \textbf{\decofourleft~Brevet des collèges Polynésie 7 septembre 2020~\decofourright}}
    
% \bigskip
    
% \textbf{Durée : 2 heures} \end{center}

% \bigskip

% \textbf{\begin{tabularx}{\linewidth}{|X|}\hline
%  L'évaluation prend en compte la clarté et la précision des raisonnements ainsi que, plus largement, la qualité de la rédaction. Elle prend en compte les essais et les démarches engagées même non abouties. Toutes les réponses doivent être justifiées, sauf mention contraire.\\ \hline
% \end{tabularx}}

% \vspace{0.5cm}\textbf{\textsc{Exercice 6 \hfill 7 points}}

\medskip

\begin{enumerate}
\item Dans la cellule O2, on a saisi la formule : = SOMME(B1 : N1)   
\item 
	\begin{enumerate}
	\item $\overline{x} = \dfrac{8\times 1 + 8\times 2 + 2 \times 3 + 2 \times 4 + 1 \times 5 + 3 \times 6 + 1\times 11 + 2 \times 13 + 1\times 14 + 1 \times 15 + 1 \times 18 + 1\times 32 + 1 \times 40}{26}$
	
$\overline{x} = \dfrac{205}{26} \approx  8$. 

La moyenne de cette série est égale à environ 8 médailles.

	\item On calcule $\dfrac{N}{2}= \dfrac{26}{2} = 13$. 
	
La médiane de cette série est comprise entre la 13\up{e} valeur et la 
  14\up{e} de la série rangée dans l’ordre croissant.
  
On cumule les effectifs jusqu’à dépasser 13 :  $8 + 2 + 2= 12$. La 13\up{e} valeur est 4 et la 14\up{e} valeur est 4.

Donc la médiane de cette série est égale à 4 médailles.
	\item Les valeurs de la moyenne et de la médiane sont différentes car l’étendue de la série est très grande : $40 - 1 = 39$. Les valeurs sont alors très dispersées.
		\end{enumerate}
\item  Soit $x$ le nombre de pays médaillés.

70\,\% des pays médaillés ont obtenu au moins une médaille d’or ; ainsi, $\dfrac{70}{100} \times x = 26$, c'est-à-dire $0,7\times x = 26$.

Par suite, $x = \dfrac{26}{0,7} \approx 37$ et $37 - 26 =11$.

Par conséquent, $11$ pays n’ont obtenu que des médailles d’argent ou de bronze.
\end{enumerate}
\end{document}\end{document}