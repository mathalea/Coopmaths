\textbf{\textsc{Exercice 1} \hfill 3 points}

\medskip

Cet exercice est un QCM (questionnaire à choix multiples). Pour chaque ligne du tableau, une seule
affirmation est juste. Sur votre copie, indiquer le numéro de la question et recopier l'affirmation juste.

On ne demande pas de justifier.

\begin{center}
\begin{tabularx}{\linewidth}{|m{4cm}|*{3}{>{\centering \arraybackslash}X|}}\hline
					&Réponse A &Réponse B &Réponse C\\ \hline
\textbf{1.~~} Si ABC est un triangle rectangle en A tel que AB = 5~cm et AC = 7~cm alors la
mesure arrondie au degré près de $\widehat{\text{ABC}}$ est  :&46~\degres& 54~\degres &36~\degres\\ \hline
\textbf{2.~~} L'antécédent de 8 par la fonction $f\::\: \: x \longmapsto  3x - 2$ est&inférieur à 3&compris entre 3 et 4&supérieur à 4\\ \hline
\textbf{3.~~} La valeur exacte de $\dfrac{1- (- 4)}{- 2 + 9}$ est :&$\dfrac{5}{7}$&  8 &\np{0,7142857143}\\ \hline
\end{tabularx}
\end{center}
 
\bigskip

\begin{tabularx}{\linewidth}{|X|}\hline
Les 8 exercices qui suivent traitent du même thème \og le macaron \fg{} mais sont indépendants.\\ \hline
\end{tabularx}

\bigskip

