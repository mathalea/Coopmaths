\documentclass[10pt]{article}
\usepackage[T1]{fontenc}
\usepackage[utf8]{inputenc}%ATTENTION codage UTF8
\usepackage{fourier}
\usepackage[scaled=0.875]{helvet}
\renewcommand{\ttdefault}{lmtt}
\usepackage{amsmath,amssymb,makeidx}
\usepackage[normalem]{ulem}
\usepackage{diagbox}
\usepackage{fancybox}
\usepackage{tabularx,booktabs}
\usepackage{colortbl}
\usepackage{pifont}
\usepackage{multirow}
\usepackage{dcolumn}
\usepackage{enumitem}
\usepackage{textcomp}
\usepackage{lscape}
\newcommand{\euro}{\eurologo{}}
\usepackage{graphics,graphicx}
\usepackage{pstricks,pst-plot,pst-tree,pstricks-add}
\usepackage[left=3.5cm, right=3.5cm, top=3cm, bottom=3cm]{geometry}
\newcommand{\R}{\mathbb{R}}
\newcommand{\N}{\mathbb{N}}
\newcommand{\D}{\mathbb{D}}
\newcommand{\Z}{\mathbb{Z}}
\newcommand{\Q}{\mathbb{Q}}
\newcommand{\C}{\mathbb{C}}
\usepackage{scratch}
\renewcommand{\theenumi}{\textbf{\arabic{enumi}}}
\renewcommand{\labelenumi}{\textbf{\theenumi.}}
\renewcommand{\theenumii}{\textbf{\alph{enumii}}}
\renewcommand{\labelenumii}{\textbf{\theenumii.}}
\newcommand{\vect}[1]{\overrightarrow{\,\mathstrut#1\,}}
\def\Oij{$\left(\text{O}~;~\vect{\imath},~\vect{\jmath}\right)$}
\def\Oijk{$\left(\text{O}~;~\vect{\imath},~\vect{\jmath},~\vect{k}\right)$}
\def\Ouv{$\left(\text{O}~;~\vect{u},~\vect{v}\right)$}
\usepackage{fancyhdr}
\usepackage[french]{babel}
\usepackage[dvips]{hyperref}
\usepackage[np]{numprint}
%Tapuscrit : Denis Vergès
%\frenchbsetup{StandardLists=true}

\begin{document}
\setlength\parindent{0mm}
% \rhead{\textbf{A. P{}. M. E. P{}.}}
% \lhead{\small Brevet des collèges}
% \lfoot{\small{Polynésie}}
% \rfoot{\small{7 septembre 2020}}
\pagestyle{fancy}
\thispagestyle{empty}
% \begin{center}
    
% {\Large \textbf{\decofourleft~Brevet des collèges Polynésie 7 septembre 2020~\decofourright}}
    
% \bigskip
    
% \textbf{Durée : 2 heures} \end{center}

% \bigskip

% \textbf{\begin{tabularx}{\linewidth}{|X|}\hline
%  L'évaluation prend en compte la clarté et la précision des raisonnements ainsi que, plus largement, la qualité de la rédaction. Elle prend en compte les essais et les démarches engagées même non abouties. Toutes les réponses doivent être justifiées, sauf mention contraire.\\ \hline
% \end{tabularx}}

% \vspace{0.5cm}\textbf{Exercice 6 \hfill 9 points}

\medskip

\parbox{0.6\linewidth}{Dans cet exercice, on considère le rectangle ABCD ci-contre tel que son périmètre soit égal à 31 cm.}\hfill
\parbox{0.38\linewidth}{ \psset{unit=2cm}
\begin{pspicture}(2.7,1.7)
\psframe(.3,0.2)(2.1,1.3)
\uput[ul](0.3,1.3){A} \uput[ur](2.1,1.3){B} \uput[dr](2.1,0.2){C} \uput[dl](.3,0.2){D} 
\end{pspicture}}

\bigskip

\begin{enumerate}
\item 
	\begin{enumerate}
		\item Si un tel rectangle a pour longueur 10 cm, quelle est sa largeur ? 
		\item Proposer une autre longueur et trouver la largeur correspondante. 
		\item On appelle $x$ la longueur AB. 

En utilisant le fait que le périmètre de ABCD est de $31$ cm, exprimer la longueur BC en fonction de $x$. 
		\item En déduire l'aire du rectangle ABCD en fonction de $x$. 
	\end{enumerate}
\item On considère la fonction $f$ définie par $f(x) = x (15,5 - x)$.
	\begin{enumerate}
		\item Calculer $f(4)$. 
		\item Vérifiez qu'un antécédent de $52,5$ est $5$. 
	\end{enumerate}
\item Sur le graphique ci-dessous, on a représenté l'aire du rectangle ABCD en fonction de la valeur de $x$. 

\begin{center}
\psset{xunit=0.65cm,yunit=0.1cm}
\begin{pspicture}(-0.5,-10)(16,65)
\psaxes[linewidth=1.5pt,Dy=10]{->}(0,0)(0,0)(16,65)
\psplot[plotpoints=5000,linewidth=1.25pt,linecolor=blue]{0}{15.5}{15.5 x sub x mul}
\rput{90}(-1.5,30){Aire de ABCD}
\uput[d](14,-5){Valeur de $x$} 
\end{pspicture}
\end{center}
 
À l'aide de ce graphique, répondre aux questions suivantes en donnant des valeurs approchées: 

\medskip

	\begin{enumerate}
		\item Quelle est l'aire du rectangle ABCD lorsque $x$ vaut 3 cm ? 
		\item Pour quelles valeurs de $x$ obtient-on une aire égale à 40 cm$^2$ ? 
		\item Quelle est l'aire maximale de ce rectangle ? Pour quelle valeur de $x$ est-elle obtenue ? 
	\end{enumerate}
\item Que peut-on dire du rectangle ABCD lorsque AB vaut $7,75$~cm ? 
\end{enumerate}
\end{document}\end{document}