\textbf{Exercice 9 : Marionnette \hfill 3 points}

\medskip

%Un marionnettiste doit faire un spectacle sur le thème de l'ombre. Pour cela il a besoin que sa marionnette de 30~cm ait une ombre de 1,2~m.
%
%La source de lumière C est située à 8 m de la toile (AB).
%
%La marionnette est représentée par le segment [DE].
%
%\medskip

\begin{enumerate}
\item %Démontrer que les droites (AB) et (DE) sont parallèles.
Les droites (AB) et (DE) sont parallèles car perpendiculaires à la droite (BC).
\item %Calculer EC pour savoir où il doit placer sa marionnette.
Les droites  (AB) et (DE) sont parallèles, C, E, B d’une part, C, D et A d’autre part sont alignés dans cet ordre ; d’après le théorème de Thalès :

$\dfrac{\text{EC}}{\text{CB}} = \dfrac{\text{DE}}{\text{AB}}$, soit $\dfrac{\text{EC}}{8} = \dfrac{0,3}{1,2} = \dfrac{1}{4}$, d’où EC $ = 2$~m. 
\end{enumerate}

%\begin{center}
%\psset{unit=1cm}
%\begin{pspicture}(12,5.5)
%%\psgrid
%\pspolygon(0.5,1.2)(9.5,1.2)(0.5,5)
%\psline(7.5,1.2)(7.5,2.05)
%\psframe(0.5,1.2)(0.8,1.5)\psframe(7.5,1.2)(7.8,1.5)
%\psline{<->}(0.5,0.6)(9.5,0.6)
%\uput[d](5,0.6){8 m}\uput[l](0.5,3.1){1,2 m}
%\uput[l](7.5,1.65){30 cm}
%\uput[ul](0.5,5){A} \uput[dl](0.5,1.2){B} \uput[d](9.5,1.2){C} \uput[u](7.5,2.05){D} \uput[d](7.5,1.2){E}
%\psset{unit=0.6cm}
%\rput{-20}(16,1.9){\pspolygon*(0,-0.3)(0,0.3)(0.8,0.2)(2,0.2)(2,-0.2)(0.8,-0.2)} 
%\end{pspicture}
%
%Cette figure n'est pas à l'échelle.
%\end{center}
\newpage
\begin{center}

\textbf{\large ANNEXE 1 - Exercice 8}

\bigskip

\begin{tabularx}{\linewidth}{|m{3cm}|*{5}{>{\centering \arraybackslash}X|}}\hline
Nombre de clips &1 &2 &5 &10 &15\\ \hline
Prix en euros pour le téléchargement direct&4 &8&20&40&60\\ \hline
Prix en euros pour le téléchargement membre&12 &14&20&30&40\\ \hline
Prix en euros pour le téléchargement premium&50 &50&50&50&50\\ \hline
\end{tabularx}

\vspace{2cm}

\textbf{\large ANNEXE 2 - Exercice 8}

\bigskip

\psset{xunit=0.5cm,yunit=0.1cm}
\begin{pspicture}(-1,-10)(24,60)
\multido{\n=0+1}{25}{\psline[linewidth=0.2pt](\n,0)(\n,60)}
\multido{\n=0+5}{13}{\psline[linewidth=0.2pt](0,\n)(24,\n)}
\psaxes[linewidth=1.25pt,Dy=100]{->}(0,0)(24,60)
\multido{\n=0+5}{13}{\uput[l](0,\n){\n}}
\uput[d](20.5,-4){Nombre de clips achetés}
\uput[r](0,57.5){Prix en euros}
\psline[linewidth=1.5pt](0,50)(24,50)\uput[u](3,50){$f$}
\psline[linewidth=1.5pt](15,60)\uput[ul](14,56){$g$}
\psplot[linewidth=1.5pt]{0}{24}{2 x mul 10 add}\uput[ul](23,57){$h$}
\end{pspicture}
\end{center}
