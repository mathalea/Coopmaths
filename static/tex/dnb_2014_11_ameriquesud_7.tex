\documentclass[10pt]{article}
\usepackage[T1]{fontenc}
\usepackage[utf8]{inputenc}%ATTENTION codage UTF8
\usepackage{fourier}
\usepackage[scaled=0.875]{helvet}
\renewcommand{\ttdefault}{lmtt}
\usepackage{amsmath,amssymb,makeidx}
\usepackage[normalem]{ulem}
\usepackage{diagbox}
\usepackage{fancybox}
\usepackage{tabularx,booktabs}
\usepackage{colortbl}
\usepackage{pifont}
\usepackage{multirow}
\usepackage{dcolumn}
\usepackage{enumitem}
\usepackage{textcomp}
\usepackage{lscape}
\newcommand{\euro}{\eurologo{}}
\usepackage{graphics,graphicx}
\usepackage{pstricks,pst-plot,pst-tree,pstricks-add}
\usepackage[left=3.5cm, right=3.5cm, top=3cm, bottom=3cm]{geometry}
\newcommand{\R}{\mathbb{R}}
\newcommand{\N}{\mathbb{N}}
\newcommand{\D}{\mathbb{D}}
\newcommand{\Z}{\mathbb{Z}}
\newcommand{\Q}{\mathbb{Q}}
\newcommand{\C}{\mathbb{C}}
\usepackage{scratch}
\renewcommand{\theenumi}{\textbf{\arabic{enumi}}}
\renewcommand{\labelenumi}{\textbf{\theenumi.}}
\renewcommand{\theenumii}{\textbf{\alph{enumii}}}
\renewcommand{\labelenumii}{\textbf{\theenumii.}}
\newcommand{\vect}[1]{\overrightarrow{\,\mathstrut#1\,}}
\def\Oij{$\left(\text{O}~;~\vect{\imath},~\vect{\jmath}\right)$}
\def\Oijk{$\left(\text{O}~;~\vect{\imath},~\vect{\jmath},~\vect{k}\right)$}
\def\Ouv{$\left(\text{O}~;~\vect{u},~\vect{v}\right)$}
\usepackage{fancyhdr}
\usepackage[french]{babel}
\usepackage[dvips]{hyperref}
\usepackage[np]{numprint}
%Tapuscrit : Denis Vergès
%\frenchbsetup{StandardLists=true}

\begin{document}
\setlength\parindent{0mm}
% \rhead{\textbf{A. P{}. M. E. P{}.}}
% \lhead{\small Brevet des collèges}
% \lfoot{\small{Polynésie}}
% \rfoot{\small{7 septembre 2020}}
\pagestyle{fancy}
\thispagestyle{empty}
% \begin{center}
    
% {\Large \textbf{\decofourleft~Brevet des collèges Polynésie 7 septembre 2020~\decofourright}}
    
% \bigskip
    
% \textbf{Durée : 2 heures} \end{center}

% \bigskip

% \textbf{\begin{tabularx}{\linewidth}{|X|}\hline
%  L'évaluation prend en compte la clarté et la précision des raisonnements ainsi que, plus largement, la qualité de la rédaction. Elle prend en compte les essais et les démarches engagées même non abouties. Toutes les réponses doivent être justifiées, sauf mention contraire.\\ \hline
% \end{tabularx}}

% \vspace{0.5cm}\textbf{\textsc{Exercice 7} \hfill 3 points}

\medskip

\textbf{Il sera tenu compte de toute trace de réponse même incomplète dans l'évaluation} 

\medskip

Joachim doit traverser une rivière avec un groupe d'amis. 

Il souhaite installer une corde afin que les personnes peu rassurées puissent se tenir. 

Il veut connaître la largeur de la rivière à cet endroit (nommé D) pour déterminer si la corde dont il dispose est assez longue. 

Pour cela il a repéré un arbre (nommé A) sur l'autre rive. 

Il parcourt 20~mètres sur la rive rectiligne où il se situe et trouve un nouveau repère : un rocher (nommé R). 

Ensuite il poursuit sur 12 mètres et s'éloigne alors de la rivière, à angle droit, jusqu'à ce que le rocher soit aligné avec l'arbre depuis son point d'observation (nommé B). Il parcourt pour cela 15 mètres. 

Il est alors satisfait: sa corde d'une longueur de 30 mètres est assez longue pour qu'il puisse l'installer entre les points D et A. 

A l'aide de la figure, confirmer sa décision. 

\begin{center}
\psset{unit=1cm}
\begin{pspicture}(10,5)
%\psgrid
\psline(0,2)(10,2)\psline(0,4.5)(10,4.5)
\psline(0.5,2)(0.5,4.5)(6,0.25)(6,2)
\rput(8,3.25){Rivière} 
\uput[d](2,2){20~m} 
\uput[u](4.9,2){12~m}
\uput[r](6,1.25){15~m} 
\rput(2.5,0.5){\small La figure n'est pas à l'échelle}
\uput[ul](0.5,4.5){A} \uput[d](0.5,2){D}\uput[u](3.75,2){R}
\uput[u](6,2){V} \uput[r](6,0.5){B} 
\end{pspicture}
\end{center}
\end{document}}\end{document}