
\medskip

 Voici les dimensions de quatre solides: 

\begin{itemize}
\item[$\bullet$] Une pyramide de 6 cm de hauteur dont la base est un rectangle de 6 cm de longueur et de 3 cm de largeur. 

\item[$\bullet$] Un cylindre de 2 cm de rayon et de 3 cm de hauteur. 


\item[$\bullet$] Un cône de 3 cm de rayon et de 3 cm de hauteur. 

\item[$\bullet$] Une boule de 2 cm de rayon. 
\end{itemize}
\parbox{0.55\linewidth}{
\begin{enumerate}
\item

%\begin{enumerate}[a)]
  \begin{enumerate}

\item Représenter approximativement les trois premiers solides comme l'exemple ci-contre : 

\item Placer les dimensions données sur les représentations. 
\end{enumerate}
\item  Classer ces quatre solides dans l'ordre croissant de leur volume. 
\end{enumerate}

}\hfill
 \parbox{0.35\linewidth}{\psset{unit=1.5cm}
\begin{pspicture}(-2,-1)(2,0)
\pscircle(0,0){1.4}
\scalebox{.99}[0.3]{\psarc[linestyle=dashed](0,0){1.4}{0}{180}}
\scalebox{.99}[0.3]{\psarc[](-0.1,0){1.4}{180}{0}}
\pstGeonode[PointName=none, dotscale=1.1](0,0){O}
\pstGeonode[PointName=none, dotscale=1.1](-1,.267){A}
\pstLineAB[linestyle=dashed,arrows=<->]{A}{O}
\rput{-12}(-0.4,0.2){{\tiny 2~cm}}
\end{pspicture}
}


\textit{Quelques formules }: 
$$\dfrac{4}{3}\times \pi\times rayon^3\qquad\hfill\qquad \pi\times rayon^2\times hauteur$$

$$\dfrac{1}{3}\times \pi\times  rayon^2\times hauteur \qquad\hfill\qquad \dfrac{1}{3}\times aire\:de\:la\:base\times hauteur$$




\bigskip

