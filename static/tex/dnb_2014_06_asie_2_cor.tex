\textbf{Exercice 2 \hfill 5 points}

\medskip

%Une corde de guitare est soumise à une tension $T$, exprimée en Newton (N), 
%qui permet d'obtenir un son quand la corde est pincée.
%  
%Ce son plus ou moins aigu est caractérisé par une fréquence $f$ exprimée en Hertz ( Hz). 
%
%\medskip
%
%\parbox{0.4\linewidth}{La fonction qui à une tension $T$ associe 
%sa fréquence  est définie par la relation : 
%$f(T) =20\sqrt{T}$.
% 
%On donne ci-contre la représentation 
%graphique de cette fonction.}\hfill
%\parbox{0.57\linewidth}{
%\psset{unit=0.004cm}
%\def\pshlabel#1{\tiny#1}
%\def\psvlabel#1{\scriptsize#1}
%\begin{pspicture}(-150,-100)(1700,1300) 
%\uput[r](0,1250){Fréquence $f$ en Hz}
%\uput[u](1400,0){Tension $T$ en N } 
%\psaxes[linewidth=1.5pt,Dx=100,Dy=100,xlabelFactor{0.5}]{->}(0,0)(-90,-90)(1700,1300)
%\multido{\n=0+100}{18}{\psline[linewidth=0.3pt,linecolor=orange](\n,0)(\n,1300)}
%\multido{\n=0+100}{14}{\psline[linewidth=0.3pt,linecolor=orange](0,\n)(1700,\n)}
%\psplot[plotpoints=4000,linewidth=1.25pt,linecolor=cyan]{0}{1700}{x sqrt 20 mul}
%\end{pspicture}}
%
%\medskip
%
%\textbf{Tableau des fréquences (en Hertz) de différentes notes de musique}
%
%\medskip
%
%\begin{tabularx}{\linewidth}{|m{1.7cm}|*{14}{>{\footnotesize \centering \arraybackslash}X|}}\hline
% Notes& Do2& Ré2& Mi2& Fa2& Sol2 &La2& Si2& Do3& Ré3& Mi3& Fa3& Sol3& La3& Si3\\\hline 
%Fréquences (en Hz)&132& 148,5& 165& 176& 198& 220& 247,5& 264 &297 &330& 352& 396& 440& 495\\\hline
%\end{tabularx}
%\medskip
 

%Déterminer graphiquement une valeur approchée de la tension à appliquer sur la corde pour obtenir un \og La3 \fg. 
On trace la droite horizontale contenant tous les points d’ordonnée 440, qui coupe la courbe en un point dont l’abscisse est environ 480.

%Déterminer par le calcul la note obtenue si on pince la corde avec une tension de 220~N environ.
On calcule $f(220) = 20 \sqrt{220} \approx 297$~Hz. On obtient la note Ré3.
 
%La corde casse lorsque la tension est supérieure à 900~N.
 
%Quelle fréquence maximale peut-elle émettre avant de casser? 
Pour $T = 900$, on obtient la fréquence maximale : $f(900) = 20 \sqrt{900} = 20 \times 30 = 600$~(Hz).

\bigskip

