\textbf{Exercice 2 \hfill 4,5 points}

\medskip

Voici un programme de calcul sur lequel travaillent quatre élèves.

\begin{center}
\begin{tabularx}{0.4\linewidth}{|X|}\hline
$\bullet~~$ Prendre un nombre\\
$\bullet~~$ Lui ajouter 8\\
$\bullet~~$ Multiplier le résultat par 3\\
$\bullet~~$ Enlever 24\\
$\bullet~~$ Enlever le nombre de départ\\\hline
\end{tabularx}
\end{center}

Voici ce qu'ils affirment :

Sophie : \og Quand je prends 4 comme nombre de
départ, j'obtiens, 8 \fg

Martin : \og En appliquant le programme à 0, je trouve 0. \fg

Gabriel : \og  Moi,j'ai pris $-3$ au départ et j'ai obtenu $-9$. \fg

Faïza : \og Pour n'importe quel nombre choisi, le résultat final est égal au double du nombre de départ. \fg

Pour chacun de ces quatre élèves expliquer s'il a raison ou tort.

\clearpage

