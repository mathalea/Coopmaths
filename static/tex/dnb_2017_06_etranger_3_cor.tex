\documentclass[10pt]{article}
\usepackage[T1]{fontenc}
\usepackage[utf8]{inputenc}%ATTENTION codage UTF8
\usepackage{fourier}
\usepackage[scaled=0.875]{helvet}
\renewcommand{\ttdefault}{lmtt}
\usepackage{amsmath,amssymb,makeidx}
\usepackage[normalem]{ulem}
\usepackage{diagbox}
\usepackage{fancybox}
\usepackage{tabularx,booktabs}
\usepackage{colortbl}
\usepackage{pifont}
\usepackage{multirow}
\usepackage{dcolumn}
\usepackage{enumitem}
\usepackage{textcomp}
\usepackage{lscape}
\newcommand{\euro}{\eurologo{}}
\usepackage{graphics,graphicx}
\usepackage{pstricks,pst-plot,pst-tree,pstricks-add}
\usepackage[left=3.5cm, right=3.5cm, top=3cm, bottom=3cm]{geometry}
\newcommand{\R}{\mathbb{R}}
\newcommand{\N}{\mathbb{N}}
\newcommand{\D}{\mathbb{D}}
\newcommand{\Z}{\mathbb{Z}}
\newcommand{\Q}{\mathbb{Q}}
\newcommand{\C}{\mathbb{C}}
\usepackage{scratch}
\renewcommand{\theenumi}{\textbf{\arabic{enumi}}}
\renewcommand{\labelenumi}{\textbf{\theenumi.}}
\renewcommand{\theenumii}{\textbf{\alph{enumii}}}
\renewcommand{\labelenumii}{\textbf{\theenumii.}}
\newcommand{\vect}[1]{\overrightarrow{\,\mathstrut#1\,}}
\def\Oij{$\left(\text{O}~;~\vect{\imath},~\vect{\jmath}\right)$}
\def\Oijk{$\left(\text{O}~;~\vect{\imath},~\vect{\jmath},~\vect{k}\right)$}
\def\Ouv{$\left(\text{O}~;~\vect{u},~\vect{v}\right)$}
\usepackage{fancyhdr}
\usepackage[french]{babel}
\usepackage[dvips]{hyperref}
\usepackage[np]{numprint}
%Tapuscrit : Denis Vergès
%\frenchbsetup{StandardLists=true}

\begin{document}
\setlength\parindent{0mm}
% \rhead{\textbf{A. P{}. M. E. P{}.}}
% \lhead{\small Brevet des collèges}
% \lfoot{\small{Polynésie}}
% \rfoot{\small{7 septembre 2020}}
\pagestyle{fancy}
\thispagestyle{empty}
% \begin{center}
    
% {\Large \textbf{\decofourleft~Brevet des collèges Polynésie 7 septembre 2020~\decofourright}}
    
% \bigskip
    
% \textbf{Durée : 2 heures} \end{center}

% \bigskip

% \textbf{\begin{tabularx}{\linewidth}{|X|}\hline
%  L'évaluation prend en compte la clarté et la précision des raisonnements ainsi que, plus largement, la qualité de la rédaction. Elle prend en compte les essais et les démarches engagées même non abouties. Toutes les réponses doivent être justifiées, sauf mention contraire.\\ \hline
% \end{tabularx}}

% \vspace{0.5cm}\textbf{\textsc{Exercice 3} \hfill 6 points}

\medskip

%Voici les dimensions de quatre solides : 
%
%\begin{itemize}
%\item[$\bullet$] Une pyramide de 6 cm de hauteur dont la base est un rectangle de 6 cm de longueur et de 3 cm de largeur. 
%
%\item[$\bullet$] Un cylindre de 2 cm de rayon et de 3 cm de hauteur. 
%
%
%\item[$\bullet$] Un cône de 3 cm de rayon et de 3 cm de hauteur. 
%
%\item[$\bullet$] Une boule de 2 cm de rayon. 
%\end{itemize}
%\parbox{0.55\linewidth}{
\begin{enumerate}
\item
\begin{enumerate}

\item %Représenter approximativement les trois premiers solides comme l'exemple ci-contre : 

\item %Placer les dimensions données sur les représentations. 
\end{enumerate}
\item %Classer ces quatre solides dans l'ordre croissant de leur volume.
$\bullet~~$Volume de la pyramide : $V_{\text{pyramide}}  = \dfrac{1}{3} S \times h = \dfrac{6 \times 3 \times 6}{3} = 36$~cm$^3$ ;

$\bullet~~$Volume du cylindre : $V_{\text{cylindre}} =  \pi \times R^2 \times h = \pi \times 2^2 \times 3 = 12 \pi \approx 37,7$~cm$^3$ ;

$\bullet~~$Volume du cône : $V_{\text{cône}}  = \dfrac{1}{3} \pi \times r^2 \times h = \dfrac{\pi \times 3^2 \times 3}{3} = 9\pi \approx 28,3$~cm$^3$ ;

$\bullet~~$Volume de la boule : $V_{\text{boule}}  = \dfrac{4}{3}\times r^3 = \dfrac{2^3 \times 4 \pi}{3} = \dfrac{32\pi}{3} \approx 33,5$~cm$^3$.

On a donc : $ < V_{\text{cône}}  < V_{\text{boule}} < V_{\text{pyramide}} < V_{\text{cylindre}}$.
\end{enumerate}

%}\hfill
% \parbox{0.35\linewidth}{\psset{unit=1.5cm}
%\begin{pspicture}(-2,-1)(2,0)
%\pscircle[fillstyle=solid,fillcolor=lightgray](0,0){1.4}
%\scalebox{.99}[0.3]{\psarc[linestyle=dashed](0,0){1.4}{0}{180}}
%\scalebox{.99}[0.3]{\psarc[](-0.1,0){1.4}{180}{0}}
%\pstGeonode[PointName=none, dotscale=1.1](0,0){O}
%\pstGeonode[PointName=none, dotscale=1.1](-1,.267){A}
%\pstLineAB[linestyle=dashed]{<->}{A}{O}\rput{-12}(-0.4,0.2){{\tiny 2~cm}}
%
%
%\end{pspicture}
%
%}
%\textit{Quelques formules }: 
%$$\dfrac{4}{3}\times \pi\times rayon^3\qquad\hfill\qquad \pi\times rayon^2\times hauteur$$
%
%$$\dfrac{1}{3}\times \pi\times  rayon^2\times hauteur \qquad\hfill\qquad \dfrac{1}{3}\times aire\:de\:la\:base\times hauteur$$

\bigskip

\end{document}