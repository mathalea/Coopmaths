
\medskip

%Les activités humaines produisent du dioxyde de carbone (CO$_2$) qui contribue au réchauffement
%climatique. Le graphique suivant représente l'évolution de la concentration atmosphérique moyenne
%en CO$_2$ (en ppm) en fonction du temps (en année).
%
%\begin{center}
%\psset{xunit=0.35cm,yunit=0.05cm}
%\begin{pspicture}(45,190)
%\psaxes[linewidth=1.25pt,Ox=1980,Dx=5,Oy=300,Dy=20](0,0)(0,0)(35,160)
%\multido{\n=0+5}{8}{\psline[linestyle=dashed](\n,0)(\n,160)}
%\multido{\n=0+20}{9}{\psline[linestyle=dashed](0,\n)(35,\n)}
%\pscurve[linewidth=1.25pt,linecolor=blue](4,42)(7.5,50)(10,54)(12.5,58)(15,60)(20,68)(25,80)(30,88)(33,97)
%\rput(17.5,188){\textbf{Concentration de CO$_2$ atmosphérique}}
%\rput(17.5,172){\footnotesize (Source: Centre Mondial de Données relatives aux Gaz à Effet de Serre sous l'égide de l'OMM)}
%\uput[r](35,150){\scriptsize 450 ppm}
%\uput[r](35,145){\scriptsize niveau moyen}
%\uput[r](35,140){\scriptsize à ne pas dépasser}
%\uput[r](35,135){\scriptsize à l'horizon 2100}
%\uput[r](0,165){parties par million - CO$_2$}
%\end{pspicture}
%
%\vspace{0.6cm}
%
%1 ppm de CO$_2$ = 1 partie par million de CO$_2$ = 1 milligramme de CO$_2$ par kilogramme d'air.
%\end{center}
%\smallskip

\begin{enumerate}
\item %Déterminer graphiquement la concentration de CO$_2$ en ppm en 1995 puis en 2005.
On lit sur le graphique :
\begin{itemize}
\item en 1995 : 360~ppm ;
\item en 2005 : 380~ppm. 
\end{itemize}

\item %On veut modéliser l'évolution de la concentration de CO$_2$ en fonction du temps à l'aide d'une
%fonction $g$ où $g(x)$ est la concentration de CO$_2$ en ppm en fonction de l'année $x$.
	\begin{enumerate}
		\item %Expliquer pourquoi une fonction affine semble appropriée pour modéliser la concentration
%en CO$_2$ en fonction du temps entre 1995 et 2005.
Les points de la courbe sont à peu près alignés : le modèle affine semble donc pertinent.
		\item  %Arnold et Billy proposent chacun une expression pour la fonction $g$ :
		
%Arnold propose l'expression $g(x) = 2x - \np{3630}$ ;
		
%Billy propose l'expression $g(x) = 2x - \np{2000}$.
		
%Quelle expression modélise le mieux l'évolution de la concentration de CO$_2$ ? Justifier.
Pour l'année 1995, l'expression de'Arnold donne $2 \times \np{1995} - \np{3630} = 360$, et celle de Billy $2 \times 1995 - \np{2000} = \np{1990}$ : ce dernier résultat est complètement erroné.

Il vaut mieux prendre l'expression d'Arnold.
		\item  %En utilisant la fonction que vous avez choisie à la question précédente, indiquer l'année pour laquelle la valeur de $450$~ppm est atteinte.
		Il faut résoudre l'équation :
		
$2x - \np{3630} = 450$ soit $2x = \np{4080}$ et $x = \np{2040}$.

La valeur de 450~ppm sera atteinte en 2040.
 	\end{enumerate}
\item  %En France, les forêts, grâce à la photosynthèse, captent environ $70$~mégatonnes de CO$_2$ par an, ce qui représente $15$\,\% des émissions nationales de carbone (année 2016).
	
%Calculer une valeur approchée à une mégatonne près de la masse M du CO$_2$ émis en France en 2016.
Si en 2016, $70$ mégatonnes représentent 15\,\% des émissions $M$ de carbone, alors :

$\dfrac{15}{100} \times M = 70$, soit $M = \dfrac{70 \times 100}{15} \approx 466,6$, soit 467~mégatonnes de CO$_2$ à la mégatonne près.
\end{enumerate}

\bigskip

