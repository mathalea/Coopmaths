
\medskip

Dans la vitrine d'un magasin A sont présentés au total $45$ modèles de chaussures. Certaines sont conçues pour la ville, d'autres pour le sport et sont de trois couleurs différentes: noire, blanche ou marron.

\medskip

\begin{enumerate}
\item Compléter le tableau suivant sur \textbf{l'annexe 1}.
\begin{center}
\begin{tabularx}{0.7\linewidth}{|*{4}{>{\centering \arraybackslash}X|}}\hline
Modèle &Pour la ville &Pour le sport &Total\\ \hline
Noir 	&	&5 	&20\\ \hline
Blanc 	&7	&	&\\ \hline
Marron 	&	&3	&\\ \hline
Total 	&27	&	& 45\\ \hline
\end{tabularx}
\end{center}

\item On choisit un modèle de chaussures au hasard dans cette vitrine.
	\begin{enumerate}
		\item Quelle est la probabilité de choisir un modèle de couleur noire ?
		\item Quelle est la probabilité de choisir un modèle pour le sport ?
		\item Quelle est la probabilité de choisir un modèle pour la ville de couleur marron ?
 	\end{enumerate}
\item Dans la vitrine d'un magasin B, on trouve $54$ modèles de chaussures dont $30$ de couleur noire.
	
On choisit au hasard un modèle de chaussures dans la vitrine du magasin A puis dans celle
du magasin B.
	
Dans laquelle des deux vitrines a-t-on le plus de chance d'obtenir un modèle de couleur noire ?
Justifier.
\end{enumerate}

\begin{center}
\textbf{\large Annexe 1}

\medskip

\textbf{(à rendre avec la copie)}

\vspace{1.5cm}

\medskip

\begin{tabularx}{0.8\linewidth}{|*{4}{>{\centering \arraybackslash}X|}}\hline
Modèle 	&Pour la ville 	&Pour le sport 	&Total\\ \hline
Noir 	&				&5				&20\\ \hline
Blanc	&7				&				&\\ \hline
Marron	&				&3				&\\ \hline
Total 	&27				&				&45\\ \hline
\end{tabularx}
\end{center}


\vspace{0,5cm}

