\textbf{Exercice 5 \hfill 4 points}

\bigskip
 
Pour son anniversaire, Julien a reçu un coffret de tir à l'arc.
 
Il tire une flèche. La trajectoire de la pointe de cette flèche est représentée ci-dessous.
 
La courbe donne la hauteur en mètres (m) en fonction de la distance horizontale en mètres (m) parcourue par la flèche. 

\begin{center}
\psset{unit=1cm}
\begin{pspicture}(-0.5,-1)(11,5)
\psgrid[gridlabels=0,subgriddiv=2,gridcolor=cyan,subgridcolor=cyan](0,0)(11,5)
\psaxes[linewidth=1pt](0,0)(11,5)
\psaxes[linewidth=1.5pt]{->}(0,0)(1,1)
\uput[d](10.8,0){$x$}
\uput[l](0,4.6){$y$}
\uput[d](9,-0.5){Distance horizontale (m)}
\rput(1,4.8){Hauteur (m)}
\psplot[plotpoints=5000,linewidth=1.25pt,linecolor=blue]{0}{10}{1 0.9 x mul add x dup mul 0.1 mul sub}
\end{pspicture}
\end{center} 

\begin{enumerate}
\item Dans cette partie, les réponses seront données grâce à des \textbf{lectures graphiques}. Aucune justification n'est attendue sur la copie. 
	\begin{enumerate}
		\item De quelle hauteur la flèche est-elle tirée ? 
		\item À quelle distance de Julien la flèche retombe-t-elle au sol? 
		\item Quelle est la hauteur maximale atteinte par la flèche ?
	\end{enumerate} 
\item Dans cette partie, les réponses seront justifiées par des \textbf{calculs} :
 
La courbe ci-dessus représente la fonction $f$ définie par 

$f(x) = - 0,1 x^2 + 0,9x + 1$. 
	\begin{enumerate}
		\item Calculer $f(5)$. 
		\item La flèche s' élève-t-elle à plus de 3 m de hauteur ?
	\end{enumerate} 
\end{enumerate} 

\bigskip
 
