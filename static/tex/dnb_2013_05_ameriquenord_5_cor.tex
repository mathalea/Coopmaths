\textbf{\textsc{Exercice 5} \hfill 4 points}

\bigskip
 
%Tom doit calculer $3,5^2$.
% 
%\og Pas la peine de prendre la calculatrice \fg, lui dit Julie, tu n'as qu'à effectuer le produit de $3$ par $4$ et rajouter $0,25$. 
%
%\medskip

\begin{enumerate}
\item %Effectuer le calcul proposé par Julie et vérifier que le résultat obtenu est bien le carré de $3,5$.
$3 \times 4  + 0,25 = 12 + 0,25 = 12,25$.

Or $3,5^2 = \left(\dfrac{7}{2} \right)^2 = \dfrac{7^2}{2^2} = \dfrac{49}{4} = \dfrac{24,5}{2} = 12,25$. Le calcul est exact. 
\item %Proposer une façon simple de calculer $7,5^2$ et donner le résultat.
Multiplier 7 par 8 et ajouter 0,25 au produit.

$7 \times 8 + 0,25 = 56,25$.

$7,5^2 = \left(\dfrac{15}{2} \right)^2 = \dfrac{15^2}{2^2} = \dfrac{225}{4} = \dfrac{112,5}{2} = 56,25$. Exact ! 
\item %Julie propose la conjecture suivante : 	$(n + 0,5)^2 = n(n + 1) + 0,25$ 

%$n$ est un nombre entier positif. 

%Prouver que la conjecture de Julie est vraie (quel que soit le nombre $n$)
Quel que soit le naturel $n$ : $(n + 0,5)^2 = n^2 + 0,5^2 + 2 \times n \times 0,5 = n^2 + n + 0,25 = n(n + 1) + 0,25$.

La  conjecture de Julie est vraie.
\end{enumerate} 

\bigskip

