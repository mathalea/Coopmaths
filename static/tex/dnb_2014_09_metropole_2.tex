\textbf{Exercice 2 \hfill 5 points}

\medskip
 
Dans cet exercice, les figures codées ne sont pas en vraie grandeur. 

Chacune des affirmations suivantes est-elle vraie ou fausse ? On rappelle que toutes les réponses doivent être justifiées.

\medskip

\begin{tabularx}{\linewidth}{|X|}\hline 
\parbox{0.4\linewidth}{\textbf{Affirmation 1 :}

Le volume de ce solide est 56 cm$^3$.} \hfill 
\parbox{0.55\linewidth}{\psset{unit=0.925cm}
\begin{pspicture}(0,-0.4)(8.2,4)
\pspolygon(1.4,0.6)(4.7,0.6)(1.4,2.25)
\psline(4.7,0.6)(8,2.2)(4.7,3.9)(1.4,2.25)
\psline[linestyle=dotted](1.4,0.6)(4.7,2.2)(8,2.2)
\psline[linestyle=dotted](4.7,2.2)(4.7,3.9)
\psframe(4.7,2.2)(4.9,2.4)
\psframe(1.4,0.6)(1.6,0.8)
\psline{<->}(1.4,0.4)(4.7,0.4)
\psline{<->}(1.2,0.6)(1.2,2.25)
\psline{<->}(4.8,0.45)(8.3,2.2)
\uput[d](3.05,0.4){4 cm}
\uput[l](1.2,1.425){2 cm}
\rput{27}(6.55,1.1){7 cm}
\end{pspicture}}\\ \hline
\parbox{0.4\linewidth}{Dans ce dessin, les points sont placés sur les sommets d'un quadrillage à maille carrée.

\textbf{Affirmation 2 :} Les droites (ML) et (NO) sont parallèles.}\hfill
\parbox{0.55\linewidth}{\psset{unit=1cm}
\begin{pspicture*}(-0.4,-0.6)(7.6,4.5)
\psgrid[subgriddiv=1,gridlabels=0,gridcolor=cyan]
\pspolygon[linewidth=1.25pt](0,0)(7,0)(7,4)
\psline[linewidth=1.25pt](5,0)(7,1)
\uput[dl](0,0){N} \uput[ur](7,4){O} \uput[dr](7,0){K} 
\uput[r](7,1){M} \uput[dl](5,0){L} 
\end{pspicture*}}\\ \hline
~\\ 
\textbf{Affirmation 3 :} La diagonale d'un carré d'aire 36 cm$^2$ a pour longueur $6\sqrt{2}$ cm.\\ 
~\\ \hline
~\\  
\textbf{Affirmation 4 :} 0 a un seul antécédent par la fonction qui à tout nombre $x$ associe $3x + 5$.\\ 
~\\ \hline
\end{tabularx}
  
\vspace{0,5cm}

