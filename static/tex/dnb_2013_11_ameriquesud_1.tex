\textbf{Exercice 1 \hfill 6 points}

\medskip

Voici trois documents: 

\medskip

\begin{tabularx}{\linewidth}{|m{6.5cm}|X|}\hline
\textbf{Document 1}&\textbf{Document 2}\\ 
Le salaire moyen brut\up{1} des Français s'établissait en 2010 à \np{2764}~\euro{} par mois.&La population française est estimée en 2010 à 65 millions d'habitants. \\
{\footnotesize \emph{Étude publiée par l'INSEE en juin 2012}}&\\
(1)  \emph{Le salaire moyen brut est le salaire non soumis aux charges}&\\ \hline
\end{tabularx} 

\medskip

\begin{tabularx}{\linewidth}{|X|}\hline
\textbf{Document 3}\\ 
\og Encore un peu moins d'argent dans le porte-monnaie des Français en 2010. Le salaire médian brut est celui qui partage la population en deux parties égales, la moitié qui gagne plus, l'autre moitié qui gagne moins ; il est égal à \np{1610}~\euro{} par mois.\\ 
Le niveau de vie des français a baissé par rapport à 2009.\\
D'ailleurs, le taux de pauvreté enregistré en cette année 2010 est le plus haut jamais observé depuis 1997. Il concerne 8,6~millions de Français qui vivent donc en dessous du seuil de pauvreté évalué à 964~\euro{} par mois. \fg\\
{\footnotesize \emph{Extrait d'un reportage diffusé sur BFM TV en septembre $2012$}} \\ \hline
\end{tabularx}

\bigskip

\begin{enumerate}
\item En France, le salaire que touche effectivement un employé est égal au salaire brut, diminué de 22\,\% et est appelé le salaire net. 

Montrer que le salaire net moyen que percevait un français en 2010 était de \np{2155,92}~\euro. 
\item Expliquer à quoi correspond le salaire médian brut. 
\item Comparer le salaire médian brut et le salaire moyen brut des Français. 

Comment peut-on expliquer cette différence ? 
\item Calculer le pourcentage de français qui vivaient en 2010 sous le seuil de pauvreté. On arrondira le résultat à l'unité. 
\end{enumerate}

\bigskip

