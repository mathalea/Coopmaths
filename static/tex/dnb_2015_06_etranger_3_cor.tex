\documentclass[10pt]{article}
\usepackage[T1]{fontenc}
\usepackage[utf8]{inputenc}%ATTENTION codage UTF8
\usepackage{fourier}
\usepackage[scaled=0.875]{helvet}
\renewcommand{\ttdefault}{lmtt}
\usepackage{amsmath,amssymb,makeidx}
\usepackage[normalem]{ulem}
\usepackage{diagbox}
\usepackage{fancybox}
\usepackage{tabularx,booktabs}
\usepackage{colortbl}
\usepackage{pifont}
\usepackage{multirow}
\usepackage{dcolumn}
\usepackage{enumitem}
\usepackage{textcomp}
\usepackage{lscape}
\newcommand{\euro}{\eurologo{}}
\usepackage{graphics,graphicx}
\usepackage{pstricks,pst-plot,pst-tree,pstricks-add}
\usepackage[left=3.5cm, right=3.5cm, top=3cm, bottom=3cm]{geometry}
\newcommand{\R}{\mathbb{R}}
\newcommand{\N}{\mathbb{N}}
\newcommand{\D}{\mathbb{D}}
\newcommand{\Z}{\mathbb{Z}}
\newcommand{\Q}{\mathbb{Q}}
\newcommand{\C}{\mathbb{C}}
\usepackage{scratch}
\renewcommand{\theenumi}{\textbf{\arabic{enumi}}}
\renewcommand{\labelenumi}{\textbf{\theenumi.}}
\renewcommand{\theenumii}{\textbf{\alph{enumii}}}
\renewcommand{\labelenumii}{\textbf{\theenumii.}}
\newcommand{\vect}[1]{\overrightarrow{\,\mathstrut#1\,}}
\def\Oij{$\left(\text{O}~;~\vect{\imath},~\vect{\jmath}\right)$}
\def\Oijk{$\left(\text{O}~;~\vect{\imath},~\vect{\jmath},~\vect{k}\right)$}
\def\Ouv{$\left(\text{O}~;~\vect{u},~\vect{v}\right)$}
\usepackage{fancyhdr}
\usepackage[french]{babel}
\usepackage[dvips]{hyperref}
\usepackage[np]{numprint}
%Tapuscrit : Denis Vergès
%\frenchbsetup{StandardLists=true}

\begin{document}
\setlength\parindent{0mm}
% \rhead{\textbf{A. P{}. M. E. P{}.}}
% \lhead{\small Brevet des collèges}
% \lfoot{\small{Polynésie}}
% \rfoot{\small{7 septembre 2020}}
\pagestyle{fancy}
\thispagestyle{empty}
% \begin{center}
    
% {\Large \textbf{\decofourleft~Brevet des collèges Polynésie 7 septembre 2020~\decofourright}}
    
% \bigskip
    
% \textbf{Durée : 2 heures} \end{center}

% \bigskip

% \textbf{\begin{tabularx}{\linewidth}{|X|}\hline
%  L'évaluation prend en compte la clarté et la précision des raisonnements ainsi que, plus largement, la qualité de la rédaction. Elle prend en compte les essais et les démarches engagées même non abouties. Toutes les réponses doivent être justifiées, sauf mention contraire.\\ \hline
% \end{tabularx}}

% \vspace{0.5cm}\textbf{\textsc{Exercice 3} \hfill 6 points}

\medskip

%Soit un cercle de diamètre [KM] avec KM = 6~cm.
%
%Soit un point L sur le cercle tel que ML = 3~cm.
%
%\medskip

\begin{enumerate}
\item %Faire une figure.
On dessine un cercle de diamètre 6~cm donc de rayon 3~cm. Le cercle de centre M et de même rayon 3~cm coupe le cercle en deux points L répondant au problème ; on en choisit un.
\item %Déterminer l'aire en cm$^2$ du triangle KLM. Donner la valeur exacte puis un arrondi au cm$^2$ près.
Le triangle KLM est inscrit dans un cercle admettant pour diamètre l'un de ses côtés ; ce triangle est donc rectangle en L, d'hypoténuse [KM].

L'aire de ce triangle est égale au demi-produit des mesures des deux côtés de l'angle droit en L :

$\mathcal{A}_{\text{KLM}} = \dfrac{\text{KL} \times \text{LM} }{2}$.

Le triangle KLM étant rectangle en L, le théorème de Pythagore permet d'élire :

$\text{KM}^2 = \text{KL}^2 + \text{LM}^2$, soit 

$6^2 = \text{KL}^2 + 3^2$ ou $\text{KL}^2 = 6^2 - 3^2 = (6 + 3)(6 - 3) = 9 \times 3$, donc 

$\text{KL} = \sqrt{9 \times 3} = \sqet{9} \times \sqrt{3} = 3\sqrt{3}$.

Donc $\mathcal{A}_{\text{KLM}} = \dfrac{3\sqrt{3} \times 3 }{2} = \dfrac{9\sqrt{3}}{2} \approx 7,79$ soit 8~cm$^2$ à 1 cm$^2$ près.
\end{enumerate}

\vspace{0,5cm}

\end{document}