\textbf{\textsc{Exercice 3} \hfill 4 points}

\medskip

\begin{enumerate}
\item Ce n'est pas une situation de proportionnalité car le graphique montrant
l'évolution de la tension en fonction du temps n'est pas une droite.
\item La tension mesurée au bout de 0,2 s, la tension mesurée est de 4,4 V.
\item Je calcule 60\,\% de la tension maximale : $\dfrac{60}{100} \times 5 = 0,6 \times 5 = 3$.

60\,\% de la tension maximale correspond à 3~V.

Par lecture graphique, on détermine que cette tension est atteinte au bout
d'environ $0,09$~s.
\end{enumerate}

\bigskip

