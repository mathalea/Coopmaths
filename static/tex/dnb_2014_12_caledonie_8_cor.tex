\documentclass[10pt]{article}
\usepackage[T1]{fontenc}
\usepackage[utf8]{inputenc}%ATTENTION codage UTF8
\usepackage{fourier}
\usepackage[scaled=0.875]{helvet}
\renewcommand{\ttdefault}{lmtt}
\usepackage{amsmath,amssymb,makeidx}
\usepackage[normalem]{ulem}
\usepackage{diagbox}
\usepackage{fancybox}
\usepackage{tabularx,booktabs}
\usepackage{colortbl}
\usepackage{pifont}
\usepackage{multirow}
\usepackage{dcolumn}
\usepackage{enumitem}
\usepackage{textcomp}
\usepackage{lscape}
\newcommand{\euro}{\eurologo{}}
\usepackage{graphics,graphicx}
\usepackage{pstricks,pst-plot,pst-tree,pstricks-add}
\usepackage[left=3.5cm, right=3.5cm, top=3cm, bottom=3cm]{geometry}
\newcommand{\R}{\mathbb{R}}
\newcommand{\N}{\mathbb{N}}
\newcommand{\D}{\mathbb{D}}
\newcommand{\Z}{\mathbb{Z}}
\newcommand{\Q}{\mathbb{Q}}
\newcommand{\C}{\mathbb{C}}
\usepackage{scratch}
\renewcommand{\theenumi}{\textbf{\arabic{enumi}}}
\renewcommand{\labelenumi}{\textbf{\theenumi.}}
\renewcommand{\theenumii}{\textbf{\alph{enumii}}}
\renewcommand{\labelenumii}{\textbf{\theenumii.}}
\newcommand{\vect}[1]{\overrightarrow{\,\mathstrut#1\,}}
\def\Oij{$\left(\text{O}~;~\vect{\imath},~\vect{\jmath}\right)$}
\def\Oijk{$\left(\text{O}~;~\vect{\imath},~\vect{\jmath},~\vect{k}\right)$}
\def\Ouv{$\left(\text{O}~;~\vect{u},~\vect{v}\right)$}
\usepackage{fancyhdr}
\usepackage[french]{babel}
\usepackage[dvips]{hyperref}
\usepackage[np]{numprint}
%Tapuscrit : Denis Vergès
%\frenchbsetup{StandardLists=true}

\begin{document}
\setlength\parindent{0mm}
% \rhead{\textbf{A. P{}. M. E. P{}.}}
% \lhead{\small Brevet des collèges}
% \lfoot{\small{Polynésie}}
% \rfoot{\small{7 septembre 2020}}
\pagestyle{fancy}
\thispagestyle{empty}
% \begin{center}
    
% {\Large \textbf{\decofourleft~Brevet des collèges Polynésie 7 septembre 2020~\decofourright}}
    
% \bigskip
    
% \textbf{Durée : 2 heures} \end{center}

% \bigskip

% \textbf{\begin{tabularx}{\linewidth}{|X|}\hline
%  L'évaluation prend en compte la clarté et la précision des raisonnements ainsi que, plus largement, la qualité de la rédaction. Elle prend en compte les essais et les démarches engagées même non abouties. Toutes les réponses doivent être justifiées, sauf mention contraire.\\ \hline
% \end{tabularx}}

% \vspace{0.5cm}\textbf{Exercice 8 : Sphères de stockage}

\begin{enumerate}
\item La plus grande sphère du dépôt a un diamètre de 19,7~m, donc un rayon de 9,85~m. \\$V_{\text{grande sphère}} = \dfrac{4}{3} \times \pi \times 9,85^3$ \\[2mm]
\phantom{$V_{\text{grande sphère}} $} $\approx 4~003$ m$^3$ \\
Le volume de stockage de la plus grande sphère du dépôt est bien d'environ \np{4000}~m$^{3}$.
\item 1 m$^3$ de butane pèse 580 kg soit 0,58 tonne. \\
On a une situation de proportionnalité : \\[2mm]
\renewcommand{\arraystretch}{1.5}
%\hspace{1.5cm} 
\begin{tabular}{|>{\raggedright\hspace{0pt}}m{3cm}|>{\centering\hspace{1pt}}m{1.7cm}|>{\centering\hspace{0pt}}m{1.7cm}|}
			\hline
			\textbf{Volume en m$^3$} & 1 & $V$  \tabularnewline
			\hline
			\textbf{Masse en tonne} & 0,58 &  1~200  \tabularnewline
			\hline
			\end{tabular}

Le volume $V$ correspondant aux \np{1200}~tonnes est :  $V=\dfrac{1\times 1~200}{0,58}\approx 2~069$ m$^3$. 
\item  Le volume total des deux plus petites sphères est de $1~000+600=1~600$ m$^3$. \\
Ce volume est inférieur aux \np{2069} m$^3$ correspondant à \np{1200}~tonnes de \linebreak butane, donc la grande sphère sera nécessaire.
\end{enumerate}
\end{document}\end{document}