\documentclass[10pt]{article}
\usepackage[T1]{fontenc}
\usepackage[utf8]{inputenc}%ATTENTION codage UTF8
\usepackage{fourier}
\usepackage[scaled=0.875]{helvet}
\renewcommand{\ttdefault}{lmtt}
\usepackage{amsmath,amssymb,makeidx}
\usepackage[normalem]{ulem}
\usepackage{diagbox}
\usepackage{fancybox}
\usepackage{tabularx,booktabs}
\usepackage{colortbl}
\usepackage{pifont}
\usepackage{multirow}
\usepackage{dcolumn}
\usepackage{enumitem}
\usepackage{textcomp}
\usepackage{lscape}
\newcommand{\euro}{\eurologo{}}
\usepackage{graphics,graphicx}
\usepackage{pstricks,pst-plot,pst-tree,pstricks-add}
\usepackage[left=3.5cm, right=3.5cm, top=3cm, bottom=3cm]{geometry}
\newcommand{\R}{\mathbb{R}}
\newcommand{\N}{\mathbb{N}}
\newcommand{\D}{\mathbb{D}}
\newcommand{\Z}{\mathbb{Z}}
\newcommand{\Q}{\mathbb{Q}}
\newcommand{\C}{\mathbb{C}}
\usepackage{scratch}
\renewcommand{\theenumi}{\textbf{\arabic{enumi}}}
\renewcommand{\labelenumi}{\textbf{\theenumi.}}
\renewcommand{\theenumii}{\textbf{\alph{enumii}}}
\renewcommand{\labelenumii}{\textbf{\theenumii.}}
\newcommand{\vect}[1]{\overrightarrow{\,\mathstrut#1\,}}
\def\Oij{$\left(\text{O}~;~\vect{\imath},~\vect{\jmath}\right)$}
\def\Oijk{$\left(\text{O}~;~\vect{\imath},~\vect{\jmath},~\vect{k}\right)$}
\def\Ouv{$\left(\text{O}~;~\vect{u},~\vect{v}\right)$}
\usepackage{fancyhdr}
\usepackage[french]{babel}
\usepackage[dvips]{hyperref}
\usepackage[np]{numprint}
%Tapuscrit : Denis Vergès
%\frenchbsetup{StandardLists=true}

\begin{document}
\setlength\parindent{0mm}
% \rhead{\textbf{A. P{}. M. E. P{}.}}
% \lhead{\small Brevet des collèges}
% \lfoot{\small{Polynésie}}
% \rfoot{\small{7 septembre 2020}}
\pagestyle{fancy}
\thispagestyle{empty}
% \begin{center}
    
% {\Large \textbf{\decofourleft~Brevet des collèges Polynésie 7 septembre 2020~\decofourright}}
    
% \bigskip
    
% \textbf{Durée : 2 heures} \end{center}

% \bigskip

% \textbf{\begin{tabularx}{\linewidth}{|X|}\hline
%  L'évaluation prend en compte la clarté et la précision des raisonnements ainsi que, plus largement, la qualité de la rédaction. Elle prend en compte les essais et les démarches engagées même non abouties. Toutes les réponses doivent être justifiées, sauf mention contraire.\\ \hline
% \end{tabularx}}

% \vspace{0.5cm}\textbf{\textsc{Exercice 6} \hfill 6 points}

\medskip
 
Pour préparer son voyage à Marseille, Julien utilise un site Internet pour choisir le meilleur itinéraire. 

\begin{center}
\begin{tabularx}{\linewidth}{|l|m{1cm}|l X|}\hline
Calculez votre itinéraire&&\multicolumn{2}{|l|}{\textbf{59 000 Lille--13000 Marseille}}\\
\textbf{Départ}&&Co\^ut estimé	&Péage 73,90~\euro\\
59 000 Lille  France&&&Carburant 89,44~\euro\\
&&Temps&8~h~47 dont\\
&&&8~h~31 sur autoroute\\
\textbf{Arrivée}&&&\\
13 000 Marseille France&&Distance&\np{1004}~km dont\\
&&&993~km sur autoroute\\ \hline
\end{tabularx}
\end{center}

\begin{enumerate}
\item Quelle vitesse moyenne, arrondie au km/h, cet itinéraire prévoit-il pour la portion de trajet sur autoroute ? 

\textit{Il parcourt 993 km en 8h31, c'est-à-dire en $8\times 60+31=511$ min.}

\textit{Sa vitesse moyenne $v$ est donc :} $v=\dfrac{993~\text{km}}{511~\text{min}}\times 60~\text{min/h}$\fbox{$\approx 117$ km/h} 
\item Sachant que la sécurité routière préconise au moins une pause de 10 à 20 minutes toutes les deux heures de conduite, quelle doit être la durée minimale que Julien doit prévoir pour son voyage ?

\textit{Dans 8h47, il y a 4 fois 2 heures, donc à rajouter au minimum 40 minutes, soit une durée de 9h27} 

\medskip

 
\item %\textbf{Pour cette question, faire apparaître sur la copie la démarche utilisée. Toute trace de recherche sera prise en compte lors de l'évaluation même si le travail n'est pas complètement abouti.}

Sachant que le réservoir de sa voiture a une capacité de 60~L et qu'un litre d'essence coûte 1,42~\euro, peut-il faire le trajet avec un seul plein d'essence en se fiant aux données du site internet ?

\textit{D'après les données du site internet, le volume d'essence qu'il dépensera sera de} $\dfrac{89,44}{1,42}\:$\fbox{$\approx 63$~L} \textit{arrondi au litre, soit plus d'un réservoir.}
\end{enumerate}
 
\vspace{0,5cm}

\end{document}