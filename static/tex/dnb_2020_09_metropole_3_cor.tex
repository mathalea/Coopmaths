
\medskip

Une entreprise fabrique des portiques pour installer des balançoires sur des aires de jeux.

\medskip

\begin{tabularx}{\linewidth}{|X m{5cm}|}\hline
\multicolumn{2}{|l|}{\textbf{Document 1 : croquis d'un portique}}\\
Vue d'ensemble&Vue de côté\\
\psset{unit=1cm}
\begin{pspicture}(6.5,4.5)
%\psgrid
\psline[linewidth=1.2pt](0.3,1)(1,3.6)(4.5,3.6)(6.4,0.2)
\psline(1,3.6)(2.9,0.2)\psline(4.5,3.6)(3.8,1)
\psline[linewidth=1.2pt,linestyle=dashed](0.7,2.3)(1.85,2.1)
\psline[linewidth=1.2pt,linestyle=dashed](4.2,2.3)(5.35,2.1)
\psline{<->}(1,3.8)(4.5,3.8)\uput[u](2.75,3.8){384 cm}
\uput[ul](1,3.6){A}\uput[dr](0.3,1){B}\uput[dr](2.9,0.2){C}
\end{pspicture}&\vspace*{-2cm}
\psset{unit=1cm}
\begin{pspicture}(-2,0)(2,4.5)
%\psgrid
\psline[linewidth=1.2pt](-1,1)(0,4.3)(1,1)
\psline[linewidth=1.2pt,linestyle=dashed](-0.5,2.8)(0.5,2.8)
\psline[linewidth=1.2pt,linestyle=dashed](0,4.3)(0,1)
\psline[linewidth=1.2pt,linestyle=dashed,dash=4pt 2pt](-1,1)(1,1)
\psframe(0,1)(0.25,1.25)
\psline{<->}(0.5,4.35)(1.8,1)\uput[r](1.15,2.7){342 cm}
\psline{<->}(-1,0.5)(1,0.5)\uput[d](0,0.5){290 cm}
\uput[ul](0,4.3){A}\uput[dl](-1,1){B}\uput[dr](1,1){C}
\uput[l](-0.55,2.8){M}\uput[r](0.55,2.8){N}\uput[dr](0,1){H}
\end{pspicture}\\
\psline[linewidth=1.2pt](0,0)(1,0) \qquad \qquad \qquad : poutres en bois de diamètre 100 mm 

\psline[linewidth=1.2pt,linestyle=dashed](0,0)(1,0) \qquad \qquad \qquad : barres de maintien latérales en bois.&
ABC est un triangle isocèle en A.

H est le milieu de [BC]
 
(MN)est parallèle à (BC).\\ \hline
\end{tabularx}

\bigskip

\begin{tabularx}{\linewidth}{|X X|}\hline
\multicolumn{2}{|l|}{\textbf{Document 2 : coût du matériel} }\\
\vspace*{-4cm}Poutres en bois de diamètre 100 mm :

-- Longueur 4 m : 12,99 \euro{} l'unité ;

-- Longueur 3,5 m : 11,75 \euro{} l'unité ;

-- Longueur 3 m : 10,25 \euro{} l'unité.

Barres de maintien latérales en bois:

-- Longueur 3 m : 6,99 \euro{} l'unité ;

-- Longueur 2 m : 4,75 \euro{} l'unité;

-- Longueur 1,5 m : 3,89 \euro{} l'unité.&\psset{unit=1cm}
\begin{pspicture}(6.5,4.5)
%\psgrid
\psline[linewidth=1.2pt](0.3,1)(1,3.6)(4.8,3.6)(5.8,0.6)
\psline(1,3.6)(2,0.6)\psline(4.8,3.6)(4.1,1)
\psline[linewidth=1.2pt,linestyle=dashed](0.7,2.3)(1.42,2.1)
\psline[linewidth=1.2pt,linestyle=dashed](4.5,2.3)(5.3,2.1)
\psline(2.2,3.6)(2.2,1.8)(2,1.6)\psline(2.7,3.6)(2.7,1.8)(2.5,1.6)
\psline(2.2,1.8)(2.3,1.4)\psline(2.7,1.8)(2.8,1.4)
\pspolygon(1.9,1.7)(2.7,1.7)(2.9,1.35)(2.1,1.35)%balancegauche
\psline(3.4,3.6)(3.4,1.8)(3.2,1.6)\psline(3.9,3.6)(3.9,1.8)(3.7,1.6)
\psline(3.4,1.8)(3.5,1.4)\psline(3.9,1.8)(4,1.4)
\pspolygon(3.1,1.7)(3.9,1.7)(4.1,1.35)(3.3,1.35)%balancedroite
\end{pspicture}
\\
&\\
\multicolumn{2}{|l|}{Ensemble des fixations nécessaires pour un portique: 80 \euro.}\\
\multicolumn{2}{|l|}{Ensemble de deux balançoires pour un portique : 50 \euro.}\\ \hline
\end{tabularx}

\medskip

\begin{enumerate}
\item %Déterminer la hauteur AH du portique, arrondie au cm près.
Dans le triangle ABC isocèle en A, la hauteur (AH) est aussi la médiane, donc BH = HC $= \dfrac{290}{2} = 145$.

Le théorème de Pythagore appliqué au triangle ACH rectangle en H s'écrit :

$\text{AC}^2 = \text{AH}^2 + \text{HC}^2$, soit $342^2 = \text{AH}^2 + 145^2$.

Donc $\text{AH}^2 = 342^2 - 145^2 = (342 + 145)\times (342 - 145) = 487 \times 197 = \np{95939}$.

Conclusion AH $ = \sqrt{\np{95939}} \approx 309,74$, soit 310~cm au centimètre près.
\item %Les barres de maintien doivent être fixées à 165 cm du sommet (AN $= 165$ cm).
%Montrer que la longueur MN de chaque barre de maintien est d'environ $140$ cm.
On a avec (MN) parallèle à (BC) une situation de Thalès. On peut donc écrire :

$\dfrac{\text{AN}}{\text{AC}} = \dfrac{\text{MN}}{\text{BC}}$ ou $\dfrac{165}{342} = \dfrac{\text{MN}}{290}$. On en déduit en multipliant chaque membre par 290 :

MN $ = 290 \times \dfrac{165}{342} = \dfrac{290 \times 165}{342} = \dfrac{2 \times 145 \times 3 \times 55}{2 \times 3 \times 57} = \dfrac{145 \times 55}{57} \approx 139,9$ soit environ 140~cm au centimètre près.
\item %Montrer que le coût minimal d'un tel portique équipé de balançoires s'élève à 196,98 \euro.
Il faut :

-- pour la poutre principale 1 poutre de 4 m ;

-- pour les pieds 4 poutres de 3,5 m ;

-- pour le maintien 2 barres de 1,5 m, soit :

$12,99 + 4 \times 11,75 + 2 \times 3,89 = 12,99 + 47 + 7,68 = 66,67$~(\euro), plus les fixations et les deux balançoires, soit :

$66,67 + 80 + 50 = 197,67$~(\euro).
Ce n'est pas le coût minimal car, pour les barres de maintien au lieu de prendre 2 barres de 1,5~m à 3,89~\euro{}, on peut en prendre une de 3~m à 6,99~\euro{} et la couper en deux.

Le coût est alors : 

$12,99 + 4 \times 11,75 +  6,99 + 80 + 50 = 196,98$~(euro).
\item %L'entreprise veut vendre ce portique équipé 20\,\% plus cher que son coût minimal. Déterminer ce prix de vente arrondi au centime près.
Ajouter 20\,\%, c'est multiplier par $1 + \dfrac{20}{100} = 1 + 0,20 = 1,2$.

Le prix de vente sera donc : $196,98 \times 1,2 = 236,376 \approx 236,38$~(\euro).
\item %Pour des raisons de sécurité, l'angle $\widehat{\text{BAC}}$ doit être compris entre 45\degres et 55\degres. 

%Ce portique respecte-t-il cette condition ?
Dans le triangle rectangle en H, AHC, on a :

$\sin \widehat{\text{HAC}} = \dfrac{\text{HC}}{\text{AC}} = \dfrac{145}{342} \approx \np{0,423977}$.

Avec la touche \fbox{$\sin^{-1} $}, on obtient $\widehat{\text{HAC}} \approx \np{25,0859}$.

La triangle BAC étant isocèle en A, on a donc $\widehat{\text{BAC}}  = 2 \times \widehat{\text{HAC}} \approx 50,17$, donc le portique respecte la condition de sécurité.
\end{enumerate}

\bigskip

