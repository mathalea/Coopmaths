
\medskip 

%Cet exercice est un questionnaire à choix multiples. Pour chaque question, une seule réponse
%est correcte.
%
%Pour chacune des questions, écrire sur la copie, le numéro de la question et la lettre de la bonne
%réponse.
%
%\emph{Aucune justification n'est attendue.}
%
%\begin{center}
%\begin{tabularx}{\linewidth}{|c|m{6cm}|*{3}{>{\centering \arraybackslash}X|}}\hline
%	&		&Réponse A &Réponse B &Réponse C\\ \hline
%1	&ABC est un triangle rectangle en A.
%
%AC = 3,5 cm et BC = 7 cm. La mesure de l'angle $\widehat{\text{ABC}}$ est :&30\degres &45\degres&60\degres\\ \hline
%2	&\psset{unit=0.8cm}
%\begin{pspicture}(-2.9,-2.5)(2.9,2.5)
%\psdots[dotstyle=+,dotangle=45,dotscale=1.6](0,0)\uput[dr](0,0){O}
%\psline(-1,1.75)(-2.42,-1)(-1,-1.75)%BAC
%\psline(1,1.75)(2.42,1)(1,-1.75)%FDE
%\psline(-1,1.75)(-1,-1.75)
%\psline(1,1.75)(1,-1.75)
%\psarc(-1,1.75){0.5}{-118}{-90}
%\uput[d](-1.24,1){\small 35\degres}
%\rput{-27}(-2.42,-1){\psframe(0.3,0.3)}
%\uput[l](-2.42,-1){A} \uput[u](-1,1.75){B} \uput[d](-1,-1.75){C} 
%\uput[r](2.42,1){D} \uput[d](1,-1.75){E} \uput[u](1,1.75){F}
%\end{pspicture}
%
%Le triangle DEF est le symétrique du triangle ABC par rapport au point O.
%La mesure de l'angle $\widehat{\text{DEF}}$ est:&35\degres &55\degres& 65\degres\\ \hline
%3	&\psset{unit=1cm}
%\begin{pspicture}(0,-0.2)(5.8,3.4)
%%\psgrid
%\def\tri{\pspolygon(0,0)(2.5,-0.2)(0.75,1)}
%\rput(0,0.5){\tri}
%\rput(1,0){Figure 1}\rput(3.5,1.2){Figure 2}
%\psset{unit=1.3cm}
%\rput(1.8,1.4){\tri}
%\end{pspicture}
%La transformation utilisée pour obtenir la
%figure 2 à partir de la figure 1 est une :&translation &homothétie &rotation\\ \hline
%\end{tabularx}
%\end{center}
\begin{enumerate}
\item Réponse A : 30\degres (le triangle rectangle est un demi-triangle équilatéral).
\item Réponse A : 35\degres ($\widehat{\text{DEF}}$ et  $\widehat{\text{ABC}}$ sont symétriques autour de O).
\item Réponse B : une homothétie
\end{enumerate}

\medskip

