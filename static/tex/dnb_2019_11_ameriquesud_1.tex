
\medskip 

Pour chacune des affirmations suivantes, indiquer sur la copie, si elle est vraie ou fausse.

On rappelle que chaque réponse doit être justifiée.

\begin{itemize}[label=$\bullet~~$]
\item \textbf{Affirmation \no 1}

\og Dans la série de valeurs ci-dessous, l'étendue est 25. 

Série : 37~;~20~;~18~;~25~;~45~;~94~;~62 \fg.

\item \textbf{Affirmation \no 2}

\og Les nombres 70 et 90 ont exactement deux diviseurs premiers en commun \fg.

\item \textbf{Affirmation \no 3}

\parbox{0.6\linewidth}
{
\og À partir du quadrilatère BUTS, on a obtenu le quadrilatère VRAC par une translation \fg.
}
\hfill 
\parbox{0.38\linewidth}
{
\psset{unit=1cm}
\def\poly{\pspolygon[fillstyle=solid,fillcolor=lightgray](0,0)(1,0)(1,2)(0,1)}
\begin{pspicture}(5.5,3.8)
\rput(0.3,0.3){\poly}\rput{180}(4.6,3.3){\poly}
\uput[dl](0.3,0.3){C}\uput[dr](1.3,0.3){A} \uput[ur](1.3,2.3){R} \uput[ul](0.3,1.3){V} 
\uput[ur](4.6,3.3){U} \uput[ul](3.6,3.3){B} \uput[dl](3.6,1.3){S} \uput[dr](4.6,2.3){T} 
\end{pspicture}
}

\item \textbf{Affirmation \no 4}

\og Quand on multiplie l'arête d'un cube par 3, son volume est multiplié par 27 \fg.
\end{itemize}

\vspace{0,5cm}

