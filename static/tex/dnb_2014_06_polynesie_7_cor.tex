\documentclass[10pt]{article}
\usepackage[T1]{fontenc}
\usepackage[utf8]{inputenc}%ATTENTION codage UTF8
\usepackage{fourier}
\usepackage[scaled=0.875]{helvet}
\renewcommand{\ttdefault}{lmtt}
\usepackage{amsmath,amssymb,makeidx}
\usepackage[normalem]{ulem}
\usepackage{diagbox}
\usepackage{fancybox}
\usepackage{tabularx,booktabs}
\usepackage{colortbl}
\usepackage{pifont}
\usepackage{multirow}
\usepackage{dcolumn}
\usepackage{enumitem}
\usepackage{textcomp}
\usepackage{lscape}
\newcommand{\euro}{\eurologo{}}
\usepackage{graphics,graphicx}
\usepackage{pstricks,pst-plot,pst-tree,pstricks-add}
\usepackage[left=3.5cm, right=3.5cm, top=3cm, bottom=3cm]{geometry}
\newcommand{\R}{\mathbb{R}}
\newcommand{\N}{\mathbb{N}}
\newcommand{\D}{\mathbb{D}}
\newcommand{\Z}{\mathbb{Z}}
\newcommand{\Q}{\mathbb{Q}}
\newcommand{\C}{\mathbb{C}}
\usepackage{scratch}
\renewcommand{\theenumi}{\textbf{\arabic{enumi}}}
\renewcommand{\labelenumi}{\textbf{\theenumi.}}
\renewcommand{\theenumii}{\textbf{\alph{enumii}}}
\renewcommand{\labelenumii}{\textbf{\theenumii.}}
\newcommand{\vect}[1]{\overrightarrow{\,\mathstrut#1\,}}
\def\Oij{$\left(\text{O}~;~\vect{\imath},~\vect{\jmath}\right)$}
\def\Oijk{$\left(\text{O}~;~\vect{\imath},~\vect{\jmath},~\vect{k}\right)$}
\def\Ouv{$\left(\text{O}~;~\vect{u},~\vect{v}\right)$}
\usepackage{fancyhdr}
\usepackage[french]{babel}
\usepackage[dvips]{hyperref}
\usepackage[np]{numprint}
%Tapuscrit : Denis Vergès
%\frenchbsetup{StandardLists=true}

\begin{document}
\setlength\parindent{0mm}
% \rhead{\textbf{A. P{}. M. E. P{}.}}
% \lhead{\small Brevet des collèges}
% \lfoot{\small{Polynésie}}
% \rfoot{\small{7 septembre 2020}}
\pagestyle{fancy}
\thispagestyle{empty}
% \begin{center}
    
% {\Large \textbf{\decofourleft~Brevet des collèges Polynésie 7 septembre 2020~\decofourright}}
    
% \bigskip
    
% \textbf{Durée : 2 heures} \end{center}

% \bigskip

% \textbf{\begin{tabularx}{\linewidth}{|X|}\hline
%  L'évaluation prend en compte la clarté et la précision des raisonnements ainsi que, plus largement, la qualité de la rédaction. Elle prend en compte les essais et les démarches engagées même non abouties. Toutes les réponses doivent être justifiées, sauf mention contraire.\\ \hline
% \end{tabularx}}

% \vspace{0.5cm}\textbf{Exercice 7 \hfill 6 points}

\medskip

%Dans tout cet exercice, on travaille avec des triangles ABC isocèles en A tels que : BC = 5 cm. La mesure de l'angle $\widehat{\text{ABC}}$ peut varier. 
%
%\smallskip
%
%On va alors s'intéresser aux angles extérieurs de ces triangles, c'est-à-dire, comme l'indique la figure ci-après, aux angles qui sont supplémentaires et adjacents avec les angles de ce triangle.
%
%\begin{center}
%\psset{unit=1cm}
%\begin{pspicture}(12,7.5)
%%\psgrid 
%\pspolygon(4,2)(6.3,2.8)(4,5.7)
%\uput[l](4,2){C} \uput[dr](6.3,2.8){B} \uput[ur](4,5.7){A}
%\rput(2.6,6.9){$x$}
%\rput(9.,3.6){$y$}
%\rput(3.85,0.2){$z$}
%\psline(3.9,3.7)(4.1,3.5)\psline(3.9,3.6)(4.1,3.4)
%\psline(4.9,4.4)(5.1,4.6)\psline(4.9,4.3)(5.1,4.5)
%\psline(6.3,2.8)(8.6,3.6)
%\psline(4,5.7)(2.85,7.15)
%\psline(4,2)(4,0.1)
%\psarc(4,2){5mm}{-90}{26}
%\psarc(6.3,2.8){5mm}{26}{122}
%\psarc(4,5.7){5mm}{125}{270}
%\rput(1.6,6){Angle extérieur}\psline{->}(5,1)(4.5,1.6) 
%\rput(7,4.8){Angle extérieur}\psline{->}(6.8,4.6)(6.4,3.4) 
%\rput(6.2,1){Angle extérieur}\psline{->}(2.7,5.8)(3.4,5.5)
%\end{pspicture}
%\end{center}

\begin{enumerate}
\item %Dans cette question uniquement, on suppose que $\widehat{\text{ABC}} = 40~\degres$. 
	\begin{enumerate}
		\item %Construire le triangle ABC en vraie grandeur. Aucune justification n'est attendue pour cette construction.
		Le triangle étant isocèle en A on a donc  $\widehat{\text{ABC}} = \widehat{\text{ACB}} = 40\degres$.
		
		On trace donc un segment [BC] de 5 cm et à chaque extrémité les deux angles de même mesure 40\degres. Les deux demi-droites tracées sont sécantes en A.
\begin{center}
\psset{unit=1cm}
\begin{pspicture}(-2,-0.4)(7.5,5)
%\psgrid
\psline(-2,0)(7.5,0) \uput[d](0.3,0){C} \uput[d](5.3,0){B}\uput[u](2.8,2.07){A}
\psarc(0.3,0){0.5cm}{0}{40}\psarc(5.3,0){0.5cm}{140}{180}
%\psarc(0.3,0){5cm}{0}{40}\psarc(5.3,0){5cm}{140}{180}
\psline(0.3,0)(4.15,3.2)
\psline(5.3,0)(1.45,3.2)
\uput[ul](4.15,3.2){$x$} \uput[u](-2,0){$y$}\uput[d](7.5,0){$z$}
\uput[u](2.8,0){5 cm}\rput(1.3,0.4){40\degres}\rput(4.3,0.4){40\degres}
\end{pspicture}
\end{center}		
		
		\item %Calculer la mesure de chacun de ses 3 angles extérieurs.
		
On a $\widehat{\text{BAC}} = 180 - (40 + 40) = 180 - 80 = 100\degres$. Donc :

On a $\widehat{x\text{AB}} = 180 - 100 = 80\degres$ ;

$\widehat{\text{AC}y} = 180 - 40 = 140\degres$ ;

$\widehat{\text{AB}z} = 180 - 40 = 140\degres$ ; 
		\item %Vérifier que la somme des mesures de ces 3 angles extérieurs est égale à 360~\degres.
		On a bien $80 + 140 + 140 = 360\degres$.
	\end{enumerate} 
\item %Est-il possible de construire un triangle ABC isocèle en A tel que la somme des mesures de ses trois angles extérieurs soit différente de $360$~\degres ?
Si l'on reprend les mêmes calculs avec un triangle isocèle dont les deux angles de même mesure ont pour mesure $a$, on a : 

$\widehat{\text{BAC}} = 180 - (a + a) = 180 - 2a\degres$. Donc :

On a $\widehat{x\text{AB}} = 180 - (180 - 2a) = 2a\degres$ ;

$\widehat{\text{AB}y} = 180 - a\degres$ ;

$\widehat{\text{AB}z} = 180 - a\degres$ ; 
La somme est donc égale à :

$2a	 + 180 - a + 180 - a = 360$.

Conclusion il n'est pas possible de construire un triangle ABC isocèle en A tel que la somme des mesures de ses trois angles extérieurs soit différente de $360$\degres.
\end{enumerate}
\end{document}\end{document}