
\medskip

Pour chaque affirmation, dire en justifiant, si elle est vraie ou fausse.

\bigskip

%\begin{tabularx}{\linewidth}{l X}
%\textbf{Affirmation 1 :} &~\\
%&\begin{tabular}{|l|}\hline
%\textbf{Programme de calcul A}\\
%Choisir un nombre\\
%Ajouter 3\\
%Multiplier le résultat par 2\\
%Soustraire le double du nombre de départ\\ \hline
%\end{tabular}\\
%&Le résultat du programme de calcul A est toujours égal à  6.\\
%\textbf{Affirmation 2 :} &Le résultat du calcul $\dfrac{7}{5} - \dfrac{4}{5} \times \dfrac{1}{3}$ est égal à  $\dfrac{1}{5}$.\\
%\textbf{Affirmation 3 :} &La solution de l'équation $4x - 5 = x + 1$ est une solution de l'équation $x^2 - 2x = 0$.\\
%\textbf{Affirmation 4 :} &Pour tous les nombres entiers $n$ compris entre $2$ et $9$, $2^n - 1$ est un nombre premier.\\
%\end{tabularx}
\textbf{Affirmation 1 :} 0 donne  3 puis 6  puis 6

1 donne 4 puis 8  et enfin 6.

$n$ donne $n + 3$ puis $2n + 6$ et enfin $2n + 6 - 2n = 6$. L'affirmation est vraie quel que soit le nombre $n$.

\textbf{Affirmation 2 :}

$\dfrac{7}{5} - \dfrac{4}{5} \times \dfrac{1}{3} = \dfrac{7}{5} - \dfrac{4}{15}  = \dfrac{21}{15} - \dfrac{4}{15} = \dfrac{17}{15}$. L'affirmation est fausse.

\textbf{Affirmation 3 :}

$4x - 5 = x + 1$ donne $4x - x = 1 + 5$, soit $3x = 6$ et enfin $x = 2$.

Or $2^2 - 2\times = 0$, donc 2 est une solution de l'équation $x^2 - 2x = 0$. L'affirmation est vraie.

\textbf{Affirmation 4 :}

$2^3 - 1 = 7$ qui est premier ;

$2^4 - 1 = 15$ qui est divisible par 3 et par 5  : il n'est pas premier. L'affirmation est fausse.

\vspace{0,5cm}


