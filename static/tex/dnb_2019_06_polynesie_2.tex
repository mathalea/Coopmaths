
\medskip

\begin{enumerate}
\item On a utilisé une feuille de calcul pour obtenir les images de différentes valeurs de $x$ par une fonction affine $f$.

Voici une copie de l'écran obtenu :

\begin{center}
\begin{tabularx}{\linewidth}{|c|*{8}{>{\centering \arraybackslash}X|}}\hline
\multicolumn{2}{|l|}{B2}&\multicolumn{7}{l|}{=3*B1$-4$}\\ \hline
	&A		&B		&C		&D		&E		&F	&G	&H\\ \hline
1	&$x$	&$-2$	&$-1$	&0		&1		&2	&3	&4\\ \hline
2	&$f(x)$	&\multicolumn{1}{>{\columncolor{lightgray}}c|}{$- 10$}	&$- 7$	&$- 4$	&$- 1$	&2	&5	&8\\ \hline
\end{tabularx}
\end{center}

	\begin{enumerate}
		\item Quelle est l'image de $- 1$ par la fonction $f$ ?
		\item Quel est l'antécédent de $5$ par la fonction $f$?
		\item Donner l'expression de $f(x)$.
		\item Calculer $f(10)$.
 	\end{enumerate}
\item  On donne le programme suivant qui traduit un programme de calcul.

\begin{center}
\begin{scratch}
\blockinit{Quand \greenflag est cliqué}
\blocksensing{demander \txtbox{Choisir un nombre} et attendre}
\blockmove{mettre \ovalnum{A \selectarrownum} à \ovalvariable{réponse}}
\blockmove{mettre \ovalnum{A\selectarrownum} à \ovalnum{\ovalnum{A} + \ovalnum{3}}}
\blockmove{mettre \ovalnum{A\selectarrownum} à \ovalnum{\ovalnum{A} * \ovalnum{2}}}
\blockmove{mettre \ovalnum{A\selectarrownum} à \ovalnum{\ovalnum{A} $- \ovalnum{5}$}}
\blockmove{dire\ovalnum{regroupe} {\txtbox{Le programme de calcul donne }}\ovalnum{A}}
\end{scratch}
\end{center}
	\begin{enumerate}
		\item Écrire sur votre copie les deux dernières étapes du programme de calcul:
\begin{center}
\begin{tabularx}{0.4\linewidth}{|X|}\hline
$\bullet~~$ Choisir un nombre.\\
$\bullet~~$ Ajouter 3 à ce nombre.\\
$\bullet~~$ \ldots\\
$\bullet~~$ \ldots\\ \hline
\end{tabularx}
\end{center}
		\item  Si on choisit le nombre $8$ au départ, quel sera le résultat ?
		\item  Si on choisit $x$ comme nombre de départ, montrer que le résultat obtenu avec ce programme de calcul sera $2x + 1$.
		\item Quel nombre doit-on choisir au départ pour obtenir 6 ?
 	\end{enumerate}
\item  Quel nombre faudrait-il choisir pour que la fonction $f$ et le programme de calcul
donnent le même résultat ?
\end{enumerate}

\bigskip

