\textbf{Exercice 8 : Clip musical \hfill 7 points}

\medskip

%Un site internet propose de télécharger légalement des clips vidéos. Pour cela, sur la page d'accueil, trois choix s'offrent à nous:
%
%\setlength\parindent{8mm}
%\begin{itemize}
%\item[$\bullet~~$] Premier choix : téléchargement \textbf{direct sans inscription}. Avec ce mode, chaque clip peut être téléchargé pour 4~euros.
%\item[$\bullet~~$] Deuxième choix: téléchargement \textbf{membre}. Ce mode nécessite une inscription à 10~euros.
%valable un mois et permet d'acheter par la suite chaque clip pour 2 euros.
%\item[$\bullet~~$] Troisième choix : téléchargement \textbf{premium}. Une inscription à 50~euros permettant de télécharger tous les clips gratuitement pendant un mois.
%\end{itemize}
%\setlength\parindent{0mm}
%
%\medskip

\begin{enumerate}
\item %Je viens pour la première fois sur ce site et je souhaite télécharger un seul clip.

%Quel est le choix le moins cher ?
Pour télécharger un seul titre le moins cher est le direct sans inscription : 4~\euro.
\item Pour cette question, utiliser l'annexe 1.
	\begin{enumerate}
		\item %Compléter le tableau.
Voir à la fin.
		\item %À partir de combien de clips devient-il intéressant de s'inscrire en tant que membre ?
		Le tableau montre que pour 5 téléchargements les deux premières possibilités coûtent 20~\euro. Donc à partir de $x = 6$, il devient intéressant de prendre l’abonnement membre
	\end{enumerate}
\item £Dans cette question, $x$ désigne le nombre de clips vidéos achetés.
	
%$f,\: g$ et $h$ sont trois fonctions définies par :
	
%\setlength\parindent{8mm}
%\begin{itemize}
%\item[$\bullet~~$]$f(x) = 50$
%\item[$\bullet~~$]$g(x) = 4x$
%\item[$\bullet~~$]$h(x) = 2x + 10$
%\end{itemize}
%\setlength\parindent{0mm}

	\begin{enumerate}
		\item %Associer chacune de ces fonctions au choix qu'elle représente (direct, membre ou premium).
$f$ correspond à l’abonnement premium.

$g$ correspond au téléchargement direct sans abonnement.

$h$ correspond à l’abonnement membre.
		\item %Dans le repère de l'annexe 2, tracer les droites représentant les fonctions $f,\: g$ et $h$.
Voir à la fin.
		\item %À l'aide du graphique, déterminer le nombre de clips à partir duquel l'offre premium devient la moins chère.
		Pour 20 téléchargement les deux abonnements reviennent au même prix. À partir de 21 téléchargements l’abonnement premium est la solution la moins onéreuse.
	\end{enumerate}
\end{enumerate}

\vspace{0,5cm}

