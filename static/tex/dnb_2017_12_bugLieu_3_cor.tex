\documentclass[10pt]{article}
\usepackage[T1]{fontenc}
\usepackage[utf8]{inputenc}%ATTENTION codage UTF8
\usepackage{fourier}
\usepackage[scaled=0.875]{helvet}
\renewcommand{\ttdefault}{lmtt}
\usepackage{amsmath,amssymb,makeidx}
\usepackage[normalem]{ulem}
\usepackage{diagbox}
\usepackage{fancybox}
\usepackage{tabularx,booktabs}
\usepackage{colortbl}
\usepackage{pifont}
\usepackage{multirow}
\usepackage{dcolumn}
\usepackage{enumitem}
\usepackage{textcomp}
\usepackage{lscape}
\newcommand{\euro}{\eurologo{}}
\usepackage{graphics,graphicx}
\usepackage{pstricks,pst-plot,pst-tree,pstricks-add}
\usepackage[left=3.5cm, right=3.5cm, top=3cm, bottom=3cm]{geometry}
\newcommand{\R}{\mathbb{R}}
\newcommand{\N}{\mathbb{N}}
\newcommand{\D}{\mathbb{D}}
\newcommand{\Z}{\mathbb{Z}}
\newcommand{\Q}{\mathbb{Q}}
\newcommand{\C}{\mathbb{C}}
\usepackage{scratch}
\renewcommand{\theenumi}{\textbf{\arabic{enumi}}}
\renewcommand{\labelenumi}{\textbf{\theenumi.}}
\renewcommand{\theenumii}{\textbf{\alph{enumii}}}
\renewcommand{\labelenumii}{\textbf{\theenumii.}}
\newcommand{\vect}[1]{\overrightarrow{\,\mathstrut#1\,}}
\def\Oij{$\left(\text{O}~;~\vect{\imath},~\vect{\jmath}\right)$}
\def\Oijk{$\left(\text{O}~;~\vect{\imath},~\vect{\jmath},~\vect{k}\right)$}
\def\Ouv{$\left(\text{O}~;~\vect{u},~\vect{v}\right)$}
\usepackage{fancyhdr}
\usepackage[french]{babel}
\usepackage[dvips]{hyperref}
\usepackage[np]{numprint}
%Tapuscrit : Denis Vergès
%\frenchbsetup{StandardLists=true}

\begin{document}
\setlength\parindent{0mm}
% \rhead{\textbf{A. P{}. M. E. P{}.}}
% \lhead{\small Brevet des collèges}
% \lfoot{\small{Polynésie}}
% \rfoot{\small{7 septembre 2020}}
\pagestyle{fancy}
\thispagestyle{empty}
% \begin{center}
    
% {\Large \textbf{\decofourleft~Brevet des collèges Polynésie 7 septembre 2020~\decofourright}}
    
% \bigskip
    
% \textbf{Durée : 2 heures} \end{center}

% \bigskip

% \textbf{\begin{tabularx}{\linewidth}{|X|}\hline
%  L'évaluation prend en compte la clarté et la précision des raisonnements ainsi que, plus largement, la qualité de la rédaction. Elle prend en compte les essais et les démarches engagées même non abouties. Toutes les réponses doivent être justifiées, sauf mention contraire.\\ \hline
% \end{tabularx}}

% \vspace{0.5cm}\textbf{Exercice 3 :  \hfill 6 points}

\medskip

\begin{minipage}{9cm}
Pour gagner le gros lot à  une kermesse, il faut d'abord tirer une boule rouge dans une urne, puis obtenir un multiple de 3 en tournant une roue de loterie numérotée de 1 à  6.

L'urne contient 3 boules vertes, 2 boules bleues et 3 boules rouges.

\begin{enumerate}
\item Sur la roue de loterie, il y a deux issues (3 et 6) sur 6 issues qui réalisent l’évènement \guill{obtenir un multiple de $3$}.

La probabilité d'obtenir un multiple de $3$ est donc égale à $\dfrac{2}{6}$ $\left( \text{ou~} \dfrac{1}{3}\right)$
\item Dans l’urne, la probabilité de tirer une boule rouge est égale à $\dfrac{3}{8}$.

la probabilité de tirer une boule rouge dans une urne, puis d’obtenir un multiple de 3 sur la roue de loterie est égale à $\dfrac{3}{8}\times\dfrac{1}{3}$, soit $\dfrac{1}{8}$.

La probabilité qu'un participant gagne le gros lot est égale à $\dfrac{1}{8}$.
\end{enumerate}
\end{minipage}
\hspace{0.5cm}\begin{minipage}{5cm}
\psset{unit=0.85cm}
\begin{pspicture}(-2.5,-2.5)(2.8,2.5)
\pscircle(0,0){2.5}
\multido{\n=0+60,\na=30+60,\nb=1+1}{6}{\psline(2.5;\n)\rput(1.5;\na){\nb}}
\rput{-35}(2.6;-35){$\blacktriangleleft$}
\end{pspicture}
\end{minipage}

\begin{enumerate}
\item[\textbf{3.}] Comme on ne change pas le nombre de boules vertes et de boules bleues, il y a 5 boules vertes ou bleues.

Il faut que la moitié des boules soient rouges, donc il faut mettre en tout 5 boules rouges dans l'urne pour que la probabilité de tirer une boule rouge soit de $0,5$.
\end{enumerate}

\newpage

\end{document}