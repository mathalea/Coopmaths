\textbf{\textsc{Exercice 3} \hfill 5 points}

\medskip

Une nouvelle boutique a ouvert à Paris. Elle vend exclusivement des
macarons (petites pâtisseries).

L'extrait de tableur ci-dessous indique le nombre de macarons vendus
une semaine.

\begin{center}
\begin{tabularx}{\linewidth}{|c||m{1.75cm}|*{8}{>{\centering \arraybackslash}X|}}\hline
	&A							&B		&C		&D			&E		&F			&G		&H		&I\\ \hline
1	&							&\scriptsize Lundi	&\scriptsize Mardi	&\scriptsize Mercredi	&\scriptsize Jeudi	&\scriptsize Vendredi	&\scriptsize Samedi	&\tiny Dimanche &\scriptsize Total\\ \hline
2	&Nombre de macarons vendus	&324	&240	&310		&204	&318		&386		&468	&\\ \hline
\end{tabularx}
\end{center}

\begin{enumerate}
\item Quelle formule doit être saisie dans la case I2 pour calculer le nombre total de macarons
vendus dans la semaine ?
\item Calculer le nombre moyen de macarons vendus par jour. Arrondir le résultat à l'unité.
\item Calculer le nombre médian de macarons.
\item Calculer la différence entre le nombre de macarons vendus le dimanche et ceux vendus le
jeudi. À quel terme statistique correspond cette valeur ?
\end{enumerate}

\bigskip

