\textbf{Exercice 3 \hfill 6 points}

\medskip

\parbox{0.48\linewidth}{Dans la figure ci-contre:

\begin{itemize}
\item[$\bullet~~$] ABE est un triangle;
\item[$\bullet~~$] AB = 6~cm, AE = 8~cm et 

BE = 10~cm ;
\item[$\bullet~~$] I et J sont les milieux respectifs des côtés [AB] et [AE] ;
\item[$\bullet~~$] le cercle $(C)$ passe par les points I, J et A.
\end{itemize}} \hfill
\parbox{0.48\linewidth}{\psset{unit=0.6cm}
\begin{pspicture}(10,5.5)
%\psgrid
\pspolygon(0.3,0.6)(9.5,0.5)(3.3,4.8)%BEA
\uput[u](3.3,4.8){A} \uput[l](0.3,0.6){B} \uput[r](9.5,0.5){E} \uput[l](1.8,2.8){I} \uput[r](6.3,2.8){J}\uput[ur](5.4,4.5){$(C)$} 
\pscircle(4.05,2.7){2.25}
\psline(1.8,2.7)(6.3,2.7)%IJ
\rput(5,0){La figure n'est pas à  l'échelle}
\end{pspicture}}
\bigskip

\begin{enumerate}
\item Peut-on affirmer que les droites (IJ) et (BE) sont parallèles ?
\item Montrer que le triangle ABE est rectangle.
\item Quelle est la mesure de l'angle $\widehat{\text{AEB}}$ ? On donnera une valeur approchée au degré près.

\item  
	\begin{enumerate}
		\item Justifier que le centre du cercle $(C)$ est le milieu du segment [IJ].
		\item Quelle est la mesure du rayon du cercle $(C)$ ?
	\end{enumerate}
\end{enumerate}

\bigskip

