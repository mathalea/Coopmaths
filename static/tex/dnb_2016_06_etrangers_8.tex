\textbf{\textsc{Exercice 8} \hfill 5 points}

\medskip

Pour cuire des macarons, la température du four doit être impérativement de 150~\degres C.

Depuis quelques temps, le responsable de la boutique n'est pas satisfait de la cuisson de ses
pâtisseries. Il a donc décidé de vérifier la fiabilité de son four en réglant sur 150~\degres C et en prenant régulièrement la température à l'aide d'une sonde.

Voici la courbe représentant l'évolution de la température de son four en fonction du temps.

\begin{center}
\psset{xunit=0.58cm,yunit=0.058cm}
\begin{pspicture}(-1.75,-15)(18,200)
\rput(9,190){Évolution de la température du four en fonction du temps}
\multido{\n=0+1}{19}{\psline[linewidth=0.3pt](\n,0)(\n,180)}
\multido{\n=0+10}{19}{\psline[linewidth=0.3pt](0,\n)(18,\n)}
\psaxes[linewidth=1.25pt,Dx=2,Dy=20](0,0)(0,0)(18,180)
\pscurve(0,0)(1,25)(2,50)(3,72)(4,90)(5,109)(6,126)(7,140)(8,150)(9,156)(10,155)(11,150)(12,145)(13,144)(14,150)(15,152.5)(16,154)
\uput[u](16,0){Temps (en minutes)}\rput{90}(-1.75,155){Température (en \degres C)}
\end{pspicture}
\end{center}

\begin{enumerate}
\item La température du four est-elle proportionnelle au temps?
\item Quelle est la température atteinte au bout de $3$ minutes? Aucune justification n'est demandée.
\item De combien de degrés Celsius, la température a-t-elle augmenté entre la deuxième et la septième
minute ?
\item Au bout de combien de temps, la température de 150~\degres C nécessaire à la cuisson des macarons
est-elle atteinte ?
\item Passé ce temps, que peut-on dire de la température du four? Expliquer pourquoi le responsable
n'est pas satisfait de la cuisson de ses macarons.
\end{enumerate}

\bigskip

