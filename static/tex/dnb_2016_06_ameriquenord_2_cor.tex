\textbf{\textsc{Exercice 2 \hfill 4 points}}

\medskip

\begin{enumerate}
\item 
	\begin{enumerate}
		\item La probabilité que la piste empruntée soit une piste rouge est $\dfrac{2}{5}$.
		\item À partir du restaurant, la probabilité que Guilhem emprunte une piste bleue est $\dfrac{1}{7}$.
	\end{enumerate}
\item La probabilité que Guilhem enchaîne cette fois-ci deux pistes noires est : 
	
$\dfrac{2}{5} \times \dfrac{3}{7} = \dfrac{2 \times 3}{5 \times 7} = \dfrac{6}{35}$.
	
Pour s'en persuader :
	
\begin{center}
\pstree[treemode=R,nodesep=8pt,levelsep=3.5cm]{\TR{}}
{\pstree{\TR{\footnotesize Piste 	noire}\taput{$\frac{2}{5}$}}
	{\TR{\footnotesize Piste noire}\taput{$\frac{3}{7}$}
	\TR{\footnotesize Piste non noire}\tbput{$\frac{4}{7}$}
	}
\pstree{\TR{\footnotesize Piste non noire}\tbput{$\frac{3}{5}$}}
	{\TR{\footnotesize Piste noire}\taput{$\frac{3}{7}$}
	\TR{\footnotesize Piste non noire}\tbput{$\frac{4}{7}$}
	}
}
\end{center}

On suit les branches du haut...
\end{enumerate}

\vspace{0,5cm}

