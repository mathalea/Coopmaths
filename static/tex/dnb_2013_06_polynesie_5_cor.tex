\textbf{Exercice 5 \hfill 4 points}

\medskip

%Voici le profil schématisé d'une \og ferme simple \fg{} de
%la maison de Moana.
%
%\og  L'entrait \fg{} et \og  le faux-entrait \fg{} sont
%parallèles.

\begin{center}
\psset{unit=1cm}
\begin{pspicture}(8,5)
%\psgrid
\pspolygon(0.5,0.5)(7.5,0.5)(4,4.5)%DEA
\uput[dl](0.5,0.5){D} \uput[dr](7.5,0.5){E} \uput[u](4,4.5){A}
\psline(2.2,2.4)(5.8,2.4)%BC
\uput[l](2.2,2.4){B}\uput[r](5.8,2.4){C}
\psline(4,4.5)(4,2.4)%AF
\uput[d](4,2.4){F}
\rput(5.8,3.6){poinçon}\psline{->}(5.1,3.6)(4,3.6)
\rput(2,4){faux-entrait}\psline{->}(2.8,3.8)(3.2,2.4)
\rput(4,0.1){entrait}\psline{->}(4,0.2)(4,0.5)
\psframe(4,2.4)(4.3,2.7)
\rput{48}(2.2,2.8){340}\uput[u](4,0.5){561}\uput[u](3.7,2.4){330}
\end{pspicture} 
\end{center}
%Toutes les longueurs sont exprimées en
%centimètres.
%
%Moana décide de calculer la longueur du segment [AB] :
%
%On donne AD = 340 ; DE = 561 et BC = 330.
%
%B est un point du segment [AD] et C est un point du
%segment [AE]. Les droites (BC) et (DE) sont
%parallèles.

\medskip

\begin{enumerate}
\item %Sur le schéma ci-contre qui n'est pas à l'échelle,
%reporter les valeurs des longueurs données.
Voir au dessus.
\item %Dans cet exercice, pour calculer AB, quelle propriété peut-on utiliser ?

%Cocher ci-dessous la bonne propriété.
\begin{itemize}
%\item[$\bullet~~$] Le théorème de Pythagore.
%\item[$\bullet~~$] Si un segment a pour extrémités les milieux de deux côtés d'un triangle, alors sa longueur est égale à la moitié de la longueur du troisième côté du triangle.
\item[$\bullet~~$] Le théorème de Thalès.
\end{itemize}
\item %Calculer AB.
D'après le théorème de Thalès les droites (BC) et (DE) étant parallèles :

$\dfrac{\text{AB}}{\text{AD}} = \dfrac{\text{BC}}{\text{DE}}$, ou en remplaçant par les valeurs connues :

$\dfrac{\text{AB}}{340} = \dfrac{330}{561}$, d'où $\text{AB} = \dfrac{340 \times 330}{561} = 200$.

\end{enumerate}

\bigskip

