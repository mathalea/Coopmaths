\textbf{\textsc{Exercice} 3 \hfill 6 points}

\medskip

Les informations suivantes concernent les salaires des hommes et des femmes d'une même entreprise : 
\medskip

\begin{tabularx}{\linewidth}{|>{\centering \arraybackslash}X|}\hline
Salaires des femmes :\\ 
\np{1200}~\euro{} ; \np{1230}~\euro{} ; \np{1250}~\euro{} ; \np{1310}~\euro{} : \np{1376}~\euro{} ; \np{1400}~\euro{} ; \np{1440}~\euro{} ; \np{1500}~\euro{} ; \np{1700}~\euro{} ; \np{2100}~\euro{}\\ \hline
\end{tabularx}

\medskip

\begin{tabularx}{\linewidth}{|>{\centering \arraybackslash}X|}\hline
Salaires des hommes : \\
Effectif total : 20\\
Moyenne : \np{1769}~\euro\\
Étendue: \np{2400}~\euro \\
Médiane: \np{2000}~\euro\\ 
Les salaires des hommes sont tous différents.\\ \hline
\end{tabularx}

\medskip

\begin{enumerate}
\item Comparer le salaire moyen des hommes et celui des femmes. 
\item On tire au sort une personne dans l'entreprise. Quelle est la probabilité que ce soit une femme ?
\item Le plus bas salaire de l'entreprise est de \np{1000}~\euro. Quel salaire est le plus élevé ?
\item Dans cette entreprise combien de personnes gagnent plus de \np{2000}~\euro ?
\end{enumerate}

\bigskip

