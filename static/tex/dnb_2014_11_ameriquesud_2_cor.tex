\textbf{\textsc{Exercice 2} \hfill 6 points}

\medskip

%\begin{center}
%\psset{unit=1cm}
%\begin{pspicture}(6,4)
%%\psgrid
%\psline(0.5,1.5)(3.8,0.5)(5.7,1.7)(5.7,2.9)(3.8,1.7)(3.8,0.5)%EFGCBF
%\psline(3.8,1.7)(0.5,2.7)(0.5,1.5)%BAE
%\psline(0.5,2.7)(2.4,3.9)(5.7,2.9)%ADC
%\psline(2.8,0.8)(3.8,1.7)(4.4,0.88)%NBM
%\psline[linestyle=dashed](2.8,0.8)(4.4,0.88)
%\psline[linestyle=dashed](2.4,3.9)(2.4,2.7)(0.5,1.5)%DHE
%\psline[linestyle=dashed](2.4,2.7)(5.7,1.7)%HG
%\uput[ul](0.5,2.7){A} \uput[u](3.8,1.7){B} \uput[ur](5.7,2.9){C}
%\uput[u](2.4,3.9){D} \uput[dl](0.5,1.5){E} \uput[d](3.8,0.5){F}
%\uput[dr](5.7,1.7){G} \uput[ur](2.4,2.7){H} \uput[dr](4.4,0.88){M}
%\uput[dl](2.8,0.8){N}
%\end{pspicture}
%\end{center}
%    
%On considère le parallélépipède rectangle ABCDEFGH. 
%
%M est un point de [FG] et N un point de [EF]. 
%
%On donne : FE = 15~cm ; FG = 10~cm ; FB = 5~cm ; FN = 4~cm ; FM = 3~cm. 
%
%\medskip

\begin{enumerate}
\item %Démontrer que l'aire du triangle FNM est égal à 6~cm$^2$.
On est dans un parallélépipède rectangle, donc [FN] et [FM] sont perpendiculaires. L'aire du triangle rectangle FMN est donc égale à : 

$\dfrac{\text{FN} \times \text{FM}}{2}  = \dfrac{4 \times 3}{2} = 6$~cm$^2$.
\item %Calculer le volume de la pyramide de sommet B et de base le triangle FNM.
Le volume  du prisme de base FMN et de hauteur [BF] est égale à 

$\dfrac{1}{3} \times \mathcal{A}(\text{FMN}) \times \text{BF}] = \dfrac{6 \times 5}{3} = 10$~cm$^3$.  

%On rappelle que le volume d'une pyramide: $V = \dfrac{(B \times h)}{3}$ où $B$ est l'aire de la base et $h$ la hauteur de la pyramide. 
\item %On considère le solide ABCDENMGH obtenu en enlevant la pyramide précédente au parallélépipède rectangle. 
	\begin{enumerate}
		\item %Calculer son volume.
Le volume du parallélépipède ABCDEFGH est égal à $15 \times 10 \times 5 = 750$~cm$^3$.

Donc le volume du solide ABCDENMGH est égal à $750 - 10 = 740$~cm$^3$. 
		\item %On appelle caractéristique d'Euler d'un solide le nombre $x$ tel que: 

%\[x =  \text{nombre de faces} - \text{nombre d'arêtes}  + \text{nombre de sommets}\] 

%Recopier et compléter le tableau suivant: 
	\end{enumerate}

\medskip
\begin{tabularx}{\linewidth}{|l|*{2}{>{\centering \arraybackslash}X|}}\hline
  					&Parallélépipède ABCDEFGH	&   Solide ABCDENMGH\\ \hline   
Nombre de faces		&6							&7\\ \hline       
Nombre d'arêtes		&12							&14\\ \hline        
Nombre de sommets	&8							&9\\ \hline        
Caractéristique $x$	&2							&2\\ \hline 
\end{tabularx}
\medskip      
\end{enumerate}

\vspace{0,5cm}

