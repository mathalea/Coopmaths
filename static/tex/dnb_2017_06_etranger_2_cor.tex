\documentclass[10pt]{article}
\usepackage[T1]{fontenc}
\usepackage[utf8]{inputenc}%ATTENTION codage UTF8
\usepackage{fourier}
\usepackage[scaled=0.875]{helvet}
\renewcommand{\ttdefault}{lmtt}
\usepackage{amsmath,amssymb,makeidx}
\usepackage[normalem]{ulem}
\usepackage{diagbox}
\usepackage{fancybox}
\usepackage{tabularx,booktabs}
\usepackage{colortbl}
\usepackage{pifont}
\usepackage{multirow}
\usepackage{dcolumn}
\usepackage{enumitem}
\usepackage{textcomp}
\usepackage{lscape}
\newcommand{\euro}{\eurologo{}}
\usepackage{graphics,graphicx}
\usepackage{pstricks,pst-plot,pst-tree,pstricks-add}
\usepackage[left=3.5cm, right=3.5cm, top=3cm, bottom=3cm]{geometry}
\newcommand{\R}{\mathbb{R}}
\newcommand{\N}{\mathbb{N}}
\newcommand{\D}{\mathbb{D}}
\newcommand{\Z}{\mathbb{Z}}
\newcommand{\Q}{\mathbb{Q}}
\newcommand{\C}{\mathbb{C}}
\usepackage{scratch}
\renewcommand{\theenumi}{\textbf{\arabic{enumi}}}
\renewcommand{\labelenumi}{\textbf{\theenumi.}}
\renewcommand{\theenumii}{\textbf{\alph{enumii}}}
\renewcommand{\labelenumii}{\textbf{\theenumii.}}
\newcommand{\vect}[1]{\overrightarrow{\,\mathstrut#1\,}}
\def\Oij{$\left(\text{O}~;~\vect{\imath},~\vect{\jmath}\right)$}
\def\Oijk{$\left(\text{O}~;~\vect{\imath},~\vect{\jmath},~\vect{k}\right)$}
\def\Ouv{$\left(\text{O}~;~\vect{u},~\vect{v}\right)$}
\usepackage{fancyhdr}
\usepackage[french]{babel}
\usepackage[dvips]{hyperref}
\usepackage[np]{numprint}
%Tapuscrit : Denis Vergès
%\frenchbsetup{StandardLists=true}

\begin{document}
\setlength\parindent{0mm}
% \rhead{\textbf{A. P{}. M. E. P{}.}}
% \lhead{\small Brevet des collèges}
% \lfoot{\small{Polynésie}}
% \rfoot{\small{7 septembre 2020}}
\pagestyle{fancy}
\thispagestyle{empty}
% \begin{center}
    
% {\Large \textbf{\decofourleft~Brevet des collèges Polynésie 7 septembre 2020~\decofourright}}
    
% \bigskip
    
% \textbf{Durée : 2 heures} \end{center}

% \bigskip

% \textbf{\begin{tabularx}{\linewidth}{|X|}\hline
%  L'évaluation prend en compte la clarté et la précision des raisonnements ainsi que, plus largement, la qualité de la rédaction. Elle prend en compte les essais et les démarches engagées même non abouties. Toutes les réponses doivent être justifiées, sauf mention contraire.\\ \hline
% \end{tabularx}}

% \vspace{0.5cm}\textbf{\textsc{Exercice 2} \hfill 7 points}

\medskip

\textbf{Partie 1 }: 

%Pour réaliser une étude sur différents isolants, une société réalise 3 maquettes de maison strictement identiques à l'exception près des isolants qui diffèrent dans chaque maquette. On place ensuite ces 3 maquettes dans une chambre froide réglée à 6~\degres C. On réalise un relevé des températures ce qui permet de construire les 3 graphiques suivants: 
%
%\begin{center}
%\psset{xunit=.12,yunit=.3}
%\begin{pspicture}(-3,-.5)(100,24)
%\multido{\n=0+5}{21}{\psline[linewidth=0.2pt,linecolor=lightgray](\n,-0)(\n,22.5)}
%\multido{\n=0+2}{12}{\psline[linewidth=0.2pt,linecolor=lightgray](-.20,\n)(100,\n)}
%\psaxes[labelsep=.8mm,linewidth=.75pt,ticksize=-2pt 2pt,Dx=5,Dy=2,labelFontSize=\scriptstyle]{->}(0,0)(0,0)(100,23)
%\rput(10,21.5){{\scriptsize Température en $^\circ$C}}
%\psline[linewidth=1.25pt](0,20)(15,20)
%\pscurve[linewidth=1.25pt](15,20)(19,19.9)(76,6.1)(80,6)
%\psline[linewidth=1.25pt](80,6)(100,6)
%\rput(60,19.2){MAQUETTE A}
%\rput(87,1.5){{\scriptsize Durée en heures}}
%\end{pspicture}
%\end{center}
%
%
%
%\begin{center}
%\psset{xunit=.12,yunit=.3}
%\begin{pspicture}(-3,-.5)(100,24)
%\multido{\n=0+5}{21}{\psline[linewidth=0.2pt,linecolor=lightgray](\n,-0)(\n,22.5)}
%\multido{\n=0+2}{12}{\psline[linewidth=0.2pt,linecolor=lightgray](-.20,\n)(100,\n)}
%\psaxes[labelsep=.8mm,linewidth=.75pt,ticksize=-2pt 2pt,Dx=5,Dy=2,labelFontSize=\scriptstyle]{->}(0,0)(0,0)(100,23)
%\rput(10,21.5){{\scriptsize Température en $^\circ$C}}
%\psline[linewidth=1.25pt](0,20)(20,20)
%\pscurve[linewidth=1.25pt](20,20)(24,19.9)(66,6.1)(70,6)
%\psline[linewidth=1.25pt](70,6)(100,6)
%\rput(60,19.2){MAQUETTE B}
%\rput(87,1.5){{\scriptsize Durée en heures}}
%\end{pspicture}
%\end{center}
%
%\begin{center}
%\psset{xunit=.12,yunit=.3}
%\begin{pspicture}(-3,-.5)(100,24)
%\multido{\n=0+5}{21}{\psline[linewidth=0.2pt,linecolor=lightgray](\n,-0)(\n,22.5)}
%\multido{\n=0+2}{12}{\psline[linewidth=0.2pt,linecolor=lightgray](-.20,\n)(100,\n)}
%\psaxes[labelsep=.8mm,linewidth=.75pt,ticksize=-2pt 2pt,Dx=5,Dy=2,labelFontSize=\scriptstyle]{->}(0,0)(0,0)(100,23)
%\rput(10,21.5){{\scriptsize Température en $^\circ$C}}
%\psline[linewidth=1.25pt](0,20)(10,20)
%\pscurve[linewidth=1.25pt](10,20)(14,19.9)(51,6.1)(55,6)
%\psline[linewidth=1.25pt](55,6)(100,6)
%\rput(60,19.2){MAQUETTE C}
%\rput(87,1.5){{\scriptsize Durée en heures}}
%\end{pspicture}
%\end{center}

%\medskip

\begin{enumerate}
\item  %Quelle était la température des maquettes avant d'être mise dans la chambre froide ?
Les trois maquettes étaient à 20~\degres C. 

\item %Cette expérience a-t-elle duré plus de 2 jours? Justifier votre réponse. 
L'expérience a duré 100 heures soit $4 \times 24 + 4$ donc 4 jours et 4 heures.

\item %Quelle est la maquette qui contient l'isolant le plus performant? Justifier votre réponse.
La maquette la plus résistante au froid est la B car il lui faut  70~h pour descendre à 6~\degres C.
\end{enumerate}

\medskip

\textbf{Partie 2 }: 

%Pour respecter la norme RT2012 des maisons BBC (Bâtiments Basse Consommation), il faut que la résistance thermique des murs notée $R$ soit supérieure ou égale à 4. Pour calculer cette résistance thermique, on utilise la relation: 
%
%$$R=\dfrac{e}{c}$$ 
%
%où $e$ désigne l'épaisseur de l'isolant en mètre et $c$ désigne le coefficient de conductivité thermique de l'isolant. Ce coefficient permet de connaître la performance de l'isolant. 

\medskip

\begin{enumerate}
\item  %Noa a choisi comme isolant la laine de verre dont le coefficient de conductivité thermique est: $c = 0,035$. Il souhaite mettre 15~cm de laine de verre sur ses murs. 

%Sa maison respecte-t-elle la normé RT2012 des maisons BBC ? 
On a $R_{\text{Noa}} = \dfrac{0,15}{0,035} = \dfrac{150}{35} = \dfrac{30}{7} \approx 4,3$ donc supérieur à 4.
\item  %Camille souhaite obtenir une résistance thermique de 5 ($R = 5$). Elle a choisi comme isolant du liège dont le coefficient de conductivité thermique est: $c = 0,04$. 

%Quelle épaisseur d'isolant doit-elle mettre sur ses murs ? 
Il faut trouver $e$ tel que :

$R = \dfrac{e}{c}$, soit $5 = \dfrac{e}{0,04}$, donc $e = 5 \times 0,04 = 0,2$~(m) soit 20~cm.
\end{enumerate}

\bigskip

\end{document}