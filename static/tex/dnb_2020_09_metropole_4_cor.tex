
\medskip

Une association propose diverses activités pour occuper les enfants pendant les vacances scolaires.

\smallskip

Plusieurs tarifs sont proposés:

\begin{itemize}
\item Tarif A : 8 \euro{} par demi-journée ;
\item Tarif B : une adhésion de 30 \euro{} donnant droit à un tarif préférentiel de 5~\euro{} par demi-journée
\end{itemize}

\medskip

Un fichier sur tableur a été préparé pour calculer le coût à payer en fonction du nombre de demi-journées d'activités pour chacun des tarifs proposés :


\begin{center}
\begin{tabularx}{\linewidth}{|c|c|*{5}{>{\centering \arraybackslash}X|}}\hline
	&A						&B	&C	&D	&E	&F\\ \hline
1	&Nombre de demi-journées&1	&2	&3	&4	&5\\ \hline
2	& Tarif A				&8 	&16	&	&	&\\ \hline
3	& Tarif B				&35	&40	&	&	&\\ \hline
\end{tabularx}
\end{center}

Les questions 1, 2, 4 et 5 ne nécessitent pas de justification. 

\medskip

\begin{enumerate}
\item %Compléter ce tableau sur l'annexe 1.
Voir l'annexe à la fin.
\item La bonne formule est $=30+5*\text{B}1$.
%Retrouver parmi les réponses suivantes la formule qui a été saisie dans la cellule B3 avant de l'étirer vers la droite :

%\begin{center}
%\begin{tabularx}{\linewidth}{|*{5}{>{\centering \arraybackslash}X|}}\hline
%Réponse A 	&Réponse B 			&Réponse C 					&Réponse D 			&Réponse E\\ \hline
%$=8*$B1		& $=30*\text{B}1+5$	& $=5*\text{B}1+30*\text{B}1$& $=30+5*\text{B}1$& $=35$\\ \hline
%\end{tabularx}
%\end{center}

\item %On considère les fonctions $f$ et $g$ qui donnent les tarifs à payer en fonction du nombre $x$ de demi-journées d'activités :

%\begin{itemize}
%\item Tarif A :\quad  $f(x) = 8x$
%\item Tarif B :\quad $g(x) =30 + 5x$
%\end{itemize}

%Parmi ces fonctions, quelle est celle qui traduit une situation de proportionnalité ?
C'est la fonction linéaire $f$
\item %Sur le graphique de l'annexe 2, on a représenté la fonction $g$. Représenter sur ce même graphique la fonction $f$.
Voir l'annexe à la fin.
\item %Déterminer le nombre de demi-journées d'activités pour lequel le tarif A est égal au tarif B.
$\bullet~~$\emph{Graphiquement} (ce qui semble demandé) : on voit que pour $x = 10$ le prix à payer est le même avec les deux formules : 80~\euro.

$\bullet~~$\emph{Par la calcul} Il faut résoudre dans $\N$ l'équation : 

$f(x) = g(x)$ ou $8x = 5x + 30$ ou $3x = 30$ et enfin en multipliant chaque membre par $\dfrac{1}{3}$, \: $x = 10$.
\item %Avec un budget de 100~\euro, déterminer le nombre maximal de demi-journées auxquelles on peut participer.

%Décrire la méthode choisie.
$\bullet~~$\emph{Graphiquement} 

La droite d'équation $y = 100$ coupe $\mathcal{C}_g$ en un point d'abscisse maximal, soit $x = 14$.

Avec 100~\euro{} il vaut mieux choisir la formule B ; on aura 14 demi-journées.

$\bullet~~$\emph{Par le calcul}

On résout $100 = f(x)$ soit $100 = 8x$ ou $25 = 2x$, soit $x = 12,5$, donc en fait 12 demi-journées.

On résout ensuite $100 = g(x)$ soit $100 = 5x + 30$ soit $70 = 5x$ c'est-à-dire $5 \times 14 = 5 \times x$, donc $14 = x$.
\end{enumerate}


\begin{center}

	\textbf{\large ANNEXES à rendre avec votre copie}
	
	\bigskip
	
	\textbf{Annexe 1 - Question 1}
	
	\medskip
	
	\begin{tabularx}{\linewidth}{|c|c|*{5}{>{\centering \arraybackslash}X|}}\hline
		&A						&B	&C	&D	&E	&F\\ \hline
	1	&Nombre de demi-journées&1	&2	&3	&4	&5\\ \hline
	2	& Tarif A				&8 	&16	&24	&32	&40\\ \hline
	3	& Tarif B				&35	&40	&45	&50	&55\\ \hline
	\end{tabularx}
	
	\vspace{2cm}
	
	\bigskip
	
	\textbf{Annexe 2 - Question 4}
	
	\bigskip
	
	\psset{xunit=0.8cm,yunit=0.08cm,arrowsize=2pt 4}
	\begin{pspicture}(-1,-10)(15,120)
	\multido{\n=0+1}{16}{\psline[linecolor=orange,linewidth=0.15pt](\n,0)(\n,120)}
	\multido{\n=0+10}{13}{\psline[linecolor=orange,linewidth=0.15pt](0,\n)(15,\n)}
	\psaxes[linewidth=1.25pt,Dy=20]{->}(0,0)(0,0)(15,120)
	\psplot[plotpoints=2000,linewidth=1.25pt,linecolor=blue]{0}{15}{5 x mul 30 add}
	\uput[dr](14,100){\blue $\mathcal{C}_g$}
	\psplot[plotpoints=2000,linewidth=1.25pt,linecolor=red]{0}{15}{8 x mul}
	\uput[ul](14,112){\red $\mathcal{C}_f$}
	\uput[r](0,118){Tarif en \euro}
	\uput[u](13,0){Nombre de demi-journées}
	\psline[linewidth=1.2pt,linestyle=dashed,ArrowInside=->](0,100)(14,100)(14,0)
	\psline[linewidth=1.2pt,linestyle=dashed,ArrowInside=->](10,80)(10,0)
	\psline[linewidth=1.2pt,linestyle=dashed,ArrowInside=->](10,80)(0,80)
	\end{pspicture}
	\end{center}
\bigskip

