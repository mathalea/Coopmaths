\textbf{Exercice 1 : Q. C. M. \hfill 4 points}

\medskip

%\emph{Cet exercice est un questionnaire à choix multiples (QCM). Pour chaque question, une seule des trois réponses proposées est exacte. Sur la copie, indiquer le numéro de la question et recopier, sans justifier, la réponse choisie. Aucun point ne sera enlevé en cas de mauvaise réponse.}
%
%\medskip 
%
%\begin{tabularx}{\linewidth}{|m{4.5cm}|*{3}{>{\centering \arraybackslash}X|}}\hline
%Questions &\multicolumn{3}{|c|}{Réponses}\\ \cline{2-4}
%&A&B&C\\ \hline
%\textbf{1.} Sur cette figure, les points K, F, R et K, M, P sont alignés.
% 
%\psset{unit=0.75cm}
%\begin{pspicture}(3.5,3.5)
%%\psgrid
%\psline(0,0.5)(1.8,2.7)(2.2,0)
%\psline(0,1)(3,0.4)
%\psline(0,2.5)(3,1.9)
%\rput(4.2,3.1){KF = 3, KR = 9,}
%\rput(4.2,2.6){KM = 4, KP = 12}
%\uput[u](1.8,2.6){K} \uput[ul](1.4,2.2){F}\uput[d](0.38,0.9){R}
%\uput[ur](1.9,2.1){M} \uput[ur](2.1,0.6){P}
%\end{pspicture}
%
%Les droites (FM) et (RP) sont-elles parallèles ? &Oui &Non&On ne peut pas savoir\\ \hline
%\textbf{2.} Si on remplace $x$ par $- 3$ dans l'expression $5 - 2x$, on trouve&$- 9$&11&$- 1$\\ \hline 
%\textbf{3.} On a représenté la fonction $f$ dans le repère ci-dessous : 
%
%\psset{unit=0.75cm}
%\begin{pspicture*}(-1,-2)(5,5)
%\psgrid[gridlabels=0pt,subgriddiv=1,griddots=5]
%\psaxes[linewidth=1.25pt](0,0)(-0.9,-1.9)(4.95,4.95)
%\psplot[plotpoints=3000,linewidth=1.25pt,linecolor=red]{-0.6}{4.56}{x 3 exp 4 mul 9 div x dup mul 23 mul 9 div sub 10 x mul 3 div add 1 add}
%%\pscurve[linewidth=1.25pt,linecolor=blue](-0.5,-1.6)(-0.25,0)(0,1)(0.5,2.2)(0.8,2.4)(1,2.3)(1.5,1.75)(2,1)(3,0)(3.5,0.7)(4,2.4)(4.4,5.3)
%\end{pspicture*}&L'image de 2 par la fonction $f$ est 1.&L'image de 1 par la fonction $f$ est 2.& 2 n'a pas d'image par la fonction $f$. \\ \hline
%\textbf{4.}  En utilisant le même graphique que la question \textbf{3.} :&
%5 est l'antécédent de $0$ par la fonction $f$.&1 n'a pas d'antécédent par la fonction $f$.&  
%2 a trois antécédents par la fonction $f$.\\ \hline
%\end{tabularx}
\begin{enumerate}
\item 

Si les droites sont parallèles le théorème de Thalès permet d'écrire l'égalité :

$\dfrac{\text{KF}}{\text{KR}} = \dfrac{\text{KM}}{\text{KP}}$, soit $\dfrac{3}{9} = \dfrac{4}{12}$ ; comme $3 \times 12 = 9 \times 4$, cette égalité est vraie. Réponse A.
\item On a $5 - 2\times (- 3) = 5 + 6 = 11$. Réponse B.
\item L'image de 2 par la fonction $f$ est 1. Réponse A.
\item 2 a trois antécédents par la fonction $f$. Réponse C.
\end{enumerate}
\vspace{0,5cm}

