\documentclass[10pt]{article}
\usepackage[T1]{fontenc}
\usepackage[utf8]{inputenc}%ATTENTION codage UTF8
\usepackage{fourier}
\usepackage[scaled=0.875]{helvet}
\renewcommand{\ttdefault}{lmtt}
\usepackage{amsmath,amssymb,makeidx}
\usepackage[normalem]{ulem}
\usepackage{diagbox}
\usepackage{fancybox}
\usepackage{tabularx,booktabs}
\usepackage{colortbl}
\usepackage{pifont}
\usepackage{multirow}
\usepackage{dcolumn}
\usepackage{enumitem}
\usepackage{textcomp}
\usepackage{lscape}
\newcommand{\euro}{\eurologo{}}
\usepackage{graphics,graphicx}
\usepackage{pstricks,pst-plot,pst-tree,pstricks-add}
\usepackage[left=3.5cm, right=3.5cm, top=3cm, bottom=3cm]{geometry}
\newcommand{\R}{\mathbb{R}}
\newcommand{\N}{\mathbb{N}}
\newcommand{\D}{\mathbb{D}}
\newcommand{\Z}{\mathbb{Z}}
\newcommand{\Q}{\mathbb{Q}}
\newcommand{\C}{\mathbb{C}}
\usepackage{scratch}
\renewcommand{\theenumi}{\textbf{\arabic{enumi}}}
\renewcommand{\labelenumi}{\textbf{\theenumi.}}
\renewcommand{\theenumii}{\textbf{\alph{enumii}}}
\renewcommand{\labelenumii}{\textbf{\theenumii.}}
\newcommand{\vect}[1]{\overrightarrow{\,\mathstrut#1\,}}
\def\Oij{$\left(\text{O}~;~\vect{\imath},~\vect{\jmath}\right)$}
\def\Oijk{$\left(\text{O}~;~\vect{\imath},~\vect{\jmath},~\vect{k}\right)$}
\def\Ouv{$\left(\text{O}~;~\vect{u},~\vect{v}\right)$}
\usepackage{fancyhdr}
\usepackage[french]{babel}
\usepackage[dvips]{hyperref}
\usepackage[np]{numprint}
%Tapuscrit : Denis Vergès
%\frenchbsetup{StandardLists=true}

\begin{document}
\setlength\parindent{0mm}
% \rhead{\textbf{A. P{}. M. E. P{}.}}
% \lhead{\small Brevet des collèges}
% \lfoot{\small{Polynésie}}
% \rfoot{\small{7 septembre 2020}}
\pagestyle{fancy}
\thispagestyle{empty}
% \begin{center}
    
% {\Large \textbf{\decofourleft~Brevet des collèges Polynésie 7 septembre 2020~\decofourright}}
    
% \bigskip
    
% \textbf{Durée : 2 heures} \end{center}

% \bigskip

% \textbf{\begin{tabularx}{\linewidth}{|X|}\hline
%  L'évaluation prend en compte la clarté et la précision des raisonnements ainsi que, plus largement, la qualité de la rédaction. Elle prend en compte les essais et les démarches engagées même non abouties. Toutes les réponses doivent être justifiées, sauf mention contraire.\\ \hline
% \end{tabularx}}

% \vspace{0.5cm}\textbf{Exercice 4 \hfill 23 points}

\medskip

%Un silo à grains permet de stocker des céréales. Un ascenseur permet d'acheminer le blé dans le silo. L'ascenseur est soutenu par un pilier.
%
%\begin{center}
%\psset{unit=0.9cm}
%\begin{pspicture}(0,-1.5)(16,6)
%%\psgrid
%\psline(11.8,0)
%\pscircle(1.8,0.3){0.1}\pscircle(3.6,0.3){0.1}\pscircle(4.3,0.3){0.1}
%\pscircle(1.8,0.3){0.3}\pscircle(3.6,0.3){0.3}\pscircle(4.3,0.3){0.3}
%\psline(2.5,0.6)(4.6,0.6)
%\rput(8,4){Ascenseur à blé}
%\rput(4.8,-1){Déversoir}
%\psframe(1.5,0.6)(2.6,1.7)
%\psframe(1.6,1)(2.3,1.5)
%\rput{60}(4.6,0.6){\psframe(0,0)(1,2.1)}
%\psline(5.1,0)(4.8,0.6)(5.8,0.6)(5.6,0)\psline[linewidth=2.5pt](3,0.6)(3.4,1.3)
%\scalebox{.99}[0.3]{\psarc[linewidth=1.5pt](13.2,0){1.275}{180}{0}}%
%\psellipse(13.1,4)(1.3,0.65)
%\psline(11.8,0)(11.8,4)\psline(14.35,0)(14.35,4)
%\psline(5.75,0.4)(12,4)\psline(5.65,0.2)(12.1,3.875)
%\pscurve(12,4)(12.05,3.95)(12.1,3.875)
%\pscurve(12.1,3.875)(12.6,3.8)(12.9,3.36)
%\pscurve(12,4)(12.6,4)(13.12,3.36)
%\pscurve(4.8,0.9)(5,0.8)(5.2,0.6)
%\pscurve(4.9,1.1)(5.15,0.95)(5.4,0.6)
%\psline{->}(4.8,-0.8)(5.3,0.3)\psline{->}(8,3.7)(9,2.3)
%\psline(8.9,0)(8.9,2.07)\psline(9.1,0)(9.1,2.17)
%\rput(9,-1){Pilier de soutien}\psline{->}(8.1,-0.8)(8.9,1.4)
%\rput(13.2,-1){Silo}\psline{->}(13.2,-0.8)(13.2,1)
%\psline(14.3,0)(15.4,0)
%\rput(4.5,4){Camion à benne}\psline{->}(4.5,3.8)(4.2,2)
%\end{pspicture}
%\end{center}
%
%On modélise l'installation par la figure ci-dessous qui n'est pas réalisée à l'échelle :
%
%\parbox{0.46\linewidth}{\begin{itemize}[label=$\bullet~~$]
%\item Les points C, E et M sont alignés.
%\item Les points C, F{}, H et P sont alignés.
%\item Les droites (EF) et (MH) sont perpendiculaires à la droite (CH).
%\item CH $=8,50$ m et CF $= 2,50$ m.
%\item Hauteur du cylindre: HM $= 20,40$ m.
%\item Diamètre du cylindre: HP $= 4,20$ m.
%\end{itemize}
%}\hfill
%\parbox{0.54\linewidth}{\psset{unit=1cm}
%\begin{pspicture}(8,5.5)
%\psline(8,0)\psline(0.7,0)(5.5,4.6)%CM
%\psline(2.6,0)(2.6,1.8)%FE
%\psframe[fillstyle=solid,fillcolor=lightgray](5.5,0)(7.2,4.6)%HPM..
%\rput(6.4,2.1){Cylindre}
%\uput[d](0.7,0){C} \uput[d](2.6,0){F} \uput[d](6.5,0){H} 
%\uput[d](7.2,0){P} \uput[ul](2.6,1.8){E} \uput[ul](5.5,4.6){M}
%\psframe(2.6,0)(2.85,0.25)\psframe(5.5,0)(5.75,0.25)
%\end{pspicture}
%}
%
%\vspace{0,7cm}
%
%\textbf{Les quatre questions suivantes sont indépendantes.}

\medskip

\begin{enumerate}
\item %Quelle est la longueur CM de l'ascenseur à blé ?
Le triangle CHM étant rectangle en H le théorème de Pythagore permet d'écrire 

$\text{CM}^2 = \text{CH}^2 + \text{HM}^2$ soit $\text{CM}^2 = 8,5^2 + 20,4^2 = 72,25 + 416,16 = 488,41$.

La calculatrice donne CM $ = \sqrt{488,41} = 22,1$ (m).
\item %Quelle est la hauteur EF du pilier ?
Les droites (EF) et (MH) sont perpendiculaires à la droite (CP) : elles sont donc parallèles.

On peut donc appliquer le théorème de Thalès :

$\dfrac{\text{CF}}{\text{CH}} = \dfrac{\text{EF}}{\text{MH}}$, soit $\dfrac{2,5}{8,5} = \dfrac{\text{EF}}{20,4}$; d'où en multipliant par 20,4 :

$\text{EF} = 20,4 \times \dfrac{2,5}{8,5} = 6$.

Le pilier [EF] mesure 6~m.
\item %Quelle est la mesure de l'angle $\widehat{\text{HCM}}$ entre le sol et l'ascenseur à blé ? On donnera une valeur approchée au degré près.
Dans le triangle CEF rectangle en F, on a :

$\tan \widehat{\text{FCE}} = \dfrac{\text{EF}}{\text{CF}} = \dfrac{6}{2,5} = 2,4$.

La calculatrice donne $\widehat{\text{FCE}}  \approx 67,3$.

L'angle $\widehat{\text{HCM}}$ mesure 67\degres{} au degré près.
\item %Un mètre-cube de blé pèse environ $800$~kg.

%Quelle masse maximale de blé peut-on stocker dans ce silo ? On donnera la réponse à une tonne près.

%
%\medskip
%
%\begin{tabularx}{\linewidth}{|X|}\hline
%Rappels :\\
%\hspace{1.2cm}$\bullet~~$1 tonne = \np{1000}~kg\\
%\hspace{1.2cm}$\bullet~~$ volume d'un cylindre de rayon $R$ et de hauteur $h$ : $\pi \times R^2 \times h$\\ \hline
%\end{tabularx}
Le rayon du cylindre est égal à 2,1~m ; son volume est donc égal à : $\pi \times 2,1^2 \times 20,4 = 89,964\pi$~m$^3$.

On peut donc mettre dans ce silo : 

$89,964\pi \times 800 \approx \np{226104}$~kg de blé, soit encore environ 226 tonnes de blé à la tonne près.
\end{enumerate}

\vspace{0,5cm}

\end{document}