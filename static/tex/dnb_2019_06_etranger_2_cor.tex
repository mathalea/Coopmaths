\documentclass[10pt]{article}
\usepackage[T1]{fontenc}
\usepackage[utf8]{inputenc}%ATTENTION codage UTF8
\usepackage{fourier}
\usepackage[scaled=0.875]{helvet}
\renewcommand{\ttdefault}{lmtt}
\usepackage{amsmath,amssymb,makeidx}
\usepackage[normalem]{ulem}
\usepackage{diagbox}
\usepackage{fancybox}
\usepackage{tabularx,booktabs}
\usepackage{colortbl}
\usepackage{pifont}
\usepackage{multirow}
\usepackage{dcolumn}
\usepackage{enumitem}
\usepackage{textcomp}
\usepackage{lscape}
\newcommand{\euro}{\eurologo{}}
\usepackage{graphics,graphicx}
\usepackage{pstricks,pst-plot,pst-tree,pstricks-add}
\usepackage[left=3.5cm, right=3.5cm, top=3cm, bottom=3cm]{geometry}
\newcommand{\R}{\mathbb{R}}
\newcommand{\N}{\mathbb{N}}
\newcommand{\D}{\mathbb{D}}
\newcommand{\Z}{\mathbb{Z}}
\newcommand{\Q}{\mathbb{Q}}
\newcommand{\C}{\mathbb{C}}
\usepackage{scratch}
\renewcommand{\theenumi}{\textbf{\arabic{enumi}}}
\renewcommand{\labelenumi}{\textbf{\theenumi.}}
\renewcommand{\theenumii}{\textbf{\alph{enumii}}}
\renewcommand{\labelenumii}{\textbf{\theenumii.}}
\newcommand{\vect}[1]{\overrightarrow{\,\mathstrut#1\,}}
\def\Oij{$\left(\text{O}~;~\vect{\imath},~\vect{\jmath}\right)$}
\def\Oijk{$\left(\text{O}~;~\vect{\imath},~\vect{\jmath},~\vect{k}\right)$}
\def\Ouv{$\left(\text{O}~;~\vect{u},~\vect{v}\right)$}
\usepackage{fancyhdr}
\usepackage[french]{babel}
\usepackage[dvips]{hyperref}
\usepackage[np]{numprint}
%Tapuscrit : Denis Vergès
%\frenchbsetup{StandardLists=true}

\begin{document}
\setlength\parindent{0mm}
% \rhead{\textbf{A. P{}. M. E. P{}.}}
% \lhead{\small Brevet des collèges}
% \lfoot{\small{Polynésie}}
% \rfoot{\small{7 septembre 2020}}
\pagestyle{fancy}
\thispagestyle{empty}
% \begin{center}
    
% {\Large \textbf{\decofourleft~Brevet des collèges Polynésie 7 septembre 2020~\decofourright}}
    
% \bigskip
    
% \textbf{Durée : 2 heures} \end{center}

% \bigskip

% \textbf{\begin{tabularx}{\linewidth}{|X|}\hline
%  L'évaluation prend en compte la clarté et la précision des raisonnements ainsi que, plus largement, la qualité de la rédaction. Elle prend en compte les essais et les démarches engagées même non abouties. Toutes les réponses doivent être justifiées, sauf mention contraire.\\ \hline
% \end{tabularx}}

% \vspace{0.5cm}\textbf{\textsc{Exercice 2 \hfill 14 points}}

\medskip

\begin{enumerate}
\item On obtient successivement :

$1 \to 1^2 = 1 \to 1 + 3 \times 1 = 1 + 3 = 4 \to 4 + 2 = 6$. 
\item De même en partant de $- 5$ : 

$- 2 \to (- 5)^2 = 25 \to 25 + 3 \times (- 5) = 25 - 15 = 10 \to 10 + 2 = 12$.
\item En partant de $x$, on obtient :

$x \to x^2 \to x^2 + 3x \to x^2 + 3x + 2$.
\item On a quel que soit le nombre $x$ : 

$(x + 2)(x + 1) = x^2 + x + 2x + 2 = x^2 + 3x + 2$, donc inversement, quel que soit le nombre $x$ :

$x^2 + 3x + 2 = (x + 1)(x + 2)$.
\item
	\begin{enumerate}
		\item La formule est =(B1 + 2)*(B1 + 1)
		\item Il faut trouver les nombres $x$ tels que $(x + 2)(x + 1) = 0$ ; or un produit est nul si l'un de ses facteurs est nul, soit :
		
		$\left\{\begin{array}{l c l}
		x + 2&=&0 \:\text{ou}\\
		x + 1&=&0
		\end{array}\right.$ ou encore $\left\{\begin{array}{l c l}
		x &=&- 2\:\text{ou}\\
		x &=&- 1
		\end{array}\right.$
		
Si l'on part de $- 1$ ou de $- 2$, le programme donne $0$.
	\end{enumerate}
\end{enumerate}

\vspace{0,5cm}

\end{document}