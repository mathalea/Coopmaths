\textbf{\textsc{Exercice 8} \hfill 4 points}

\medskip 

Le débit moyen $q$ d'un fluide dépend de la vitesse moyenne $v$ du fluide et de l'aire de la section d'écoulement d'aire $S$. Il est donné par la formule suivante : 

\[q = S \times v\]
 
où $q$ est exprimé en m$^3$.s$^{-1}$ ; $S$ est exprimé en m$^2$ ; $v$ est exprimé en m.s$^{-1}$. 

Pour cette partie, on considérera que la vitesse moyenne d'écoulement de l'eau à travers la vantelle durant le remplissage est $v = 2,8$ m.s$^{-1}$.
 
La vantelle a la forme d'un disque de rayon $R = 30$cm. 

\medskip

\begin{enumerate}
\item Quelle est l'aire exacte, en m$^2$, de la vantelle ? 
\item Déterminer le débit moyen arrondi au millième de cette vantelle durant le 
remplissage. 
\item Pendant combien de secondes, faudra-t-il patienter pour le remplissage d'une écluse de capacité 756 m$^3$ ? Est-ce qu'on attendra plus de $15$~minutes ?
\end{enumerate}
	 
\vspace{0,5cm}

