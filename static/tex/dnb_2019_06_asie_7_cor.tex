
\medskip

%Les représentations graphiques $C_1$ et $C_2$ de deux fonctions sont données dans le repère
%ci-dessous.
%
%Une de ces deux fonctions est la fonction $f$ définie par $f(x) = -2x + 8$.
%
%\medskip
%
%\parbox{0.48\linewidth}{\psset{unit=0.6cm}
%\begin{pspicture*}(-2,-1.5)(8.5,12.5)
%\psgrid[gridlabels=0,subgriddiv=1,gridwidth=0.2pt]
%\psaxes[linewidth=1.25pt,labelFontSize=\scriptstyle]{->}(0,0)(-2,-1.5)(8.5,12.5)
%\psplot[plotpoints=2000,linewidth=1.25pt]{-2}{4.5}{8 2 x mul sub}
%\psplot[plotpoints=2000,linewidth=1.25pt,linecolor=blue]{-2}{6.5}{x 3 sub dup mul 1 sub}
%\rput(-1.5,10){$C_2$}\rput(6.8,10){\blue $C_1$}
%\end{pspicture*}}\hfill
%\parbox{0.48\linewidth}{\textbf{1.} Laquelle de ces deux représentations est celle de la fonction $f$ ?
%
%\textbf{2.} Que vaut $f(3)$ ?
%
%\textbf{3.}  Calculer le nombre qui a pour image 6 par la fonction $f$.
%
%\textbf{4.}  La feuille de calcul ci -dessous permet de calculer des images par la fonction $f$.
%}
%
%\medskip
%
%\begin{tabularx}{\linewidth}{|c|*{7}{>{\centering \arraybackslash}X|}}\hline
%	&A		&B		&C		&D	&E	&F	&G\\ \hline
%1	&$x$	&$- 2$	&$- 1$	&0	&1	&2	&3\\ \hline
%2	&$f(x)$	&		&		&	&	&	&\\ \hline
%\end{tabularx}
%
%\medskip
%
%Quelle formule peut-on saisir dans la cellule B2 avant de l'étirer vers la droite
%jusqu'à la cellule G2 ?
\begin{enumerate}
\item $f$ est une fonction affine dont la représentation graphique est une droite qui est donc la droite C$_2$.
\item $f(3) = - 2 \times 3 + 8 = - 6 + 8 = 2$ (lisible sur la représentation graphique.
\item Il faut trouver $x$ tel que :
$- 2x + 8 = 6$ soit en ajoutant à chaque membre $2x$ :

$8 = 6 + 2x$, puis en ajoutant $- 6$ :

$2 = 2x$ ou $2\times 1 = 2\times x$ et en simplifiant par 2 : 

$1 = x$. $1$ a pour image $6$ par $f$ (lisible sur la représentation graphique).
\item On peut écrire dans la cellule B2 : = 8-2*B1.
\end{enumerate}
