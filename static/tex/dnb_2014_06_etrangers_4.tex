\textbf{\textsc{Exercice 4} \hfill 5 points}

\medskip

\parbox{0.65\linewidth}{Paul en visite à Paris admire la Pyramide, réalisée en verre feuilleté au centre de la cour intérieure du Louvre. 
Cette pyramide régulière a :

\setlength\parindent{6mm} 
\begin{itemize}
\item[$\bullet~~$] pour base un carré ABCD de côté 35 mètres ; 
\item[$\bullet~~$] pour hauteur le segment [SO] de longueur 22 mètres.
\end{itemize}
\setlength\parindent{0mm}}
\hfill 	\parbox{0.3\linewidth}{\psset{unit=1cm}
\begin{pspicture}(3.3,4.4)
\psline(0.3,0.3)(2.3,0.3)(3,1.5)%ABC
\psline[linestyle=dashed](0.3,0.3)(1,1.5)(3,1.5)%ADC
\psline[linestyle=dashed](0.3,0.3)(3,1.5)%AC
\psline[linestyle=dashed](2.3,0.3)(1,1.5)%BD
\psline[linestyle=dashed](1.65,0.9)(1.65,3.9)(1,1.5)%OSD
\psline(0.3,0.3)(1.65,3.9)(2.3,0.3)%ASB
\psline(1.65,3.9)(3,1.5)%SC
\uput[dl](0.3,0.3){A} \uput[dr](2.3,0.3){B} \uput[r](3,1.5){C} \uput[ul](1,1.5){D} \uput[u](1.65,3.9){S} 
\end{pspicture}}

\medskip
	 
Paul a tellement apprécié cette pyramide qu'il achète comme souvenir de sa visite une lampe à huile dont le réservoir en verre est une réduction à l'échelle $\dfrac{1}{500}$ de la  vraie pyramide.
 
Le mode d'emploi de la lampe précise que, une fois allumée, elle brûle 4 cm$^3$ d'huile par heure.
 
Au bout de combien de temps ne restera-t-il plus d'huile dans le réservoir ? Arrondir à l'unité d'heures.
 
\textbf{Rappel :} \emph{Volume d'une pyramide = un tiers du produit de l'aire de la base par la hauteur}

\medskip
 
\textbf{Faire apparaitre sur la copie la démarche utilisée. Toute trace de recherche sera prise en compte lors de l'évaluation même si le travail n'est pas complètement abouti.}
 
\vspace{0,5cm}

