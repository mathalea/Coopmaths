\textbf{\textsc{Exercice 5} \hfill 3 points}

\medskip

Pascale, Alexis et Carole se partagent deux boîtes de 12 macarons chacune, soit 24
macarons au total.

Soit $x$ le nombre de macarons mangés par Pascale. Le nombre de macarons mangés par
Alexis est donc de $x + 4$ et celui de Carole $2x$.

On peut écrire et résoudre l'équation :

$x + x + 4 + 2x = 24$

$4x + 4 = 24$

$4 x + 4 - 4 = 24 - 4$

$4x = 20$

$\dfrac{4x}{4} = \dfrac{20}{4}$

$x = 5$.

Pascale a donc mangé 5 macarons, Alexis 9 macarons (4 de plus que Pascale) et Carole
10 (2 fois plus que Pascale).

On a bien : $5 + 9 + 10=24$.
\vspace{0,5cm}

