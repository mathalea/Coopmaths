\documentclass[10pt]{article}
\usepackage[T1]{fontenc}
\usepackage[utf8]{inputenc}%ATTENTION codage UTF8
\usepackage{fourier}
\usepackage[scaled=0.875]{helvet}
\renewcommand{\ttdefault}{lmtt}
\usepackage{amsmath,amssymb,makeidx}
\usepackage[normalem]{ulem}
\usepackage{diagbox}
\usepackage{fancybox}
\usepackage{tabularx,booktabs}
\usepackage{colortbl}
\usepackage{pifont}
\usepackage{multirow}
\usepackage{dcolumn}
\usepackage{enumitem}
\usepackage{textcomp}
\usepackage{lscape}
\newcommand{\euro}{\eurologo{}}
\usepackage{graphics,graphicx}
\usepackage{pstricks,pst-plot,pst-tree,pstricks-add}
\usepackage[left=3.5cm, right=3.5cm, top=3cm, bottom=3cm]{geometry}
\newcommand{\R}{\mathbb{R}}
\newcommand{\N}{\mathbb{N}}
\newcommand{\D}{\mathbb{D}}
\newcommand{\Z}{\mathbb{Z}}
\newcommand{\Q}{\mathbb{Q}}
\newcommand{\C}{\mathbb{C}}
\usepackage{scratch}
\renewcommand{\theenumi}{\textbf{\arabic{enumi}}}
\renewcommand{\labelenumi}{\textbf{\theenumi.}}
\renewcommand{\theenumii}{\textbf{\alph{enumii}}}
\renewcommand{\labelenumii}{\textbf{\theenumii.}}
\newcommand{\vect}[1]{\overrightarrow{\,\mathstrut#1\,}}
\def\Oij{$\left(\text{O}~;~\vect{\imath},~\vect{\jmath}\right)$}
\def\Oijk{$\left(\text{O}~;~\vect{\imath},~\vect{\jmath},~\vect{k}\right)$}
\def\Ouv{$\left(\text{O}~;~\vect{u},~\vect{v}\right)$}
\usepackage{fancyhdr}
\usepackage[french]{babel}
\usepackage[dvips]{hyperref}
\usepackage[np]{numprint}
%Tapuscrit : Denis Vergès
%\frenchbsetup{StandardLists=true}

\begin{document}
\setlength\parindent{0mm}
% \rhead{\textbf{A. P{}. M. E. P{}.}}
% \lhead{\small Brevet des collèges}
% \lfoot{\small{Polynésie}}
% \rfoot{\small{7 septembre 2020}}
\pagestyle{fancy}
\thispagestyle{empty}
% \begin{center}
    
% {\Large \textbf{\decofourleft~Brevet des collèges Polynésie 7 septembre 2020~\decofourright}}
    
% \bigskip
    
% \textbf{Durée : 2 heures} \end{center}

% \bigskip

% \textbf{\begin{tabularx}{\linewidth}{|X|}\hline
%  L'évaluation prend en compte la clarté et la précision des raisonnements ainsi que, plus largement, la qualité de la rédaction. Elle prend en compte les essais et les démarches engagées même non abouties. Toutes les réponses doivent être justifiées, sauf mention contraire.\\ \hline
% \end{tabularx}}

% \vspace{0.5cm}\textbf{\textsc{Exercice 3} \hfill 6 points}

\medskip

%\parbox{0.4\linewidth}{\psset{unit=0.75cm}
%\begin{pspicture}(-3.8,-4.5)(3.8,4.5)
%\pscircle(0,0){3.5}\psline[linestyle=dashed](0.8,3.4)(-0.8,-3.4)
%\pspolygon(1.8,-3)(-3.1,-1.6)(0.8,3.4)%CBA
%\uput[ur](0.8,3.4){A} \uput[dl](-3.1,-1.6){B} \uput[dr](1.8,-3){C} \uput[ur](0,0){O}\uput[d](-0.8,-3.4){M}
%\psdots(0,0)
%\rput(-1.6,1){5} 
%\end{pspicture}}\hfill
%\parbox{0.55\linewidth}{On considère un triangle ABC isocèle en A tel que l'angle $\widehat{\text{BAC}}$ mesure 50\degres{} et AB est égal à 5~cm.
% 
%On note O le centre du cercle circonscrit au triangle ABC. La droite (OA) coupe ce cercle, noté ($C$), en un autre point M.
%
%\medskip

\begin{enumerate}
\item %Quelle est la mesure de l'angle $\widehat{\text{BAM}}$ ? Aucune justification n'est demandée. 
Le triangle est isocèle en A, donc AB = AC.

O est le centre du cercle circonscrit au triangle, donc OA = OC.

Les deux points A et O sont équidistants de A et de C, donc la droite (AO) est la médiatrice de [BC]. C’est aussi la bissectrice de $\widehat{\text{BAC}}$, donc $\widehat{\text{BAM}} = 25\degres$.
\item %Quelle est la nature du triangle BAM ? Justifier. 
A et M sont diamétralement opposés. [AM] est un diamètre, donc le triangle ABM est un triangle rectangle en B.
\item %Calculer la longueur AM et en donner un arrondi au dixième de centimètre près.
Dans le triangle ABM rectangle en M, on a $\cos   \widehat{\text{BAM}} = \dfrac{\text{AB}}{\text{AM}}$ ; donc AM $ = \dfrac{\text{AB}}{ \cos  \widehat{\text{BAM}}} = \dfrac{5}{ \cos 25} \approx 5,51$ soit environ 5,5~cm au dixième près.
\item %La droite (BO) coupe le cercle ($C$) en un autre point K. Quelle est la mesure de l'angle $\widehat{\text{BKC}}$ ?
$\widehat{\text{BAC}} = \widehat{\text{BKC}}$ car ce sont des angles inscrits qui interceptent le même arc. Donc $\widehat{\text{BKC}} = 50\degres$.

%Justifier. 
\end{enumerate}

\bigskip

\end{document}