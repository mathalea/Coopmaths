
\medskip
		
%Recopier la bonne réponse (aucune justification n'est attendue).
%
%\medskip
%\renewcommand\arraystretch{2.2}
%\begin{tabularx}{\linewidth}{|c|m{3.5cm}|*{3}{>{\centering\arraybackslash}X|}}\cline{3-5}
%\multicolumn{2}{c|}{~}&Réponse A &Réponse B &Réponse C\\ \hline
%1&La somme $\dfrac{7}{4} + \dfrac{2}{3}$ est égale à : &$\dfrac{9}{7}$&$\dfrac{29}{12}$&$\dfrac{9}{12}$\\ \hline
%2&L'équation $5x + 12 = 3$ a pour solution :& 1,8 &3 &$- 1,8$\\ \hline
%3&Une valeur approchée, au dixième près, du nombre $\dfrac{\sqrt{5} + 1}{2}$ est :&2,7 &1,6 &1,2\\ \hline
%\end{tabularx}
%\renewcommand\arraystretch{1.}
\begin{enumerate}
\item $\dfrac{7}{4} + \dfrac{2}{3} = \dfrac{7\times 3}{4\times 3} + \dfrac{2\times 4}{3\times 4} = \dfrac{21 + 8}{4\times 3} = \dfrac{29}{12}$.
\item $5x + 12 = 3$ entraine $5x = 3 - 12$ ou $5x = - 9$, d'où $x = - \dfrac{9}{5} = - \dfrac{18}{10} = - 1,8$.
\item $2,23 < \sqrt{5} < 2,24$, donc $3,23 < \sqrt{5} + 1 < 3,24$ et $1,615 < \dfrac{\sqrt{5} + 1}{2}< 1,62$, donc $\dfrac{\sqrt{5} + 1}{2} \approx 1,6$ au dixième près.
\end{enumerate}
		
\vspace{0,5cm}

