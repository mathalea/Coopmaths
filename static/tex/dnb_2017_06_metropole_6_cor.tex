\textbf{\textsc{Exercice 6} \hfill 4 points}

\medskip
\begin{enumerate}
\item 
	\begin{enumerate}
		\item Il y a 3 personnes sur 6 en situation de surpoids ou d'obésité.
		\item La formule écrite en B3 et recopiée à droite est =B2/(B1*B1).
	\end{enumerate}
\item 
	\begin{enumerate}
		\item $m = \dfrac{9 \times 20 + 12 \times 22 + 6\times 23 +8 \times 24- + 2\times 25 + 29 + 30 + 2 \times 33}{41} = \dfrac{949}{41} \approx  23$
		
L'IMC moyen des employés de cette entreprise est d'environ 23.
		\item L'effectif de cette entreprise est de 41, la médiane est donc la 21\up{e} valeur de la série ordonnée, c'est-à-dire 22. 
		
L'IMC médian est donc de 22, cela signifie qu'au moins 50\,\% des salariés ont un IMC inférieur ou égal à (respectivement supérieur ou égal à) 22.
		\item $2 + 1 + 1 + 2 = 6$.

Il y a 6 personnes en situation de surpoids ou d'obésité dans cette entreprise.

$\dfrac{6}{41} \times  100 = 15 > 5$.

Environ 15\,\% des employés de cette entreprise sont en situation de surpoids ou
d'obésité, donc plus de 5\,\%. L'affirmation du magazine est vraie pour cette
entreprise.
	\end{enumerate}
\end{enumerate}

\bigskip

