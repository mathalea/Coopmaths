
\medskip

Mathilde fait tourner deux roues de loterie A et B comportant chacune quatre secteurs numérotés comme sur le schéma ci-dessous:

\begin{center}

\psset{unit=1cm} 
\begin{pspicture}(-2,-2.5)(2,2)
\pscircle(0,0){2}
\psline(-2,0)(2,0)\psline(0,2)(0,-2)
\uput[d](0,-2){Roue A}
\rput(1;135){1} \rput(1;45){2} \rput(1;-45){3} \rput(1;-135){4} 
\pspolygon[fillstyle=solid,fillcolor=gray](1.6;60)(2.3;55)(2.3;65)
\end{pspicture}\hspace{1.5cm}
\begin{pspicture}(-2,-2.5)(2,2)
\pscircle(0,0){2}
\psline(-2,0)(2,0)\psline(0,2)(0,-2)
\uput[d](0,-2){Roue B}
\rput(1;135){6} \rput(1;45){7} \rput(1;-45){8} \rput(1;-135){9} 
\pspolygon[fillstyle=solid,fillcolor=gray](1.6;60)(2.3;55)(2.3;65)
\end{pspicture}
\end{center} 

La probabilité d'obtenir chacun des secteurs d'une roue est la même. Les flèches indiquent les deux secteurs obtenus. 

L'expérience de Mathilde est la suivante: elle fait tourner les deux roues pour obtenir un nombre à deux chiffres. Le chiffre obtenu avec la roue A est le chiffre des dizaines et celui avec la roue B est le chiffre des unités. 

\emph{Dans l'exemple ci-dessus, elle obtient le nombre $27$ (Roue A : $2$ et Roue B : $7$)}.

\medskip
 
\begin{enumerate}
\item Écrire tous les nombres possibles issus de cette expérience. 
\item Prouver que la probabilité d'obtenir un nombre supérieur à 40 est 0,25. 
\item Quelle est la probabilité que Mathilde obtienne un nombre divisible par 3 ? 
\end{enumerate} 

\vspace{0.5cm}

