\textbf{Exercice 1 \hfill 5 points}

\medskip

%\emph{Cet exercice est un questionnaire à choix multiples (QCM). Aucune justification n'est
%demandée.\\
%Pour chaque question, trois réponses (A, B et C) sont proposées. Une seule d'entre elles
%est exacte. Recopier sur la copie le numéro de la question et la réponse exacte.\\
%Une bonne réponse rapporte $1$ point.\\
%Une mauvaise réponse ou l'absence de réponse n'enlève aucun point.}
%
%\begin{center}
%\begin{tabularx}{\linewidth}{|c|m{4cm}|*{3}{>{\centering \arraybackslash}X|}}\cline{3-5}
%\multicolumn{2}{c|}{~}&A &B &C\\ \hline
%1&L'écriture en notation scientifique du nombre \np{587000000} est :&$5,87\times 10^{- 8}$& $587 \times 10^6$& $5,87 \times 10^8$\\ \hline
%2&Si on développe et réduit l'expression $(x + 2)(3x -1)$ on obtient:& $3x^2 + 5x - 2$ &$3x^2 + 6x +2$ &$3x^2 - 1$\\ \hline
%3&Dans un parking il y a des motos  et des voitures. On compte 28 véhicules et 80 roues. Il y a donc :&20 voitures& 16 voitures &12 voitures\\ \hline
%4& Le produit de 18 facteurs égaux à $- 8$ s'écrit:&$- 8^{18}$&$(- 8)^{18}$& $18 \times  (- 8)$\\ \hline
%5& La section d'un cylindre de révolution de diamètre 4 cm et de  hauteur 10 cm par un plan parallèle à son axe peut être :&un rectangle de dimensions 3 cm et 10 cm&un rectangle de  dimensions 5 cm et 10 cm&un rectangle de dimensions 3 cm et 8 cm\\ \hline
%\end{tabularx}
%\end{center}
\begin{enumerate}
\item Réponse C : $\np{587000000} = 5,87 \times 10^8.$
\item Réponse A : $(x + 2)(3x - 1) = 3x^2 - x + 6x - 2 = 3x^2 + 5x - 2$.
\item $12 \times 4 + 16 \times 2 = 48 + 32 = 80$.
\item Réponse B : $- 8 \times (- 8) \times \cdots \times(- 8) =  (- 8)^{18}$.
\item Réponse A : Si on coupe par un plan parallèle à son axe, la longueur du
rectangle obtenu est 10~cm, la largeur est inférieur ou égale à 4~cm ; seule la réponse A
convient.
\end{enumerate}

\vspace{0,5cm}

