\textbf{\textsc{Exercice 6} \hfill 6 points}

\medskip 

Pendant le remplissage d'une écluse, Jules et Paul, à bord de leur péniche, patientent en jouant aux dés. Ces dès sont équilibrés. 

\medskip

\begin{enumerate}
\item Est-ce que, lors du jet d'un dé, la probabilité d'obtenir un \og 1 \fg{} est la même que celle d'obtenir un \og 5 \fg{} ? Expliquer. 
\item Jules lance en même temps un dé rouge et un dé jaune. Par exemple il peut obtenir 3 au dé rouge et 4 au dé jaune, c'est l'une des issues possibles. Expliquer pourquoi le nombre d'issues possibles quand il lance ses deux dés est de 36.

\medskip

\begin{tabularx}{\linewidth}{|X|}\hline 
Jules propose à Paul de jouer avec ces deux dés (un jaune et un rouge), Il lui explique la règle :

\setlength\parindent{8mm} 
\begin{itemize}
\item Le gagnant est le premier à remporter un total de 1000 points.
\item Si, lors d'un lancer, un joueur fait deux \og 1 \fg, c'est-à-dire une paire* de \og 1 \fg, il remporte \np{1000} points (et donc la partie).
\item Si un joueur obtient une paire de 2, il obtient 100 fois la valeur du 2, soit  $2 \times  100 = 200$~points.
\item De même, si un joueur obtient une paire de 3 ou de 4 ou de 5 ou 6, il obtient 100 fois la valeur du dé soit $3 \times  100 = 300$, ou \ldots
\item Si un joueur obtient un résultat autre qu'une paire (exemple 3 sur le dé jaune et 5 sur le dé rouge), il obtient $50$ points.
\end{itemize}
\setlength\parindent{0mm}

* On appelle une paire de 1 quand on obtient deux 1, une paire de 2 quand on obtient deux 2 \ldots\\ \hline
\end{tabularx}

\medskip
 
\item Paul a déjà fait 2 lancers et a obtenu $650$~points.
 
Quelle est la probabilité qu'il gagne la partie à son troisième lancer ?
 
\textbf{Dans cette question, si le travail n'est pas terminé, laisser tout de même sur la copie une trace de la recherche. Elle sera prise en compte dans la notation.} 
\end{enumerate}

\vspace{0,5cm}

