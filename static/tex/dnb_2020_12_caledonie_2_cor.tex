
\medskip

%Un prix TTC (Toutes Taxes Comprises) s'obtient en ajoutant la taxe appelée TGC (Taxe Générale sur la Consommation) au prix HT (Hors Taxes).
%
%En Nouvelle-Calédonie, il existe quatre taux de TGC selon les cas : 22\,\%, 11\,\%, 6\,\% et 3\,\%.
%
%\smallskip
%
%Alexis vient de faire réparer sa voiture chez un carrossier.
%
%Voici un extrait de sa facture qui a été tâchée par de la peinture. 
%
%Les colonnes B, D et E désignent des prix en francs.

\begin{center}
\begin{tabularx}{\linewidth}{|c|l|*{4}{>{\centering \arraybackslash}X|}}\hline
&A&B&C&D&E\\ \hline
1& \textbf{Référence}	&Prix HT&TGC (en \,\%)&Montant TGC&Prix TTC\\ \hline
2& Phare avant			&\np{64000}	&22\,\%	&\np{14080}	&\np{78080} \\ \hline
3& Pare choc			&\np{18000}	&22\,\%	&\pscurve*(-0.5,-0.1)(0,-0.1)(0.6,0.12)(0.8,0.2)(0.5,0.28)(0,0.25)(-0.25,0.3)(-0.5,-0.1)			&\np{21960}\\ \hline
4& Peinture				&\np{11700}	&11\,\%	&\np{1287}	&\np{12987}\\ \hline 
5& Main d'œuvre			&\np{24000}	&\pscurve*(-0.5,0)(0,-0.1)(0.6,0.1)(0.8,0.18)(0.5,0.28)(0,0.25)(-0.25,0.3)(-0.5,0)		&\np{1440}	&\np{25440}\\ \hline
6&\multicolumn{2}{c}{~}&\multicolumn{2}{r|}{\textbf{TOTAL À RÉGLER (en Francs)}}&\textbf{\np{138467}}\\ \hline
\end{tabularx}
\end{center}

\begin{enumerate}
\item %Quel est le montant TGC pour le pare-chocs ?
Le montant TGC pour le pare-chocs est égal à la différence $\np{21960} - \np{18000} = \np{3960}$~(francs).

On peut aussi calculer $\np{18000} \times \dfrac{22}{100} = \np{3960}$~(francs).
\item %Quel est le pourcentage de la TGC qui s'applique à la main d'œuvre ?
On a $\frac{\np{1440}}{\np{24000}} \times 100 = \dfrac{\np{1440}}{24} = 6$~(\,\%)
\item %La facture a été faite à l'aide d'un tableur.

%Quelle formule a été saisie dans la cellule E6 pour obtenir le total à payer ?
Dans la case C6 on écrit : = SOMME(E2:E5)
\end{enumerate}

\vspace{0,5cm}

