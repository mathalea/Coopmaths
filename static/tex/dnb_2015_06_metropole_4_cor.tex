\textbf{\textsc{Exercice 4 \hfill 7,5 points}}

\medskip

\begin{enumerate}
\item On a $f(3) = - 6 \times 3 + 7 = -18 + 7 = -11$.
\item La probabilité qu'Arthur choisisse une chemisette verte est de $\dfrac{1}{3}$. Celle qu'il choisisse un short vert est de $\dfrac{1}{2}$.
  
La probabilité qu'il soit habillé uniquement en vert est donc de $\dfrac{1}{3} \times \dfrac{1}{2} = \dfrac{1}{6}$.
\item On a $2^{40} = 2^{1 + 39} = 2^1 \times 2^{39} = 2 \times 2^{39}$.
  
Ariane a donc bien raison.
\item Le PGCD de 15 et 12 est 3. Loïc n'a donc pas raison.
\item On a $5x - 2 = 3x + 7$ d'où $5x - 3x = 7 + 2$.
  
On a donc $2x = 9$ d'où $x = \dfrac{9}{2} = 4,5$.
  
La solution de cette équation est donc 4,5.
\end{enumerate}

\newpage

