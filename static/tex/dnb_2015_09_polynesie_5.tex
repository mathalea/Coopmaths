\textbf{Exercice 5 \hfill 7 points}

\medskip

Laurent s'installe comme éleveur de chèvres pour produire du lait afin de fabriquer des fromages.

\medskip

\textbf{PARTIE 1 : La production de lait}

\medskip

\textbf{Document 1}

\textbf{Chèvre de race alpine :} 

\textbf{Production de lait :} 1,8 litre de lait par jour et par chèvre en moyenne

\textbf{Pâturage :} 12 chèvres maximum par hectare 

\medskip

\textbf{Document 2}

Plan simplifié des surfaces de pâturage.

\begin{center}
\psset{unit=0.75cm}
\begin{pspicture}(9.5,7.8)
\pspolygon(0.2,0.5)(3.8,0.5)(3.8,4.1)(9.4,4.1)(9.4,7.5)(0.2,7.5)
\psframe(0.2,0.5)(0.6,0.9)
\psframe(3.8,0.5)(3.4,0.9)
\psframe(9.4,4.1)(9,4.5)
\psframe(9.4,7.5)(9,7.1)
\psframe(0.2,7.5)(0.6,7.1)
\uput[u](4.8,7.5){620 m}
\uput[d](2,0.5){240 m}
\psline(2,0.7)(2,0.3)
\psline(3.6,2.3)(4,2.3)
\psline(9.2,5.8)(9.6,5.8)
\end{pspicture}
\end{center}

\textbf{Document 3}

1 hectare = \np{10000} m$^2$

\begin{enumerate}
\item Prouver que Laurent peut posséder au maximum 247 chèvres.
\item Dans ces conditions, combien de litres de lait peut-il espérer produire par jour en moyenne ?
\end{enumerate}

\bigskip
 
\textbf{PARTIE 2 : Le stockage du lait}
 
\medskip

Laurent veut acheter une cuve cylindrique pour stocker le lait de ses
chèvres.

Il a le choix entre 2 modèles :

\setlength\parindent{6mm}
\begin{itemize}
\item[$\bullet~~$] cuve A : contenance 585 litres
\item[$\bullet~~$] cuve B : diamètre 100 cm, hauteur 76 cm
\end{itemize}
\setlength\parindent{0mm} 
 
Formule du volume du cylindre : $V = \pi \times  r^2 \times h$
 
Conversion : 1 dm$^3$ = 1 L
 
 \medskip
 
Il choisit la cuve ayant la plus grande contenance. Laquelle va-t-il acheter ?
 
\vspace{0,5cm}

